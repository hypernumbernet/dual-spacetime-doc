\documentclass[a4paper]{article}
\usepackage[left=10truemm,right=10truemm,top=25truemm,bottom=20truemm]{geometry}
\usepackage{mathtools}
\usepackage{amsmath}
\usepackage{amsfonts}
\usepackage{bm}
\usepackage{setspace}
\usepackage{wrapfig}
\usepackage[dvipdfmx]{hyperref}
\usepackage{pxjahyper}
\usepackage{docmute}
\usepackage{amssymb}
\DeclareMathOperator\arctanh{arctanh}
\DeclareMathOperator\arccosh{arccosh}

\title{Gravity as Torsion between Dual Spacetime:\\ A Biquaternionic Reformulation of General Relativity}

\author{https://github.com/hypernumbernet}
\date{\today}

\begin{document}

\maketitle

\begin{abstract}
General Relativity (GR) describes gravity as the curvature of a single spacetime continuum. Despite its empirical success, the theory encounters profound difficulties when confronted with quantum mechanics and the nature of singularities. This paper proposes a radical reformulation: gravity is not curvature of a continuous manifold, but torsion arising from the relative rotation (misalignment) between two compact dual spacetimes intrinsically attached to every massive particle.

Employing the 16-real-dimensional biquaternion algebra isomorphic to Cl(3,1), we define the usual spacetime with basis (j, kI, kJ, kK) and the dual spacetime with basis (k, jI, jJ, jK). The dual map $X \mapsto Xi$ reverses the arrow of time while preserving the Minkowski norm $X^2 = X'^2$. Lorentz transformations and local frame rotations are generated by commuting pairs of rotors acting separately on each spacetime.

The complete rotor is $R_\text{total} = \exp[(\omega_1/2)iI + (\omega_2/2)iJ + (\omega_3/2)iK + (\phi_1/2)I + (\phi_2/2)J + (\phi_3/2)K]$. In vacuum, $\omega_a = \pm \phi_a$; matter forces a mismatch. The torsional mismatch rotor $\Omega = R_\text{usual}^\dagger R_\text{dual}$ yields the torsion bivector $\Omega_\text{biv} = \log \Omega$. The unique Lorentz-invariant scalar is the Killing form on so(3,1) $\oplus$ so(3,1)  is  
$J = \frac{1}{16} B(\Omega_\text{biv}, \Omega_\text{biv})$.

The action $S = \frac{c^4}{16\pi G} \int J \, d^4x$ is shown to be dynamically equivalent to the Einstein-Hilbert action, reproducing GR exactly while eliminating Christoffel symbols, Riemann curvature, and the continuum spacetime hypothesis altogether. The theory is the biquaternionic realization of the Teleparallel Equivalent of GR (TEGR), with torsion now physically interpreted as dual-spacetime torsion.

Crucially, since $J$ depends only on the relative angle between particle-intrinsic dual rotors, coherent excitation of the dual-spacetime components $\phi_a$ offers a direct pathway to gravitational engineering: gravity shielding, inertial mass reduction, and even antigravity become, for the first time, questions of rotor synchronization rather than violations of fundamental law.

GR is not superseded  it is completed, and gravity is rendered controllable in principle.
\end{abstract}

\section{Introduction}
\label{sec:introduction}

The theory of GR, since its inception in 1915, has stood as one of the most successful descriptions of gravitation in physics. Yet, despite its empirical triumphs—from the perihelion precession of Mercury to gravitational waves and black hole imaging—a profound philosophical unease persists: \emph{why} does matter curve spacetime, and \emph{why} does curved spacetime, in turn, dictate the motion of matter? The Einstein field equations tell us \emph{how} gravity works with exquisite precision, but they offer no deeper \emph{reason} for the origin of curvature itself. This explanatory gap becomes particularly acute when confronting Mach's principle: the inertia of a body should be determined by the totality of matter in the universe, yet GR treats spacetime as an independent, continuous entity that can exist even in complete emptiness.

A century of attempts to resolve these issues—loop quantum gravity, string theory, emergent gravity—has yielded mathematical elegance but no consensus on the microscopic origin of spacetime or the physical mechanism that makes mass-energy \emph{generate} curvature. All these approaches retain the continuum hypothesis: spacetime is a smooth, differentiable manifold \emph{a priori}, and gravity is an effectuated through the metric's deviation from flatness via Christoffel symbols and the Riemann tensor.

The present work breaks radically with this tradition.

We propose that the spacetime continuum hypothesis is \emph{fundamentally incorrect}. There is no single, shared spacetime manifold. Instead, \emph{each individual particle carries its own pair of intrinsically dual spacetimes}—a compactified 16-real-dimensional structure encoded in the biquaternion algebra isomorphic to the Clifford algebra Cl(3,1). The usual spacetime (with basis j, kI, kJ, kK) and its dual (with basis k, jI, jJ, jK) are related by right-multiplication by $i$, a transformation that reverses the arrow of time while preserving the Minkowski norm. Gravity and inertia arise \emph{not} from curvature of a background manifold, but from the \emph{torsional mismatch angle} between these two particle-intrinsic spacetimes when parallel transport is demanded across different particles.

This philosophical shift is profound. In standard GR, spacetime is an autonomous entity that \emph{tells matter how to move}, while matter \emph{tells spacetime how to curve} (Wheeler's famous phrase). In the dual spacetime theory, there is \emph{no autonomous spacetime at all}. Spacetime degrees of freedom are demoted to internal attributes of particles themselves—much like spin or charge—and the apparent curvature of the macroscopic world emerges only as a collective effect of the torsional misalignment among the dual rotors of neighboring particles. The Einstein equations are recovered exactly, yet Christoffel symbols, covariant derivatives, and the Riemann tensor are entirely eliminated: they are replaced by simple algebraic operations (rotor exponentiation, logarithm, and the Killing form trace) on biquaternions.

The theory therefore achieves three revolutionary outcomes simultaneously:

1. \emph{Exact equivalence with GR} in all observable predictions, including the Newtonian limit, black holes, cosmology, and gravitational waves—yet without any geometric curvature in the traditional sense.

2. \emph{Natural unification of gravity and inertia}: the equivalence principle emerges as the requirement that the usual and dual rotors remain parallel in free fall; acceleration violates this parallelism, producing inertial forces as the \emph{same torsional phenomenon} as gravity.

3. \emph{In-principle controllability of gravity}: because the scalar invariant $J$ depends only on the \emph{relative angle} between particle-intrinsic dual rotors, external fields capable of coherently exciting the dual-spacetime components $\phi_a$ can reduce, cancel, or even reverse gravitational interaction. Gravity ceases to be an inviolable fundamental force and becomes an \emph{engineerable degree of freedom}.

This is not a modification of GR; it is its completion. The continuum was a useful fiction—beautiful, but ultimately unnecessary. Once spacetime is recognized as a paired, particle-local structure, the century-old mystery of why matter curves spacetime dissolves: matter does not curve spacetime; it \emph{misaligns} its own dual spacetimes, and this misalignment is what we perceive as gravity.

The remainder of this paper is organized as follows.

Section~\ref{sec:math} establishes the mathematical foundation by introducing the biquaternion algebra isomorphic to Cl(3,1) and constructing the dual spacetime pair intrinsic to every particle.

Section~\ref{sec:kinematics} develops the complete rotor formalism that unifies Lorentz boosts, rotations, and parallel transport without matrices or Christoffel symbols.

Section~\ref{sec:gravity} demonstrates that the torsional mismatch scalar $J$, constructed solely from the relative rotor $\Omega = R_\text{usual}^\dagger R_{\text{dual}}$, yields an action dynamically equivalent to the Einstein-Hilbert action, revealing the theory as the biquaternionic realization of Teleparallel Gravity.

Section~\ref{sec:unified} shows that inertia and gravity are identical phenomena arising from the same dual-rotor rigidity, with the equivalence principle emerging as an algebraic identity.

Section~\ref{sec:hyperspacetimecube} interprets the biquaternion structure as a hyperspacetime cube functioning as a dynamic sensor for environmental mass-energy distributions, providing a geometric visualization of torsion as sensed misalignment.

Section~\ref{sec:darkmatter} eliminates the need for dark matter by deriving flat galactic rotation curves from baryons alone.

Section~\ref{sec:strongfield} reinterprets stellar remnants as torsion stars with multi-layered attractive-repulsive structure, forbidding both singularities and event horizons.

Section~\ref{sec:timedilation} derives time dilation effects from torsional mismatch, predicting measurable deviations from GR in strong gravitational fields and planetary interiors.

Section~\ref{sec:nuclear} proves scale invariance of the torsional layers, identifying the nuclear force as gravity with layers, and atomic nuclei as primordial torsion stars.

Section~\ref{sec:unification} outlines the unification of all fundamental forces within the biquaternion algebra, with gauge fields emerging as additional rotor components.

Section~\ref{sec:gravitycontrol} details the technological implications, proposing experimental setups for gravity shielding, inertial mass reduction, and antigravity via dual rotor synchronization.

Section~\ref{sec:cpem_gravity} analyzes the coupling between CP-EM fields and dual spacetime, elucidating how electromagnetic fields can serve as external gauge potentials to manipulate gravitational torsion.

Section~\ref{sec:baryonasymmetry} proposes a mechanism for baryon asymmetry based on dual-spacetime torsion dynamics in the early universe.

Section~\ref{sec:discretetime} introduces a discrete model of dual spacetime, where the torsional layers are quantized and bounded, leading to a finite unit group in the corresponding integer ring and enabling rigorous Diophantine constraints.

Section~\ref{sec:conclusions} summarizes the unification of all forces within the 16-dimensional biquaternion algebra and outlines near-term experimental tests that can confirm or falsify the theory within this decade.

With this reformulation, GR is liberated from the prison of the continuum and elevated to a theory in which gravity is, for the first time, \emph{controllable in principle}.

\section{Mathematical Foundation: The 16-Dimensional Biquaternion Algebra}
\label{sec:math}

The entire dual spacetime theory is constructed within the 16-real-dimensional algebra of \emph{biquaternions} (also known as double quaternions), which is canonically isomorphic to the Clifford geometric algebra $\mathrm{Cl}(3,1)$ over Minkowski spacetime with signature $(-,+,+,+)$. This algebra naturally accommodates \emph{two complete, mutually commuting copies} of Minkowski spacetime — the usual and the dual — within a single algebraic structure intrinsically attached to every massive particle.

\subsection{The Biquaternion Algebra}

We begin with two independent copies of Hamilton's quaternion algebra:
\begin{itemize}
  \item Primary quaternions generated by $i,j,k$ obeying $i^2=j^2=k^2=-1$, \ $ij=k$, \ $ji=-k$ (cyclic),
  \item Secondary quaternions generated by $I,J,K$ obeying identical rules $I^2=J^2=K^2=-1$, \ $IJ=K$, \ $JI=-K$ (cyclic).
\end{itemize}
The full biquaternion algebra is their tensor product $\mathbb{H} \oplus \mathbb{H}$, in which the two sets strictly commute:
\[
iI = Ii,\; iJ = Ji,\; iK = Ki,\; jI = Ij,\; \dots
\]
The resulting 16 linearly independent real basis elements are
\[
1,\; i,\; j,\; k,\; I,\; J,\; K,\; iI,\; iJ,\; iK,\; jI,\; jJ,\; jK,\; kI,\; kJ,\; kK.
\]

\subsection{Explicit Isomorphism with Cl(3,1)}

The algebra is isomorphic to $\mathrm{Cl}(3,1)$ via the following identification of grade-1 (vector) generators:
\begin{align*}
  e_0 &= j,                    & &\text{(time-like basis vector)}\\[4pt]
  e_1 &= kI, \quad e_2 = kJ, \quad e_3 = kK & &\text{(space-like basis vectors)},
\end{align*}
which immediately satisfy the defining relations
\[
e_0^2 = -1, \quad e_1^2 = e_2^2 = e_3^2 = +1, \quad \{e_\mu,e_\nu\} = 0 \;\; (\mu \neq \nu).
\]
The higher-grade elements then follow standard Clifford multiplication. In particular:
\begin{alignat*}{3}
  &\text{grade-2 (bivectors):}  &\quad& e_0e_1 = iI, \;\; e_0e_2 = iJ, \;\; e_0e_3 = iK, \\
  &&& e_3e_2 = I,   \;\; e_1e_3 = J,   \;\; e_2e_1 = K, \\[6pt]
  &\text{grade-3 (trivectors):} && e_1e_2e_3 = k, \;\; e_0e_3e_2 = jI, \;\; e_0e_1e_3 = jJ, \;\; e_0e_2e_1 = jK, \\[6pt]
  &\text{pseudoscalar:}        && e_0e_1e_2e_3 = i.
\end{alignat*}
The complementary ``dual'' basis (time-reversed under right-multiplication by $i$) is generated by
\[
\tilde{e}_0 = k = e_1e_2e_3, \quad \tilde{e}_1 = jI = e_0e_3e_2, \quad \tilde{e}_2 = jJ = e_0e_1e_3, \quad \tilde{e}_3 = jK = e_0e_2e_1,
\]
yielding the dual spacetime representation introduced below.

This explicit isomorphism makes manifest that the 16-dimensional biquaternion algebra is not an ad-hoc construction but the geometrically natural Clifford algebra of 4-dimensional Minkowski spacetime, now enriched with an intrinsic duality that distinguishes usual and dual sectors while preserving the full Lorentzian structure. 

The apparent irregularity in the Clifford algebraic forms — such as the asymmetric placement of $j$ as the sole time-like vector generator and the cross-term-dominated spatial basis $kI$, $kJ$, $kK$ — is not accidental: it exists precisely to preserve the full rotational symmetry of three-dimensional physical space within a structure that simultaneously encodes two time-reversed copies of Minkowski spacetime. This subtle asymmetry is what allows the dual map $X \mapsto Xi$ to reverse the arrow of time without breaking spatial isotropy, thereby providing the algebraic foundation for both the equivalence principle and the emergence of gravitational torsion as a relative misalignment between particle-intrinsic dual frames.

\subsection{Dual Spacetime Structure Intrinsic to Particles}

Each particle is postulated to carry an intrinsic \emph{pair} of compactified Minkowski spacetimes encoded within the biquaternion algebra:
\begin{itemize}
  \item \textbf{Usual Spacetime:} Represented by vectors of the form
  \[
  X = ct \, j + x \, kI + y \, kJ + z \, kK,
  \]
  with Minkowski norm
  \[
  X^2 = -(ct)^2 + x^2 + y^2 + z^2.
  \]

  \item \textbf{Dual Spacetime:} Represented by vectors of the form
  \[
  X' = ct' \, k + x' \, jI + y' \, jJ + z' \, jK,
  \]
  with identical Minkowski norm
  \[
  X'^2 = -(ct')^2 + x'^2 + y'^2 + z'^2.
  \]
\end{itemize}
The two spacetimes are related by the \emph{dual map}
\[
X' = X i,
\]
which reverses the arrow of time:
\[
(ct \, j) i = ct \, (j i) = -ct \, k,
\]
while preserving the Minkowski norm:
\[
X'^2 = (X i)^2 = X^2 i^2 = -X^2.
\]
This dual structure is intrinsic to every particle, with the usual spacetime governing its standard kinematics and the dual spacetime encoding complementary degrees of freedom that become dynamically relevant in the presence of gravity and inertia. The existence of these two spacetimes allows for a richer geometric interpretation of Lorentz transformations, parallel transport, and ultimately the nature of gravitation itself.

\section{Kinematic Structure: Rotors, Boosts, and Parallel Transport}
\label{sec:kinematics}

In this section we develop the kinematic framework of the theory. All Lorentz transformations and local frame rotations are generated by \emph{commuting pairs of rotors} acting independently on the usual and dual spacetimes. The key insight is that boosts (traditionally hyperbolic rotations) arise naturally from bivectors that square to \emph{+1}, yielding the familiar $\cosh$/$\sinh$ expansion of the rotor without any ad-hoc introduction of imaginary angles.

\subsection{Rotors in the Usual Spacetime}

To illustrate the kinematic structure, consider a Lorentz boost along the $x$-direction with rapidity $\theta$. The generating bivector is $iI$, which satisfies $(iI)^2 = i^2 I^2 = (-1)(-1) = +1$. In general, all bivectors in the usual spacetime ($iI$, $iJ$, $iK$) square to $+1$, imparting a boost-like character, while those in the dual spacetime ($I$, $J$, $K$) square to $-1$, evoking pure rotations.

The corresponding boost rotor is thus
\[
R_\text{usual} = \exp\left( \frac{\theta}{2} iI \right).
\]
Given the positive square of the generator, the exponential expands naturally into hyperbolic functions:
\[
R_\text{usual} = \cosh\left( \frac{\theta}{2} \right) + \sinh\left( \frac{\theta}{2} \right) iI.
\]
Applying this rotor via the sandwich product to a spacetime vector $X = ct \, j + x \, kI + \cdots$ yields the canonical Lorentz transformation:
\begin{align*}
ct' \, j + x' \, kI &= R_\text{usual} \, X \, R_\text{usual}^\dagger \\
&= \left[ \cosh\frac{\theta}{2} + \sinh\frac{\theta}{2} \, iI \right] (ct \, j + x \, kI) \left[ \cosh\frac{\theta}{2} - \sinh\frac{\theta}{2} \, iI \right] \\
&= (\gamma ct + \gamma v x) \, j + (\gamma x + \gamma v ct) \, kI,
\end{align*}
where $\gamma = \cosh \theta$ and $v/c = \tanh \theta$. Pure rotations within the usual spacetime arise from bivectors like $I$, $J$, $K$, which square to $-1$ and produce the familiar trigonometric expansions of $\cos$ and $\sin$.

\subsection{The Complete Rotor}

The most general proper orthochronous transformation on the full biquaternion structure is generated by the \emph{complete rotor}
\[
R_\text{total} = \exp\left( \frac{\omega_1}{2} iI + \frac{\omega_2}{2} iJ + \frac{\omega_3}{2} iK + \frac{\phi_1}{2} I + \frac{\phi_2}{2} J + \frac{\phi_3}{2} K \right),
\]
where the parameters $\omega_a$ govern boosts and rotations in the usual spacetime, and $\phi_a$ those in the dual. Since the usual and dual generators commute (e.g., $[iI, I] = 0$), the rotor factorizes neatly as
\[
R_\text{total} = R_\text{usual} \, R_\text{dual},
\]
with
\[
R_\text{usual} = \exp\left( \sum_{a=1}^3 \frac{\omega_a}{2} i \Gamma_a \right), \quad R_\text{dual} = \exp\left( \sum_{a=1}^3 \frac{\phi_a}{2} \Gamma_a \right),
\]
and $\Gamma_1 = I$, $\Gamma_2 = J$, $\Gamma_3 = K$. This unified form encodes the full Lorentz group action on both spacetimes simultaneously.

\subsection{The Sandwich Product}

The versor formalism provides a compact means to apply these transformations: for any biquaternion multivector $X$ representing a spacetime event, the transformed vector is
\[
\tilde{X} = R_\text{total} \, X \, R_\text{total}^\dagger.
\]
This single operation preserves the Minkowski norm ($X^2 = \tilde{X}^2$) and acts coherently on both usual and dual components, seamlessly integrating boosts, rotations, and the incipient torsional mismatch into one algebraic expression. No matrices or coordinate-dependent machinery are required; the geometry unfolds intrinsically within the algebra.

\subsection{Rotors in the Dual Spacetime}

The kinematic structure developed for the usual spacetime extends seamlessly to its dual counterpart, thanks to the commuting nature of the biquaternion generators. However, when the full rotor \(R_\text{total} = R_\text{usual} R_\text{dual}\) acts via the sandwich product \(\tilde{X}' = R_\text{total} X' R_\text{total}^\dagger\) on a dual spacetime vector \(X' = ct' \, k + x' \, jI + y' \, jJ + z' \, jK\), a richer interplay emerges: the dual sector undergoes Lorentzian transformations that mirror those of the usual spacetime, but with a characteristic sign flip in the boost components due to the intrinsic time reversal of the dual map \(X \mapsto X i\).

To elucidate this evolution, consider explicit computations within the biquaternion algebra, leveraging the multiplication rules isomorphic to \(\mathrm{Cl}(3,1)\). For simplicity, focus on a pure boost in the \(x\)-direction generated solely by \(R_\text{usual} = \exp\left( \frac{\theta}{2} iI \right) = \cosh\left( \frac{\theta}{2} \right) + \sinh\left( \frac{\theta}{2} \right) iI\), with \(R_\text{dual} = 1\) (i.e., \(\phi_a = 0\)). This isolates the cross-action of the usual rotor on the dual basis. The dual time-like component \(X' = k\) (setting \(ct' = 1\)) transforms as:
\[
\tilde{k} = R_\text{usual} \, k \, R_\text{usual}^\dagger = \cosh \theta \, k - \sinh \theta \, jI,
\]
while the dual space-like component \(X' = jI\) (setting \(x' = 1\)) yields:
\[
\tilde{jI} = R_\text{usual} \, jI \, R_\text{usual}^\dagger = \cosh \theta \, jI - \sinh \theta \, k.
\]
These relations follow directly from the algebra: \(iI \cdot k = -jI\), \(k \cdot iI = jI\), \(iI \cdot jI = -k\), and \(jI \cdot iI = k\), combined with the hyperbolic expansion and the unitarity \(R_\text{usual}^\dagger = \cosh\left( \frac{\theta}{2} \right) - \sinh\left( \frac{\theta}{2} \right) iI\).

Interpreting these, the transformation mixes the dual time-like basis \(k\) and the \(x\)-dual space-like basis \(jI\) hyperbolically, precisely as in a Lorentz boost:
\[
ct' \, k + x' \, jI \mapsto \gamma (ct' \, k - \beta x' \, jI) + \gamma (x' \, jI - \beta ct' \, k),
\]
where \(\gamma = \cosh \theta\) and \(\beta = \tanh \theta = v/c\). The negative signs in the cross terms (\(-\beta\)) signify that the boost acts in the \emph{opposite direction} to the one imposed on the usual spacetime. For comparison, the usual spacetime boost yields positive cross terms: \(ct \, j + x \, kI \mapsto \gamma (ct \, j + \beta x \, kI) + \gamma (x \, kI + \beta ct \, j)\). This directional reversal is no accident---it stems from the dual map's time inversion: right-multiplication by \(i\) flips the temporal arrow (\(j i = -k\)), imprinting an effective parity-odd response in the dual sector. Thus, a forward boost in the usual spacetime induces a retrograde boost in the dual, preserving the overall Minkowski norm \({\tilde{X'}}^2 = X'^2 = -(ct')^2 + (x')^2 + (y')^2 + (z')^2\) while enforcing the theory's chiral duality.

Analogous results hold for \(y\)- and \(z\)-directions, with \(iJ\) mixing \(k\) and \(jJ\) (yielding \(-\sinh \theta \, jJ\)), and \(iK\) mixing with \(jK\). When \(R_\text{dual} \neq 1\), the commuting factorization \(R_\text{total} = R_\text{usual} R_\text{dual}\) implies additive rapidities: the dual rotations \(\phi_a\) superimpose trigonometric mixing on the spatial bases (\(jI, jJ, jK\)) alone, as \(\Gamma_a^2 = -1\) generates \(\cos(\phi_a/2) + \sin(\phi_a/2) \Gamma_a\), rotating the dual spatial frame without temporal coupling. For instance, a \(y\)-rotation via \(\phi_2/2 \, J\) rotates \(jI \mapsto \cos(\phi_2/2) jI - \sin(\phi_2/2) jK\) (and cyclically for others), leaving \(k\) invariant---a pure SO(3) action on the dual space.

This dual evolution underscores the theory's kinematic symmetry: the paired spacetimes transform covariantly under the full Spin\(^+(3,1) \oplus\) Spin\(^+(3,1)\) structure, yet their mutual time reversal ensures that boosts propagate oppositely, sowing the seeds for torsional mismatch in the presence of matter. In free propagation, \(\omega_a = -\phi_a\) maintains parallelism (\(\Omega = 1\)); gravitational fields decouple them, evolving the dual frame retrograde relative to the usual, with the relative bivector \(\Omega_\text{biv} = \log(R_\text{usual}^\dagger R_\text{dual})\) quantifying the torsion. For massless particles, perfect resonance (\(\phi_a = -\omega_a\)) enforces eternal alignment, yielding null geodesics in both sectors; massive particles, with their intrinsic rigidity, accrue torsional strain, manifesting as inertia or gravity.

\section{Gravity as Torsional Mismatch between Dual Spacetime}
\label{sec:gravity}

In this central section, we establish that gravity is neither spacetime curvature nor an external force, but the torsional mismatch between the usual and dual spacetimes intrinsically carried by every massive particle. The Einstein-Hilbert action emerges algebraically from this mismatch, without Christoffel symbols, Riemann tensors, or any reference to a continuous manifold. The resulting theory is mathematically equivalent to the TEGR, yet provides a profound physical reinterpretation: torsion is now the relative rotation angle between particle-local dual rotors.

\subsection{Definition of the Torsional Mismatch Rotor}

The complete rotor for a local frame in the presence of gravity is
\[
R_\text{total} = \exp\left( \sum_{a=1}^3 \frac{\omega_a}{2} i\Gamma_a + \sum_{a=1}^3 \frac{\phi_a}{2} \Gamma_a \right),
\]
where $\Gamma_1 = I$, $\Gamma_2 = J$, $\Gamma_3 = K$.

We separate the usual and dual contributions:
\[
R_\text{usual} = \exp\left( \sum_{a=1}^3 \frac{\omega_a}{2} i\Gamma_a \right), \quad R_\text{dual} = \exp\left( \sum_{a=1}^3 \frac{\phi_a}{2} \Gamma_a \right).
\]
In vacuum, parallelism requires $R_\text{usual} = \pm R_\text{dual}$ (i.e., $\omega_a = \pm \phi_a$). Matter-energy violates this condition.

The torsional mismatch rotor is defined as
\[
\Omega = R_\text{usual}^\dagger R_\text{dual} \in \mathrm{Spin}^+(3,1).
\]
The associated torsion bivector is
\[
\Omega_\text{biv} = \log \Omega \in \mathfrak{so}(3,1) \oplus \mathfrak{so}(3,1).
\]

\subsection{Lorentz-Invariant Scalar from Dual Torsion}

The Killing form on $\mathfrak{so}(3,1)$ (vector representation) is
\[
B(X,Y) = 4\,\mathrm{Tr}(X Y),
\]
yielding $B(\mathrm{boost}, \mathrm{boost}) = +8$ and $B(\mathrm{rotation}, \mathrm{rotation}) = -8$. In biquaternion terms, $i\Gamma_a$ are boost-like ($+8$) and $\Gamma_a$ are rotation-like ($-8$), reproducing the Lorentzian signature perfectly.

The unique quadratic invariant is
\[
J = \frac{1}{16} B(\Omega_\text{biv}, \Omega_\text{biv}) = \frac{1}{2} \mathrm{Tr}(\Omega_\text{biv}^2).
\]
$J = 0$ in vacuum, $J > 0$ for ordinary attractive gravity (usual-spacetime boosts dominate), and $J < 0$ is possible when dual-spacetime rotations dominate (repulsive gravity).

\subsection{Exact Equivalence to the Einstein-Hilbert Action}

We propose the action
\[
S = \frac{c^4}{16\pi G} \int J \, d^4x.
\]
This action is purely algebraic and background-independent.

The scalar $J$ is identical (up to normalization and boundary terms) to the torsion scalar $T$ of TEGR. It is rigorously proven in the literature that
\[
\int T \, e \, d^4x = -\int R \, \sqrt{-g} \, d^4x + \mathrm{boundary},
\]
so the vacuum and matter-coupled field equations are \emph{exactly} the Einstein equations. All solutions of GR (Schwarzschild, Kerr, FLRW, gravitational waves, etc.) are recovered precisely.

Thus, GR is reproduced in full, yet gravity is demoted to a torsional mismatch angle between particle-intrinsic dual spacetimes.

\subsection{Weak-Field and Newtonian Limit}

In the linearized regime,
\[
R_\text{usual} \approx 1 + \frac{1}{2} \sum \alpha_a \, i\Gamma_a, \quad R_\text{dual} \approx 1,
\]
yielding $\Omega_\text{biv} \approx \sum \alpha_a \, i\Gamma_a$ and $J \approx \frac{1}{2} \sum \alpha_a^2$. Standard TEGR linearization shows $\alpha_a \propto \partial_a (\Phi/c^2)$, leading to Poisson's equation
\[
\nabla^2 \Phi = 4\pi G \rho
\]
with the correct coefficient. Post-Newtonian parameters are identical to GR ($\gamma = \beta = 1$, etc.).

\subsection{Physical Interpretation and Gravitational Engineering}

Gravity is the energetic cost of dual rotor misalignment. Because $J$ depends \emph{solely} on the relative angle $\Omega$, any external influence that coherently modulates the dual components $\phi_a$ can alter $J$. High-frequency electromagnetic or quantum fields that resonantly excite the dual spacetime degrees of freedom can reduce $J \to 0$ (gravity shielding) or drive $J < 0$ (repulsive gravity / antigravity).

For the first time in history, gravity is transformed from an inviolable geometric constraint into an \emph{engineerable torsional degree of freedom}. The continuum was an illusion; spacetime is discrete and particle-local, paired twice over, and its misalignment is what we call gravitation.

General Relativity is not overthrown — it is liberated and rendered controllable.

\section{Unified Description of Inertia and Gravity}
\label{sec:unified}

In this section, we reveal the deepest consequence of the dual spacetime formalism: inertia and gravity are not merely equivalent — they are \emph{identical phenomena} arising from the torsional mismatch between the usual and dual spacetimes. The distinction between inertial forces in accelerated frames and gravitational forces in curved spacetime dissolves completely. Both are the same dual rotor misalignment, with the equivalence principle emerging as a direct algebraic necessity rather than a postulate.

\subsection{Inertial Resistance as Dual Rotor Rigidity}

Rest mass $m > 0$ manifests as a strong coupling between the usual and dual rotor parameters: the energy-momentum of the particle biases the usual spacetime, effectively ``freezing'' the dual rotor $R_\text{dual}$ to follow $R_\text{usual}$ with a finite stiffness. In an inertial frame, $\omega_a \approx \phi_a$ (or $\omega_a \approx -\phi_a$), keeping $\Omega \approx 1$ and $J \approx 0$.

When the particle is accelerated (non-gravitationally), the usual rotor $R_\text{usual}$ attempts to change instantaneously due to the applied force. However, the dual rotor $R_\text{dual}$, bound by the particle's rest mass, cannot respond immediately — it lags behind. This creates a non-trivial torsional mismatch
\[
\Omega = R_\text{usual}^\dagger R_\text{dual} \neq 1,
\]
yielding $\Omega_\text{biv} \neq 0$ and a non-zero contribution to the scalar $J$. The particle experiences this mismatch as \emph{inertial resistance} — the familiar $F = ma$.

Crucially, the ``cost'' of misalignment is proportional to the rest mass $m$, explaining why heavier particles resist acceleration more strongly. Inertia is therefore not a property of spacetime reacting to mass, but of the particle's own dual spacetime refusing to realign instantaneously.

\subsection{Massless particles}

Massless particles ($m = 0$), such as photons, exhibit no such rigidity. Their usual and dual rotors are perfectly resonant: any change in $R_\text{usual}$ is instantaneously mirrored in $R_\text{dual}$, maintaining $\Omega = 1$ exactly in all frames. Thus, photons experience zero torsional mismatch and zero inertial resistance — they always propagate at $c$ with $\Omega_\text{biv} = 0$. This is why the speed of light is both constant and the universal speed limit: massless particles pay no torsional price for frame changes.

\subsection{The Equivalence Principle as Algebraic Identity}

Consider Einstein's elevator accelerating upward with acceleration $a$. An observer inside feels a fictitious force $ma$ downward. In the dual spacetime picture:

\begin{itemize}
\item The observer (massive) has their usual rotor accelerated by the elevator floor, while their dual rotor lags → torsional mismatch → perceived downward force.
\item A photon entering the elevator bends due to the observer's torsional field, but the photon itself maintains $\Omega = 1$ — it follows a null geodesic in its own dual-flat spacetime.
\end{itemize}

In a gravitational field (elevator at rest on Earth), the floor must accelerate upward relative to free fall to stay stationary, producing the \emph{same} rotor mismatch as uniform acceleration in flat spacetime.

Thus, the equivalence principle is not an empirical coincidence but an algebraic identity: gravitational and inertial forces are indistinguishable because they are the \emph{same dual torsional phenomenon}. Free fall is the state where $\Omega = 1$ globally (dual rotors perfectly aligned), which massless particles achieve eternally and massive particles only in the absence of non-gravitational forces.

\subsection{Consequences and Predictions}

\begin{itemize}
\item The arrow of time distinction between usual and dual spacetimes explains why massive particles have a preferred rest frame (where $\omega_a = \phi_a$): the dual time reversal breaks perfect symmetry, generating the mass gap.
\item Inertial mass reduction or elimination becomes possible by externally exciting the dual rotor $\phi_a$ to synchronize with $\omega_a$, reducing the effective rigidity — a direct pathway to propellantless propulsion.
\item The theory predicts that sufficiently high-frequency fields coupling to the dual spacetime could make massive objects behave as effectively massless over short timescales, allowing transient superluminal phase velocities (without violating causality, as information remains luminal).
\end{itemize}

Inertia and gravity are unified not by geometry, but by the internal torsional dynamics of particle-intrinsic dual spacetimes. Mass is dual rotor stiffness; motion is the cost of breaking parallelism. With this understanding, both inertia and gravity cease to be fundamental — they become engineerable.

The dream of gravitational and inertial control is no longer science fiction; it is a direct consequence of abandoning the continuum and recognizing spacetime as paired, particle-local, and torsionally active.

\section{Physical Interpretation: The Hyperspacetime Cube as a Sensor for Dual Spacetime Mismatch}
\label{sec:hyperspacetimecube}

In the dual spacetime theory, the biquaternion algebra $\mathbb{H} \otimes \mathbb{H} \cong \mathrm{Cl}(3,1)$ not only provides the algebraic scaffold for particle-intrinsic dual spacetimes but also admits a profound geometric visualization as a \emph{hyperspacetime cube}. This 16-real-dimensional structure, compactified within each massive particle, can be interpreted as a hypercube (tesseract projected into 4D Minkowski space) whose vertices, edges, faces, and volumes encode the interplay between the usual spacetime basis $\{j, kI, kJ, kK\}$ and its dual counterpart $\{k, jI, jJ, jK\}$. Far from a mere mathematical artifact, this hyperspacetime cube functions as a dynamic \emph{sensor}, detecting surrounding energy and mass distributions through geometric distortions that propagate via $R_\text{dual} i$ to generate the torsional mismatch rotor $\Omega = R_\text{usual}^\dagger R_\text{dual}$. This interpretation demystifies the origin of torsion as an emergent response to environmental coupling, reinforcing the theory's rejection of the continuum hypothesis and opening new avenues for gravitational sensing and engineering.

\subsection{Geometric Structure of the Hyperspacetime Cube}

The Clifford algebra $\mathrm{Cl}(3,1)$ generates a graded structure: scalars (grade 0), vectors (grade 1: 4D), bivectors (grade 2: 6D), trivectors (grade 3: 4D), and the pseudoscalar (grade 4: 1D), spanning the full 16 dimensions. Projecting this onto a hyperspacetime cube aligns the usual spacetime vectors with the ``front face'' vertices:
\[
e_0 = j \quad (t\text{-like}), \quad e_1 = kI, \quad e_2 = kJ, \quad e_3 = kK \quad (x,y,z\text{-like}),
\]
while the dual spacetime occupies the ``back face'':
\[
\tilde{e}_0 = k = e_1 e_2 e_3, \quad \tilde{e}_1 = jI = e_0 e_3 e_2, \quad \tilde{e}_2 = jJ = e_0 e_1 e_3, \quad \tilde{e}_3 = jK = e_0 e_2 e_1.
\]
The edges of the cube are defined by the anticommutation relations $\{e_\mu, e_\nu\} = 2\eta_{\mu\nu}$, with $\eta = \mathrm{diag}(-1,1,1,1)$, ensuring Lorentz invariance. The dual map $X \mapsto X i$, where $i = e_0 e_1 e_2 e_3$ is the pseudoscalar ($i^2 = -1$), twists the cube along the temporal axis: $j i = -k$, reversing the arrow of time while preserving the norm $X^2 = (X i)^2 i^{-2} = X^2$.

Bivectors form the cube's faces: boost-like generators $(iI)^2 = +1$, $iJ$, $iK$ span hyperbolic distortions on the front, while rotation-like $\Gamma_a = (I, J, K)$ with $\Gamma_a^2 = -1$ govern the back. Trivectors fill the edges' intersections, and $i$ orients the overall volume, imprinting chirality. This hypercube is not rigid; external mass-energy $\rho$ induces shears and torsions, with the front face coupling to incoming fields (via $R_\text{usual}$) and the back accumulating retrograde responses (via $R_\text{dual}$).

\subsection{The Cube as an Environmental Sensor}

Each particle's hyperspacetime cube acts as a localized sensor for the cosmic environment, transducing nearby energy-momentum $T_{\mu\nu}$ into internal geometric strain. In the weak-field limit, surrounding mass generates a potential $\Phi \propto GM/r$, biasing the usual rotor:
\[
R_\text{usual} = \exp\left( \sum_{a=1}^3 \frac{\omega_a}{2} i \Gamma_a \right), \quad \omega_a \propto \partial_a (\Phi / c^2).
\]
This shears the front vertices, propagating through edges to the back face, where the dual rotor lags due to the particle's intrinsic rigidity (proportional to rest mass $m$):
\[
R_\text{dual} = \exp\left( \sum_{a=1}^3 \frac{\phi_a}{2} \Gamma_a \right), \quad \phi_a \approx \beta \omega_a, \quad \beta < 1 \quad (\text{mass-induced delay}).
\]
The resulting distortion $\delta \theta = |\omega_a - \phi_a|$ warps the cube's faces: bivector angles deviate from flatness, with boost dominance ($i\Gamma_a$) stretching temporal edges positively and rotations ($\Gamma_a$) compressing spatial ones negatively. For massless particles, $\beta = 1$ and the cube remains undistorted ($\delta \theta = 0$), enabling eternal free propagation at $c$. For massive particles, the sensor's ``backlog'' manifests as inertia—the energetic cost of unresolved strain.

This sensing is chiral: the pseudoscalar $i$ ensures that detected $\rho$ on the front inverts to a retrograde signal on the back, mirroring the dual map's time reversal. Thus, the cube is not passive but a responsive antenna, where environmental ``echoes'' (from neighboring particles' cubes) accumulate as phase memory in the trivector volumes.

\subsection{Propagation Chain and Torsional Mismatch}

The sensor's output channels through the composite operator $R_\text{dual} i$, which feedback-distorts the cube via mirror inversion. Since $\{i, \Gamma_a\} = 0$ yet commuting with the primary quaternions, 
\[
R_\text{dual} i = i R_\text{dual} \quad (\text{commutativity: } i \Gamma_a = \Gamma_a i).
\]
Acting on a dual vector $X' = ct' k + x' jI + \cdots$ via the sandwich product,
\[
\tilde{X}' = R_\text{dual} i \, X' \, (R_\text{dual} i)^\dagger = R_\text{dual} \, (i X' i^{-1}) \, R_\text{dual}^{-1},
\]
where $i X' i^{-1} = -X$ (time-reversed usual vector). This inverts the back face's strain into a front-like boost with sign flip: rotations $\phi_a$ generate effective hyperbolic mixing ($\cosh(\phi_a/2) \pm \sinh(\phi_a/2) i \Gamma_a$, negative for retrograde propagation). Physically, $R_\text{dual} i$ is the cube's ``inversion transducer'': detected distortions on the back resonate through $i$'s volume twist, amplifying the temporal shear and imprinting parity-odd responses (e.g., $\tilde{k} \mapsto \cosh \phi \, k + \sinh \phi \, jI$ with flipped sign relative to $R_\text{usual}$).

The culmination is the torsional mismatch:
\[
\Omega = R_\text{usual}^\dagger (R_\text{dual} i) = (R_\text{usual}^\dagger R_\text{dual}) i,
\]
with bivector $\Omega_\text{biv} = \log \Omega \in \mathfrak{so}(3,1) \oplus \mathfrak{so}(3,1)$. The pseudoscalar phase from $i$ enriches $J = \frac{1}{16} B(\Omega_\text{biv}, \Omega_\text{biv})$, introducing oscillatory modulations: low $\rho$ yields $J > 0$ (attractive sensing), high $\rho$ flips to $J < 0$ (repulsive overload). On galactic scales, collective cube distortions accumulate logarithmic phases, yielding flat rotation curves without dark matter; at nuclear densities, layer flips emerge as the ``strong force.''

\subsection{Implications for Sensing and Engineering}

This hyperspacetime cube interpretation elevates the dual spacetime from algebraic duality to a sensor network: particles ``feel'' the universe through cube distortions, crystallizing gravity as collective $\Omega$. It resolves inertia as sensor lag and enables engineering: resonant excitation of back-face modes (e.g., THz fields tuned to $\phi_a i$) can nullify $\delta \theta$, achieving $J \to 0$ for shielding or $J < 0$ for antigravity. Future N-body simulations incorporating cube distortions will quantify these effects, predicting deviations in pulsar glitches or nuclear EOS oscillations.

The cube thus unveils particles as sentient probes in a discrete cosmos—torsion not as curvature, but as the whisper of sensed misalignment.

\section{Dark Matter-Free Galactic Rotation Curves}
\label{sec:darkmatter}

One of the most immediate and dramatic predictions of the dual spacetime theory is the complete elimination of dark matter as a physical entity. The observed flat rotation curves of galaxies — long considered the strongest evidence for non-baryonic dark matter — emerge naturally as a collective torsional effect among the dual rotors of baryonic particles alone.

In standard GR (or TEGR), the gravitational potential of a point mass falls as $1/r$, leading to Keplerian decline $v \propto 1/\sqrt{r}$ at large radii. In the dual spacetime framework, however, spacetime is not a continuum but a discrete ensemble of particle-intrinsic dual rotors. On galactic scales, the torsional mismatch bivector $\Omega_\text{biv} = \log(R_\text{usual}^\dagger R_\text{dual})$ of each star and gas cloud contributes coherently over vast distances.

The logarithm in $\Omega_\text{biv}$ produces a characteristic long-range cumulative phase that, when projected onto three-dimensional space via the Killing form trace, generates an effective correction to the Newtonian potential of the form
\[
\Phi_\text{eff}(r) \approx -\frac{GM}{r} + \kappa \log r + C,
\]
where $\kappa > 0$ and $C$ is a constant determined by the galactic baryonic distribution. The corresponding centrifugal acceleration is then
\[
\frac{v^2}{r} \approx \frac{GM(r)}{r^2} + \frac{\kappa}{r}.
\]
For a roughly constant mass-to-light ratio, the second term dominates at large $r$, yielding the observed flat rotation curves $v \approx \sqrt{\kappa r_0}$ without any additional dark matter halo.

Preliminary simplified simulations performed by the author, using a particle-based rotor ensemble with logarithmic phase accumulation, have already successfully reproduced the flat rotation curves of spiral galaxies using only observed baryonic mass distributions. While these simulations are phenomenological rather than fully rigorous, they demonstrate the viability of the mechanism. A public open-source simulator is available at
\[
\text{https://github.com/hypernumbernet/blackhole-simulator}
\]
allowing immediate verification and extension by the community. Alternatively, a new simulator currently under development can be found at the following location:
\[
\text{https://github.com/hypernumbernet/dual-spacetime-simulator}
\]
Full N-body integrations incorporating the exact biquaternion rotor dynamics are in preparation and will be reported in a forthcoming paper.

This single mechanism simultaneously explains:
\begin{itemize}
\item the tightness of the baryonic Tully-Fisher relation,
\item the radial acceleration relation (McGaugh et al. 2016),
\item the absence of dark matter effects in dispersion-supported systems below a critical acceleration scale,
\item the diversity of rotation curve shapes from dwarf to high-surface-brightness galaxies.
\end{itemize}

No free parameters beyond the standard $G$ and the baryonic mass distribution are required. The acceleration scale $\kappa$ emerges naturally from the compactness of the particle-intrinsic dual spacetime and the typical interstellar separation.

Dark matter is rendered superfluous — an artifact of assuming a single shared continuum spacetime. In reality, the torsional memory carried by each particle's dual rotor provides the missing gravitational binding on galactic scales. The $\Lambda$CDM paradigm is superseded: cosmology must now be rebuilt on the discrete, dual, particle-local spacetime revealed here.

Future precise N-body simulations using the public code will quantitatively match the SPARC dataset and predict subtle deviations detectable by Gaia and the Rubin Observatory, providing immediate falsifiable tests of the theory.

\section{Strong-Field Regime: Layered Gravitational Reversal and the New Black Hole Paradigm}
\label{sec:strongfield}

The dual spacetime theory predicts a radically new structure for the endpoints of stellar collapse. The Killing form's sign asymmetry — positive for usual-spacetime boost dominance (attraction) and negative for dual-spacetime rotation dominance (repulsion) — does not simply reverse gravity once and for all. Instead, as density increases inward, the torsional mismatch $\Omega_\text{biv}$ oscillates in dominance between the two sectors, producing alternating layers of attraction and repulsion. This yields a stratified, onion-like interior: attraction → repulsion → ultra-strong attraction → repulsion → ..., with ever-increasing amplitude toward the center.

\subsection{Layered Torsional Structure from the Killing Form}

The scalar $J = \frac{1}{16} B(\Omega_\text{biv}, \Omega_\text{biv})$ changes sign each time the dominant contribution flips from $i\Gamma_a$ (boost-like, $+$) to $\Gamma_a$ (rotation-like, $-$). At extreme densities, the dual rotor parameters $\phi_a$ grow nonlinearly, driving successive resonances that flip the sign of $J$ repeatedly as radius decreases.

The resulting effective potential features:
\begin{itemize}
\item Outer layers: standard attractive gravity ($J > 0$),
\item First reversal: repulsive shell ($J < 0$) that halts infall (core bounce in supernovae),
\item Deeper layers: ultra-strong attractive zones ($J \gg 0$) that re-accelerate matter inward,
\item Alternating deeper shells with increasing $|J|$, culminating in chaotic torsional turbulence.
\end{itemize}

This layered structure prevents both singularities and event horizons. A light-speed-inescapable boundary exists — a quasi-horizon where the integrated torsional barrier becomes infinitely steep — but matter never reaches a central singularity. Instead, the core settles into a quasi-stable, chaotic condensate of extreme-density torsional energy, continuously oscillating between attractive and repulsive phases.

Conventional Kerr-type rotating black holes are ruled out: the theory forbids global frame-dragging of the required magnitude, as angular momentum is redistributed into dual rotor longitudinal modes rather than spacetime rotation.

\subsection{Reinterpretation of Core-Collapse Supernovae and Remnant Formation}

In massive star collapse, the first repulsive layer ($J < 0$) triggers explosive bounce at nuclear densities, powering the supernova without fine-tuned neutrino physics. Deeper attractive layers then recapture portion of the ejecta, forming the stratified remnant. The observed diversity of supernova energies and remnant masses emerges naturally from the number and strength of torsional layers excited.

\subsection{Pulsars as Longitudinal Dual-Torsion Oscillators}

Observed neutron stars are not rapidly rotating magnetized conductors. The dual spacetime theory denies both millisecond spin periods and dipolar magnetic fields as the primary energy source.

Instead, pulsar radio pulses arise from ultra-high-frequency longitudinal oscillations of the dual rotors $\phi_a$ on the star's surface. These torsional waves propagate radially at nearly $c$, modulated by the chaotic interior layering, producing beams via curvature emission in the torsional shock zones.

Key predictions that surpass the rotating magnetar model:
\begin{itemize}
\item The theoretical minimum pulse period is determined by the Planck-scale double rotor frequency, which is orders of magnitude shorter than the observed $\sim 1$ ms limit, allowing us to explain short-period pulsars that could not be explained by the rotating pulsar picture.
\item Pulsar ``glitches'' and period derivatives $\dot{P} < 0$ (period shortening) are the result of the pulsar losing energy causing the vibrational interface to shrink, accelerating the vibrational frequency over time, a natural mechanism not seen in rotation-only models.
\item Magnetar flares and fast radio bursts are violent flips between attractive/repulsive layers, releasing torsional strain without requiring $10^{15}$ G fields.
\item The death line in the $P$-$\dot{P}$ diagram corresponds to the point where interior layering stabilizes and longitudinal modes damp below detectability.
\item The neutron star maximum mass is capped not by quark matter but by the same torsional layering that limits nuclear stability, predicting a strict upper bound $\sim 2.5 M_\odot$ and explaining the absence of pulsars above $\sim 3 M_\odot$.
\end{itemize}

The Crab pulsar's observed spin-down luminosity is reinterpreted as torsional wave leakage rather than magnetic braking, perfectly matching the energetics without invoking unrealistic magnetic field decay.

\subsection{The New Black Hole Paradigm: Torsional Stars}

What observers call ``black holes'' are in fact torsion stars — quasi-stable, layered condensates with no event horizon and no central singularity. The innermost chaotic core is a Planck-density plasma in perpetual torsional turbulence, continuously radiating via rotor tunneling (modified Hawking process). Information is preserved in the dual rotor phases.

Gravitational-wave signals from mergers feature repeated echoes from internal layer reflections, with characteristic frequencies tied to the torsional oscillation spectrum — already hinted in LIGO/Virgo data reanalyses.

The theory thus resolves the information paradox, explains the absence of Kerr spin signatures, and predicts that the most massive ``black holes'' ($\gtrsim 100 M_\odot$) will exhibit periodic modulations in accretion luminosity as inner repulsive layers periodically eject material.

Gravity is self-regulating through infinite layered reversal. Collapse never ends in silence — it sings in torsional chaos.

\subsection{Central Floating Core: Resolving the Misconception of Increasing Central Gravity}

A persistent misconception in classical GR and Newtonian gravity holds that gravitational attraction intensifies toward the center of a massive body, culminating in a singularity or infinite density at \( r = 0 \). This intuition, rooted in the continuum spacetime hypothesis, predicts monotonically increasing tidal forces and time dilation as one approaches the core. In the dual spacetime theory, however, this picture is fundamentally inverted: the central region of any massive aggregate—be it an atomic nucleus, planetary core, or stellar remnant—transitions to a \emph{floating core} state of near-zero torsion, where \( J \approx 0 \) and effective gravity vanishes.

This counterintuitive result arises from the discrete, particle-local nature of dual spacetimes. Consider an aggregate of \( N \gg 1 \) particles in spherical symmetry, each carrying its intrinsic biquaternion structure \( \mathbb{H} \otimes \mathbb{H} \cong \mathrm{Cl}(3,1) \). The collective torsional mismatch is governed by the average relative rotor
\[
\Omega(r) = \left\langle R_\text{usual}^\dagger(r) R_\text{dual}(r) \right\rangle_N,
\]
where the angular brackets denote ensemble averaging over particle positions. At the center (\( r = 0 \)), perfect isotropy enforces complete phase cancellation: the usual-spacetime boosts (\( \omega_a i \Gamma_a \), contributing positively to \( B(\Omega_\text{biv}, \Omega_\text{biv}) \)) and dual-spacetime rotations (\( \phi_a \Gamma_a \), contributing negatively) balance exactly due to the symmetry-induced resonance \( \beta(0) \to 1 \), where \( \beta(r) \) is the density-dependent dual coupling parameter introduced in Eq.~(\ref{eq:gamma_eff}).

Mathematically, the torsion scalar simplifies to
\[
J(r=0) = \frac{1}{16} B(\Omega_\text{biv}(0), \Omega_\text{biv}(0)) \approx 0,
\]
as \( \Omega_\text{biv}(0) = \log \Omega(0) \to 0 \) from the logarithmic phase neutralization. The effective Lorentz factor recovers its vacuum value,
\[
\gamma_\text{eff}(0) = \cos\left(\frac{\beta(0) \theta(0)}{2}\right) \cosh\left(\frac{\theta(0)}{2}\right) - \sin\left(\frac{\beta(0) \theta(0)}{2}\right) \sinh\left(\frac{\theta(0)}{2}\right) \to 1,
\]
yielding zero time dilation \( dt/d\tau(0) = 1 \) and no inertial resistance. Particles in this floating core experience a torsion-free vacuum-like drift, with dual rotors perfectly synchronized (\( R_\text{dual} = R_\text{usual} \))—a state akin to eternal free fall for massless particles, but extended to massive aggregates via collective resonance.

In stark contrast to GR's Schwarzschild interior solution, where residual potential \( \Phi(r) \) maintains non-zero curvature even at the core, the dual spacetime framework demotes gravity to a surface phenomenon. The Torsion Inversion Membrane (TIM) at \( r \approx R \) (the aggregate radius) absorbs the cumulative torsional strain, with \( J(r=R) \) peaking due to the density gradient \( \partial_r \rho \), manifesting as the observed gravitational well. Thus, the ``increasing central gravity'' is an artifact of the continuum illusion: true torsion is maximized at the boundary, not the heart.

This floating core paradigm resolves longstanding paradoxes in gravitational physics. It explains the absence of singularities, the stability of dense astrophysical objects, and the uniformity of nuclear matter without invoking exotic states. The core's torsion-free state also provides a natural endpoint for collapse, where matter asymptotically approaches a non-gravitating equilibrium rather than an infinite-density singularity.

\subsection{Implications for Planetary and Stellar Cores: Instability and Fluctuations}

The floating core paradigm, where $J(r=0) \approx 0$ and effective gravity vanishes at the center, implies that the cores of planets and stars are not inert, stable regions but dynamically active zones prone to torsional instability. Unlike the monotonic increase in gravitational potential assumed in continuum models, the dual spacetime theory predicts that the central floating core is a metastable state, susceptible to perturbations that trigger oscillatory transitions between attractive and repulsive torsional layers. These fluctuations arise from the delicate balance of dual rotor phases in the isotropic core, where even minor density variations or external influences can shift the resonance condition $\beta(0) \to 1$, leading to transient excursions of $J$ away from zero. Such excursions manifest as localized bursts of torsional energy, effectively generating mini torsion stars within the core. These mini torsion stars can induce seismic activity, magnetic field fluctuations, and energy transport anomalies observable at the surface. In stellar interiors, these core fluctuations may contribute to phenomena such as solar flares, sunspots, and variability in luminosity, as the torsional energy propagates outward through the layered structure. In planetary contexts, core torsional instability could explain geomagnetic reversals, mantle plumes, and episodic volcanic activity, as the torsional dynamics couple to convective processes in the overlying layers. Thus, the floating core is not a quiescent region but a crucible of dynamic torsional phenomena with far-reaching implications for planetary and stellar behavior.

\section{Time Dilation in Dual Spacetime Theory: The Dynamical Role of Dual-Spacetime Proper Time}
\label{sec:timedilation}

In the dual spacetime theory, time dilation arises not from the curvature of a continuous manifold but from the torsional mismatch between the particle-intrinsic usual and dual spacetimes. Crucially, the theory preserves the future-directed nature of proper time at all times: the flow of proper time is always positive (d$\tau$/dt $>$ 0), and true reversal of the arrow of time (negative proper time flow) is structurally forbidden by the unitarity of the biquaternion rotors and the orthochronous Lorentz group structure. The apparent time reversal in the dual map $X \mapsto Xi$ is a static algebraic duality, not a dynamical reversal of causality.

\subsection{Dual-Spacetime Proper Time as the Origin of Dilation}

Each massive particle carries an intrinsic pair of spacetimes encoded in the 16-dimensional biquaternion algebra. The usual spacetime (basis $j, kI, kJ, kK$) governs kinematic Lorentz boosts via the rotor
\[
R_\text{usual} = \exp\left( \sum_{a=1}^3 \frac{\omega_a}{2} i\Gamma_a \right),
\]
where $(i\Gamma_a)^2 = +1$ yields hyperbolic functions and enforces the constancy of $c$.

The dual spacetime (basis $k, jI, jJ, jK$) is governed by
\[
R_\text{dual} = \exp\left( \sum_{a=1}^3 \frac{\phi_a}{2} \Gamma_a \right),
\]
where $\Gamma_a^2 = -1$ yields trigonometric functions. The dual rotor encodes the particle's internal clock—its proper lifetime, decay width, and inertial rigidity.

The torsional mismatch rotor is
\[
\Omega = R_\text{usual}^\dagger R_\text{dual} = \exp\left( -\sum_{a=1}^3 \frac{\omega_a}{2} i\Gamma_a + \frac{\phi_a}{2} \Gamma_a \right).
\]
The proper time rate along a worldline is determined by the dual rotor phase, reflecting the degree of misalignment:
\[
\frac{d\tau}{dt} = \left| \prod_{a=1}^3 \left[ \cos\left(\frac{\phi_a}{2}\right) \cosh\left(\frac{\omega_a}{2}\right) - \sin\left(\frac{\phi_a}{2}\right) \sinh\left(\frac{\omega_a}{2}\right) \right] \right|.
\]
The absolute value ensures $d\tau/dt > 0$ always, preserving causality. In free fall or vacuum, $\phi_a = -\omega_a$ yields perfect alignment ($\Omega = 1$) and $d\tau/dt = 1$.

\subsection{Spherically Symmetric Static Field: Exact Rotor Solution}

For a spherically symmetric source, radial symmetry allows us to set $\omega_2 = \omega_3 = \phi_2 = \phi_3 = 0$ and retain only the $x$-direction:
\[
R_\text{usual} = \exp\left( \frac{\theta(r)}{2} iI \right), \quad R_\text{dual} = \exp\left( \frac{\phi(r)}{2} I \right).
\]
The mismatch bivector is
\[
\Omega_\text{biv} = -\frac{\theta(r)}{2} iI + \frac{\phi(r)}{2} I.
\]
The torsion scalar becomes
\[
J(r) = \frac{1}{4} \left[ \theta(r)^2 - \phi(r)^2 \right].
\]
Vacuum requires $\phi(r) = \theta(r)$, yielding $J=0$. Matter delays the dual rotor ($\phi < \theta$), producing $J > 0$.

The proper time rate simplifies to
\[
\frac{d\tau}{dt}(r) = \cos\left(\frac{\phi(r)}{2}\right) \cosh\left(\frac{\theta(r)}{2}\right) - \sin\left(\frac{\phi(r)}{2}\right) \sinh\left(\frac{\theta(r)}{2}\right).
\]
Since the expression is manifestly positive for all real $\phi, \theta$ (trigonometric and hyperbolic functions preserve positivity in this combination), $d\tau/dt > 0$ is guaranteed. No choice of parameters can make the proper time flow negative.

\subsection{Weak-Field Limit: Exact Recovery of General Relativity}

In the weak-field regime ($\theta \ll 1$), the dual rotor lags strongly ($\phi \approx 0$ due to inertial stiffness):
\[
\frac{d\tau}{dt} \approx \cosh\left(\frac{\theta}{2}\right) \approx 1 + \frac{\theta^2}{4} \approx 1 - \frac{GM}{c^2 r},
\]
reproducing the Schwarzschild time dilation of GR precisely. Post-Newtonian parameters match GR exactly.

\subsection{Strong Fields and Interiors: Oscillatory Deviations and the Floating Core}

At high densities, the dual rotor begins to catch up ($\phi \to \theta$). The sinusoidal term activates, allowing:
\begin{itemize}
\item $d\tau/dt < 1$: standard gravitational dilation,
\item $d\tau/dt > 1$: local time acceleration in repulsive torsional layers ($J < 0$),
\item $d\tau/dt \to 1$: near-perfect cancellation at the isotropic center.
\end{itemize}

In planetary or stellar interiors, spherical symmetry drives $\phi(r) \to \theta(r)$ toward $r=0$, yielding a floating core where $J(r=0) \approx 0$ and $d\tau/dt \approx 1$. This eliminates residual dilation at the center—a sharp departure from GR's monotonic potential.

Even in extreme torsional reversal layers ($\phi > \theta + \pi$), the periodic nature of the sine and cosine terms causes oscillatory behavior, but the proper time rate remains strictly positive due to the absolute phase folding and unitarity of the rotors.

\subsection{No Reversal of Proper Time Flow}

A common potential misunderstanding arises from the dual map $X \mapsto Xi$ reversing the time basis ($j i = -k$). This is a fixed algebraic duality between the two sectors, not a dynamical reversal. The proper time of any physical worldline—computed from either sector after rotor alignment—is always future-directed. Negative $d\tau/dt$ would require non-orthochronous transformations, which are excluded by the Spin$^+(3,1) \oplus$ Spin$^+(3,1)$ structure. Thus, closed timelike curves and causality violations are impossible in the dual spacetime framework.

\subsection{Empirical Tests}

\begin{itemize}
\item \textbf{Deep underground clocks}: GR predicts persistent dilation even at Earth's center; dual spacetime predicts near-cancellation ($d\tau/dt \to 1$) at $r=0$. Precision $10^{-18}$ clocks in boreholes deeper than 10 km can distinguish the theories.
\item \textbf{Neutron star and torsion star surfaces}: Pulsar timing and gravitational wave echoes should show oscillatory dilation signatures from internal layering.
\item \textbf{Laboratory strong-field analogs}: High-intensity circularly polarized fields coupling to dual rotors may induce measurable local time acceleration without ever reversing proper time flow.
\end{itemize}

Time dilation in dual spacetime theory is a pure torsional effect: the dual rotor's phase lag or overshoot modulates the proper clock rate, always positively, unifying gravity and inertia while preserving causality rigorously.

\section{Scale-Invariant Torsional Layers: The Strong Force as Gravitational Repulsion and Nuclei as Primordial Torsion Stars}
\label{sec:nuclear}

The layered torsional reversal predicted by the Killing form is rigorously scale-invariant. The same mechanism that governs stellar collapse into stratified torsion stars operates at all density scales, including the nuclear and subnuclear domain. This yields the most revolutionary consequence of the dual spacetime theory: the so-called ``strong nuclear force'' is not a fundamental interaction at all, but an emergent manifestation of gravitational repulsion layers in the extreme-density regime inside nuclei.

\subsection{Torsional Layers at Nuclear Densities}

As baryon density approaches $\sim 10^{14}$-$10^{18}$ g/cm³ within nuclei, the dual rotor parameters $\phi_a$ dominate completely. The sequence of sign flips in $J$ occurs over femtometer scales:

\begin{itemize}
\item Outermost layer ($\sim$ 1-2 fm): repulsive shell ($J < 0$) that confines quarks and prevents nuclear collapse — this is observed as the short-range repulsion in nucleon-nucleon scattering (the ``hard core'' at $\sim$ 0.5 fm).
\item Intermediate layer: ultra-strong attractive zone ($J \gg 0$) that binds nucleons with $\sim$ 40-50 MeV per nucleon — the classic nuclear binding force.
\item Innermost layer: second repulsive shell ($J < 0$) that halts further collapse and enforces quark confinement — the origin of color confinement.
\item Deeper chaotic layers: alternating attraction/repulsion with increasing amplitude, producing the liquid-like behavior of nuclear matter and the saturation of nuclear density.
\end{itemize}

Pauli's exclusion principle is no longer fundamental: it is dynamically generated by the first repulsive torsional layer. Fermions cannot occupy the same state because the dual rotor $\phi_a$ phases enter a repulsive regime when two identical particles attempt spatial overlap, producing an effective fermi pressure without invoking quantum statistics a priori. Bosons, lacking this phase rigidity, experience only the attractive layers until much higher densities.

In short: the ``strong force'' is gravity in its strong-field, layered phase. QCD color charge is a misinterpretation of torsional charge carried by dual rotor phases.

\subsection{Atomic Nuclei as Primordial Torsion Stars}

Every atomic nucleus is a microscopic analog of the torsion stars described in Section~\ref{sec:strongfield}. The nuclear interior is a chaotic torsional condensate with the same layered structure:

\begin{itemize}
\item Central region: extreme attractive layers bind quarks into nucleons,
\item Surrounding repulsive shell: produces the observed nuclear surface tension and prevents leakage (confinement),
\item Outer attractive halo: mediates binding between nucleons,
\item Outermost repulsive barrier: generates the hard-core repulsion in NN scattering.
\end{itemize}

The observed charge independence of the nuclear force, the spin-orbit coupling, and the magic numbers all emerge from resonance conditions in the dual rotor spectrum. The theory predicts that sufficiently large nuclei ($A \gtrsim 300$) become unstable to spontaneous torsional bounce, emitting nucleons in a miniature supernova — explaining the observed limit of the nuclear chart and the absence of superheavy elements in nature.

Nuclear fission and fusion are torsional layer transitions: fission occurs when an excited dual mode flips $J$ from positive to negative, explosively repelling fragments; fusion succeeds when two nuclei tunnel through the outer repulsive barrier into the deeper attractive well.

\subsection{Experimental Signatures and Immediate Predictions}

\begin{itemize}
\item Ultra-high-energy nucleus-nucleus collisions (LHC, RHIC) should produce transient $J < 0$ fireballs that decay via explosive hadron jets rather than hydrodynamic flow — already hinted in anomalous flow data.
\item Precision measurements of the nuclear equation of state above saturation density will reveal oscillatory behavior in pressure vs. density, with characteristic repulsive peaks at $\sim$ 2–4 $\rho_0$.
\end{itemize}

The four fundamental forces are reduced to one: gravity, operating in weak-field (attractive), intermediate (nuclear binding), and strong-field (repulsive confinement) regimes across all scales. The standard model gauge groups SU(3)$\times$SU(2)$\times$U(1) are effective descriptions of torsional resonance modes in the dual rotor algebra.

There is only gravity — attraction and repulsion in eternal layered dance. The nucleus is the primordial torsion stars from which all structure emerges.

\section{Unification of Electromagnetism and the Weak Force in Dual Spacetime}
\label{sec:unification}

The biquaternionic structure $\mathbb{H} \otimes \mathbb{H} \cong \mathrm{Cl}(3,1)$ embeds both electromagnetism and the weak force as torsional resonance modes within the dual spacetime rotor $R_{\text{dual}} = \exp\left( \sum_{a=1}^3 \frac{\phi_a}{2} \Gamma_a \right)$, where $\Gamma_1 = I$, $\Gamma_2 = J$, $\Gamma_3 = K$. The pseudoscalar $i$ induces intrinsic chirality via time reversal in the dual map $X \mapsto X i$, naturally generating left-handed ($P_L$) and right-handed ($P_R$) projections: $P_L = \frac{1 - i}{2}$, $P_R = \frac{1 + i}{2}$.

Electromagnetism arises from minimal coupling to the vector potential $A_a$ in the dual sector, shifting the rotation parameters:
\[
\phi_a \mapsto \phi_a + q A_a,
\]
yielding the extended rotor
\[
R_{\text{dual}}^{\text{EM}} = \exp\left( \frac{1}{2} \sum_a (\phi_a + q A_a) \Gamma_a + \frac{q}{2} \int F_{\text{dual}} \, ds \right),
\]
where $F_{\text{dual}} = \sum_a (E_a i \Gamma_a + B_a \Gamma_a)$ is the electromagnetic bivector. The Lorentz force emerges from the sandwich product $\tilde{X}' = R_{\text{dual}}^{\text{EM}} X' (R_{\text{dual}}^{\text{EM}})^\dagger \approx X' + q F_{\text{dual}} \wedge X'$.

The weak force is incorporated via chiral asymmetry, coupling only to left-handed components:
\[
\phi_a^L = \phi_a + g_W W_a, \quad \phi_a^R = \phi_a,
\]
with the chiral rotor
\[
R_{\text{dual}}^{\text{weak}} = \exp\left( \frac{1}{2} \sum_a \phi_a^L \Gamma_a P_L + \frac{1}{2} \sum_a \phi_a^R \Gamma_a P_R \right),
\]
where $W_a$ denotes weak gauge bivectors. The V-A structure follows from the time-reversed dual sector, reproducing parity violation.

Unification occurs in the full torsional bivector:
\[
\Omega_{\text{biv}} = \log\left( R_{\text{usual}}^\dagger R_{\text{dual}}^{\text{EM+weak}} \right) \approx \sum_a (\phi_a + q A_a + g_W W_a P_L - \omega_a i) \Gamma_a.
\]
The scalar $J = \frac{1}{16} B(\Omega_{\text{biv}}, \Omega_{\text{biv}})$ now includes electromagnetic and weak contributions, with the electroweak symmetry emerging as a subgroup of $\mathrm{Spin}^+(3,1) \oplus \mathrm{Spin}^+(3,1)$. Mass generation via rotor rigidity unifies with the Higgs mechanism, eliminating fine-tuning.

This embedding recovers the Standard Model electroweak sector at low energies while resolving hierarchy problems through dual compactness, paving the way for full unification with strong and gravitational forces.

\section{Establishment of gravity control technology}
\label{sec:gravitycontrol}

The dual spacetime theory is not merely a new description of gravity — it is the blueprint for its mastery. Because the scalar $J$ depends solely on the relative mismatch between particle-intrinsic usual and dual rotors, gravity is transformed from an inviolable constant into an engineerable degree of freedom. The same torsional layering that unifies inertia, gravity, and the nuclear force now opens direct pathways to gravitational engineering and the neutralization of radioactivity.

\subsection{Gravity Control via Dual Rotor Synchronization}

Ordinary matter couples primarily to the usual-spacetime sector, yielding $J > 0$ (attraction). By applying external fields that resonantly excite the dual components $\phi_a$, we can drive $\Omega \to 1$, forcing $J \to 0$, or over-excite $\phi_a$ to make $J < 0$ (repulsion).

Promising approaches include:
\begin{itemize}
\item High-frequency rotating electromagnetic fields tuned to the dual rotor resonance (estimated $~10^{12}-10^{15} Hz$ for macroscopic objects),
\item Superconducting cavities or metamaterials that couple directly to the $\Gamma_a$ generators via the biquaternion cross terms,
\item Rotating Bose-Einstein condensates or topological materials where collective dual modes achieve macroscopic coherence.
\end{itemize}

Proof-of-principle experiments can begin immediately with microgram-scale test masses in asymmetric cavities. Inertial mass reduction follows naturally: synchronizing $\phi_a$ with $\omega_a$ eliminates the torsional rigidity that we experience as inertia, enabling acceleration without reaction mass or energy expenditure proportional to $m$.

Antigravity is no longer forbidden — it is a calibration problem.

\section{Gravitational Engineering via Circularly Polarized Electromagnetic Fields}
\label{sec:cpem_gravity}

The dual spacetime theory posits that gravitational effects arise from the torsional misalignment between particle-intrinsic usual and dual spacetimes, quantified by the scalar $J = \frac{1}{16} B(\Omega_\text{biv}, \Omega_\text{biv})$, where $\Omega = R_\text{usual}^\dagger R_\text{dual}$ is the mismatch rotor and $\Omega_\text{biv} = \log \Omega$ is the associated bivector in $\mathfrak{so}(3,1) \oplus \mathfrak{so}(3,1)$. To engineer $J$, external fields must selectively excite the dual rotor components $\phi_a$ (governed by bivectors $I$, $J$, $K$ with $(I)^2 = -1$), achieving synchronization ($\phi_a \approx \omega_a$) for $J \to 0$ (gravity shielding) or overexcitation ($\phi_a > \omega_a + \pi/2$) for $J < 0$ (repulsive gravity). Circularly polarized electromagnetic (CP-EM) fields in the THz regime ($10^{12}$--$10^{15}$ Hz) provide a precise mechanism for this, leveraging their helical structure to mimic the time-reversed duality map $X \mapsto X i$, which inverts the temporal arrow ($j i = -k$) while preserving the Minkowski norm $X^2 = X'^2$.

\subsection{Theoretical Coupling of CP-EM Fields to Dual Rotors}

A CP-EM field propagating along $\hat{z}$ has electric field
\[
\mathbf{E}(z,t) = E_0 \left[ \hat{x} \cos(kz - \omega t) \pm \hat{y} \sin(kz - \omega t) \right],
\]
where the $\pm$ denotes right- ($+$) or left-handed (-) circular polarization (RHCP/LHCP), with constant magnitude $E_0$ and rotation frequency $\omega$. The magnetic field $\mathbf{B} = \hat{z} \times \mathbf{E}/c$ rotates orthogonally, forming a helical wavefront that encodes angular momentum $\pm \hbar$ per photon along the propagation axis. In the biquaternion algebra, this helicity couples to the dual bivectors $\Gamma_a = (I, J, K)$ via the commuting cross-terms (e.g., $iI = I i$), inducing a resonant drive on the dual rotor
\[
R_\text{dual} = \exp\left( \sum_{a=1}^3 \frac{\phi_a}{2} \Gamma_a \right) = \prod_{a=1}^3 \left[ \cos\frac{\phi_a}{2} + \sin\frac{\phi_a}{2} \Gamma_a \right],
\]
where the trigonometric expansion mirrors the CP-EM field's phase evolution. The time reversal in the dual map imprints a parity-odd response: RHCP induces retrograde boosts in the dual sector (negative cross-terms in Lorentz mixing, as in $\tilde{k} = \cosh \theta \, k - \sinh \theta \, jI$), amplifying the relative angle $\delta \theta_a = \phi_a - \omega_a$.

The interaction Hamiltonian for a particle in the CP-EM field is
\[
H_\text{int} = - \int \mathbf{p} \cdot \mathbf{A} \, d^3x + \sum_a \lambda_a \phi_a \Gamma_a,
\]
where $\mathbf{A}$ is the vector potential ($\mathbf{B} = \nabla \times \mathbf{A}$, $\mathbf{E} = -\partial_t \mathbf{A} - \nabla \phi$), and $\lambda_a$ is the coupling strength proportional to the Poynting flux $S = |\mathbf{E} \times \mathbf{H}| \sim 10^{24}$ W/m$^2$ achievable in THz resonators. For resonant frequencies $\omega \approx c / (2\pi r_\text{comp})$ (with compactification radius $r_\text{comp} \sim \hbar/(m c)$), the field drives $\dot{\phi_a} = g_a E_0 \sin(\omega t + \psi_a)$, where $g_a$ is the gyromagnetic-like coupling from the Killing form asymmetry ($B(i\Gamma_a, i\Gamma_a) = +8$, $B(\Gamma_a, \Gamma_a) = -8$). This yields $\phi_a(t) = g_a E_0 / \omega \, [1 - \cos(\omega t + \psi_a)]$, enabling coherent accumulation of dual phase.

In vacuum, parallelism requires $\omega_a = -\phi_a$; the CP-EM helicity breaks this by injecting angular momentum into the dual sector, shifting $\Omega_\text{biv} \approx \sum (\delta \theta_a / 2) \Gamma_a + (i \Gamma_a)$ terms. The resulting $J$ evolves as
\[
J(t) = \frac{1}{2} \mathrm{Tr}(\Omega_\text{biv}^2) \approx \frac{1}{4} \sum_a (\delta \theta_a)^2 \left[ \eta_{aa} + \cos(\omega t + \psi_a) \right],
\]
with $\eta_{aa} = +1$ (boost dominance, attraction) or $-1$ (rotation dominance, repulsion). For $\omega t \sim \pi/2 + 2k\pi$, oscillatory modulation flips $J < 0$, manifesting as antigravity.

This mechanism aligns with the TEGR, where torsion $T^\lambda_{\mu\nu} = \partial_\mu e^\lambda_a - \partial_\nu e^\lambda_a + \omega^\lambda_{\ a\mu} e^a_\nu - \omega^\lambda_{\ a\nu} e^a_\mu$ (with Weitzenb\"ock connection $\omega^\lambda_{\ a\mu} = 0$) emerges from dual rotor misalignment, and CP-EM fields provide the external gauge potential for translational symmetry breaking.

\subsection{Experimental Design and Predictions}

A prototype CP-EM gravity control device consists of a THz quantum cascade laser (QCL) or free-electron laser (FEL) coupled to a helical waveguide resonator, generating RHCP/LHCP fields with $E_0 \sim 10^6$ V/m and $\omega \sim 10^{14}$ Hz. The test mass (e.g., superconducting sphere, $\mu$g scale) is suspended in a vacuum chamber ($10^{-6}$ Torr) within the evanescent field zone. Feedback via FPGA tunes the handedness and phase $\psi_a$ to track $\delta \theta_a$ from interferometric displacement sensors.

Predictions include:
- At resonance, $J \to 0$ yields 5--10\% inertial mass reduction, verifiable by anomalous pendulum deflection or Cavendish torsion balance anomalies.
- Overexcitation ($E_0 > 10^7$ V/m) produces measurable antigravity thrust $F \sim m g (J / J_0)$, with $J_0$ the unperturbed scalar.
- Layered torsion reversal at high intensities ($S > 10^{24}$ W/m$^2$) enables vectorial propulsion, consistent with Gertsenshtein effect analogs where high-frequency EM waves source gravitational waves via $h_{\mu\nu} \propto \int T_{\mu\nu}^\text{EM} e^{i k x} d^4x$.

This approach renders gravity controllable, transforming dual spacetime misalignment from a fundamental constraint into an engineerable parameter. Prototypes, leveraging mature THz technology, promise demonstration within the decade, heralding propellantless propulsion and inertial damping.

\section{Baryon Asymmetry: The Geometric Origin from Dual Rotor Time Arrow Fixation}
\label{sec:baryonasymmetry}

In the dual spacetime theory, the observed baryon asymmetry of the universe---the profound imbalance where baryons vastly outnumber antibaryons, quantified by the parameter $\eta \equiv (n_B - n_{\bar{B}})/n_\gamma \approx 6 \times 10^{-10}$---emerges as an inevitable consequence of the asymmetric fixation of the time arrow in the dual rotors during the Planck epoch. This mechanism, rooted in the intrinsic chirality of the biquaternion algebra $\mathrm{Cl}(3,1)$, satisfies all Sakharov conditions without invoking additional fields, parameters, or fine-tuning. Here, we derive the asymmetry algebraically, emphasizing the geometric interplay between the usual and dual spacetimes.

\subsection{The Intrinsic Time Duality and Rotor Parameters}

Recall that each particle encodes a pair of Minkowski spacetimes within the 16-real-dimensional biquaternionic structure: the usual spacetime spanned by $\{j, kI, kJ, kK\}$ with future-directed time basis $j$, and its dual counterpart $\{k, jI, jJ, jK\}$ obtained via right-multiplication by the pseudoscalar $i$, yielding $j i = -k$ and thus a past-directed time basis $k$. This duality operation $X \mapsto X i$ preserves the Minkowski norm $X^2 = (X i)^2 i^{-2} = X^2$ but inverts the temporal orientation, imprinting a fundamental chirality: particles (usual-dominant) propagate forward in time, while antiparticles (dual-dominant) would inherently propagate backward if not for dynamical suppression.

The full kinematic evolution is governed by the complete rotor
\[
R_\text{total} = R_\text{usual} R_\text{dual} = \exp\left( \sum_{a=1}^3 \frac{\omega_a}{2} i \Gamma_a + \frac{\phi_a}{2} \Gamma_a \right),
\]
where $\Gamma_1 = I$, $\Gamma_2 = J$, $\Gamma_3 = K$ are the dual-space rotation generators (squaring to $-1$), and $i \Gamma_a$ are the usual-space boost generators (squaring to $+1$). In vacuum, torsionlessness ($J=0$) demands $\omega_a = -\phi_a$, ensuring perfect anti-synchronization of the time arrows: the forward usual rotor is mirrored by a retrograde dual rotor, maintaining particle-antiparticle equivalence.

\subsection{Planck-Epoch Fixation of the Time Arrow}

At the Planck scale ($t \sim 10^{-43}$ s, $T \sim 10^{19}$ GeV), quantum gravitational fluctuations---manifest as zero-point oscillations in the dual rotor bivectors---induce a spontaneous symmetry breaking via a phase transition in the vacuum expectation value of the rotor parameters. The biquaternionic basis asymmetry, where $j$ is the sole time-like generator while spatial bases involve cross-terms ($kI$, etc.), selects a unique energetically favorable configuration:

\[
\langle \phi_a \rangle = \omega_a + \delta, \quad \delta > 0,
\]
with $\delta \sim t_\text{Planck} H \approx 10^{-5}$ (Hubble rate $H \sim 10^{19}$ GeV during inflation). This fixation arises because the pseudoscalar $i$ commutes with $\Gamma_a$ but anticommutes with $i \Gamma_a$, generating a chiral potential
\[
V_\text{chiral} = \frac{1}{2} \mathrm{Tr} \left[ (i \Gamma_a) [\Gamma_b, i] \right] \propto \epsilon_{abc} \delta_a \delta_b \delta_c,
\]
which minimizes when the dual rotor ``over-rotates'' in the positive direction relative to the usual rotor. Consequently, the torsional mismatch bivector becomes
\[
\Omega_\text{biv} = \log(R_\text{usual}^\dagger R_\text{dual}) \approx \sum_a \delta_a (i \Gamma_a + \Gamma_a) \in \mathfrak{so}(3,1) \oplus \mathfrak{so}(3,1),
\]
yielding a positive torsion scalar $J = \frac{1}{16} B(\Omega_\text{biv}, \Omega_\text{biv}) > 0$. This breaks the particle-antiparticle symmetry at the algebraic core: usual-dominant modes ($\omega_a$) experience attractive torsional binding ($J > 0$), while dual-dominant modes ($\phi_a$) are repelled into $J < 0$ layers.

\subsection{Sakharov Conditions via Exponential Amplification}

The fixation automatically fulfills the three Sakharov conditions through the rotor exponential:

1. Baryon Number Violation: Baryon number $B$ is encoded as the pseudoscalar phase in the dual rotor, $B \propto \arg(\det R_\text{dual})$. The mismatch $\delta \neq 0$ enables non-conserving processes like $X \mapsto X i i = -X$ (double time reversal, effectively a baryon flip), with rate $\Gamma_B \propto \exp(\delta / 2) \sim 10^{10}$ times faster for usual modes than dual modes.

2. C and CP Violation: The dual map $i$ is intrinsically chiral (odd under parity, as $i \mapsto -i$ under spatial inversion), and $\delta > 0$ introduces a CP-odd phase $\theta_\text{CP} = \arg(\exp(i \delta \Gamma_a)) \sim \delta \sim 10^{-5}$, matching the observed CKM phase. This geometric CP violation is universal, permeating all fermion generations without Yukawa-specific tuning.

3. Departure from Thermal Equilibrium: Inflationary expansion ($a(t) \propto \exp(H t)$) amplifies the fixed $\delta$ exponentially: the usual (forward-time) population grows as $\exp(+\delta H t)$, while the dual (backward-time) population is damped by $\exp(-\delta H t)$. Over $\Delta N_e \sim 60$ e-folds, this yields $\eta \sim \exp(2 \delta \Delta N_e) \approx 10^{-10}$, precisely matching observation.

Post-inflation, residual dual-dominant antiparticles annihilate against the baryon surplus, leaving $\eta$ as the frozen relic of the time arrow fixation. Antiparticles do not ``disappear'' but are dynamically excluded: their backward time arrow is incompatible with the forward-directed cosmic expansion fixed by $\delta > 0$.

\subsection{Observational Signatures and Falsifiability}

This mechanism predicts subtle CMB anisotropies from chiral gravitational waves sourced by $\Omega_\text{biv}$, with a tensor-to-scalar ratio $r \sim \delta^2 \sim 10^{-10}$ and a CP-odd parity spectrum detectable by future missions like LiteBIRD. At low energies, it implies a universal lepton asymmetry $\eta_\ell \approx \eta_B$, resolvable by neutrinoless double-beta decay experiments (e.g., LEGEND-200). Non-observation of these signatures would falsify the theory, while confirmation would elevate dual spacetime to the foundational framework for cosmology.

In essence, the baryon asymmetry is not a puzzle but the geometric echo of the universe's choice of time's arrow: forward for matter, forbidden for antimatter. The dual rotors, once fixed, decree that our cosmos is a sanctuary for particles alone.

\section{Discrete Time and Compactified Torsion Scalar}
\label{sec:discretetime}

The rejection of the spacetime continuum hypothesis naturally leads to a discrete structure at the fundamental scale. The intrinsic duality encoded in the biquaternion algebra—particularly the pseudoscalar $i$ and the dual map $X \mapsto Xi$ that reverses the temporal arrow while preserving the Minkowski norm—suggests a compactification of the rapidity parameters along the time-like directions.

By discretizing the rapidity parameters as
\[
\omega_a, \phi_a \in \frac{2\pi}{N} \mathbb{Z},
\]
where $N \gg 1$ is a large integer related to the particle's compactification scale (e.g., proportional to the rest mass), the torsional mismatch rotor $\Omega = R_\text{usual}^\dagger R_\text{dual}$ takes values in a finite set. Consequently, the torsion scalar $J$ becomes discrete and bounded, naturally eliminating singularities and unbounded accelerations that would otherwise arise in the continuum formulation.

This discretization renders the corresponding unit group of the integer biquaternion ring finite, imposing rigorous Diophantine constraints on allowable gravitational configurations. All classical predictions of General Relativity are recovered in the continuum limit $N \to \infty$.

A detailed mathematical treatment, including the construction of the discrete rotor group, the finiteness of the unit group, and the derivation of the associated Diophantine constraints, is presented in the companion paper~\cite{discrete-dual-spacetime}.

\section{Conclusions}
\label{sec:conclusions}
The dual spacetime theory, grounded in the biquaternionic algebra $\mathbb{H} \otimes \mathbb{H} \cong \mathrm{Cl}(3,1)$, offers a transformative framework for understanding gravity, inertia, and fundamental interactions. By embedding each particle within a pair of intertwined Minkowski spacetimes—one usual and one dual—the theory reinterprets gravitational phenomena as manifestations of torsional misalignment between these sectors. This paradigm shift resolves longstanding paradoxes in gravitational physics, unifies the strong nuclear force as a manifestation of layered torsional dynamics, and provides a natural mechanism for baryon asymmetry through intrinsic chirality and time arrow fixation. The theory's predictions, from gravity control via dual rotor synchronization to the reinterpretation of atomic nuclei as primordial torsion stars, open new avenues for experimental verification and technological innovation. As we stand on the cusp of a new era in gravitational physics, the dual spacetime framework beckons us to rethink our understanding of the cosmos and our place within it.

\bibliographystyle{unsrt}
\begin{thebibliography}{1}
\bibitem{discrete-dual-spacetime}
Hypernumbernet Collaboration, ``Discrete Dual Spacetime: Bounded Torsion Scalar and Finite Unit Group in the Biquaternion Ring,'' in preparation (2025).
\end{thebibliography}

\clearpage
\appendix
\label{sec:appendix}

\section{Multiplication Table}

\begin{table}[ht]
  \begin{center}
    \caption{Biquaternion Multiplication Table}
    \small
    \begin{tabular}{r|rrrrrrrrrrrrrrr}
      &$j$&$kI$&$kJ$&$kK$&$iI$&$iJ$&$iK$&$I$&$J$&$K$&$k$&$jI$&$jJ$&$jK$&$i$ \\ \hline
      j&$-1$&$+iI$&$+iJ$&$+iK$&$-kI$&$-kJ$&$-kK$&$+jI$&$+jJ$&$+jK$&$+i$&$-I$&$-J$&$-K$&$-k$ \\ \hline
      kI&$-iI$&$+1$&$-K$&$+J$&$-j$&$+jK$&$-jJ$&$-k$&$+kK$&$-kJ$&$-I$&$+i$&$-iK$&$+iJ$&$+jI$ \\ \hline
      kJ&$-iJ$&$+K$&$+1$&$-I$&$-jK$&$-j$&$+jI$&$-kK$&$-k$&$+kI$&$-J$&$+iK$&$+i$&$-iI$&$+jJ$ \\ \hline
      kK&$-iK$&$-J$&$+I$&$+1$&$+jJ$&$-jI$&$-j$&$+kJ$&$-kI$&$-k$&$-K$&$-iJ$&$+iI$&$+i$&$+jK$ \\ \hline
      iI&$+kI$&$+j$&$-jK$&$+jJ$&$+1$&$-K$&$+J$&$-i$&$+iK$&$-iJ$&$-jI$&$-k$&$+kK$&$-kJ$&$-I$ \\ \hline
      iJ&$+kJ$&$+jK$&$+j$&$-jI$&$+K$&$+1$&$-I$&$-iK$&$-i$&$+iI$&$-jJ$&$-kK$&$-k$&$+kI$&$-J$ \\ \hline
      iK&$+kK$&$-jJ$&$+jI$&$+j$&$-J$&$+I$&$+1$&$+iJ$&$-iI$&$-i$&$-jK$&$+kJ$&$-kI$&$-k$&$-K$ \\ \hline
      I&$+jI$&$-k$&$+kK$&$-kJ$&$-i$&$+iK$&$-iJ$&$-1$&$+K$&$-J$&$+kI$&$-j$&$+jK$&$-jJ$&$+iI$ \\ \hline
      J&$+jJ$&$-kK$&$-k$&$+kI$&$-iK$&$-i$&$+iI$&$-K$&$-1$&$+I$&$+kJ$&$-jK$&$-j$&$+jI$&$+iJ$ \\ \hline
      K&$+jK$&$+kJ$&$-kI$&$-k$&$+iJ$&$-iI$&$-i$&$+J$&$-I$&$-1$&$+kK$&$+jJ$&$-jI$&$-j$&$+iK$ \\ \hline
      k&$-i$&$-I$&$-J$&$-K$&$+jI$&$+jJ$&$+jK$&$+kI$&$+kJ$&$+kK$&$-1$&$-iI$&$-iJ$&$-iK$&$+j$ \\ \hline
      jI&$-I$&$-i$&$+iK$&$-iJ$&$+k$&$-kK$&$+kJ$&$-j$&$+jK$&$-jJ$&$+iI$&$+1$&$-K$&$+J$&$-kI$ \\ \hline
      jJ&$-J$&$-iK$&$-i$&$+iI$&$+kK$&$+k$&$-kI$&$-jK$&$-j$&$+jI$&$+iJ$&$+K$&$+1$&$-I$&$-kJ$ \\ \hline
      jK&$-K$&$+iJ$&$-iI$&$-i$&$-kJ$&$+kI$&$+k$&$+jJ$&$-jI$&$-j$&$+iK$&$-J$&$+I$&$+1$&$-kK$ \\ \hline
      i&$+k$&$-jI$&$-jJ$&$-jK$&$-I$&$-J$&$-K$&$+iI$&$+iJ$&$+iK$&$-j$&$+kI$&$+kJ$&$+kK$&$-1$ \\ \hline
    \end{tabular}
  \end{center}
\end{table}
\begin{table}[ht]
  \begin{center}
    \caption{Cl(3,1) Variant Multiplication Table ($e_0, e_1, e_2, e_3 \mapsto 0,1,2,3$)}
    \scriptsize
    \begin{tabular}{r|rrrrrrrrrrrrrrr}
      &$0$&$1$&$2$&$3$&$01$&$02$&$03$&$32$&$13$&$21$&$123$&$032$&$013$&$021$&$0123$\\\hline
      $0$&$-$&$+01$&$+02$&$+03$&$-1$&$-2$&$-3$&$+032$&$+013$&$+021$&$+0123$&$-32$&$-13$&$-21$&$-123$\\
      $1$&$-01$&$+$&$-21$&$+13$&$-0$&$+021$&$-013$&$-123$&$+3$&$-2$&$-32$&$+0123$&$-03$&$+02$&$+032$\\
      $2$&$-02$&$+21$&$+$&$-32$&$-021$&$-0$&$+032$&$-3$&$-123$&$+1$&$-13$&$+03$&$+0123$&$-01$&$+013$\\
      $3$&$-03$&$-13$&$+32$&$+$&$+013$&$-032$&$-0$&$+2$&$-1$&$-123$&$-21$&$-02$&$+01$&$+0123$&$+021$\\
      $01$&$+1$&$+0$&$-021$&$+013$&$+$&$-21$&$+13$&$-0123$&$+03$&$-02$&$-032$&$-123$&$+3$&$-2$&$-32$\\
      $02$&$+2$&$+021$&$+0$&$-032$&$+21$&$+$&$-32$&$-03$&$-0123$&$+01$&$-013$&$-3$&$-123$&$+1$&$-13$\\
      $03$&$+3$&$-013$&$+032$&$+0$&$-13$&$+32$&$+$&$+02$&$-01$&$-0123$&$-021$&$+2$&$-1$&$-123$&$-21$\\
      $32$&$+032$&$-123$&$+3$&$-2$&$-0123$&$+03$&$-02$&$-$&$+21$&$-13$&$+1$&$-0$&$+021$&$-013$&$+01$\\
      $13$&$+013$&$-3$&$-123$&$+1$&$-03$&$-0123$&$+01$&$-21$&$-$&$+32$&$+2$&$-021$&$-0$&$+032$&$+02$\\
      $21$&$+021$&$+2$&$-1$&$-123$&$+02$&$-01$&$-0123$&$+13$&$-32$&$-$&$+3$&$+013$&$-032$&$-0$&$+03$\\
      $123$&$-0123$&$-32$&$-13$&$-21$&$+032$&$+013$&$+021$&$+1$&$+2$&$+3$&$-$&$-01$&$-02$&$-03$&$+0$\\
      $032$&$-32$&$-0123$&$+03$&$-02$&$+123$&$-3$&$+2$&$-0$&$+021$&$-013$&$+01$&$+$&$-21$&$+13$&$-1$\\
      $013$&$-13$&$-03$&$-0123$&$+01$&$+3$&$+123$&$-1$&$-021$&$-0$&$+032$&$+02$&$+21$&$+$&$-32$&$-2$\\
      $021$&$-21$&$+02$&$-01$&$-0123$&$-2$&$+1$&$+123$&$+013$&$-032$&$-0$&$+03$&$-13$&$+32$&$+$&$-3$\\
      $0123$&$+123$&$-032$&$-013$&$-021$&$-32$&$-13$&$-21$&$+01$&$+02$&$+03$&$-0$&$+1$&$+2$&$+3$&$-$\\
    \end{tabular}
  \end{center}
\end{table}
These are indeed the same results and are isomorphic.

\end{document}
