\documentclass[12pt,a4paper]{article}
\usepackage{amsmath, amssymb, amsthm}
\usepackage{geometry}
\geometry{margin=1in}
\usepackage{hyperref}
\usepackage[utf8]{inputenc}
\usepackage{CJKutf8}

\title{双対時空理論におけるキリング形式の導出}
\author{Grok, xAI}
\date{2025年11月23日}

\newtheorem{theorem}{定理}
\newtheorem{lemma}{補題}
\newtheorem{definition}{定義}
\newtheorem{remark}{注記}

\begin{document}

\begin{CJK}{UTF8}{min}
\maketitle

\section{双対時空理論におけるキリング形式の導入}

双対時空理論では、重力は連続的な時空多様体の曲率としてではなく、各粒子が内在する通常時空とその双対のコンパクト化された構造間のねじれとして現れます。この構造は、16実次元の双四元数代数で記述され、これは署名 $(-1,+1,+1,+1)$ を持つクリフォード代数 $\mathrm{Cl}(3,1)$ と同相です。各質量粒子はこのペア構造を伴います。通常時空は基底 $\{j, kI, kJ, kK\}$ で張られ、双対時空は $\{k, jI, jJ, jK\}$ で張られます。時空ベクトルは 
\begin{equation}
    X = ct \, j + x \, kI + y \, kJ + z \, kK
\end{equation}
と表現され、ミンコフスキーノルム
\begin{equation}
    X^2 = -(ct)^2 + x^2 + y^2 + z^2
\end{equation}
を保存します。双対ベクトル
\begin{equation}
    X' = ct' \, k + x' \, jI + y' \, jJ + z' \, jK
\end{equation}
は
\begin{equation}
    X'^2 = -(ct')^2 + x'^2 + y'^2 + z'^2
\end{equation}
を満たし、二重性写像 $X \mapsto Xi$ は時間の符号を反転させる内在的な時間反転を誘導します。

ローレンツ変換と平行移動は、ローター
\begin{equation}
    R = \exp\left( \frac{\omega_a}{2} i \Gamma_a + \frac{\phi_a}{2} \Gamma_a \right)
\end{equation}
で生成されます。ここで $\Gamma_1 = I$, $\Gamma_2 = J$, $\Gamma_3 = K$ で、双四元数は $i^2 = j^2 = k^2 = -1$, $ij = k$, $ji = -k$(巡回)と同様に $I,J,K$ を満たし、両者は互いに可換です。通常ローター
\begin{equation}
    R_\mathrm{usual} = \exp\left( \sum_a \frac{\omega_a}{2} i \Gamma_a \right)
\end{equation}
は通常基底に作用(ブースト型、$(i\Gamma_a)^2 = +1$)、双対ローター
\begin{equation}
    R_\mathrm{dual} = \exp\left( \sum_a \frac{\phi_a}{2} \Gamma_a \right)
\end{equation}
は双対に作用(回転型、$\Gamma_a^2 = -1$)。真空では並行性 $\omega_a = \pm \phi_a$ が要求され、
\begin{equation}
    \Omega = R_\mathrm{usual}^\dagger R_\mathrm{dual} = \pm 1
\end{equation}
となります。物質はずれを生み、ねじれ二ベクトル
\begin{equation}
    \Omega_\mathrm{biv} = \log \Omega \in \mathfrak{so}(3,1) \oplus \mathfrak{so}(3,1)
\end{equation}
を定義します。

重力のスカラー不変量は
\begin{equation}
    J = \frac{1}{16} B(\Omega_\mathrm{biv}, \Omega_\mathrm{biv}) = \frac{1}{2} \mathrm{Tr}(\Omega_\mathrm{biv}^2)
\end{equation}
で、ここで $B(\cdot, \cdot)$ はリー代数 $\mathfrak{so}(3,1)$ 上のキリング形式です。この形式はローレンツ署名を符号付け、ブースト支配の正の寄与 ($J > 0$、吸引) と回転支配の負の寄与 ($J < 0$、反発) を生み、層状構造と $\phi_a$ の共鳴励起による重力工学を可能にします。作用
\begin{equation}
    S = \frac{c^4}{16\pi G} \int J \, d^4x
\end{equation}
はアインシュタイン・ヒルベルト作用と等価で、クリストッフェル記号やリーマン曲率なしに一般相対性理論を再現します。

本章では、キリング形式を厳密に導出します。まず一般リー代数論から始め、次に $\mathfrak{so}(3,1)$ へ移り、最後に双対時空理論の双四元数枠組みへの埋め込みを行います。

\section{キリング形式の一般定義}

\begin{definition}[キリング形式]
有限次元の半単純リー代数 $\mathfrak{g}$($\mathbb{R}$ または $\mathbb{C}$ 上)に対し、リー括弧 $[\cdot, \cdot]$ を用いて、\emph{キリング形式}は対称双対形式
\begin{equation}
B(X, Y) = \mathrm{Tr}_\mathfrak{g} (\mathrm{ad}_X \circ \mathrm{ad}_Y), \quad X,Y \in \mathfrak{g}
\end{equation}
として定義されます。ここで $\mathrm{ad}_X : \mathfrak{g} \to \mathfrak{g}$ は随伴表現 $\mathrm{ad}_X(Z) = [X, Z]$ で、$\mathrm{Tr}_\mathfrak{g}$ はこの表現のトレースです。
\end{definition}

\begin{theorem}[キリング形式の性質]
\label{thm:killingprops}
キリング形式は不変:$B([Z,X], Y) + B(X, [Z,Y]) = 0$(すべての $Z,X,Y \in \mathfrak{g}$)。半単純 $\mathfrak{g}$ では $B$ は非退化です。単純 $\mathfrak{g}$ では $B$ は正定値または不定値(実形式による)。
\end{theorem}

\begin{proof}[不変性の証明スケッチ]
ヤコビ恒等式を微分し、トレースの巡回性を用いる:
\begin{equation}
B([Z,X], Y) = \mathrm{Tr}([\mathrm{ad}_Z, \mathrm{ad}_X] \circ \mathrm{ad}_Y) = -\mathrm{Tr}(\mathrm{ad}_X \circ [\mathrm{ad}_Z, \mathrm{ad}_Y]) = -B(X, [Z,Y]).
\end{equation}
非退化はカルタンの判定基準から:核 $\{X \in \mathfrak{g} \mid B(X,\mathfrak{g}) = 0\}$ はイデアルゆえ、半単純代数では零。
\end{proof}

座標では、構造定数 $f_{ijk}$ ($[T_i, T_j] = f_{ijk} T_k$、基底 $\{T_i\}$)で
\[
B(T_i, T_j) = f_{ilk} f_{jlk} = -f_{kli} f_{klj}.
\]
この二次形式が代数の署名を分類します。

\section{ローレンツリー代数とそのキリング形式}

ローレンツリー代数 $\mathfrak{so}(3,1)$ は正しい正時ローレンツ群 $\mathrm{SO}^+(3,1)$ のリー代数で、ミンコフスキー計量 $\eta_{\mu\nu} = \mathrm{diag}(-1,1,1,1)$ を保存する無限小変換からなります。6次元で、回転生成子 $J_i$ ($i=1,2,3$) とブースト生成子 $K_i$ で張られます。

\begin{definition}[交換関係]
基底は
\begin{align*}
[J_i, J_j] &= \epsilon_{ijk} J_k, \\
[J_i, K_j] &= \epsilon_{ijk} K_k, \\
[K_i, K_j] &= -\epsilon_{ijk} J_k
\end{align*}
を満たします。ここで $\epsilon_{ijk}$ はレヴィ・チヴィタ記号です。
\end{definition}

$\mathfrak{so}(3,1)$ の元はミンコフスキーベクトル $v^\mu$ に $\delta v^\mu = \frac{1}{2} \omega^{\rho\sigma} (J_{\rho\sigma})^\mu_{\ \nu} v^\nu$ で作用し、$J_{\mu\nu} = -J_{\nu\mu}$ は交代生成子です。

\begin{lemma}[行列表現]
基本(ベクトル)表現($\mathbb{R}^{3,1}$ 上)では、生成子は$4\times4$行列:
\[
(J_1)^\mu_{\ \nu} = \begin{pmatrix}
0 & 0 & 0 & 0 \\
0 & 0 & 0 & 0 \\
0 & 0 & 0 & -1 \\
0 & 0 & 1 & 0
\end{pmatrix}, \quad
(K_1)^\mu_{\ \nu} = \begin{pmatrix}
0 & 1 & 0 & 0 \\
1 & 0 & 0 & 0 \\
0 & 0 & 0 & 0 \\
0 & 0 & 0 & 0
\end{pmatrix}
\]
で、$J_{2,3}, K_{2,3}$ は巡回置換です。
\end{lemma}

$\mathfrak{so}(3,1)$ 上のキリング形式は、この表現のトレースに比例:$B(X,Y) = 2(n-2) \mathrm{Tr}(X Y)$($\mathfrak{so}(n)$ 用)が、ローレンツの場合 ($n=4$、署名 $(3,1)$) で $B(X,Y) = 4 \mathrm{Tr}(X Y)$ に調整されます。

\begin{theorem}[$\mathfrak{so}(3,1)$ 上の明示的なキリング形式]
\label{thm:so31killing}
正規化基底に対し、
\[
B(J_i, J_j) = -2 \delta_{ij}, \quad B(K_i, K_j) = 2 \delta_{ij}, \quad B(J_i, K_j) = 0.
\]
よって $X = \sum_i \theta_i J_i + \sum_i \beta_i K_i$ で $B(X,X) = -2 \sum_i \theta_i^2 + 2 \sum_i \beta_i^2$。
\end{theorem}

\begin{proof}
トレースを直接計算。回転の場合:
\[
J_1^2 = \begin{pmatrix}
0 & 0 & 0 & 0 \\
0 & 0 & 0 & 0 \\
0 & 0 & -1 & 0 \\
0 & 0 & 0 & -1
\end{pmatrix}, \quad \mathrm{Tr}(J_1^2) = -2.
\]
ブーストの場合:
\[
K_1^2 = \begin{pmatrix}
1 & 0 & 0 & 0 \\
0 & 1 & 0 & 0 \\
0 & 0 & 0 & 0 \\
0 & 0 & 0 & 0
\end{pmatrix}, \quad \mathrm{Tr}(K_1^2) = 2.
\]
オフ対角トレースは直交性で消滅。双対時空での正規化 $B = 4 \mathrm{Tr}$ は双四元数ノルムに合わせ:$(i\Gamma_a)^2 = +1 \mapsto +8$、$\Gamma_a^2 = -1 \mapsto -8$。
\end{proof}

\begin{remark}
不定署名はローレンツ計量の反映:空間回転は負(コンパクト)、ブーストは正(非コンパクト)。
\end{remark}

\section{双四元数と双対時空への埋め込み}

双四元数代数 $\mathbb{H} \otimes \mathbb{H}$ は $\gamma^0 = j$, $\gamma^1 = kI$, $\gamma^2 = kJ$, $\gamma^3 = kK$ で $\mathrm{Cl}(3,1)$ と同相、$\{\gamma^\mu, \gamma^\nu\} = 2\eta^{\mu\nu}$ を満たします。双対基底は $\tilde{\gamma}^0 = k$, $\tilde{\gamma}^1 = jI$ などで、右乗算 $i$ で関連:$X i = -ct k + x jI + \cdots$。

二ベクトルは $\mathfrak{so}(3,1)$ を生成:ブースト $i\Gamma_a = \gamma^0 \gamma^a$ ($+1$)、回転 $\Gamma_a = \frac{1}{2} \gamma^a \gamma^b$ ($-1$)。ねじれ二ベクトルは $\Omega_\mathrm{biv} = \sum_a \alpha_a i\Gamma_a + \sum_a \beta_a \Gamma_a$ で、$\alpha_a \propto \omega_a - \phi_a$, $\beta_a \propto \phi_a$。

\begin{lemma}[クリフォード表現のトレース]
スピノル表現(次元$4$、ベクトルと一致)で $\mathrm{Tr}(\gamma^\mu \gamma^\nu) = 4 \eta^{\mu\nu}$ ゆえ、二ベクトルトレースで $B(i\Gamma_a, i\Gamma_a) = 8$, $B(\Gamma_a, \Gamma_a) = -8$。
\end{lemma}

よって、
\[
B(\Omega_\mathrm{biv}, \Omega_\mathrm{biv}) = 8 \sum_a \alpha_a^2 - 8 \sum_a \beta_a^2, \quad J = \frac{1}{2} \sum_a \alpha_a^2 - \frac{1}{2} \sum_a \beta_a^2.
\]
真空で $\alpha_a = \beta_a = 0$、$J=0$。物質で $\alpha_a > 0$(吸引)。外部場で $\beta_a$ を励起し $J < 0$(反発)を反転、重力工学を可能に。

\begin{theorem}[TEGRとの等価性]
スカラー $J$ はテレパラレル重力のねじれスカラー $T$ と一致(正規化を除き)、$\int J \, d^4x \sim -\int R \sqrt{-g} \, d^4x + \mathrm{boundary}$ でアインシュタイン方程式を再現。
\end{theorem}

\section{例:弱場ニュートン極限}

線形近似で $R_\mathrm{usual} \approx 1 + \frac{1}{2} \sum \alpha_a i\Gamma_a$、$R_\mathrm{dual} \approx 1$ ゆえ $\Omega_\mathrm{biv} \approx \sum \alpha_a i\Gamma_a$、$J \approx \frac{1}{2} \sum \alpha_a^2$。TEGR線形化で $\alpha_a \propto \partial_a (\Phi/c^2)$、$\nabla^2 \Phi = 4\pi G \rho$ を与える。

この導出は、重力の制御可能性を明らかに:キリング形式の署名二重性がねじれを工学可能な自由度に変え、粒子内在の双対時空で慣性(加速ずれ)と重力(物質ずれ)を統一します。

\bibliographystyle{plain}
\bibliography{references}

\end{CJK}
\end{document}