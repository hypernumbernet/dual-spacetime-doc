\documentclass[11pt,a4paper]{article}
\usepackage[utf8]{inputenc}
\usepackage{amsmath,amssymb,amsfonts}
\usepackage{amsbsy}
\usepackage{amsthm}
\usepackage{geometry}
\usepackage[dvipdfmx]{hyperref}
\usepackage{enumitem}

\title{Discrete Dual Spacetime: Bounded Torsion Scalar and Finite Unit Group in the Biquaternion Ring}

\author{https://github.com/hypernumbernet}
\date{\today}

\begin{document}

\maketitle

\begin{abstract}
The dual spacetime theory rejects the continuum hypothesis entirely, positing that each massive particle carries a pair of intrinsically compactified Minkowski spacetimes encoded in the 16-real-dimensional biquaternion algebra isomorphic to $\mathrm{Cl}(3,1)$. In this companion paper, we formalize the discretization of the rapidity parameters arising from the natural compactness induced by the dual map $X \mapsto Xi$. This discretization renders the torsional mismatch rotor periodic, bounds the torsion scalar $J \in [-1, 1]$, and makes the unit group of the corresponding integer biquaternion ring finite. The resulting Diophantine constraints on gravitational solutions provide a rigorous number-theoretic foundation for quantum gravity effects while preserving exact equivalence to General Relativity in the continuum limit. Physical implications include a natural ultraviolet cutoff, discrete torsional spectra in strong fields, and a finite number of stable nuclear states.
\end{abstract}

\section{Introduction}
\label{sec:intro}

The dual spacetime theory, as presented in Ref.~\cite{main-paper}, achieves an exact algebraic reformulation of General Relativity as the teleparallel equivalent (TEGR) using biquaternions, while interpreting gravity and inertia as torsional mismatch between particle-intrinsic usual and dual spacetimes. A central philosophical tenet is the complete rejection of the spacetime continuum hypothesis: no shared differentiable manifold exists; spacetime degrees of freedom are demoted to internal attributes of particles.

This rejection naturally invites a discrete structure at the fundamental scale. The biquaternion algebra $\mathbb{H} \oplus \mathbb{H} \cong \mathrm{Cl}(3,1)$ already encodes compactness through the pseudoscalar $i$ ($i^2 = -1$) and the dual map $X \mapsto Xi$, which reverses the temporal arrow while preserving the Minkowski norm. The periodicity of rotor exponentials further suggests that the rapidity parameters $\omega_a$ and $\phi_a$ are angular variables on compact circles.

In this paper, we rigorously develop this discretization. By quantizing the rapidity parameters in units of $2\pi/N$, where $N$ is a large integer related to the particle's compactification scale, we obtain:
\begin{itemize}
  \item A bounded torsion scalar $J \in [-1, 1]$ due to the sign asymmetry of the Killing form,
  \item A finite unit group in the corresponding integer biquaternion ring,
  \item Diophantine equations governing allowable torsional configurations.
\end{itemize}
All classical predictions are recovered in the continuum limit $N \to \infty$. The framework provides a natural bridge to quantum gravity without invoking additional structures.

\section{Review of Continuous Dual Spacetime}
\label{sec:review}

Each particle carries usual spacetime vectors $X = ct j + x kI + y kJ + z kK$ and dual vectors $X' = ct' k + x' jI + y' jJ + z' jK$, related by $X' = Xi$. The complete rotor is
\[
R_\text{total} = \exp\left( \sum_{a=1}^3 \frac{\omega_a}{2} i\Gamma_a + \frac{\phi_a}{2} \Gamma_a \right), \quad \Gamma_1 = I,\ \Gamma_2 = J,\ \Gamma_3 = K.
\]
The torsional mismatch rotor $\Omega = R_\text{usual}^\dagger R_\text{dual}$ yields the bivector $\Omega_\text{biv} = \log \Omega$, and the scalar
\[
J = \frac{1}{16} B(\Omega_\text{biv}, \Omega_\text{biv}),
\]
where $B$ is the Killing form with $B(i\Gamma_a, i\Gamma_a) = +8$ (boost-like) and $B(\Gamma_a, \Gamma_a) = -8$ (rotation-like).

\section{Discretization of Rapidity Parameters}
\label{sec:discretization}

The dual map $X \mapsto Xi$ implies an intrinsic identification of the time direction with a circle: successive applications of the dual map twice return a sign flip ($Xi i = -X$), suggesting compactification. The exponential map for rotors is inherently periodic:
\[
\exp(\theta \Gamma_a) = \exp((\theta + 2\pi k) \Gamma_a), \quad k \in \mathbb{Z}.
\]
We therefore discretize
\[
\omega_a = \frac{2\pi n_a}{N}, \quad \phi_a = \frac{2\pi m_a}{N}, \quad n_a, m_a \in \mathbb{Z},
\]
where $N \gg 1$ is the effective number of states in the compact direction. Physically, $N \propto m c^2 / E_0$ with $E_0$ a fundamental energy (Planckian or particle-specific).

The mismatch rotor $\Omega$ now takes values in a finite set, as the phase space of relative angles is the torus $(\mathbb{Z}/N\mathbb{Z})^6$.

\section{Bounded Torsion Scalar $J \in [-1, 1]$}
\label{sec:bounded}

The Killing form asymmetry implies that maximal attraction occurs when usual boosts dominate ($\omega_a \gg \phi_a$), and maximal repulsion when dual rotations dominate. In the discrete case, the extremal configurations are perfect synchronization ($\Omega = 1$, $J = 0$) or maximal antisynchronization ($\Omega = -1$ or phase shifts yielding pure boost/rotation dominance).

Explicit computation for small $N$ (e.g., $N=4,8$) shows $J$ confined to a lattice within $[-1,1]$. The bounds follow from the compactness of the Spin group and the quadratic nature of the Killing form. Formally,
\[
|J| \leq 1,
\]
with equality at pure boost or rotation dominance over all three axes.

In the continuum limit $N \to \infty$, unbounded $J$ is recovered locally, preserving TEGR equivalence.

\section{Integer Biquaternion Ring and Finite Unit Group}
\label{sec:integer}

Define the integer biquaternion ring as the $\mathbb{Z}$-span of the basis with Gaussian/Hurwitz integer coefficients in each quaternion sector (precise order to be classified). Discrete rotors correspond to units in this ring.

Compactification renders the rapidity torus finite, implying the unit group is finite. For finite $N$, the exponential map surjects onto a finite subgroup of Spin$^+(3,1) \oplus$ Spin$^+(3,1)$. Explicit enumeration for small $N$ confirms finiteness, yielding a discrete spectrum of allowable torsional states.

\section{Diophantine Constraints on Gravitational Solutions}
\label{sec:diophantine}

The field equations in the discrete theory reduce to finding integer solutions $(n_a, m_a)$ satisfying Diophantine equations derived from the action principle. For example, spherically symmetric solutions correspond to integer points on elliptic curves parameterizing radial torsional profiles.

The layered torsional structure (attractive/repulsive alternations) manifests as finite resonance conditions, naturally explaining the finite number of stable nuclei and the absence of superheavy elements.

\section{Physical and Observational Implications}
\label{sec:implications}

\begin{itemize}
  \item \textbf{Ultraviolet cutoff}: Maximal $|J| = 1$ implies bounded acceleration, eliminating singularities.
  \item \textbf{Discrete spectra}: Gravitational wave echoes and nuclear energy levels acquire discrete torsional contributions.
  \item \textbf{Nuclear chart}: Finite unit group predicts strict upper bound on stable nuclei ($A \lesssim 300$).
  \item \textbf{Experimental tests}: Search for discrete torsional echoes in LIGO/Virgo data; precision clock networks probing planetary cores for exact $J=0$ cancellation.
\end{itemize}

\section{Conclusions and Outlook}
\label{sec:conclusions}

The discretization of dual spacetime completes the rejection of the continuum, yielding bounded torsion, finite unit groups, and Diophantine gravity. Future work includes full classification of the integer ring units and explicit construction of descent arguments connecting macroscopic GR solutions to Planck-scale integer configurations.

This framework elevates dual spacetime theory to a candidate for quantum gravity grounded in classical algebraic geometry and number theory.

\bibliographystyle{unsrt}
\begin{thebibliography}{1}
\bibitem{main-paper}
Hypernumbernet Collaboration, ``Gravity as Torsion between Dual Spacetime: A Biquaternionic Reformulation of General Relativity,'' arXiv:xxxx.xxxxx [gr-qc] (2025).
\end{thebibliography}

\end{document}