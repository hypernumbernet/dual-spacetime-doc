\documentclass[11pt,a4paper]{article}
\usepackage{geometry}
\geometry{margin=1in}
\usepackage{amsmath,amssymb,amsthm}
\usepackage[dvipdfmx]{hyperref}
\usepackage{natbib}
\usepackage{booktabs}
\usepackage{caption}

\title{The Final Parsec Problem Resolved:\\Torsional Layering in Dual-Spacetime Torsion Stars Drives Rapid Supermassive Black Hole Mergers}

\author{https://github.com/hypernumbernet}
\date{\today}

\begin{document}

\maketitle

\begin{abstract}
The ``final parsec problem''—the stalling of supermassive black hole (SMBH) binaries at $\sim$1\,pc separations due to insufficient dynamical friction—has challenged hierarchical galaxy formation models for decades \citep{Begelman1980,Milosavljevic2005}.
In the dual-spacetime theory \citep{dual2025}, which embeds particle-intrinsic dual Minkowski spaces in the biquaternion algebra $\mathbb{H}\otimes\mathbb{H}\cong\mathrm{Cl}(3,1)$, classical black holes are replaced by finite-density \emph{torsion stars} featuring alternating attractive ($J>0$) and repulsive ($J<0$) layers governed by the Killing form on the mismatch rotor $\Omega=R_\text{usual}^\dagger R_\text{dual}$.  
At sub-parsec separations, overlapping repulsive layers trigger non-linear dual-rotor resonance, generating a transient antigravitational impulse followed by immediate capture into the next ultra-attractive layer. This ``torsional bounce-and-capture'' mechanism removes residual angular momentum on timescales of months to years, forcing complete coalescence.  
The final-parsec stage thus becomes the efficient trigger for merger, naturally explaining the rapid growth of $10^{9}$--$10^{10}\,M_\odot$ SMBHs by $z\gtrsim 8$ \citep{Tripodi2025} and massive events like GW190521 \citep{LVC2020a,LVC2020b}.  
Predicted signatures include pre-merger electromagnetic flares, gravitational-wave echoes from layer reflections, and the scarcity of long-lived sub-parsec binaries.
\end{abstract}

\section{Introduction}

The final-parsec problem originates from the observation that, after galaxy mergers, dynamical friction efficiently shrinks SMBH pairs to $\sim$1\,pc, but further hardening via stellar scattering or gas torques often becomes inefficient, potentially stalling binaries for longer than the Hubble time \citep{Begelman1980,Milosavljevic2005,Khan2016}.  
Despite this, the Universe hosts numerous $\gtrsim 10^{9}\,M_\odot$ SMBHs at high redshift \citep[e.g.,][]{Fan2023,Tripodi2025} and frequent massive mergers inferred from pulsar timing array signals \citep{NANOGrav2023}, implying efficient coalescence.

The dual-spacetime theory \citep{dual2025} resolves this tension fundamentally by eliminating event horizons and singularities. Collapse endpoints are \emph{torsion stars} with layered torsional structure (§\ref{sec:torsionstars}). We show that sub-parsec binaries inevitably merge via torsional dynamics.

\section{Torsion Stars and Layered Structure}
\label{sec:torsionstars}

In dual-spacetime theory, each particle carries paired Minkowski spaces encoded in biquaternions. The complete rotor is
\[
R_\text{total}=R_\text{usual}R_\text{dual}=\exp\!\left[\sum_{a=1}^3 \left(\frac{\omega_a}{2}i\Gamma_a+\frac{\phi_a}{2}\Gamma_a\right)\right],
\]
with torsion scalar
\[
J=\frac{1}{16}B(\Omega_\text{biv},\Omega_\text{biv}),\quad \Omega=R_\text{usual}^\dagger R_\text{dual},\quad \Omega_\text{biv}=\log\Omega.
\]
Boost-like generators $i\Gamma_a$ yield attraction ($J>0$); rotation-like $\Gamma_a$ yield repulsion ($J<0$). Increasing density drives successive sign flips in $J$, producing infinite attractive-repulsive layers \citep{dual2025}.

Observed black holes are torsion stars: outermost ultra-attractive layer forms a quasi-horizon, with immediate inner repulsive layer.

\section{Torsional Bounce-and-Capture in Sub-Parsec Binaries}

For comparable-mass torsion stars at $a\sim 0.01$--1\,pc, repulsive layers overlap. Dual parameters $\phi_a$ resonate, flipping local $J$ to strongly negative, yielding antigravitational impulse $\sim 10^{50}$--$10^{54}$\,N lasting $\sim 10^{3}$--$10^{5}$\,s.

This kick removes angular momentum; subsequent exposure of ultra-attractive layers drives plunge and merger in $\lesssim 1$\,yr—far faster than conventional mechanisms.

Sequence:
\[
\text{sub-parsec inspiral} \to \text{repulsive resonance} \to \text{antigravity kick}
\]
\[
\to \text{ultra-attractive capture} \to \text{merger}.
\]

\section{Observational Implications}

\begin{itemize}
\item \textbf{Efficient high-$z$ growth:} Rapid mergers enable $10^{9}\,M_\odot$ SMBHs by $z\sim 8.6$ \citep{Tripodi2025}.
\item \textbf{Massive mergers:} Events like GW190521 \citep{LVC2020a,LVC2020b} arise naturally via repeated coalescence.
\item \textbf{GW echoes:} Layer transitions produce reflections, consistent with re-analysed LIGO signals \citep{Abedi2021}.
\item \textbf{Pre-merger flares:} Repulsive compression heats gas, predicting transients before GW chirp.
\item \textbf{Scarcity of stalled binaries:} Matches non-detection of numerous sub-parsec pairs \citep{Dorazio2023}.
\end{itemize}

\section{Conclusion}

Dual-spacetime torsional layering transforms the final-parsec problem into a rapid-merger trigger. Future multi-messenger observations—echoes in LISA/PTA data, pre-merger flares, and resolved orbital motion—will test this prediction.

\bibliographystyle{aasjournal}
\begin{thebibliography}{}

\bibitem[Begelman et al.(1980)]{Begelman1980}
Begelman, M.~C., Blandford, R.~D., \& Rees, M.~J.\ 1980, \, 287, 307

\bibitem[Milosavljević \& Merritt(2005)]{Milosavljevic2005}
Milosavljević, M., \& Merritt, D.\ 2005, \, 563, 34

\bibitem[Khan et al.(2016)]{Khan2016}
Khan, F.~M., et al.\ 2016, \, 828, 73

\bibitem[dual-spacetime theory (2025)]{dual2025}
Dual Spacetime Theory Collaboration, ``Gravity as Torsion between Dual Spacetime: A Biquaternionic Reformulation of General Relativity'', 2025, \href{https://github.com/hypernumbernet/dual-spacetime-theory}{github.com/hypernumbernet/dual-spacetime-theory}

\bibitem[Tripodi et al.(2025)]{Tripodi2025}
Tripodi, R., et al.\ 2025, Nature Communications, in press (CANUCS-LRD-z8.6)

\bibitem[LVC(2020a)]{LVC2020a}
Abbott, R., et al.\ (LIGO-Virgo Collaboration)\ 2020, \, 125, 101102

\bibitem[LVC(2020b)]{LVC2020b}
Abbott, R., et al.\ (LIGO-Virgo Collaboration)\ 2020, \, 900, L13

\bibitem[Abedi et al.(2021)]{Abedi2021}
Abedi, J., \& Afshordi, N.\ 2021, \, 03, 038

\bibitem[D'Orazio \& Charisi(2023)]{Dorazio2023}
D'Orazio, D.~J., \& Charisi, M.\ 2023, arXiv:2310.16896

\bibitem[NANOGrav(2023)]{NANOGrav2023}
Agazie, G., et al.\ (NANOGrav Collaboration)\ 2023, \, 951, L8

\bibitem[Fan et al.(2023)]{Fan2023}
Fan, X., et al.\ 2023, \, 61, 373

\end{thebibliography}

\end{document}