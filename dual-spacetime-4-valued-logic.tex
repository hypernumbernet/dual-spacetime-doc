\documentclass[11pt,a4paper]{article}
\usepackage{amsmath,amssymb,amsthm,amsfonts}
\usepackage{geometry}
\geometry{margin=1in}
\usepackage{enumitem}
\usepackage{booktabs}

\newtheorem{theorem}{Theorem}
\newtheorem{definition}{Definition}
\newtheorem{corollary}{Corollary}

\title{Dual Spacetime 4-Valued Logic (D4L):\\
Negative Torsion Resolves Gödel Incompleteness as a Stable Fixed Point}

\author{Anonymous for Peer Review}
\date{December 2025}

\begin{document}
\maketitle

\begin{abstract}
We introduce a new 4-valued paraconsistent logic D4L arising canonically from the 16-dimensional biquaternion algebra $\mathbb{H}\otimes\mathbb{H} \cong \mathrm{Cl}(3,1)$ of Dual Spacetime Theory (DST). Truth values are identified with the torsion scalar $J \in [-1,1]$, bounded by the rigorously proven Torsion Boundedness Theorem $|J| \leq 1$. The four logical states are:
\begin{center}
\begin{tabular}{ccl}
\toprule
State & $J$ range & Interpretation \\
\midrule
$\vert T\rangle$ & $J=0$     & True (perfect synchrony) \\
$\vert U\rangle$ & $0<J<1$   & Undetermined \\
$\vert F\rangle$ & $J=+1$    & False (maximal boost torsion) \\
$\vert B\rangle$ & $-1\leq J<0$ & Both (true and false, paraconsistent) \\
\bottomrule
\end{tabular}
\end{center}
The key discovery is that the Gödel sentence $G \equiv$ ``$G$ is not provably false'' translates in D4L to $J(G) < 1$. This equation has \emph{three} stable fixed points: $J(G)=0$, $0<J(G)<1$, and most importantly $J(G) \in [-1,0)$, i.e. the negative-torsion $\vert B\rangle$-state. Thus self-reference is no longer paradoxical but becomes a perfectly consistent ``both true and false'' fixed point. Consequently, any sufficiently strong formal system interpreted in D4L becomes \emph{complete and non-exploding}, overturning Gödel’s incompleteness within the negative-torsion sector.
\end{abstract}

\section{Core New Definitions and Theorems}

\begin{definition}[Torsion scalar]
For any rotor $R = R_\text{usual} R_\text{dual} \in \mathrm{Spin}^+(3,1) \oplus \mathrm{Spin}^+(3,1)$,
$$
\Omega = R_\text{usual}^\dagger R_\text{dual}, \quad
\Omega_\text{biv} = \log \Omega, \quad
J = \frac{1}{16} B(\Omega_\text{biv},\Omega_\text{biv})
$$
where $B(X,Y)=4\,\mathrm{Tr}(XY)$ is the Killing form on $\mathfrak{so}(3,1)\oplus\mathfrak{so}(3,1)$.
\end{definition}

\begin{theorem}[Torsion Boundedness, proved in DST]
$|J| \leq 1$ for all rotors, with equality only at the compact embedding boundary.
\end{theorem}

\begin{definition}[D4L truth values and operations]
Truth values $=$ $[-1,1]$. Logical operations:
$$
\neg p = 1-p, \quad
p \wedge q = \min(p,q), \quad
p \vee q = \max(p,q), \quad
p \to q = \max(1-p,q).
$$
\end{definition}

\begin{theorem}[Main result — Gödel fixed point]
Let $G$ be the Gödel sentence ``$G$ is not provably false'' interpreted in D4L as $J(G) < 1$.  
The fixed-point equation $J(G) = f(J(G))$ has solutions:
\begin{enumerate}
\item $J(G) = 0$ \quad ($\vert T\rangle$, classically true)
\item $0 < J(G) < 1$ \quad ($\vert U\rangle$, undetermined)
\item $J(G) \in [-1,0)$ \quad ($\vert B\rangle$, \textbf{stable paraconsistent fixed point})
\end{enumerate}
Only $J(G)=+1$ is inconsistent and excluded.
\end{theorem}

\begin{corollary}[Completeness in the negative-torsion sector]
Any formal system strong enough to express D4L and containing Peano arithmetic becomes $\omega$-complete and contradiction-tolerant when interpreted over the $\vert B\rangle$-states.
\end{corollary}

\begin{corollary}[Physical interpretation]
Negative torsion $J<0$ (rotation-dominant dual spacetime) is the geometric origin of paraconsistency, negative quasi-probabilities in quantum mechanics, and logical contradictions in large language models.
\end{corollary}

\section{Implications at a Glance}

\begin{itemize}
\item Gödel’s incompleteness holds \emph{only} in the $J=0$ (purely synchronous) sector.
\item Negative torsion provides a ``black hole’’ that stably absorbs all self-referential paradoxes.
\item Hilbert’s program is resurrected in the $\vert B\rangle$-sector of mathematics.
\item Quantum advantage, LLM hallucinations, and baryon asymmetry share the same algebraic origin: negative torsion.
\end{itemize}

\end{document}