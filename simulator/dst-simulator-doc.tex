\documentclass[a4paper,12pt,notitlepage]{jsreport}
\usepackage[left=10truemm,right=10truemm,top=25truemm,bottom=20truemm]{geometry}
\usepackage{mathtools}
\usepackage{amsmath}
\usepackage{amsfonts}
\usepackage{bm}
\usepackage{setspace}
\usepackage{wrapfig}
\usepackage[dvipdfmx]{hyperref}
\usepackage{pxjahyper}
\usepackage{docmute}
\DeclareMathOperator\arctanh{arctanh}
\DeclareMathOperator\arccosh{arccosh}

\title{dual-spacetime-simulator documents}

\author{https://github.com/hypernumbernet}
\date{\today}

\begin{document}

\maketitle

\begin{abstract}
  重力多体シミュレーションを行うに当たって数学的基盤を整えておく目的でまとめを書いています。
  プログラミングですぐに活用できるように計算手順を具体化し実係数で式を展開しています。
\end{abstract}

\clearpage
\tableofcontents
\clearpage

\documentclass[a4paper,12pt,notitlepage]{jsreport}
\usepackage[left=10truemm,right=10truemm,top=25truemm,bottom=20truemm]{geometry}
\usepackage{mathtools}
\usepackage{amsmath}
\usepackage{amsfonts}
\usepackage{bm}
\usepackage{setspace}
\usepackage{wrapfig}
\usepackage[dvipdfmx]{hyperref}
\usepackage{pxjahyper}
\usepackage{docmute}
\DeclareMathOperator\arctanh{arctanh}
\DeclareMathOperator\arccosh{arccosh}

\begin{document}

\chapter{複四元数による特殊相対性理論}

\section{複四元数とは}

まず、複素数は以下のように定義されることを復習しましょう。
\begin{equation}
  \texttt{Complex number}:\mathbb{C}\coloneq ~r_0+r_1i,\quad r_0,r_1\in\mathbb{R},\quad i^2=-1
\end{equation}

四元数は以下のようになります。クォータニオンとも呼びます。
\begin{equation}
  \begin{split}
    \texttt{Quaternion}:\mathbb{H}\coloneq ~&r_0+r_1i+r_2j+r_3k,\quad r_0,r_1,r_2,r_3\in\mathbb{R}\\
    &i^2=j^2=k^2=-1
  \end{split}
\end{equation}

四元数の演算規則です。
\begin{gather}
  ij=k,~jk=i,~ki=j,~ji=-k,~kj=-i,~ik=-j
\end{gather}

複四元数とは四元数の4つの係数を複素数にしたもので、8つの実数を内包します。
\begin{equation}
  \begin{split}
    \texttt{Complex Quaternion}: \mathbb{H} \otimes \mathbb{C} \coloneq ~&w_0+w_1i+w_2j+w_3k,\quad w_0,w_1,w_2,w_3\in\mathbb{C}\\
    =~&r_0+r_1I+(r_2+r_3I)i+(r_4+r_5I)j+(r_6+r_7I)k\\
    &r_0,...,r_7\in\mathbb{R},\quad I^2=-1
  \end{split}
\end{equation}

新たに導入した虚数$I$と四元数部分の虚数$i, j, k$は互いに干渉しません。
\begin{equation}
  iI=Ii,~jI=Ij,~kI=Ik
\end{equation}

\section{複四元数の共役は2種類}

複素数の共役と絶対値は以下のようでした。
\begin{equation}
  \begin{split}
    w=~&r_0+r_1i,\quad w\in\mathbb{C}\\
    \overline{w}=~&r_0-r_1i\\
    |w|=~&\sqrt{w\overline{w}}=\sqrt{r_0^2+r_1^2}
  \end{split}
\end{equation}

四元数での共役とノルムです。ここでは*を使って表現します。ノルムは絶対値の二乗とします。
\begin{equation}
  \begin{split}
    h=~&r_0+r_1i+r_2j+r_3k,\quad h\in\mathbb{H}\\
    h^*=~&r_0-r_1i-r_2j-r_3k\\
    N(h)=~&hh^*=r_0^2+r_1^2+r_2^2+r_3^2\\
    |h|=~&\sqrt{N(h)}=\sqrt{r_0^2+r_1^2+r_2^2+r_3^2}
  \end{split}
\end{equation}

複四元数では共役は以下のように2種類が定義されます。
\begin{equation}
  \begin{split}
    b=~&w_0+w_1i+w_2j+w_3k,\quad w_0,w_1,w_2,w_3\in\mathbb{C},\quad b\in\mathbb{B}\\
    =~&r_0+r_1I+(r_2+r_3I)i+(r_4+r_5I)j+(r_6+r_7I)k,\quad r_0,...,r_7\in\mathbb{R}\\
    b^*=~&w_0-w_1i-w_2j-w_3k\\
    =~&r_0+r_1I-(r_2+r_3I)i-(r_4+r_5I)j-(r_6+r_7I)k\\
    \overline{b}=~&\overline{w_0}+\overline{w_1}i+\overline{w_2}j+\overline{w_3}k\\
    =~&r_0-r_1I+(r_2-r_3I)i+(r_4-r_5I)j+(r_6-r_7I)k
  \end{split}
\end{equation}

複四元数のノルムの定義は四元数としての共役の定義を使用して定義します。
\begin{equation}
  N(b)=bb^*=w_0^2+w_1^2+w_2^2+w_3^2
\end{equation}

\section{複四元数で表現する特殊相対論的時空}

複四元数の一部を使って相対論の4次元時空を表現できます。
\begin{equation}
  \begin{split}
    \texttt{Spacetime}:\mathbb{M}\coloneq &\{m:m^*=\overline{m}\}\\
    =&\{t+xIi+yIj+zIk,\quad t,x,y,z\in\mathbb{R}\}
  \end{split}
\end{equation}

ノルムが丁度不変量になります。
\begin{equation}
  \texttt{Invariant}:mm^*=t^2-x^2-y^2-z^2,\quad m\in\mathbb{M}
\end{equation}
このような集合をミンコフスキー空間とも呼びます。

以下のような複四元数の集合を定義します。ノルムが1になるという意味で単位複四元数と呼びます。
\begin{equation}
  \mathbb{G}\coloneq\{g:gg^*=1,\quad g\in\mathbb{B}\}
\end{equation}

するとローレンツ変換は以下の計算で成立します。
\begin{equation}
  \texttt{Lorentz transformation}:T(m)=g^*m\overline{g},\quad m\in\mathbb{M}
\end{equation}

変換後の値も時空を表しています。
\begin{equation}
  T(m)\in\mathbb{M}
\end{equation}

変換後も不変量は変化しません。
\footnote{簡潔な証明が可能です。$N(g)=1$より、$N(g^*)=1$も成り立ちます。従って、$N(g^*m\overline{g})=N(g^*)N(m)N(\overline{g})=N(m)$となります。}
\begin{equation}
  T(m)(T(m))^*=mm^*
\end{equation}

以上のように、時空の不変量を変化させないように時空を変換する道具として複四元数が使えます。
光速度不変の原理を壊すことなくローレンツ変換を行う複四元数の条件を得ています。
任意の単位複四元数です。

\section{特殊相対論的速度との関係}

単位複四元数の集合全体について一般的に位相群を解明する事は困難です。
しかし、その一部である$\mathbb{G}\cap\mathbb{M}$については以下のように計算できることが分かっています。

三次元空間の方向を表現する単位四元数の集合を定義します。ノルムが1の純虚四元数となっており、ベルソルとも呼びます。
\begin{equation}
  \begin{split}
    \texttt{Versor}:\mathbb{V}\coloneq&\{d:d=\sqrt{-1},\quad d\in\mathbb{H}\}\\
    =&\{d_1i+d_2j+d_3k,\quad d_1,d_2,d_3\in\mathbb{R},\quad d_1^2+d_2^2+d_3^2=1\}
  \end{split}
\end{equation}

以下のように計算すると、$Id$が分解型複素数(Split-complex number)を表現している事が分かります。
\begin{equation}
  (Id)^2=I^2d^2=(-1)(-1)=+1
\end{equation}

分解型複素数のオイラーの公式を調べてみます。
\begin{equation}
  \begin{split}
    \exp(aId)=&\cosh a+Id\sinh a,\quad a\in\mathbb{R}\\
    &\cosh a+(d_1Ii+d_2Ij+d_3Ik)\sinh a
  \end{split}
\end{equation}

この至宝の公式は四次元時空の集合の範囲内にあることがわかります。
\begin{equation}
  \exp(aId)\in\mathbb{M}
\end{equation}

四次元時空の双曲幾何学的な性質からこの公式内での速さは丁度以下のようになります。
\begin{equation}
  \texttt{Speed}:v=c\tanh a,\quad c:\texttt{Speed of light},\quad v\in\mathbb{R}
\end{equation}

速さから逆に角度を求めると以下のようになります。この角度はラピディティという名前があります。
\begin{equation}
  \texttt{Rapidity}:a=\arctanh\frac{|\bm{v}|}{c}
\end{equation}

ラピディティには単純な足し算が出来るというメリットがあります。

\section{複四元数によるローレンツ変換の計算方法}

四元数による回転計算からの類推で複四元数によるローレンツ変換の角度も半角化すれば丁度良いことがわかります。
\begin{equation}
  g=\exp(0.5aId) \to g^*=\overline{g}=\exp(-0.5aId)
\end{equation}

と置けば、
\begin{equation}
  \begin{split}
    T(\exp(aId))=&g^*\exp(aId)\overline{g}\\
    =&\exp(-0.5aId)\exp(aId)\exp(-0.5aId)\\
    =&\exp(-0.5aId+aId-0.5aId)\\
    =&1
  \end{split}
\end{equation}

この変換を複四元数の実数成分で表現してみると、
\begin{equation}
  g=\exp(0.5aId)=r_0+r_1Ii+r_2Ij+r_3Ik
\end{equation}

と置いて、
\begin{equation}
  r_0=\cosh 0.5a,\quad (r_1,r_2,r_3)=(d_1,d_2,d_3)\sinh 0.5a,\quad d_1^2+d_2^2+d_3^2=1
\end{equation}

となります。変換に使用する単位複四元数を得ました。変換元の四次元時空を
\begin{equation}
  m=t+xIi+yIj+zIk
\end{equation}

と置くと、
\begin{equation}
  \begin{split}
    T(m)=~&g^*m\overline{g}\\
    =~&(r_0-r_1Ii-r_2Ij-r_3Ik)(t+xIi+yIj+zIk)(r_0-r_1Ii-r_2Ij-r_3Ik)\\
    =~&(r_0t-r_1tIi-r_2tIj-r_3tIk+r_0xIi-r_1xIiIi-r_2xIjIi-r_3xIkIi+r_0yIj-r_1yIiIj-r_2yIjIj-r_3yIkIj\\
    &+r_0zIk-r_1zIiIk-r_2zIjIk-r_3zIkIk)(r_0-r_1Ii-r_2Ij-r_3Ik)\\
    =~&(r_0t-r_1tIi-r_2tIj-r_3tIk+r_0xIi-r_1x-r_2xk+r_3xj+r_0yIj+r_1yk-r_2y-r_3yi\\
    &+r_0zIk+r_1zj+r_2zi-r_3z)(r_0-r_1Ii-r_2Ij-r_3Ik)\\
    =~&(r_0t-r_1x-r_2y-r_3z+(r_2z-r_3y)i+(r_1z+r_3x)j+(r_1y-r_2x)k\\
    &+(r_0x-r_1t)Ii+(r_0y-r_2t)Ij+(r_0z-r_3t)Ik)(r_0-r_1Ii-r_2Ij-r_3Ik)\\
  \end{split}
\end{equation}

\begin{equation}
  \begin{split}
    =~&r_0^2t-r_0r_1x-r_0r_2y-r_0r_3z+r_0(r_2z-r_3y)i+r_0(r_1z+r_3x)j+r_0(r_1y-r_2x)k\\
    &+r_0(r_0x-r_1t)Ii+r_0(r_0y-r_2t)Ij+r_0(r_0z-r_3t)Ik\\
    &+(-r_0r_1t+r_1^2x+r_1r_2y+r_1r_3z)Ii+r_1(r_2z-r_3y)I+r_1(r_1z+r_3x)Ik-r_1(r_1y-r_2x)Ij\\
    &-r_1(r_0x-r_1t)-r_1(r_0y-r_2t)k+r_1(r_0z-r_3t)j\\
    &+(-r_0r_2t+r_1r_2x+r_2^2y+r_2r_3z)Ij-r_2(r_2z-r_3y)Ik+r_2(r_1z+r_3x)I+r_2(r_1y-r_2x)Ii\\
    &+r_2(r_0x-r_1t)k-r_2(r_0y-r_2t)-r_2(r_0z-r_3t)i\\
    &+(-r_0r_3t+r_1r_3x+r_2r_3y+r_3^2z)Ik+r_3(r_2z-r_3y)Ij-r_3(r_1z+r_3x)Ii+r_3(r_1y-r_2x)I\\
    &-r_3(r_0x-r_1t)j+r_3(r_0y-r_2t)i-r_3(r_0z-r_3t)\\
    =~&r_0^2t-r_0r_1x-r_0r_2y-r_0r_3z-r_1(r_0x-r_1t)-r_2(r_0y-r_2t)-r_3(r_0z-r_3t)\\
    &+r_0(r_2z-r_3y)i-r_2(r_0z-r_3t)i+r_3(r_0y-r_2t)i\\
    &+r_0(r_1z+r_3x)j+r_1(r_0z-r_3t)j-r_3(r_0x-r_1t)j\\
    &+r_0(r_1y-r_2x)k-r_1(r_0y-r_2t)k+r_2(r_0x-r_1t)k\\
    &+r_1(r_2z-r_3y)I+r_2(r_1z+r_3x)I+r_3(r_1y-r_2x)I\\
    &+r_0(r_0x-r_1t)Ii+(-r_0r_1t+r_1^2x+r_1r_2y+r_1r_3z)Ii+r_2(r_1y-r_2x)Ii-r_3(r_1z+r_3x)Ii\\
    &+r_0(r_0y-r_2t)Ij-r_1(r_1y-r_2x)Ij+(-r_0r_2t+r_1r_2x+r_2^2y+r_2r_3z)Ij+r_3(r_2z-r_3y)Ij\\
    &+r_0(r_0z-r_3t)Ik+r_1(r_1z+r_3x)Ik-r_2(r_2z-r_3y)Ik+(-r_0r_3t+r_1r_3x+r_2r_3y+r_3^2z)Ik\\
  \end{split}
\end{equation}

$i,j,k,I$の項はきれいに消えます。
\begin{equation}
  \begin{split}
    T(m)=~&(r_0^2+r_1^2+r_2^2+r_3^2)t-2r_0(r_1x+r_2y+r_3z)\\
    &+((r_0^2+r_1^2-r_2^2-r_3^2)x-2r_1(r_0t-r_2y-r_3z))Ii\\
    &+((r_0^2-r_1^2+r_2^2-r_3^2)y-2r_2(r_0t-r_1x-r_3z))Ij\\
    &+((r_0^2-r_1^2-r_2^2+r_3^2)z-2r_3(r_0t-r_1x-r_2y))Ik
  \end{split}
\end{equation}

従って、四次元時空各成分に注目したローレンツ変換は以下となります。
\begin{equation}
  \begin{split}
    t'=~&(r_0^2+r_1^2+r_2^2+r_3^2)t-2r_0(r_1x+r_2y+r_3z)\\
    x'=~&(r_0^2+r_1^2-r_2^2-r_3^2)x-2r_1(r_0t-r_2y-r_3z)\\
    y'=~&(r_0^2-r_1^2+r_2^2-r_3^2)y-2r_2(r_0t-r_1x-r_3z)\\
    z'=~&(r_0^2-r_1^2-r_2^2+r_3^2)z-2r_3(r_0t-r_1x-r_2y)
  \end{split}
\end{equation}
コンピューターによる計算にとってはこの最適化が有用になるでしょう。

\section{ラピディティによるもう一つの計算方法}

ラピディティの足し算による計算方法もあります。以下のように変換対象の時空を定義します。
\begin{equation}
  \begin{split}
    mv=~&\exp(qIp),\quad p=p_1i+p_2j+p_3k\\
    t=~&\cosh q,\quad (x,y,z)=(p_1,p_2,p_3)\sinh q\\
    &m\in\mathbb{M},\quad q,p_1,p_2,p_3\in\mathbb{R},\quad p\in\mathbb{D}
  \end{split}
\end{equation}

$t,x,y,z$から$q,p_1,p_2,p_3$を算出する必要があり、方法はいくつか考えられますが、一つとしては以下のようになります。
\begin{equation}
  q=\arccosh t,\quad (p_1,p_2,p_3)=\frac{(x,y,z)}{\sqrt{x^2+y^2+z^2}}
\end{equation}

この変換では原点座標ではゼロの除算になるので特別な配慮が必要です。

ここまで準備できればローレンツ変換は単純に加法減法になります。

変換した分の速度をラピディティ$a\in\mathbb{R}$方向$d\in\mathbb{D}$として、
\begin{equation}
  \begin{split}
    T(m)=~&g^*m\overline{g}\\
    =~&\exp(-0.5aId)\exp(qIp)\exp(-0.5aId)\\
    =~&\exp((qp-ad)I)
  \end{split}
\end{equation}

全ての計算をラピディティで完結させることが出来ればこれ程単純な計算方法はないです。
しかし、実際は現実世界の座標に以下のように変換する必要があります。
\begin{equation}
  q'p'=qp-ad,\quad q'\in\mathbb{R},\quad p'\in\mathbb{D},\quad p'=p'_1i+p'_2j+p'_3k
\end{equation}

\begin{equation}
  q'=|qp-ad|,\quad p'=\frac{qp-ad}{|qp-ad|}
\end{equation}

\begin{equation}
  t'=\cosh q',\quad (x',y',z')=(p'_1,p'_2,p'_3)\sinh q'\\
\end{equation}

\section{行列を使ったローレンツ変換}

比較のために行列を使った算出方法を掲載しておきます。
\begin{equation}
  \bm{v}=(v_1,v_2,v_3),\quad v_1,v_2,v_3\in\mathbb{R}
\end{equation}

\begin{equation}
  \beta=\frac{|\bm{v}|}{c},\quad \gamma=\frac{1}{\sqrt{1-\beta^2}}
\end{equation}

\begin{equation}
  \setstretch{2.7}
  \begin{bmatrix}
    ct'\\x'\\y'\\z'
  \end{bmatrix}
  =
  \setstretch{1.6}
  \begin{bmatrix}
    \gamma & -\cfrac{v_1}{c}\gamma & -\cfrac{v_2}{c}\gamma & -\cfrac{v_3}{c}\gamma \\[3pt]
    -\cfrac{v_1}{c}\gamma & 1+\cfrac{v_1^2}{|\bm{v}|}(\gamma-1) & \cfrac{v_1v_2}{|\bm{v}|}(\gamma-1) & \cfrac{v_1v_3}{|\bm{v}|}(\gamma-1)\\
    -\cfrac{v_2}{c}\gamma & \cfrac{v_1v_2}{|\bm{v}|}(\gamma-1) & 1+\cfrac{v_2^2}{|\bm{v}|}(\gamma-1) & \cfrac{v_2v_3}{|\bm{v}|}(\gamma-1)\\
    -\cfrac{v_3}{c}\gamma & \cfrac{v_1v_3}{|\bm{v}|}(\gamma-1) & \cfrac{v_2v_3}{|\bm{v}|}(\gamma-1) & 1+\cfrac{v_3^2}{|\bm{v}|}(\gamma-1)
  \end{bmatrix}
  \setstretch{2.7}
  \begin{bmatrix}
    ct\\x\\y\\z
  \end{bmatrix}
\end{equation}

\section{単位複四元数は高次元の回転}

任意の複四元数$b_0,b_1$の積をその実数成分で表現しておきます。
\begin{equation}
  \begin{split}
    b_0=~&p_0+p_1I+(p_2+p_3I)i+(p_4+p_5I)j+(p_6+p_7I)k,\quad p_0,...,p_7\in\mathbb{R}\\
    =~&w_0+w_1i+w_2j+w_3k,\quad w_0,w_1,w_2,w_3\in\mathbb{H}\\
    b_1=~&q_0+q_1I+(q_2+q_3I)i+(q_4+q_5I)j+(q_6+q_7I)k,\quad p_0,...,p_7\in\mathbb{R}\\
    =~&u_0+u_1i+u_2j+u_3k,\quad u_0,u_1,u_2,u_3\in\mathbb{H}
  \end{split}
\end{equation}

まずは、通常の四元数の乗算を行います。
\begin{equation}
  \begin{split}
    b_0b_1=~&(w_0+w_1i+w_2j+w_3k)(u_0+u_1i+u_2j+u_3k)\\
    =~&w_0u_0+w_0u_1i+w_0u_2j+w_0u_3k\\
    &+w_1iu_0+w_1iu_1i+w_1iu_2j+w_1iu_3k\\
    &+w_2ju_0+w_2ju_1i+w_2ju_2j+w_2ju_3k\\
    &+w_3ku_0+w_3ku_1i+w_3ku_2j+w_3ku_3k\\
    =~&w_0u_0+w_0u_1i+w_0u_2j+w_0u_3k\\
    &+w_1u_0i-w_1u_1+w_1u_2k-w_1u_3j\\
    &+w_2u_0j-w_2u_1k-w_2u_2+w_2u_3i\\
    &+w_3u_0k+w_3u_1j-w_3u_2i-w_3u_3\\
    =~&w_0u_0-w_1u_1-w_2u_2-w_3u_3\\
    &+(w_0u_1+w_1u_0+w_2u_3-w_3u_2)i\\
    &+(w_0u_2-w_1u_3+w_2u_0+w_3u_1)j\\
    &+(w_0u_3+w_1u_2-w_2u_1+w_3u_0)k
  \end{split}
\end{equation}

次に実数に展開します。
\begin{equation}
  \begin{split}
    b_0b_1=~&(p_0+p_1I)(q_0+q_1I)-(p_2+p_3I)(q_2+q_3I)-(p_4+p_5I)(q_4+q_5I)-(p_6+p_7I)(q_6+q_7I)\\
    &+((p_0+p_1I)(q_2+q_3I)+(p_2+p_3I)(q_0+q_1I)+(p_4+p_5I)(q_6+q_7I)-(p_6+p_7I)(q_4+q_5I))i\\
    &+((p_0+p_1I)(q_4+q_5I)-(p_2+p_3I)(q_6+q_7I)+(p_4+p_5I)(q_0+q_1I)+(p_6+p_7I)(q_2+q_3I))j\\
    &+((p_0+p_1I)(q_6+q_7I)+(p_2+p_3I)(q_4+q_5I)-(p_4+p_5I)(q_2+q_3I)+(p_6+p_7I)(q_0+q_1I))k
  \end{split}
\end{equation}
\begin{equation}
  \begin{split}
    =~&(p_0q_0-p_1q_1+(p_0q_1+p_1q_0)I)-(p_2q_2-p_3q_3+(p_2q_3+p_3q_2)I)\\
    &-(p_4q_4-p_5q_5+(p_4q_5+p_5q_4)I)-(p_6q_6-p_7q_7+(p_6q_7+p_7q_6)I)\\
    &+((p_0q_2-p_1q_3+(p_0q_3+p_1q_2)I)+(p_2q_0-p_3q_1+(p_2q_1+p_3q_0)I)\\
    &+(p_4q_6-p_5q_7+(p_4q_7+p_5q_6)I)-(p_6q_4-p_7q_5+(p_6q_5+p_7q_4)I))i\\
    &+((p_0q_4-p_1q_5+(p_0q_5+p_1q_4)I)-(p_2q_6-p_3q_7+(p_2q_7+p_3q_6)I)\\
    &+(p_4q_0-p_5q_1+(p_4q_1+p_5q_0)I)+(p_6q_2-p_7q_3+(p_6q_3+p_7q_2)I))j\\
    &+((p_0q_6-p_1q_7+(p_0q_7+p_1q_6)I)+(p_2q_4-p_3q_5+(p_2q_5+p_3q_4)I)\\
    &-(p_4q_2-p_5q_3+(p_4q_3+p_5q_2)I)+(p_6q_0-p_7q_1+(p_6q_1+p_7q_0)I))k
  \end{split}
\end{equation}
\begin{equation}
  \begin{split}
    =~&p_0q_0-p_1q_1-p_2q_2+p_3q_3-p_4q_4+p_5q_5-p_6q_6+p_7q_7\\
    &+(p_0q_1+p_1q_0-p_2q_3-p_3q_2-p_4q_5-p_5q_4-p_6q_7-p_7q_6)I\\
    &+(p_0q_2-p_1q_3+p_2q_0-p_3q_1+p_4q_6-p_5q_7-p_6q_4+p_7q_5)i\\
    &+(p_0q_3+p_1q_2+p_2q_1+p_3q_0+p_4q_7+p_5q_6-p_6q_5-p_7q_4)Ii\\
    &+(p_0q_4-p_1q_5-p_2q_6+p_3q_7+p_4q_0-p_5q_1+p_6q_2-p_7q_3)j\\
    &+(p_0q_5+p_1q_4-p_2q_7-p_3q_6+p_4q_1+p_5q_0+p_6q_3+p_7q_2)Ij\\
    &+(p_0q_6-p_1q_7+p_2q_4-p_3q_5-p_4q_2+p_5q_3+p_6q_0-p_7q_1)k\\
    &+(p_0q_7+p_1q_6+p_2q_5+p_3q_4-p_4q_3-p_5q_2+p_6q_1+p_7q_0)Ik
  \end{split}
\end{equation}

複四元数のノルムを実数成分で表しておくと以下のようになります。
\begin{equation}
  g=a_0+a_1I+(a_2+a_3I)i+(a_4+a_5I)j+(a_6+a_7I)k,\quad a_0,..., a_7\in\mathbb{R}
\end{equation}

\begin{equation}
  \begin{split}
    gg^*=~&(a_0+a_1I)^2+(a_2+a_3I)^2+(a_4+a_5I)^2+(a_6+a_7I)^2\\
    =~&a_0^2-a_1^2+a_2^2-a_3^2+a_4^2-a_5^2+a_6^2-a_7^2+2(a_0a_1+a_2a_3+a_4a_5+a_6a_7)I
  \end{split}
\end{equation}

参考までに
\begin{equation}
  gg^*=1\iff
  \begin{cases}
    a_0^2-a_1^2+a_2^2-a_3^2+a_4^2-a_5^2+a_6^2-a_7^2=1\\
    a_0a_1+a_2a_3+a_4a_5+a_6a_7=0
  \end{cases}
\end{equation}

複四元数のノルムは複四元数の積に対して乗法的である事が示せます。
つまり、任意の複四元数$(b_0,b_1)$に対して次の式が成り立ちます。
\begin{equation}
  N(b_0b_1)=N(b_0)N(b_1)
\end{equation}

幸いな事に実数に展開しなくても証明できます。
\begin{equation}
  N(b_0b_1)=(b_0b_1)(b_0b_1)^*=b_0b_1b_1^*b_0^*=b_0N(b_1)b_0^*=N(b_1)b_0b_0^*=N(b_1)N(b_0)=N(b_0)N(b_1)
\end{equation}

$b_0$と$b_1$のノルムがどちらも$1$の場合、
\begin{equation}
  b_0,b_1\in\mathbb{G}\iff N(b_0)=N(b_1)=1\iff N(b_0b_1)=1
\end{equation}
従って、単位複四元数の乗算は回転を意味しています。双曲的な高次元の角度となっています。

特殊相対性理論のローレンツ変換は光速を1とした場合、三次元の角度と捉える事ができます。
それを回転させる単位複四元数についても角度でありますから、
ここまでの所は物理的有意性のある計量として角度のみでの記述が可能という事になります。

\end{document}

\documentclass[a4paper,12pt,notitlepage]{jsreport}
\usepackage[left=10truemm,right=10truemm,top=25truemm,bottom=20truemm]{geometry}
\usepackage{mathtools}
\usepackage{amsmath}
\usepackage{amsfonts}
\usepackage{bm}
\usepackage{setspace}
\usepackage{wrapfig}
\usepackage[dvipdfmx]{hyperref}
\usepackage{pxjahyper}
\usepackage{docmute}
\DeclareMathOperator\arctanh{arctanh}
\DeclareMathOperator\arccosh{arccosh}

\begin{document}

\chapter{多元数による回転計算}

\section{複素数による回転計算}

オイラーの公式により角度と絶対値が1の複素数の関係が明らかになっています。
\begin{equation}
  \exp(aI)=\cos a+I\sin a,\quad a\in\mathbb{R},\quad I^2=-1
\end{equation}
\begin{equation}
  |\exp(aI)|=\sqrt{\cos^2 a+\sin^2 a}=1
\end{equation}
ここでは複素数の虚数単位をあえて$I$で表現しています。

任意の二次元座標$(x,y)$を原点回りに角度$a$回転させる計算は、$x+yI,\quad x,y\in\mathbb{R}$を構築して、
\begin{equation}
  x'+y'I=(x+yI)\exp(aI)
\end{equation}
で算出できます。

\section{四元数による三次元の回転計算}

任意の三次元座標を原点回りに回転させる計算は二次元ほど単純ではありません。
二次元での回転と違いどの方向に回転させるかのパラメーターも必要になります。
行列を使って三次元の回転の計算をすることは一応できますが、
ジンバルロックの問題を回避して完全な回転計算を行うには四元数を使う計算が必須になります。

身近な例で言うと、天と地の軸が定まっている3Dゲームでは四元数は必須ではないですが、
宙返りを自由に行うフライトシミュレーターでは四元数は必須になってきます。
ゲームエンジンのライブラリに含まれるQuaternionクラスには四元数による回転計算が完全に実装されていない例があるようです。
その場合、以下の計算方法を独自に実装する必要があります。

方向を表す三次元単位ベクトル$\bm{a},\bm{b}$があり、$\bm{a}$から$\bm{b}$の方向に$t\in\mathbb{R}$の割合だけ回転させる計算は以下のようになります。
この計算方法は球面線形補間(Slerp)と呼ばれます。

$\bm{a}$と$\bm{b}$の成す角度$\theta$を内積から求めます。
\begin{equation}
  \begin{split}
    \bm{a}=~&(a_x,a_y,a_z),\quad \bm{b}=~(b_x,b_y,b_z),\quad a_x,a_y,a_z,b_x,b_y,b_z\in\mathbb{R}\\
    \theta=~&\arccos(\bm{a}\cdot\bm{b})=\arccos(a_xb_x+a_yb_y+a_zb_z),\quad \theta\in\mathbb{R}
  \end{split}
\end{equation}

回転軸\footnote{一般的に$N$次元の回転軸は$N-1$次元になります。プログラム中の名称はaxis。}を表す三次元単位ベクトルを外積から求めます。
\begin{equation}
  \bm{w}_{cross}=~\bm{a}\times\bm{b}=a_yb_z-a_zb_y,~a_zb_x-a_xb_z,~a_xb_y-a_yb_x
\end{equation}

この外積の結果を正規化し単位ベクトルにしておきます。
\begin{equation}
  \bm{w}_{unit}=\frac{\bm{w}_{cross}}{|\bm{w}_{cross}|}
\end{equation}
ここの計算では$\bm{a}$と$\bm{b}$が一致する場合と正反対を向く場合については特別な配慮が必要です。

この回転軸をベルソル化しておきます。
\begin{equation}
  w=\bm{w}_{unit}\begin{pmatrix}i\\j\\k\end{pmatrix}=w_xi+w_yj+w_zk,\quad w\in\mathbb{V}
\end{equation}

回転子\footnote{プログラム中の名称はrotator。}としての四元数は以下のように構成されます。
\footnote{$h$のノルムは$1$になっていますので、$h$は単位四元数です。}
\begin{equation}
  h=\exp(-0.5t\theta w)=\cos 0.5t\theta-w\sin 0.5t\theta,\quad h\in\mathbb{H}
\end{equation}

回転させたい三次元座標$(x,y,z)$を純虚四元数化させておきます。
\footnote{純虚四元数とは実数部分がゼロになっている四元数の事で四元数のベクトルパートとも呼ばれます。}
\begin{equation}
  f=xi+yj+zk
\end{equation}

回転後の三次元座標$(x',y',z')$は、
\begin{equation}
  x'i+y'j+z'k=h^*fh
\end{equation}
という計算で算出できます。
右側を共役として説明される場合もありますが、どちらでも回転計算はできます。

コンピューターでの計算速度を考えると実数成分に展開して最適化しておいた方が良さそうです。

$h=r_0+r_1i+r_2j+r_3k$と置いて、
\begin{equation}
  \begin{split}
    h^*fh=~&(r_0-r_1i-r_2j-r_3k)(xi+yj+zk)(r_0+r_1i+r_2j+r_3k)\\
    =~&(r_0xi+r_0yj+r_0zk+r_1x-r_1yk+r_1zj+r_2xk+r_2y-r_2zi-r_3xj+r_3yi+r_3z)\\
    &(r_0+r_1i+r_2j+r_3k)\\
    =~&(r_1x+r_2y+r_3z+(r_0x-r_2z+r_3y)i+(r_0y-r_3x+r_1z)j+(r_0z-r_1y+r_2x)k)\\
    &(r_0+r_1i+r_2j+r_3k)\\
    =~&(r_1x+r_2y+r_3z)r_0+(r_1x+r_2y+r_3z)r_1i+(r_1x+r_2y+r_3z)r_2j+(r_1x+r_2y+r_3z)r_3k\\
    &+(r_0x-r_2z+r_3y)r_0i-(r_0x-r_2z+r_3y)r_1+(r_0x-r_2z+r_3y)r_2k-(r_0x-r_2z+r_3y)r_3j\\
    &+(r_0y-r_3x+r_1z)r_0j-(r_0y-r_3x+r_1z)r_1k-(r_0y-r_3x+r_1z)r_2+(r_0y-r_3x+r_1z)r_3i\\
    &+(r_0z-r_1y+r_2x)r_0k+(r_0z-r_1y+r_2x)r_1j-(r_0z-r_1y+r_2x)r_2i-(r_0z-r_1y+r_2x)r_3\\
    =~&r_0r_1x+r_0r_2y+r_0r_3z-r_0r_1x-r_1r_2z+r_1r_3y-r_0r_2y-r_2r_3x+r_1r_2z-r_0r_3z-r_1r_3y+r_2r_3x\\
    &+(r_1^2x+r_1r_2y+r_1r_3z)i+(r_0^2x-r_0r_2z+r_0r_3y)i+(r_0r_3y-r_3^2x+r_1r_3z)i-(r_0r_2z-r_1r_2y+r_2^2x)i\\
    &+(r_1r_2x+r_2^2y+r_2r_3z)j-(r_0r_3x-r_2r_3z+r_3^2y)j+(r_0^2y-r_0r_3x+r_0r_1z)j+(r_0r_1z-r_1^2y+r_1r_2x)j\\
    &+(r_1r_3x+r_2r_3y+r_3^2z)k+(r_0r_2x-r_2^2z+r_2r_3y)k-(r_0r_1y-r_1r_3x+r_1^2z)k+(r_0^2z-r_0r_1y+r_0r_2x)k
  \end{split}
\end{equation}
ここでスカラーパートはきれいに消えます。
\footnote{四元数の実数部分をスカラーパートと呼びます。}
\begin{equation}
  \begin{split}
    h^*fh=~&(r_1^2x+r_1r_2y+r_1r_3z+r_0^2x-r_0r_2z+r_0r_3y+r_0r_3y-r_3^2x+r_1r_3z-r_0r_2z+r_1r_2y-r_2^2x)i\\
    &+(r_1r_2x+r_2^2y+r_2r_3z-r_0r_3x+r_2r_3z-r_3^2y+r_0^2y-r_0r_3x+r_0r_1z+r_0r_1z-r_1^2y+r_1r_2x)j\\
    &+(r_1r_3x+r_2r_3y+r_3^2z+r_0r_2x-r_2^2z+r_2r_3y-r_0r_1y+r_1r_3x-r_1^2z+r_0^2z-r_0r_1y+r_0r_2x)k\\
    =~&(r_1^2x+r_0^2x-r_3^2x-r_2^2x+r_1r_2y+r_0r_3y+r_0r_3y+r_1r_2y+r_1r_3z-r_0r_2z+r_1r_3z-r_0r_2z)i\\
    &+(r_1r_2x-r_0r_3x-r_0r_3x+r_1r_2x+r_2^2y-r_3^2y+r_0^2y-r_1^2y+r_2r_3z+r_2r_3z+r_0r_1z+r_0r_1z)j\\
    &+(r_1r_3x+r_0r_2x+r_1r_3x+r_0r_2x+r_2r_3y+r_2r_3y-r_0r_1y-r_0r_1y+r_3^2z-r_2^2z-r_1^2z+r_0^2z)k\\
    =~&((r_0^2+r_1^2-r_2^2-r_3^2)x+2((r_1r_2+r_0r_3)y+(r_1r_3-r_0r_2)z))i\\
    &+((r_0^2-r_1^2+r_2^2-r_3^2)y+2((r_1r_2-r_0r_3)x+(r_2r_3+r_0r_1)z))j\\
    &+((r_0^2-r_1^2-r_2^2+r_3^2)z+2((r_1r_3+r_0r_2)x+(r_2r_3-r_0r_1)y))k
  \end{split}
\end{equation}

最終的に、三次元座標の単位四元数による回転は以下の式になります。
\begin{equation}
  \begin{split}
    x'=~&(r_0^2+r_1^2-r_2^2-r_3^2)x+2((r_1r_2+r_0r_3)y+(r_1r_3-r_0r_2)z)\\
    y'=~&(r_0^2-r_1^2+r_2^2-r_3^2)y+2((r_1r_2-r_0r_3)x+(r_2r_3+r_0r_1)z)\\
    z'=~&(r_0^2-r_1^2-r_2^2+r_3^2)z+2((r_1r_3+r_0r_2)x+(r_2r_3-r_0r_1)y)
  \end{split}
\end{equation}
CGのコンピューティングではあらかじめこの計算をしておき四次元行列に組み込む事になるでしょう。

\section{超球の形をする宇宙}

三次元での回転については球の表面で描くことができるので想像しやすいです。
例えば地球の表面については経度と緯度の二次元で表現できます。
同様に、四次元単位ベクトルの回転についても三次元で表現できます。
三次元超球面(3-sphere)
\footnote{ここで言う三次元超球面は球の中心が原点にあり半径1の超球の事です。}
という言葉を使います。グロームとも呼ばれています。
グロームは宇宙空間そのものであると考えられており、一般相対性理論をグロームに適用すると
時間と共に宇宙が拡大しているという結論が導き出されるのでよく議論の対象になります。
グロームは四次元単位ベクトルに相当します。
従って、四次元単位ベクトルの回転計算とは宇宙空間における移動そのものと考える事ができます。
さらに、四次元単位ベクトルは単位四元数に相当するので、単位四元数の回転が宇宙空間の移動であるとする事もできます。

整理すると、三次元超球面(3-sphere)=グローム(glome)=四次元単位ベクトル=単位四元数=宇宙空間となります。
\begin{equation}
  \begin{split}
    \texttt{3-sphere}:\mathbb{S}\coloneq&\{s:ss^*=1,\quad s\in\mathbb{H}\}\\
    =~&\{\exp(ae)=\cos a+e\sin a,\quad a\in\mathbb{R},\quad e\in\mathbb{V}\}
  \end{split}
\end{equation}

地球上での移動が実は平面的でなく球の上を曲がって移動しているように、
宇宙空間での移動も四次元内の超球面を曲がって移動していると考えられます。
宇宙が巨大なのでそれを観測することが難しいのです。
このビジョンでは宇宙の果てには壁はなく逆方向から元の位置に戻ってくる事になります。
地上での移動の本質が経度と緯度の角度であると同様に宇宙空間での移動も本質は角度である事に注目してください。

\section{単位四元数の視覚化}

四次元の回転を視覚化して直感的に把握するようにしたい所です。
単位四元数の指数関数によって三次元空間に三次元超球面をプロットする事ができます。

以下の計算で単位四元数を三次元化できます。
\begin{equation}
  \begin{split}
    &s=s_0+s_1i+s_2j+s_3k,\quad s\in\mathbb{S}\mapsto (x,y,z)\\
    &v=s_1i+s_2j+s_3k\\
    &xi+yj+zk=\frac{v}{|v|}\arccos s_0
  \end{split}
\end{equation}

ただし、コンピューターでの計算では誤差や値域の関係から、\texttt{atan2}()を使った方が良いです。
\begin{gather}
  xi+yj+zk=\frac{v}{|v|}\texttt{atan2}(|v|, s_0)
\end{gather}

また、四元数の指数関数について以下のように変形しておくとイメージしやすいです。
普通に四元数$h=a+bi+cj+dk$の指数関数を求めると、
\begin{gather}
  \exp(h)=\exp(a)\left(\cos\sqrt{b^2+c^2+d^2}+\frac{bi+cj+dk}{\sqrt{b^2+c^2+d^2}}\sin\sqrt{b^2+c^2+d^2}\right)
\end{gather}

という形になりますが、予め$\theta=\sqrt{b^2+c^2+d^2}$と置いて整理しておくと、
\begin{gather}
  h=a+bi+cj+dk=a+\theta v,\quad v\in\mathbb{V}\\
  \exp(h)=\exp(a+\theta v)=\exp(a)(\cos \theta+v\sin\theta)
\end{gather}
というように簡潔になります。特に純虚四元数ではさらに簡潔になります。
\begin{gather}
  \exp(\theta v)=\cos \theta+v\sin\theta
\end{gather}
三次元の方向が$v$で移動量が$\theta$というイメージで把握できるようになります。

また、以下のように計算すれば三次元の角度になるのではないかと思われるかもしれませんが、実際に計算をしてみると似て非なる物になります。
\begin{gather}
  \exp(xi)\exp(yj)\exp(zk)
\end{gather}

\section{四次元の回転計算は単純ではない}

四次元を四元数を使って回転させる事はできるのでしょうか?

まず、単純に複素数の回転の計算方法でどうなるのかを見てみましょう。
回転させたい四次元ベクトルを四元数で$h=h_0+h_1i+h_2j+h_3k$、
回転角を$\theta\in\mathbb{R}$、回転方向を$v\in\mathbb{V}$と定義すると、
回転後の単位四元数$h'=h_0'+h_1'i+h_2'j+h_3'k$は、
\begin{equation}
  \begin{split}
    x'=~&x\exp(\theta v)\\
    h_0'+h_1'i+h_2'j+h_3'k=~&(h_0+h_1i+h_2j+h_3k)\exp(\theta v)
  \end{split}
\end{equation}

となる事が期待されます。
この乗算では確かに矛盾なく循環するように見えますが、回転して欲しい方向以外の回転も含まれてしまうので、
捻りが加わってしまう事になります。
単位四元数の視覚化を使うと良く分かります。

より一般的な四元数による回転は等斜線分解(Isoclinic decomposition)により表現できると考えられています。
\begin{equation}
  h_0'+h_1'i+h_2'j+h_3'k=\exp(au)(h_0+h_1i+h_2j+h_3k)\exp(bv),\quad a,b\in\mathbb{R},\quad u,v\in\mathbb{V}
\end{equation}

単位四元数を指数関数形式で表現しています。
2つの直交する回転面とその回転面上の2つの回転角により四次元の回転になるという事なので、
ベルソル$u,v$は直交し内積が$0$という条件が加わる事になると考えられます。
\begin{equation}
  u \perp v \iff u \cdot v=0
\end{equation}

指数関数の式で表すと等斜線分解では回転として5つの角度になっているという事が分かります。
まず$u$を決めるのに2角度です。直交する$v$を決めるのに1角度で、角度$a,b$の合計で5角度です。

では、具体的にどのように回転面と回転角を解釈すれば良いのでしょうか?
三次元超球面を自由に航海したいのです。
この数式だけでは実用的な航海術としては満足できる物ではありません。
より詳しく調査を進めていきましょう。
まず四次元の2つ要素だけを考え、式を展開してみます。
まず以下のように素材を置きます。
\begin{gather}
  h=h_0+h_1i+h_2j+h_3k \mapsto h'=h_0'+h_1'i+h_2'j+h_3'k\\
  a^2+b^2=1,\quad a=\cos\frac{\theta}{2},\quad b=\sin\frac{\theta}{2},
  \quad a,b,\theta\in\mathbb{R}
\end{gather}

倍角の公式から
\begin{equation}
  a^2-b^2=\cos\theta,\quad 2ab=\sin\theta
\end{equation}

色々な組み合わせが考えられますが、以下について計算すると$h_3$から$h_2$方向への回転が得られます。
\begin{equation}
  \begin{split}
    &(a-bi)(h_0+h_1i+h_2j+h_3k)(a+bi)\\
    =~&(ah_0+ah_1i+ah_2j+ah_3k-bh_0i+bh_1-bh_2k+bh_3j)(a+bi)\\
    =~&a^2h_0+a^2h_1i+a^2h_2j+a^2h_3k-abh_0i+abh_1-abh_2k+abh_3j\\
    &+abh_0i-abh_1-abh_2k+abh_3j+b^2h_0+b^2h_1i-b^2h_2j-b^2h_3k\\
    =~&a^2h_0+abh_1-abh_1+b^2h_0+a^2h_1i-abh_0i+abh_0i+b^2h_1i\\
    &+a^2h_2j+abh_3j+abh_3j-b^2h_2j+a^2h_3k-abh_2k-abh_2k-b^2h_3k\\
    =~&a^2h_0+b^2h_0+a^2h_1i+b^2h_1i\\
    &+a^2h_2j+2abh_3j-b^2h_2j+a^2h_3k-2abh_2k-b^2h_3k\\
    =~&h_0+h_1i+((a^2-b^2)h_2+2abh_3)j+((a^2-b^2)h_3-2abh_2)k\\
    =~&h_0+h_1i+(h_2\cos\theta+h_3\sin\theta)j+(h_3\cos\theta-h_2\sin\theta)k
  \end{split}
\end{equation}

\begin{equation}
  \begin{pmatrix}h_2'\\h_3'\end{pmatrix}
  =\begin{pmatrix}\cos\theta&\sin\theta\\-\sin\theta&\cos\theta\end{pmatrix}
  \begin{pmatrix}h_2\\h_3\end{pmatrix}
\end{equation}

同様に
\begin{equation}
  \begin{split}
    &(a-bj)(h_0+h_1i+h_2j+h_3k)(a+bj)\\
    =~&(ah_0+ah_1i+ah_2j+ah_3k-bh_0j+bh_1k+bh_2-bh_3i)(a+bj)\\
    =~&a^2h_0+a^2h_1i+a^2h_2j+a^2h_3k-abh_0j+abh_1k+abh_2-abh_3i\\
    &+abh_0j+abh_1k-abh_2-abh_3i+b^2h_0-b^2h_1i+b^2h_2j-b^2h_3k\\
    =~&a^2h_0+abh_2-abh_2+b^2h_0+a^2h_1i-abh_3i-abh_3i-b^2h_1i\\
    &+a^2h_2j-abh_0j+abh_0j+b^2h_2j+a^2h_3k+abh_1k+abh_1k-b^2h_3k\\
    =~&h_0+((a^2-b^2)h_1-2abh_3)i+h_2j+((a^2-b^2)h_3+2abh_1)k\\
    =~&h_0+(h_1\cos\theta-h_3\sin\theta)i+h_2j+(h_3\cos\theta+h_1\sin\theta)k\\
  \end{split}
\end{equation}

\begin{equation}
  \begin{pmatrix}h_1'\\h_3'\end{pmatrix}
  =\begin{pmatrix}\cos\theta&-\sin\theta\\\sin\theta&\cos\theta\end{pmatrix}
  \begin{pmatrix}h_1\\h_3\end{pmatrix}
\end{equation}
$h_1$から$h_3$方向への回転になります。

\begin{equation}
  \begin{split}
    &(a-bk)(h_0+h_1i+h_2j+h_3k)(a+bk)\\
    =~&(ah_0+ah_1i+ah_2j+ah_3k-bh_0k-bh_1j+bh_2i+bh_3)(a+bk)\\
    =~&a^2h_0+a^2h_1i+a^2h_2j+a^2h_3k-abh_0k-abh_1j+abh_2i+abh_3\\
    &+abh_0k-abh_1j+abh_2i-abh_3+b^2h_0-b^2h_1i-b^2h_2j+b^2h_3k\\
    =~&a^2h_0+abh_3-abh_3+b^2h_0+a^2h_1i+abh_2i+abh_2i-b^2h_1i\\
    &+a^2h_2j-abh_1j-abh_1j-b^2h_2j+a^2h_3k-abh_0k+abh_0k+b^2h_3k\\
    =~&h_0+((a^2-b^2)h_1+2abh_2)i+((a^2-b^2)h_2-2abh_1)j+h_3k\\
    =~&h_0+(h_1\cos\theta+h_2\sin\theta)i+(h_2\cos\theta-h_1\sin\theta)j+h_3k\\
  \end{split}
\end{equation}

\begin{equation}
  \begin{pmatrix}h_1'\\h_2'\end{pmatrix}
  =\begin{pmatrix}\cos\theta&\sin\theta\\-\sin\theta&\cos\theta\end{pmatrix}
  \begin{pmatrix}h_1\\h_2\end{pmatrix}
\end{equation}
$h_2$から$h_1$方向への回転になります。

この三つの回転角を座標軸$x,y,z$として移動を考えてみましょう。
\begin{gather}
  x,y,z\in\mathbb{R}\\
  e_x=\exp\left(\frac{x}{2}i\right)=\cos\frac{x}{2}+i\sin\frac{x}{2}=a_x+b_xi\\
  e_y=\exp\left(\frac{y}{2}j\right)=\cos\frac{y}{2}+j\sin\frac{y}{2}=a_y+b_yj\\
  e_z=\exp\left(\frac{z}{2}k\right)=\cos\frac{z}{2}+k\sin\frac{z}{2}=a_z+b_zk
\end{gather}

と置くと、任意の四元数$h$の$x,y,z$方向への移動とは
\begin{equation}
  h'=e_z^*e_y^*e_x^*he_xe_ye_z
\end{equation}

四元数$h=h_0+h_1i+h_2j+h_3k$のベクトルパートについては$h_1^2+h_2^2+h_3^2>0$
の条件が有意な空間表現の為に必要なことが直ちに分かります。
つまり、スカラーパートだけの四元数から移動を開始する事ができません。
何か様子がおかしいですがこのまま進めてみます。
$h$のノルムについては特に制限はなさそうです。$h'$を実成分で展開してみます。
\begin{equation}
  \begin{split}
    e_z^*e_y^*e_x^*=~&(a_z-b_zk)(a_y-b_yj)(a_x-b_xi)\\
    =~&(a_ya_z-a_zb_yj-a_yb_zk-b_yb_zi)(a_x-b_xi)\\
    =~&a_xa_ya_z-a_ya_zb_xi-a_xa_zb_yj-a_zb_xb_yk-a_xa_yb_zk+a_yb_xb_zj-a_xb_yb_zi-b_xb_yb_z\\
    =~&a_xa_ya_z-b_xb_yb_z-(a_xb_yb_z+a_ya_zb_x)i-(a_xa_zb_y-a_yb_xb_z)j-(a_xa_yb_z+a_zb_xb_y)k\\
  \end{split}
\end{equation}

右側は
\begin{equation}
  \begin{split}
    e_xe_ye_z=~&(a_x+b_xi)(a_y+b_yj)(a_z+b_zk)\\
    =~&(a_xa_y+a_xb_yj+a_yb_xi+b_xb_yk)(a_z+b_zk)\\
    =~&a_xa_ya_z+a_xa_yb_zk+a_xa_zb_yj+a_xb_yb_zi+a_ya_zb_xi-a_yb_xb_zj+a_zb_xb_yk-b_xb_yb_z\\
    =~&a_xa_ya_z-b_xb_yb_z+(a_xb_yb_z+a_ya_zb_x)i+(a_xa_zb_y-a_yb_xb_z)j+(a_xa_yb_z+a_zb_xb_y)k\\
  \end{split}
\end{equation}

ここで
\begin{equation}
  e_0=a_xa_ya_z-b_xb_yb_z,\quad e_1=a_xb_yb_z+a_ya_zb_x,\quad e_2=a_xa_zb_y-a_yb_xb_z,\quad e_3=a_xa_yb_z+a_zb_xb_y
\end{equation}

と置くと
\begin{equation}
  h'=(e_0-e_1i-e_2j-e_3k)h(e_0+e_1i+e_2j+e_3k)
\end{equation}

互いに共役な四元数にまとまりました。さらに展開して
\begin{equation}
  \begin{split}
    e_z^*e_y^*e_x^*h=~&(e_0-e_1i-e_2j-e_3k)(h_0+h_1i+h_2j+h_3k)\\
    =~&e_0h_0+e_0h_1i+e_0h_2j+e_0h_3k-e_1h_0i+e_1h_1-e_1h_2k+e_1h_3j\\
    &-e_2h_0j+e_2h_1k+e_2h_2-e_2h_3i-e_3h_0k-e_3h_1j+e_3h_2i+e_3h_3\\
    =~&e_0h_0+e_1h_1+e_2h_2+e_3h_3+(e_0h_1-e_1h_0-e_2h_3+e_3h_2)i\\
    &+(e_0h_2+e_1h_3-e_2h_0-e_3h_1)j+(e_0h_3-e_1h_2+e_2h_1-e_3h_0)k\\
  \end{split}
\end{equation}

\begin{equation}
  \begin{split}
    h'=~&(e_0h_0+e_1h_1+e_2h_2+e_3h_3+(e_0h_1-e_1h_0-e_2h_3+e_3h_2)i\\
    &+(e_0h_2+e_1h_3-e_2h_0-e_3h_1)j+(e_0h_3-e_1h_2+e_2h_1-e_3h_0)k)(e_0+e_1i+e_2j+e_3k)\\
    =~&e_0^2h_0+e_0e_1h_1+e_0e_2h_2+e_0e_3h_3+(e_0^2h_1-e_0e_1h_0-e_0e_2h_3+e_0e_3h_2)i\\
    &+(e_0^2h_2+e_0e_1h_3-e_0e_2h_0-e_0e_3h_1)j+(e_0^2h_3-e_0e_1h_2+e_0e_2h_1-e_0e_3h_0)k\\
    &+(e_0e_1h_0+e_1^2h_1+e_1e_2h_2+e_1e_3h_3)i-(e_0e_1h_1-e_1^2h_0-e_1e_2h_3+e_1e_3h_2)\\
    &-(e_0e_1h_2+e_1^2h_3-e_1e_2h_0-e_1e_3h_1)k+(e_0e_1h_3-e_1^2h_2+e_1e_2h_1-e_1e_3h_0)j\\
    &+(e_0e_2h_0+e_1e_2h_1+e_2^2h_2+e_2e_3h_3)j+(e_0e_2h_1-e_1e_2h_0-e_2^2h_3+e_2e_3h_2)k\\
    &-(e_0e_2h_2+e_1e_2h_3-e_2^2h_0-e_2e_3h_1)-(e_0e_2h_3-e_1e_2h_2+e_2^2h_1-e_2e_3h_0)i\\
    &+(e_0e_3h_0+e_1e_3h_1+e_2e_3h_2+e_3^2h_3)k-(e_0e_3h_1-e_1e_3h_0-e_2e_3h_3+e_3^2h_2)j\\
    &+(e_0e_3h_2+e_1e_3h_3-e_2e_3h_0-e_3^2h_1)i-(e_0e_3h_3-e_1e_3h_2+e_2e_3h_1-e_3^2h_0)\\
    =~&(e_0^2+e_1^2+e_2^2+e_3^2)h_0\\
    &+((e_0^2+e_1^2-e_2^2-e_3^2)h_1+2(e_0e_3+e_1e_2)h_2+2(e_1e_3-e_0e_2)h_3)i\\
    &+(2(e_1e_2-e_0e_3)h_1+(e_0^2-e_1^2+e_2^2-e_3^2)h_2+2(e_0e_1+e_2e_3)h_3)j\\
    &+(2(e_0e_2+e_1e_3)h_1+2(e_2e_3-e_0e_1)h_2+(e_0^2-e_1^2-e_2^2+e_3^2)h_3)k\\
  \end{split}
\end{equation}

ここで$e_0,e_1,e_2,e_3$の置き方から$e_0^2+e_1^2+e_2^2+e_3^2=1$なので
\begin{equation}
  \begin{split}
    h_0'=~&h_0\\
    h_1'=~&(e_0^2+e_1^2-e_2^2-e_3^2)h_1+2(e_0e_3+e_1e_2)h_2+2(e_1e_3-e_0e_2)h_3\\
    h_2'=~&2(e_1e_2-e_0e_3)h_1+(e_0^2-e_1^2+e_2^2-e_3^2)h_2+2(e_0e_1+e_2e_3)h_3\\
    h_3'=~&2(e_0e_2+e_1e_3)h_1+2(e_2e_3-e_0e_1)h_2+(e_0^2-e_1^2-e_2^2+e_3^2)h_3\\
  \end{split}
\end{equation}

$h,h'$は単位四元数ではないですが、ノルムは保存されます。
\begin{equation}
  N(h')=N(h)
\end{equation}

$h_0'=h_0$なので
\begin{equation}
  h_1'^2+h_2'^2+h_3'^2=h_1^2+h_2^2+h_3^2
\end{equation}

これは三次元内の球の表面の移動を意味しています。
四次元の計算のつもりが三次元になってしまいました。
この数式は三次元の回転計算としては有用そうです。
しかし、四次元の回転計算にはならない事が分かりました。

以上から等斜線分解の計算式で互いに共役な単位四元数を左右にセットすると三次元の回転になる事が、
実成分に展開して確認できましたが、求める航海術は得られませんでした。

\section{四元数の実軸からの回転}

改めて等斜線分解の計算式を考えてみます。
指数関数を使うと以下のように変形できます。
\begin{equation}
  \begin{split}
    N(h')\exp(\theta'w')=N(h)\exp(au)\exp(\theta w)\exp(bv)\\
    a,b,\theta,\theta'\in\mathbb{R},\quad u,v,w,w'\in\mathbb{V}\\
  \end{split}
\end{equation}

$N(h')=N(h)\in\mathbb{R}$なので結局は
\begin{equation}
  \exp(\theta'w')=\exp(au)\exp(\theta w)\exp(bv)
\end{equation}

指数関数を外すと
\begin{equation}
  \theta'w'=au+\theta w+bv
\end{equation}

ここで以下のように置いて
\begin{equation}
  \begin{split}
    u=u_1i+u_2j+u_3k,\quad v=v_1i+v_2j+v_3k,\quad w=w_1i+w_2j+w_3k\\
    u,v\in\mathbb{V},\quad u_1,u_2,u_3,v_1,v_2,v_3,w_1,w_2,w_3\in\mathbb{R}
  \end{split}
\end{equation}

実軸から$i$軸への回転だけを表すようになる条件は
\begin{equation}
  \begin{split}
    \theta'w_1'=~&au_1+\theta w_1+bv_1\\
    \theta'w_2'=~&\theta w_2\\
    \theta'w_3'=~&\theta w_3\\
  \end{split}
\end{equation}

になるようにするのだから
\begin{gather}
  au_2+bv_2=0 \quad\land\quad au_3+bv_3=0\iff au_2=-bv_2 \quad\land\quad au_3=-bv_3
\end{gather}

つまり以下のような$j,k$に関してだけ共役な単位四元数のペアによる変換式になります。
\begin{gather}
  h'=\exp(au_1i-bv_2j-bv_3k)h\exp(bv_1i+bv_2j+bv_3k)
\end{gather}

直交条件から
\begin{equation}
  \begin{split}
    au_1bv_1-b^2v_2^2-b^2v_3^2=0\\
    au_1v_1-bv_2^2-bv_3^2=0\\
    au_1v_1-b(v_2^2+v_3^2)=0\\
    au_1v_1-b(1-v_1^2)=0\\
    bv_1^2+au_1v_1+b=0
  \end{split}
\end{equation}

$v_1$が実数である条件は
\begin{gather}
  a^2u_1^2-4b^2 \ge 0 \iff au_1 \ge 2b
\end{gather}

$a,b$が満たすべき条件がこれでしょうか?次に右側を$a$で置き換えてみます。
\begin{gather}
  h'=\exp(au_1i+au_2j+au_3k)h\exp(bv_1i-au_2j-au_3k)
\end{gather}

直交条件から
\begin{equation}
  \begin{split}
    au_1bv_1-a^2u_2^2-a^2u_3^2=0\\
    bu_1v_1-au_2^2-au_3^2=0\\
    bu_1v_1-a(u_2^2+u_3^2)=0\\
    bu_1v_1-a(1-u_1^2)=0\\
    au_1^2+bu_1v_1+a=0
  \end{split}
\end{equation}

$u_1$が実数である条件は
\begin{gather}
  b^2v_1^2-4a^2 \ge 0 \iff bv_1 \ge 2a
\end{gather}

以上から実数$a,b$の条件をまとめると
\begin{equation}
  au_1 \ge 2b \quad\land\quad bv_1 \ge 2a  \quad\land\quad u_1,v_1 \le 1
\end{equation}

これらを満たす実数$a,b$は存在しません。
従って、四元数の実軸から$i$軸にだけ回転させる計算は少なくとも単位四元数の回転計算の範囲には存在しないことになります。

\section{八元数の定義}

八元数は以下のように定義されます。
\begin{equation}
  \texttt{Octonion}:\mathbb{O}\coloneq r_0+r_1e_1+r_2e_2+r_3e_3+r_4e_4+r_5e_5+r_6e_6+r_7e_7,
  \quad r_0,...,r_7\in\mathbb{R}
\end{equation}

\begin{wraptable}{r}{8cm}
  \caption{八元数の乗積表}
  \centering
  \begin{tabular}{rrrrrrrr}\hline
        & $e_1$& $e_2$& $e_3$& $e_4$& $e_5$& $e_6$& $e_7$\\\hline
    $e_1$& $ -1$& $e_3$&-$e_2$& $e_5$&-$e_4$&-$e_7$& $e_6$\\\hline
    $e_2$&-$e_3$& $ -1$& $e_1$& $e_6$& $e_7$&-$e_4$&-$e_5$\\\hline
    $e_3$& $e_2$&-$e_1$& $ -1$& $e_7$&-$e_6$& $e_5$&-$e_4$\\\hline
    $e_4$&-$e_5$&-$e_6$&-$e_7$& $ -1$& $e_1$& $e_2$& $e_3$\\\hline
    $e_5$& $e_4$&-$e_7$& $e_6$&-$e_1$& $ -1$&-$e_3$& $e_2$\\\hline
    $e_6$& $e_7$& $e_4$&-$e_5$&-$e_2$& $e_3$& $ -1$&-$e_1$\\\hline
    $e_7$&-$e_6$& $e_5$& $e_4$&-$e_3$&-$e_2$& $e_1$& $ -1$\\\hline
  \end{tabular}
\end{wraptable}

$e_1,...,e_7$の乗法は乗積表から算出できます。
乗法は順番を変えてはならず、3つ以上並んだ場合は左から順番に計算しないといけません。

乗積表には480通りのバリエーションがあり、ここではその一つを書いています。
しかし、その他の479通りの八元数についての本質的な違いはないと考えられています。
また、虚数単位$e_1,...,e_7$の書き方についても、$i,j,k,l$や$h,i,j,k$を組み合わせたりする事もあります。

八元数の共役$o^*$とノルム$N(o)$は以下のようになります。
\begin{equation}
  \begin{split}
    o=~&r_0+r_1e_1+r_2e_2+r_3e_3+r_4e_4+r_5e_5+r_6e_6+r_7e_7\\
    o^*=~&r_0-r_1e_1-r_2e_2-r_3e_3-r_4e_4-r_5e_5-r_6e_6-r_7e_7\\
    N(o)=~&o^*o=r_0^2+r_1^2+r_2^2+r_3^2+r_4^2+r_5^2+r_6^2+r_7^2
  \end{split}
\end{equation}

八元数とは別に分解型八元数という物が定義できますが、それはまた別途触れたいと思います。

とりあえず、任意の八元数$o_0,o_1$の積をその実数成分で表現しておきます。
\begin{equation}
  \begin{split}
    o_0=~&p_0+p_1e_1+p_2e_2+p_3e_3+p_4e_4+p_5e_5+p_6e_6+p_7e_7\\
    o_1=~&q_0+q_1e_1+q_2e_2+q_3e_3+q_4e_4+q_5e_5+q_6e_6+q_7e_7
  \end{split}
\end{equation}

$o_0o_1$を実直に計算しても良いですがスカラーパートを分ける事で少し楽をします。
\begin{equation}
  \begin{split}
    u=~&p_1e_1+p_2e_2+p_3e_3+p_4e_4+p_5e_5+p_6e_6+p_7e_7\\
    v=~&q_1e_1+q_2e_2+q_3e_3+q_4e_4+q_5e_5+q_6e_6+q_7e_7
  \end{split}
\end{equation}

\begin{equation}
  \begin{split}
    o_0o_1=~&(p_0+u)(q_0+v)\\
    =~&p_0q_0+p_0v+q_0u+uv
  \end{split}
\end{equation}

\begin{equation}
  \begin{split}
    uv=~&(p_1e_1+p_2e_2+p_3e_3+p_4e_4+p_5e_5+p_6e_6+p_7e_7)(q_1e_1+q_2e_2+q_3e_3+q_4e_4+q_5e_5+q_6e_6+q_7e_7)\\
    =~&p_1e_1q_1e_1+p_1e_1q_2e_2+p_1e_1q_3e_3+p_1e_1q_4e_4+p_1e_1q_5e_5+p_1e_1q_6e_6+p_1e_1q_7e_7\\
    &+p_2e_2q_1e_1+p_2e_2q_2e_2+p_2e_2q_3e_3+p_2e_2q_4e_4+p_2e_2q_5e_5+p_2e_2q_6e_6+p_2e_2q_7e_7\\
    &+p_3e_3q_1e_1+p_3e_3q_2e_2+p_3e_3q_3e_3+p_3e_3q_4e_4+p_3e_3q_5e_5+p_3e_3q_6e_6+p_3e_3q_7e_7\\
    &+p_4e_4q_1e_1+p_4e_4q_2e_2+p_4e_4q_3e_3+p_4e_4q_4e_4+p_4e_4q_5e_5+p_4e_4q_6e_6+p_4e_4q_7e_7\\
    &+p_5e_5q_1e_1+p_5e_5q_2e_2+p_5e_5q_3e_3+p_5e_5q_4e_4+p_5e_5q_5e_5+p_5e_5q_6e_6+p_5e_5q_7e_7\\
    &+p_6e_6q_1e_1+p_6e_6q_2e_2+p_6e_6q_3e_3+p_6e_6q_4e_4+p_6e_6q_5e_5+p_6e_6q_6e_6+p_6e_6q_7e_7\\
    &+p_7e_7q_1e_1+p_7e_7q_2e_2+p_7e_7q_3e_3+p_7e_7q_4e_4+p_7e_7q_5e_5+p_7e_7q_6e_6+p_7e_7q_7e_7\\
    =~&-p_1q_1+p_1q_2e_3-p_1q_3e_2+p_1q_4e_5-p_1q_5e_4-p_1q_6e_7+p_1q_7e_6\\
    &-p_2q_1e_3-p_2q_2+p_2q_3e_1+p_2q_4e_6+p_2q_5e_7-p_2q_6e_4-p_2q_7e_5\\
    &+p_3q_1e_2-p_3q_2e_1-p_3q_3+p_3q_4e_7-p_3q_5e_6+p_3q_6e_5-p_3q_7e_4\\
    &-p_4q_1e_5-p_4q_2e_6-p_4q_3e_7-p_4q_4+p_4q_5e_1+p_4q_6e_2+p_4q_7e_3\\
    &+p_5q_1e_4-p_5q_2e_7+p_5q_3e_6-p_5q_4e_1-p_5q_5-p_5q_6e_3+p_5q_7e_2\\
    &+p_6q_1e_7+p_6q_2e_4-p_6q_3e_5-p_6q_4e_2+p_6q_5e_3-p_6q_6-p_6q_7e_1\\
    &-p_7q_1e_6+p_7q_2e_5+p_7q_3e_4-p_7q_4e_3-p_7q_5e_2+p_7q_6e_1-p_7q_7\\
    =~&-p_1q_1-p_2q_2-p_3q_3-p_4q_4-p_5q_5-p_6q_6-p_7q_7\\
    &+p_2q_3e_1-p_3q_2e_1+p_4q_5e_1-p_5q_4e_1-p_6q_7e_1+p_7q_6e_1\\
    &-p_1q_3e_2+p_3q_1e_2+p_4q_6e_2+p_5q_7e_2-p_6q_4e_2-p_7q_5e_2\\
    &+p_1q_2e_3-p_2q_1e_3+p_4q_7e_3-p_5q_6e_3+p_6q_5e_3-p_7q_4e_3\\
    &-p_1q_5e_4-p_2q_6e_4-p_3q_7e_4+p_5q_1e_4+p_6q_2e_4+p_7q_3e_4\\
    &+p_1q_4e_5-p_2q_7e_5+p_3q_6e_5+p_4q_1e_5-p_6q_3e_5+p_7q_2e_5\\
    &+p_1q_7e_6+p_2q_4e_6-p_3q_5e_6-p_4q_2e_6+p_5q_3e_6-p_7q_1e_6\\
    &-p_1q_6e_7+p_2q_5e_7+p_3q_4e_7-p_4q_3e_7-p_5q_2e_7+p_6q_1e_7
  \end{split}
\end{equation}

\begin{equation}
  \begin{split}
    uv=~&-p_1q_1-p_2q_2-p_3q_3-p_4q_4-p_5q_5-p_6q_6-p_7q_7\\
    &+(p_2q_3-p_3q_2+p_4q_5-p_5q_4+p_7q_6-p_6q_7)e_1\\
    &+(p_3q_1-p_1q_3+p_4q_6-p_6q_4+p_5q_7-p_7q_5)e_2\\
    &+(p_1q_2-p_2q_1+p_4q_7-p_7q_4+p_6q_5-p_5q_6)e_3\\
    &+(p_5q_1-p_1q_5+p_6q_2-p_2q_6+p_7q_3-p_3q_7)e_4\\
    &+(p_1q_4-p_4q_1+p_3q_6-p_6q_3+p_7q_2-p_2q_7)e_5\\
    &+(p_1q_7-p_7q_1+p_2q_4-p_4q_2+p_5q_3-p_3q_5)e_6\\
    &+(p_2q_5-p_5q_2+p_3q_4-p_4q_3+p_6q_1-p_1q_6)e_7
  \end{split}
\end{equation}

\begin{equation}
  \begin{split}
    o_0o_1=~&p_0q_0-p_1q_1-p_2q_2-p_3q_3-p_4q_4-p_5q_5-p_6q_6-p_7q_7\\
    &+(p_0q_1+p_1q_0+p_2q_3-p_3q_2+p_4q_5-p_5q_4+p_7q_6-p_6q_7)e_1\\
    &+(p_0q_2+p_2q_0+p_3q_1-p_1q_3+p_4q_6-p_6q_4+p_5q_7-p_7q_5)e_2\\
    &+(p_0q_3+p_3q_0+p_1q_2-p_2q_1+p_4q_7-p_7q_4+p_6q_5-p_5q_6)e_3\\
    &+(p_0q_4+p_4q_0+p_5q_1-p_1q_5+p_6q_2-p_2q_6+p_7q_3-p_3q_7)e_4\\
    &+(p_0q_5+p_5q_0+p_1q_4-p_4q_1+p_3q_6-p_6q_3+p_7q_2-p_2q_7)e_5\\
    &+(p_0q_6+p_6q_0+p_1q_7-p_7q_1+p_2q_4-p_4q_2+p_5q_3-p_3q_5)e_6\\
    &+(p_0q_7+p_7q_0+p_2q_5-p_5q_2+p_3q_4-p_4q_3+p_6q_1-p_1q_6)e_7
  \end{split}
\end{equation}

$uv$は七次元の内積と外積の形をしていますので、七次元の角度や回転軸を計算する事ができると考えられます。
四元数を使って三次元の回転を計算できたように、八元数を使って七次元を回転できるのではないかと期待したい所です。
しかし、目下の所、四次元だけを回転させる計算方法を確立させたいので、まずは四次元に着目します。

\section{八元数による四次元の回転計算}

四次元ベクトル$(s_0,s_1,s_2,s_3)$を八元数の$e_4,e_5,e_6,e_7$に割り当てて回転計算を行ってみます。
\begin{equation}
  s=s_0e_4+s_1e_5+s_2e_6+s_3e_7
\end{equation}

方向を表す四次元単位ベクトル$\bm{a},\bm{b}$があり、$\bm{a}$から$\bm{b}$の方向に$t\in\mathbb{R}$の割合だけ回転させます。
$\bm{a}$と$\bm{b}$が成す角度$\theta$を内積から求めます。
\begin{equation}
  \begin{split}
    \bm{a}=~&(a_0,a_1,a_2,a_3),\quad \bm{b}=~(b_0,b_1,b_2,b_3),\quad a_0,a_1,a_2,a_3,b_0,b_1,b_2,b_3\in\mathbb{R}\\
    \theta=~&\arccos(\bm{a}\cdot\bm{b})=\arccos(a_0b_0+a_1b_1+a_2b_2+a_3b_3),\quad 0<\theta<\pi,\quad \theta\in\mathbb{R}
  \end{split}
\end{equation}

回転軸を表す八元数を七次元の外積から求めます。
今$e_4,e_5,e_6,e_7$の内に四次元があると仮定しているので$\bm{a},\bm{b}$を以下の八元数に置き換えます。
\begin{equation}
  a=a_0e_4+a_1e_5+a_2e_6+a_3e_7,\quad b=b_0e_4+b_1e_5+b_2e_6+b_3e_7
\end{equation}

積を計算し外積を抽出します。
\begin{equation}
  \begin{split}
    ab=~&-\cos\theta\\
    &+(a_0b_1-a_1b_0+a_3b_2-a_2b_3)e_1\\
    &+(a_0b_2-a_2b_0+a_1b_3-a_3b_1)e_2\\
    &+(a_0b_3-a_3b_0+a_2b_1-a_1b_2)e_3
  \end{split}
\end{equation}

$|a|=|b|=|ab|=1$なので以下であることが確認できます。
\begin{equation}
  \begin{split}
    \cos^2\theta+(a_0b_1-a_1b_0+a_3b_2-a_2b_3)^2+(a_0b_2-a_2b_0+a_1b_3-a_3b_1)^2+(a_0b_3-a_3b_0+a_2b_1-a_1b_2)^2=1\\
    (a_0b_1-a_1b_0+a_3b_2-a_2b_3)^2+(a_0b_2-a_2b_0+a_1b_3-a_3b_1)^2+(a_0b_3-a_3b_0+a_2b_1-a_1b_2)^2=\sin^2\theta
  \end{split}
\end{equation}

$e_1,e_2,e_3$の係数から八元数を構成し直して正規化しておきます。
\begin{gather}
  w_1=\frac{a_0b_1-a_1b_0+a_3b_2-a_2b_3}{\sin\theta},\quad
  w_2=\frac{a_0b_2-a_2b_0+a_1b_3-a_3b_1}{\sin\theta},\quad
  w_3=\frac{a_0b_3-a_3b_0+a_2b_1-a_1b_2}{\sin\theta}\\
  w=w_1e_1+w_2e_2+w_3e_3,\quad |w|=1
\end{gather}

こうして八元数より抽出した外積の``回転軸''$w$を観察してみると、元の軸$e_4,e_5,e_6,e_7$が構成する空間の外側にある
$e_1,e_2,e_3$軸が構成する空間で三次元として存在していると読み取れます。
従って、四次元を回転させる軸は七次元内の別の三次元に存在すると解釈できます。
四次元を四元数で回転させようとしても上手くいかなかった原因がここにありそうです。

他の軸の組み合わせはどうなるかが気なります。
7つのうち4つを選ぶ組み合わせの数は35通りあります。
$e_1,e_2,e_3,e_4$内に四次元を取ると
\begin{gather}
  a=a_0e_1+a_1e_2+a_2e_3+a_3e_4,\quad b=b_0e_1+b_1e_2+b_2e_3+b_3e_4\\
  ab+|ab|=(a_1b_2-a_2b_1)e_1+(a_2b_0-a_0b_2)e_2+(a_0b_1-a_1b_0)e_3+(a_0b_3-a_3b_0)e_5+(a_1b_3-a_3b_1)e_6+(a_2b_3-a_3b_2)e_7
\end{gather}
というように6軸になります。

どのような回転軸になるかは多様性と規則性がありそうですが、とりあえず回転計算を継続します。
以下のようにノルムが1の純虚八元数を七次元ベルソルとして定義しておきます。
\begin{equation}
  \begin{split}
    \texttt{Versor7}:\mathbb{V}_7\coloneq&\{d:d=\sqrt{-1},\quad d\in\mathbb{O}\}\\
    =&\{d_1e_1+d_2e_2+d_3e_3+d_4e_4+d_5e_5+d_6e_6+d_7e_7,\quad d_1,d_2,d_3,d_4,d_5,d_6,d_7\in\mathbb{R},\quad |d|=1\}
  \end{split}
\end{equation}

一般的に、四次元単位ベクトル$\bm{a},\bm{b}$を純虚八元数の係数のどれかに割り当てて七次元ベルソル$a,b$とした時、
これらを含む回転の回転軸を表す七次元ベルソル$w$は、$\bm{a},\bm{b}$が成す角度を$\theta$として、
\begin{equation}
  w=\frac{ab+\cos\theta}{\sin\theta},\quad w\in\mathbb{V}_7
\end{equation}
となる事が見えてきました。
この回転軸によって回転を計算する為の八元数は以下のようになります。
\begin{equation}
  o=\exp(-0.5t\theta w)=\cos 0.5t\theta-w\sin 0.5t\theta
\end{equation}
回転させたい四次元ベクトルを純虚八元数のいずれかに当てはめた八元数を$s$とすると、
回転後の四次元座標を示す八元数$s'$が以下のように算出される事が期待されます。
\begin{equation}
  s'=o^*so
\end{equation}

実数成分で${}_7 \mathrm{C}_4=35$の組み合わせで展開してみるのは流石に冗長なので、コンピューターを使って計算をしてみましょう。
するとすべての組み合わせで、この回転計算の結果$s'$が$s$が選択した八元数の4つの軸による空間に戻ることが確認できます。
また、3Dグラフで視覚化をしてみると、確かに循環する輪が描かれる事が分かります。
循環する輪の形は(1)右回りにねじれる、(2)左回りにねじれる、(3)閉じている、(4)開いているの4種類あるようです。
これらの内、(1)(2)のねじれるパターンは、四元数の乗算と同じ形になっており、ねじれるという欠点があります。
そして、平行線を伸ばしたら合流してしまう(3)は除外できるでしょう。
(4)の開いているという回転のパターンが求める三次元超球面の移動だと考えられます。
しかし、これらの回転パターンは注目したある二次元の面にだけ成り立つ特徴があり、
三次元の移動を考えた時に3軸全てで(4)の開いているという形状をしているパターンは35種類の組み合わせの中には発見できません。
つまり、三次元超球面から切り取った原点を含む二次元面については回転計算が成立しますが、そこから外れる事ができません。

\section{分解型八元数の定義}

\begin{wraptable}{r}{8cm}
  \caption{分解型八元数の乗積表\label{split-octonion-multiplication-table}}
  \centering
  \begin{tabular}{rrrrrrrr}\hline
        & $e_1$& $e_2$& $e_3$& $e_4$& $e_5$& $e_6$& $e_7$\\\hline
    $e_1$& $ -1$& $e_3$&-$e_2$&-$e_5$& $e_4$&-$e_7$& $e_6$\\\hline
    $e_2$&-$e_3$& $ -1$& $e_1$&-$e_6$& $e_7$& $e_4$&-$e_5$\\\hline
    $e_3$& $e_2$&-$e_1$& $ -1$&-$e_7$&-$e_6$& $e_5$& $e_4$\\\hline
    $e_4$& $e_5$& $e_6$& $e_7$& $  1$& $e_1$& $e_2$& $e_3$\\\hline
    $e_5$&-$e_4$&-$e_7$& $e_6$&-$e_1$& $  1$& $e_3$&-$e_2$\\\hline
    $e_6$& $e_7$&-$e_4$&-$e_5$&-$e_2$&-$e_3$& $  1$& $e_1$\\\hline
    $e_7$&-$e_6$& $e_5$&-$e_4$&-$e_3$& $e_2$&-$e_1$& $  1$\\\hline
  \end{tabular}
\end{wraptable}

分解型八元数についても可能性を調査してみましょう。
分解型八元数は以下のように定義されます。
\begin{multline}
  \texttt{Split-octonion}:\\
  \mathbb{O}_s\coloneq r_0+r_1e_1+r_2e_2+r_3e_3+r_4e_4+r_5e_5+r_6e_6+r_7e_7\\
  \quad r_0,...,r_7\in\mathbb{R}
\end{multline}

$e_1,...,e_7$の乗法は表\ref{split-octonion-multiplication-table}に従います。

共役は八元数と同じ定義です。
\begin{equation}
  \begin{split}
    s=~&r_0+r_1e_1+r_2e_2+r_3e_3+r_4e_4+r_5e_5+r_6e_6+r_7e_7\\
    s^*=~&r_0-r_1e_1-r_2e_2-r_3e_3-r_4e_4-r_5e_5-r_6e_6-r_7e_7
  \end{split}
\end{equation}

ノルムについては少し様子が違って以下のようになります。
\begin{equation}
  \begin{split}
    N(s)=s^*s=(r_0^2+r_1^2+r_2^2+r_3^2)-(r_4^2+r_5^2+r_6^2+r_7^2)
  \end{split}
\end{equation}
八元数同様に乗積表の右上と左下の符号が反転しているので共役をかけるとそれらが消えて実数部分のみになります。

まず、任意の分解型八元数$s_0,s_1$の積をその実数成分で表現しておきましょう。
\begin{equation}
  \begin{split}
    s_0=~&p_0+p_1e_1+p_2e_2+p_3e_3+p_4e_4+p_5e_5+p_6e_6+p_7e_7\\
    s_1=~&q_0+q_1e_1+q_2e_2+q_3e_3+q_4e_4+q_5e_5+q_6e_6+q_7e_7
  \end{split}
\end{equation}

スカラーパートを分けておきます。
\begin{equation}
  \begin{split}
    u=~&p_1e_1+p_2e_2+p_3e_3+p_4e_4+p_5e_5+p_6e_6+p_7e_7\\
    v=~&q_1e_1+q_2e_2+q_3e_3+q_4e_4+q_5e_5+q_6e_6+q_7e_7
  \end{split}
\end{equation}

\begin{equation}
  \begin{split}
    s_0s_1=~&(p_0+u)(q_0+v)\\
    =~&p_0q_0+p_0v+q_0u+uv
  \end{split}
\end{equation}

\begin{equation}
  \begin{split}
    uv=~&(p_1e_1+p_2e_2+p_3e_3+p_4e_4+p_5e_5+p_6e_6+p_7e_7)(q_1e_1+q_2e_2+q_3e_3+q_4e_4+q_5e_5+q_6e_6+q_7e_7)\\
    =~&p_1e_1q_1e_1+p_1e_1q_2e_2+p_1e_1q_3e_3+p_1e_1q_4e_4+p_1e_1q_5e_5+p_1e_1q_6e_6+p_1e_1q_7e_7\\
    &+p_2e_2q_1e_1+p_2e_2q_2e_2+p_2e_2q_3e_3+p_2e_2q_4e_4+p_2e_2q_5e_5+p_2e_2q_6e_6+p_2e_2q_7e_7\\
    &+p_3e_3q_1e_1+p_3e_3q_2e_2+p_3e_3q_3e_3+p_3e_3q_4e_4+p_3e_3q_5e_5+p_3e_3q_6e_6+p_3e_3q_7e_7\\
    &+p_4e_4q_1e_1+p_4e_4q_2e_2+p_4e_4q_3e_3+p_4e_4q_4e_4+p_4e_4q_5e_5+p_4e_4q_6e_6+p_4e_4q_7e_7\\
    &+p_5e_5q_1e_1+p_5e_5q_2e_2+p_5e_5q_3e_3+p_5e_5q_4e_4+p_5e_5q_5e_5+p_5e_5q_6e_6+p_5e_5q_7e_7\\
    &+p_6e_6q_1e_1+p_6e_6q_2e_2+p_6e_6q_3e_3+p_6e_6q_4e_4+p_6e_6q_5e_5+p_6e_6q_6e_6+p_6e_6q_7e_7\\
    &+p_7e_7q_1e_1+p_7e_7q_2e_2+p_7e_7q_3e_3+p_7e_7q_4e_4+p_7e_7q_5e_5+p_7e_7q_6e_6+p_7e_7q_7e_7\\
    =~&-p_1q_1+p_1q_2e_3-p_1q_3e_2-p_1q_4e_5+p_1q_5e_4-p_1q_6e_7+p_1q_7e_6\\
    &-p_2q_1e_3-p_2q_2+p_2q_3e_1-p_2q_4e_6+p_2q_5e_7+p_2q_6e_4-p_2q_7e_5\\
    &+p_3q_1e_2-p_3q_2e_1-p_3q_3-p_3q_4e_7-p_3q_5e_6+p_3q_6e_5+p_3q_7e_4\\
    &+p_4q_1e_5+p_4q_2e_6+p_4q_3e_7+p_4q_4+p_4q_5e_1+p_4q_6e_2+p_4q_7e_3\\
    &-p_5q_1e_4-p_5q_2e_7+p_5q_3e_6-p_5q_4e_1+p_5q_5+p_5q_6e_3-p_5q_7e_2\\
    &+p_6q_1e_7-p_6q_2e_4-p_6q_3e_5-p_6q_4e_2-p_6q_5e_3+p_6q_6+p_6q_7e_1\\
    &-p_7q_1e_6+p_7q_2e_5-p_7q_3e_4-p_7q_4e_3+p_7q_5e_2-p_7q_6e_1+p_7q_7\\
    =~&-p_1q_1-p_2q_2-p_3q_3+p_4q_4+p_5q_5+p_6q_6+p_7q_7\\
    &+(p_2q_3-p_3q_2+p_4q_5-p_5q_4+p_6q_7-p_7q_6)e_1\\
    &+(p_3q_1-p_1q_3+p_4q_6-p_6q_4+p_7q_5-p_5q_7)e_2\\
    &+(p_1q_2-p_2q_1+p_4q_7-p_7q_4+p_5q_6-p_6q_5)e_3\\
    &+(p_1q_5-p_5q_1+p_2q_6-p_6q_2+p_3q_7-p_7q_3)e_4\\
    &+(p_3q_6-p_6q_3+p_4q_1-p_1q_4+p_7q_2-p_2q_7)e_5\\
    &+(p_1q_7-p_7q_1+p_4q_2-p_2q_4+p_5q_3-p_3q_5)e_6\\
    &+(p_2q_5-p_5q_2+p_4q_3-p_3q_4+p_6q_1-p_1q_6)e_7\\
  \end{split}
\end{equation}
この展開された式は八元数の七次元の外積とは別物のようです。何を表すのでしょうか?
掘り下げて調査する必要がありそうです。

\section{分解型八元数の角度}

分解型八元数に内在する角度がオイラーの公式からどうなるかを調べましょう。
\begin{equation}
  s_L=~r_0+r_1e_1+r_2e_2+r_3e_3,\quad s_R=r_4e_4+r_5e_5+r_6e_6+r_7e_7
\end{equation}
というように分解すると$s_L$は四元数と同様になり、$r_1^2+r_2^2+r_3^2\ne 0$の時、
\begin{equation}
  \alpha v_L=r_1e_1+r_2e_2+r_3e_3,\quad v_L^2=-1,\quad \alpha\in\mathbb{R}
\end{equation}
と置いて、
\begin{equation}
  \exp(s_L)=\exp(r_0+\alpha v_L)=\exp(r_0)(\cos \alpha+v_L\sin\alpha)
\end{equation}
$s_R$については、$|s_R|\ne 0$の時、
\begin{equation}
  \beta v_R=r_4e_4+r_5e_5+r_6e_6+r_7e_7,\quad v_R^2=+1,\quad \beta\in\mathbb{R}
\end{equation}
と置くと、分解型複素数のオイラーの公式から、
\begin{equation}
  \exp(s_R)=\exp(\beta v_R)=\cosh \beta+v_R\sinh \beta
\end{equation}
従って、分解型八元数の角度表現は以下のようになります。
\begin{equation}
  \begin{split}
    \exp(s)=~&\exp(s_L+s_R)=\exp(s_L)\exp(s_R)
    =\exp(r_0)(\cos\alpha+v_L\sin\alpha)(\cosh\beta+v_R\sinh\beta)
  \end{split}
\end{equation}
純虚分解型八元数について書き直すと、
\begin{equation}
  \exp(s)=\exp(\alpha v_L+\beta v_R)=(\cos\alpha+v_L\sin\alpha)(\cosh\beta+v_R\sinh\beta),\quad r_0=0
\end{equation}
以下のように展開してみます。
\begin{equation}
  \exp(s)=\cos\alpha\cosh\beta+v_Lv_R\sin\alpha\sinh\beta+v_L\sin\alpha\cosh\beta+v_R\cos\alpha\sinh\beta
\end{equation}
複素数による三角関数と加法定理より、$I^2=-1$という別の虚数単位を導入して、
\begin{equation}
  \begin{split}
    \cos(\alpha-I\beta)=~&\cos\alpha\cos I\beta+\sin\alpha\sin I\beta\\
    =~&\cos\alpha\cosh\beta+I\sin\alpha\sinh\beta\\
    \sin(\alpha-I\beta)=~&\sin\alpha\cos I\beta-\cos\alpha\sin I\beta\\
    =~&\sin\alpha\cosh\beta-I\cos\alpha\sinh\beta\\
  \end{split}
\end{equation}
さらに$J^2=-1$という別の虚数単位を導入すると、
\begin{equation}
  J\sin(\alpha-I\beta)=J\sin\alpha\cosh\beta-JI\cos\alpha\sinh\beta
\end{equation}
$I=v_Lv_R,J=v_L$というように当てはめると、
\begin{equation}
  \begin{split}
    \exp(s)=~&\cos\alpha\cosh\beta+v_Lv_R\sin\alpha\sinh\beta+v_L\sin\alpha\cosh\beta-v_Lv_Lv_R\cos\alpha\sinh\beta\\
    =~&\cos(\alpha-v_Lv_R\beta)+v_L\sin(\alpha-v_Lv_R\beta)\\
    =~&\exp(v_L(\alpha-v_Lv_R\beta))\\
    =~&\exp(v_L\alpha+v_R\beta)\\
    =~&\exp(s)\\
  \end{split}
\end{equation}
元に戻ります。ここで$\theta=\alpha-v_Lv_R\beta$と置けば、四元数のオイラーの公式と同じ形が得られます。
\begin{equation}
  \exp(v_L\theta)=\cos\theta+v_L\sin\theta
\end{equation}

\section{純虚分解型八元数の指数関数}

改めて純虚分解型八元数を以下のように定義し直します。
\begin{equation}
  \begin{split}
    \exp(s)=~&\exp(\alpha u+\beta v)=p_0+p_1e_1+p_2e_2+p_3e_3+p_4e_4+p_5e_5+p_6e_6+p_7e_7\\
    &\alpha,\beta,p_0,...,p_7\in\mathbb{R}\\
    u=~&a_1e_1+a_2e_2+a_3e_3,\quad a_1^2+a_2^2+a_3^2=1\\
    v=~&a_4e_4+a_5e_5+a_6e_6+a_7e_7,\quad a_4^2+a_5^2+a_6^2+a_7^2=1\\
    &a_1,...,a_7\in\mathbb{R}\\
  \end{split}
\end{equation}
$p_0,...,p_7$を角度と方向からなる実数で表現してみましょう。
\begin{equation}
  \begin{split}
    \exp(s)=~&(\cos\alpha+u\sin\alpha)(\cosh\beta+v\sinh\beta)\\
    =~&\cos\alpha\cosh\beta+uv\sin\alpha\sinh\beta+u\sin\alpha\cosh\beta+v\cos\alpha\sinh\beta\\
  \end{split}
\end{equation}
$uv$を先に展開して、
\begin{equation}
  \begin{split}
    uv=~&(a_1e_1+a_2e_2+a_3e_3)(a_4e_4+a_5e_5+a_6e_6+a_7e_7)\\
    =~&(a_1a_5+a_2a_6+a_3a_7)e_4\\
    &+(a_3a_6-a_1a_4-a_2a_7)e_5\\
    &+(a_1a_7-a_2a_4-a_3a_5)e_6\\
    &+(a_2a_5-a_3a_4-a_1a_6)e_7\\
  \end{split}
\end{equation}
$e_1,e_2,e_3$の項が消えてしまうのが印象的です。
\begin{equation}
  \begin{split}
    \exp(s)=~&\cos\alpha\cosh\beta+a_1e_1\sin\alpha\cosh\beta+a_2e_2\sin\alpha\cosh\beta+a_3e_3\sin\alpha\cosh\beta\\
    &+((a_1a_5+a_2a_6+a_3a_7)\sin\alpha+a_4\cos\alpha)e_4\sinh\beta\\
    &+((a_3a_6-a_1a_4-a_2a_7)\sin\alpha+a_5\cos\alpha)e_5\sinh\beta\\
    &+((a_1a_7-a_2a_4-a_3a_5)\sin\alpha+a_6\cos\alpha)e_6\sinh\beta\\
    &+((a_2a_5-a_3a_4-a_1a_6)\sin\alpha+a_7\cos\alpha)e_7\sinh\beta\\
  \end{split}
\end{equation}

\begin{equation}
  \begin{split}
    p_0=~&\cos\alpha\cosh\beta\\
    p_1=~&a_1\sin\alpha\cosh\beta\\
    p_2=~&a_2\sin\alpha\cosh\beta\\
    p_3=~&a_3\sin\alpha\cosh\beta\\
    p_4=~&((a_1a_5+a_2a_6+a_3a_7)\sin\alpha+a_4\cos\alpha)\sinh\beta\\
    p_5=~&((a_3a_6-a_1a_4-a_2a_7)\sin\alpha+a_5\cos\alpha)\sinh\beta\\
    p_6=~&((a_1a_7-a_2a_4-a_3a_5)\sin\alpha+a_6\cos\alpha)\sinh\beta\\
    p_7=~&((a_2a_5-a_3a_4-a_1a_6)\sin\alpha+a_7\cos\alpha)\sinh\beta\\
  \end{split}
\end{equation}

\section{分解型八元数による回転}

分解型八元数に内在する角度が2つある事が分かってきました。
方向を表す2つのベクトルもあります。
三次元単位ベクトルと四次元単位ベクトルです。
これらが組み合わさって七次元を構成しています。

2つの純虚分解型八元数$u,v$を以下のように定義して回転を計算してみます。
\begin{equation}
  \begin{split}
    u=~&p_1e_1+p_2e_2+p_3e_3+p_4e_4+p_5e_5+p_6e_6+p_7e_7\\
    &p_1^2+p_2^2+p_3^2=1,\quad p_4^2+p_5^2+p_6^2+p_7^2=1\\
    v=~&q_1e_1+q_2e_2+q_3e_3+q_4e_4+q_5e_5+q_6e_6+q_7e_7\\
    &q_1^2+q_2^2+q_3^2=1,\quad q_4^2+q_5^2+q_6^2+q_7^2=1\\
    \cos\alpha=~&p_1q_1+p_2q_2+p_3q_3\\
    \cos\beta=~&p_4q_4+p_5q_5+p_6q_6+p_7q_7\\
  \end{split}
\end{equation}
$u,v$のノルムは$0$になります。
\begin{equation}
  N(u)=N(v)=N(uv)=0
\end{equation}
積の計算結果から、
\begin{equation}
  uv=-\cos\alpha+\cos\beta+s
\end{equation}
となるように$s$を置くと、
\begin{gather}
  N(uv)=(-\cos\alpha+\cos\beta)^2+N(s)=0\\
  N(s)=-(-\cos\alpha+\cos\beta)^2
\end{gather}
というような関係性が見られます。回転軸$w$は、
\begin{equation}
  \begin{split}
    &s=s_1e_1+s_2e_2+s_3e_3+s_4e_4+s_5e_5+s_6e_6+s_7e_7\\
    &s_L=s_1e_1+s_2e_2+s_3e_3,\quad s_R=s_4e_4+s_5e_5+s_6e_6+s_7e_7\\
  \end{split}
\end{equation}
として、
\begin{equation}
  \begin{split}
    w_L=~&\frac{s_L}{|s_L|},\quad w_R=\frac{s_R}{|s_R|}\\
    w=~&w_L+w_R,\quad N(w)=0\\
  \end{split}
\end{equation}
回転子$\rho$は、$w_L$方向に$a\alpha$、$w_R$方向に$b\beta$回転させるとして、
\begin{equation}
  \rho=\exp(-0.5a\alpha w_L+b\beta w_R)=(\cos 0.5a\alpha-w_L\sin0.5a\alpha)(\cosh 0.5b\beta-w_R\sinh 0.5b\beta)
\end{equation}
純虚分解型八元数$f$を回転させて$f'$になるとすると、
\begin{equation}
  \begin{split}
    f'=~&\rho^*f\rho\\
    =~&(\cosh 0.5b\beta+w_R\sinh 0.5b\beta)(\cos 0.5a\alpha+w_L\sin0.5a\alpha)f
    (\cos 0.5a\alpha-w_L\sin0.5a\alpha)(\cosh 0.5b\beta-w_R\sinh 0.5b\beta)
  \end{split}
\end{equation}

\end{document}

\chapter{多体シミュレーションにおける物理法則}

\section{時間進捗の考え方の導入}

多体物理シミュレーションに適用する物理法則を考えるに当たって、最も重要な事は時間に対する考え方になります。
通常の物理方程式の解の求め方では、時間を連続した直線的な概念として把握します。
時間を直線的な$t$のパラメーターとして考えてある時刻を$t$に当てはめれば、どんな未来も過去も物理的実体を厳密に算出する事ができます。

しかし、このような物理方程式を多体について厳密に解いていく事は不可能です。
コンピューターによる多体シミュレーションでは時間は非連続的であると考えます。
フィルム映画やアニメーションのようにコマ送りで物理法則を計算していきます。
計算する内容は現在の物理的パラメーターに基づいた次の瞬間の物理パラメーターの遷移についてだけです。

このシミュレーションの方法では任意に時刻を指定して状態を厳密に算出する事はできません。
もし任意の時刻の状態を知りたいのであれば、一度知りたい時刻の範囲の計算を行っておいて、そのデータを全部記録しておく事になります。

\section{時間進捗で計算する内容}

全ての物体が分解したり合体したりしないで総数が一定の粒子であるとすれば、
それぞれの粒子について位置と運動量をデータとして持っておき、それらの相互作用を計算することで多体シミュレーションを成立させます。
具体的に行う計算には2種類あります。

一つは、状態変化の計算になります。
具体的には速度を位置の変化に反映させる計算になります。
$N$体あれば$N$回の計算で済みます。

もう一つは、他の粒子との相互作用になります。
この計算は、$N$体あれば$N(N-1)/2$回の計算になり、計算量オーダー的に$O(n^2)$となりますので、粒子数が増えると計算量が激増します。
具体的には、運動量の交換がこの相互作用で行われます。

\chapter{クリフォード代数の導入}

\section{クリフォード代数の構成}

クリフォード代数は、2種類の基底ベクトルで構成されます。二乗して$1$になる基底ベクトルと、二乗して$-1$になる基底ベクトルです。
これらの基底ベクトルを任意の個数だけ使う事ができます。通常は、$e$に添え字で書いて$e_1, e_2, ...$というように基底ベクトルを表記します。
異なる基底ベクトル間の積に交換法則はありません。積の順序を変えると符号が反転します。

\begin{equation}
  e_1e_2=-e_2e_1,\quad e_2e_3=-e_3e_2
\end{equation}

この符号反転の法則から添え字の大きい方から開始する基底ベクトルの積は、小さい方から開始される基底ベクトルの積に入れ替えが可能です。
小さい添え字から開始される積により、新たな基底ベクトルが定義されます。
例えば、基底ベクトルの数が3個あるとすると、以下のような構成になります。

\begin{equation}
  e_1,\quad e_2,\quad e_3,\quad e_1e_2,\quad e_1e_3,\quad e_2e_3,\quad e_1e_2e_3
\end{equation}

実係数を加えると、

\begin{equation}
  a_0+a_1e_1+a_2e_2+a_3e_3+a_4e_1e_2+a_5e_1e_3+a_6e_2e_3+a_7e_1e_2e_3,\quad a_0,..., a_7\in\mathbb{R}
\end{equation}

という8次元の数になります。2つの基底ベクトルの積の事を特別に双ベクトル(bivector)と呼び、
3つの基底ベクトルの積は、3-ベクトル(trivector)と呼びます。

\begin{itemize}
  \item 双ベクトル(bivector): $e_1e_2,\quad e_1e_3,\quad e_2e_3$
  \item 3-ベクトル(trivector): $e_1e_2e_3$
\end{itemize}
これらに対して単一構成の基底ベクトルを、1-ベクトルと言う事があります。

組み合わせて使用する1-ベクトルの内、二乗して$1$になる1-ベクトルの数を$m$個、
二乗して$-1$になる1-ベクトルの数を$n$個とした時のクリフォード代数を$Cl(m,n)$と表現します。
$Cl(3,0)$では通常の3次元空間を扱うことができます。
$Cl(0,0)$は実数$\mathbb{R}$と同型で、$Cl(0,1)$は複素数$\mathbb{C}$と同型です。
四元数は$Cl(0,2)$により構成可能ですが、八元数は$Cl(0,3)$と同型にはなりません。
クリフォード代数が結合的である一方、八元数は非結合だからです。

\begin{itemize}
  \item クリフォード代数は結合的で括弧の付け方で結果が変わらない。$(e_ie_j)e_k = e_i(e_je_k)$
  \item 八元数は非結合で括弧の付け方で結果が変わる。$(e_ie_j)e_k \neq e_i(e_je_k)$
\end{itemize}

\section{クリフォード代数による時空表現}

クリフォード代数で特殊相対性理論の時空を表現する場合、$Cl(3,1)$もしくは、$Cl(1,3)$が適合すると考えられます。
物理学的には$(+,-,-,-)$が習慣的に良く使用されるので、$Cl(1,3)$を選択します。

\begin{itemize}
  \item 1-ベクトル:\quad $e_0,\quad e_1,\quad e_2,\quad e_3$
\end{itemize}

\begin{equation}
  e_0^2=+1(時間的)\quad e_1^2=e_2^2=e_3^2=-1(空間的)\quad e_ie_j=-e_je_i\quad (i\neq j)
\end{equation}

\begin{itemize}
  \item 双ベクトル:\quad $e_0e_1,\quad e_0e_2,\quad e_0e_3,\quad e_1e_2,\quad e_1e_3,\quad e_2e_3$
  \item 3-ベクトル:\quad $e_0e_1e_2,\quad e_0e_1e_3,\quad e_0e_2e_3,\quad e_1e_2e_3$
  \item 4-ベクトル:\quad $e_0e_1e_2e_3$
\end{itemize}

これらの要素でクリフォード代数の16次元を構成します。
\begin{equation}
  \begin{split}
    \texttt{Cl(1,3)}\coloneq &r_0+r_1e_0+r_2e_1+r_3e_2+r_4e_3\\
    &+r_5e_0e_1+r_6e_0e_2+r_7e_0e_3+r_8e_1e_2+r_9e_1e_3+r_{10}e_2e_3\\
    &+r_{11}e_0e_1e_2+r_{12}e_0e_1e_3+r_{13}e_0e_2e_3+r_{14}e_1e_2e_3\\
    &+r_{15}e_0e_1e_2e_3\\
    &\quad r_0,...,r_{15}\in\mathbb{R}
  \end{split}
\end{equation}

時間$t$と空間$x,y,z$はこの一部を使い、
\begin{equation}
  te_0+xe_1+ye_2+ze_3
\end{equation}
と表されます。

その他の係数は主に回転を算出するための要素となります。詳しく見ていきます。

\section{クリフォード代数の積}

基底ベクトルの積は、添え字を小さい順になるように入れ替えて符号を変えるという操作と、
二乗して$1$か$-1$にするという操作を組み合わせて行います。幾何積(geometric product)とも呼びます。
例えば、$e_0e_2$と$e_0e_1e_3$の積は、
\begin{equation}
  e_0e_2e_0e_1e_3=e_0(e_2e_0)e_1e_3=-e_0(e_0e_2)e_1e_3=-(e_0e_0)e_2e_1e_3=e_2e_1e_3=-e_1e_2e_3
\end{equation}
となります。この規則により$Cl(1,3)$の乗積表を作ると以下の通りです。
プログラムで作成しています。基底ベクトルをビットパターンに当てはめる事で、比較的単純な規則で算出可能です。
\begin{itemize}
  \item $e_0$を2進数で[0001]、$e_1$を[0010]とすると、$e_0e_1$が[0011]、$e_1e_2e_3$は[1110]
\end{itemize}

\begin{table}[ht]
  \begin{center}
    \caption{Cl(1,3)の乗積表1}
    \small
    \begin{tabular}{r|rrrrrrrrrrrrrrr}
      &$e_0$&$e_1$&$e_2$&$e_3$&$e_0e_1$&$e_0e_2$&$e_0e_3$\\\hline
      $e_0$&$+1$&$+e_0e_1$&$+e_0e_2$&$+e_0e_3$&$+e_1$&$+e_2$&$+e_3$\\
      $e_1$&$-e_0e_1$&$-1$&$+e_1e_2$&$+e_1e_3$&$+e_0$&$-e_0e_1e_2$&$-e_0e_1e_3$\\
      $e_2$&$-e_0e_2$&$-e_1e_2$&$-1$&$+e_2e_3$&$+e_0e_1e_2$&$+e_0$&$-e_0e_2e_3$\\
      $e_3$&$-e_0e_3$&$-e_1e_3$&$-e_2e_3$&$-1$&$+e_0e_1e_3$&$+e_0e_2e_3$&$+e_0$\\
      $e_0e_1$&$-e_1$&$-e_0$&$+e_0e_1e_2$&$+e_0e_1e_3$&$+1$&$-e_1e_2$&$-e_1e_3$\\
      $e_0e_2$&$-e_2$&$-e_0e_1e_2$&$-e_0$&$+e_0e_2e_3$&$+e_1e_2$&$+1$&$-e_2e_3$\\
      $e_0e_3$&$-e_3$&$-e_0e_1e_3$&$-e_0e_2e_3$&$-e_0$&$+e_1e_3$&$+e_2e_3$&$+1$\\
      $e_1e_2$&$+e_0e_1e_2$&$+e_2$&$-e_1$&$+e_1e_2e_3$&$+e_0e_2$&$-e_0e_1$&$+e_0e_1e_2e_3$\\
      $e_1e_3$&$+e_0e_1e_3$&$+e_3$&$-e_1e_2e_3$&$-e_1$&$+e_0e_3$&$-e_0e_1e_2e_3$&$-e_0e_1$\\
      $e_2e_3$&$+e_0e_2e_3$&$+e_1e_2e_3$&$+e_3$&$-e_2$&$+e_0e_1e_2e_3$&$+e_0e_3$&$-e_0e_2$\\
      $e_0e_1e_2$&$+e_1e_2$&$+e_0e_2$&$-e_0e_1$&$+e_0e_1e_2e_3$&$+e_2$&$-e_1$&$+e_1e_2e_3$\\
      $e_0e_1e_3$&$+e_1e_3$&$+e_0e_3$&$-e_0e_1e_2e_3$&$-e_0e_1$&$+e_3$&$-e_1e_2e_3$&$-e_1$\\
      $e_0e_2e_3$&$+e_2e_3$&$+e_0e_1e_2e_3$&$+e_0e_3$&$-e_0e_2$&$+e_1e_2e_3$&$+e_3$&$-e_2$\\
      $e_1e_2e_3$&$-e_0e_1e_2e_3$&$-e_2e_3$&$+e_1e_3$&$-e_1e_2$&$+e_0e_2e_3$&$-e_0e_1e_3$&$+e_0e_1e_2$\\
      $e_0e_1e_2e_3$&$-e_1e_2e_3$&$-e_0e_2e_3$&$+e_0e_1e_3$&$-e_0e_1e_2$&$+e_2e_3$&$-e_1e_3$&$+e_1e_2$\\
    \end{tabular}
  \end{center}
\end{table}

\begin{table}[ht]
  \begin{center}
    \caption{Cl(1,3)の乗積表2}
    \small
    \begin{tabular}{r|rrrrrrrrrrrrrrr}
      &$e_1e_2$&$e_1e_3$&$e_2e_3$&$e_0e_1e_2$&$e_0e_1e_3$&$e_0e_2e_3$&$e_1e_2e_3$&$e_0e_1e_2e_3$\\\hline
      $e_0$&$+e_0e_1e_2$&$+e_0e_1e_3$&$+e_0e_2e_3$&$+e_1e_2$&$+e_1e_3$&$+e_2e_3$&$+e_0e_1e_2e_3$&$+e_1e_2e_3$\\
      $e_1$&$-e_2$&$-e_3$&$+e_1e_2e_3$&$+e_0e_2$&$+e_0e_3$&$-e_0e_1e_2e_3$&$-e_2e_3$&$+e_0e_2e_3$\\
      $e_2$&$+e_1$&$-e_1e_2e_3$&$-e_3$&$-e_0e_1$&$+e_0e_1e_2e_3$&$+e_0e_3$&$+e_1e_3$&$-e_0e_1e_3$\\
      $e_3$&$+e_1e_2e_3$&$+e_1$&$+e_2$&$-e_0e_1e_2e_3$&$-e_0e_1$&$-e_0e_2$&$-e_1e_2$&$+e_0e_1e_2$\\
      $e_0e_1$&$-e_0e_2$&$-e_0e_3$&$+e_0e_1e_2e_3$&$+e_2$&$+e_3$&$-e_1e_2e_3$&$-e_0e_2e_3$&$+e_2e_3$\\
      $e_0e_2$&$+e_0e_1$&$-e_0e_1e_2e_3$&$-e_0e_3$&$-e_1$&$+e_1e_2e_3$&$+e_3$&$+e_0e_1e_3$&$-e_1e_3$\\
      $e_0e_3$&$+e_0e_1e_2e_3$&$+e_0e_1$&$+e_0e_2$&$-e_1e_2e_3$&$-e_1$&$-e_2$&$-e_0e_1e_2$&$+e_1e_2$\\
      $e_1e_2$&$-1$&$+e_2e_3$&$-e_1e_3$&$-e_0$&$+e_0e_2e_3$&$-e_0e_1e_3$&$-e_3$&$-e_0e_3$\\
      $e_1e_3$&$-e_2e_3$&$-1$&$+e_1e_2$&$-e_0e_2e_3$&$-e_0$&$+e_0e_1e_2$&$+e_2$&$+e_0e_2$\\
      $e_2e_3$&$+e_1e_3$&$-e_1e_2$&$-1$&$+e_0e_1e_3$&$-e_0e_1e_2$&$-e_0$&$-e_1$&$-e_0e_1$\\
      $e_0e_1e_2$&$-e_0$&$+e_0e_2e_3$&$-e_0e_1e_3$&$-1$&$+e_2e_3$&$-e_1e_3$&$-e_0e_3$&$-e_3$\\
      $e_0e_1e_3$&$-e_0e_2e_3$&$-e_0$&$+e_0e_1e_2$&$-e_2e_3$&$-1$&$+e_1e_2$&$+e_0e_2$&$+e_2$\\
      $e_0e_2e_3$&$+e_0e_1e_3$&$-e_0e_1e_2$&$-e_0$&$+e_1e_3$&$-e_1e_2$&$-1$&$-e_0e_1$&$-e_1$\\
      $e_1e_2e_3$&$-e_3$&$+e_2$&$-e_1$&$+e_0e_3$&$-e_0e_2$&$+e_0e_1$&$+1$&$-e_0$\\
      $e_0e_1e_2e_3$&$-e_0e_3$&$+e_0e_2$&$-e_0e_1$&$+e_3$&$-e_2$&$+e_1$&$+e_0$&$-1$\\
    \end{tabular}
  \end{center}
\end{table}

この積の規則により、時空ベクトルについては二乗が不変量になります。
\begin{equation}
  \begin{split}
    (te_0+xe_1+ye_2+ze_3)^2=~&t^2e_0e_0+txe_0e_1+tye_0e_2+tze_0e_3\\
    &+txe_1e_0+x^2e_1e_1+xye_1e_2+xze_1e_3\\
    &+tye_2e_0+xye_2e_1+y^2e_2e_2+yze_2e_3\\
    &+tze_3e_0+xze_3e_1+yze_3e_2+z^2e_3e_3\\
    =~&t^2+txe_0e_1+tye_0e_2+tze_0e_3\\
    &-txe_0e_1-x^2+xye_1e_2+xze_1e_3\\
    &-tye_0e_2-xye_1e_2-y^2+yze_2e_3\\
    &-tze_0e_3-xze_1e_3-yze_2e_3-z^2\\
    =~&t^2-x^2-y^2-z^2
  \end{split}
\end{equation}

双ベクトルや3-ベクトルなどのベクトルの階層の事を、ブレード(Blade)という言葉で呼ぶことがあります。
また、末尾の基底ベクトルの積については、擬スカラーと呼び、
末尾から2番目の基底ベクトルの積については、擬ベクトルと呼ぶこともあります。
\begin{itemize}
  \item 0-ブレード:スカラー
  \item 1-ブレード:1-ベクトル
  \item 2-ブレード:双ベクトル(bivector)
  \item 3-ブレード:3-ベクトル、擬ベクトル(trivector、pseudo-vector)
  \item 4-ブレード:4-ベクトル、擬スカラー(4-vector、pseudo-scalar)
\end{itemize}
さらにそれぞれを階層に見立ててグレードと呼んだりもします。
偶数グレードや奇数グレードとまとめて表現する事が出来るようになります。

\section{クリフォード代数の内積と外積}

クリフォード代数の内積は通常、幾何積のスカラー部分を取り出す形で定義されます。外積はそれ以外の部分とみなせます。
\begin{equation}
  ab=a\cdot b+a\wedge b
\end{equation}
正確な定義としては幾何積の可換部分が内積、反可換部分が外積になります。
反可換部分とは順序を逆にすると符号が反転するという意味になります。
\begin{equation}
  a\cdot b=\frac{1}{2}(ab+ba)
\end{equation}
\begin{equation}
  a\wedge b=\frac{1}{2}(ab-ba)
\end{equation}

交換子$[a,b]=ab-ba$、反交換子$\{a,b\}=ab+ba$の記法を使うと、
\begin{equation}
  2a\cdot b=\{a,b\},\quad 2a\wedge b=[a,b]
\end{equation}
と表記する事もできます。

\section{クリフォード代数の共役}

一般的には、クリフォード代数の共役には3種類あると考えられます。
\begin{table}[ht]
  \centering
  \caption{クリフォード代数の共役}
  \begin{tabular}{|l|l|l|} \hline
    名前 & 表記 & 操作 \\ \hline
    リバース(Reverse) & $\widetilde{A}$ & 基底ベクトルの順序を逆にする \\ \hline
    グレード対合(Grade Involution) & $\overline{A}$ & 奇数グレードの符号反転 \\ \hline
    クリフォード共役(Clifford Conjugate) & $A^*$ & 他の共役の掛け合わせ \\ \hline
  \end{tabular}
\end{table}

通常、クリフォード代数の共役にはリバースが使用されます。
クリフォード代数においてリバース演算は、要素の基底ベクトルの順序を逆にしたものです。例えば、
\begin{itemize}
  \item スカラー: $\widetilde{a}=a$(変化なし)
  \item 1-ベクトル: $\widetilde{e_0}=e_0$(変化なし)
  \item 双ベクトル: $\widetilde{e_0e_1}=-e_1e_0$($-1$倍)
  \item 3-ベクトル: $\widetilde{e_0e_1e_2}=e_2e_1e_0=-e_1e_2e_0=e_1e_0e_2=-e_0e_1e_2$($-1$倍)
  \item 4-ベクトル: $\widetilde{e_0e_1e_2e_3}=e_3e_2e_1e_0=e_0e_1e_2e_3$(変化なし)
\end{itemize}

各共役の符号についてまとめると以下の表になります。
\begin{table}[ht]
  \centering
  \caption{グレード$k$の符号の一般式}
  \begin{tabular}{|l|c|c|c|c|l|} \hline
    k mod 4 & 0 & 1 & 2 & 3 & \\ \hline
    $\widetilde{A}$ & $+$ & $+$ & $-$ & $-$ & $(-1)^{k(k-1)/2}$ \\ \hline
    $\overline{A}$ & $+$ & $-$ & $+$ & $-$ & $(-1)^k$ \\ \hline
    $A^*$ & $+$ & $-$ & $-$ & $+$ & $(-1)^{k(k+1)/2}$ \\ \hline
  \end{tabular}
\end{table}

\section{クリフォード代数の回転子}

クリフォード代数の回転子(Rotor)は偶数グレードから形成され、以下の条件を満たします。
\begin{equation}
  R\widetilde{R}=\widetilde{R}R=1
\end{equation}
ここで$\widetilde{R}$は$R$のリバース(Reverse)です。

この回転子を以下の様に対象$X$に作用させます。
\begin{equation}
  R X \widetilde{R} = X'
\end{equation}
この様な挟み込みの掛け合わせで対象を変換することを、サンドイッチ積(Sandwich Product)と呼びます。
通常、サンドイッチ積による操作は対象の示す座標を回転させる事になります。

\subsection{空間的双ベクトルの回転子}

空間的双ベクトル$e_1e_2,e_1e_3,e_2e_3$を使って回転子(Rotor)を定義します。
二乗について確認します。
\begin{equation}
  (e_1e_2)^2 = (e_1e_3)^2 = (e_2e_3)^2 = -1
\end{equation}
よってベルソル$\bm{v}$が定義できます。回転角を$2\theta$とすると、
\begin{equation}
  \bm{v} = v_1 e_1e_2 + v_2 e_1e_3 + v_3 e_2e_3 , \quad
  \bm{v}^2 = - 1 , \quad v_1 , v_2 , v_3 \in \mathbb{R}
\end{equation}
\begin{equation}
  B_x = \theta (v_1 e_1e_2 + v_2 e_1e_3 + v_3 e_2e_3) , \quad \theta \in \mathbb{R}
\end{equation}
と定義して、回転子は以下の通りです。
\begin{equation}
  \texttt{Rotor} : R_x = \exp(B_x) = \cos \theta + \bm{v} \sin \theta
\end{equation}
$\bm{v}^2$を展開すると、
\begin{equation}
  \begin{split}
    \bm{v}^2 = ~ & (v_1 e_1e_2 + v_2 e_1e_3 + v_3 e_2e_3)^2\\
    = ~ & v_1^2 e_1e_2e_1e_2 + v_1 v_2 e_1e_2e_1e_3 + v_1 v_3 e_1e_2e_2e_3\\
    & + v_1 v_2 e_1e_3e_1e_2 + v_2^2 e_1e_3e_1e_3 + v_2 v_3 e_1e_3e_2e_3\\
    & + v_1 v_3 e_2e_3e_1e_2 + v_2 v_3 e_2e_3e_1e_3 + v_3^2 e_2e_3e_2e_3\\
    = ~ & - v_1^2 + v_1 v_2 e_2e_3 - v_1 v_3 e_1e_3
    - v_1 v_2 e_2e_3 - v_2^2 + v_2 v_3 e_1e_2
    + v_1 v_3 e_1e_3 - v_2 v_3 e_1e_2 - v_3^2\\
    = ~ & - v_1^2 - v_2^2 - v_3^2\\
  \end{split}
\end{equation}

\subsection{時間的双ベクトルの回転子}

次に、双ベクトルの内、時間的双ベクトル$e_0e_1,e_0e_2,e_0e_3$を使って回転子(Rotor)を定義します。
二乗について確認します。
\begin{equation}
  (e_0e_1)^2 = (e_0e_2)^2 = (e_0e_3)^2 = 1
\end{equation}
よって双曲的ベルソル$\bm{u}$が定義できます。回転角を$2\phi$とすると、
\begin{equation}
  \bm{u} = u_1 e_0e_1 + u_2 e_0e_2 + u_3 e_0e_3 , \quad
  \bm{u}^2 = 1 , \quad u_1 , u_2 , u_3 \in \mathbb{R}
\end{equation}
\begin{equation}
  B_t = \phi (u_1 e_0e_1 + u_2 e_0e_2 + u_3 e_0e_3) , \quad \phi \in \mathbb{R}
\end{equation}
と定義して、回転子は以下の通りです。
\begin{equation}
  \texttt{Rotor} : R_t = \exp(B_t) = \cosh \phi + \bm{u} \sinh \phi
\end{equation}
$\bm{u}^2$を展開すると、
\begin{equation}
  \begin{split}
    \bm{u}^2 = ~ & (u_1 e_0e_1 + u_2 e_0e_2 + u_3 e_0e_3)^2\\
    = ~ & u_1^2 e_0e_1e_0e_1 + u_1 u_2 e_0e_1e_0e_2 + u_1 u_3 e_0e_1e_0e_3\\
    & + u_1 u_2 e_0e_2e_0e_1 + u_2^2 e_0e_2e_0e_2 + u_2 u_3 e_0e_2e_0e_3\\
    & + u_1 u_3 e_0e_3e_0e_1 + u_2 u_3 e_0e_3e_0e_2 + u_3^2 e_0e_3e_0e_3\\
    = ~ & u_1^2 - u_1 u_2 e_1e_2 - u_1 u_3 e_1e_3
     + u_1 u_2 e_1e_2 + u_2^2 - u_2 u_3 e_2e_3
     + u_1 u_3 e_1e_3 + u_2 u_3 e_2e_3 + u_3^2\\
    = ~ & u_1^2 + u_2^2 + u_3^2\\
  \end{split}
\end{equation}

\section{クリフォード代数のローレンツ変換}

$Cl(1,3)$におけるローレンツ変換の計算を説明します。改めて時空については以下の定義です。
\begin{equation}
  \texttt{Spacetime}:\mathbb{M}\coloneq \{te_0+xe_1+ye_2+ze_3,\quad t,x,y,z\in\mathbb{R}\}
\end{equation}
\begin{equation}
  \texttt{Invariant}:m^2=t^2-x^2-y^2-z^2,\quad m\in\mathbb{M}
\end{equation}

1-ベクトルよりなる時空を双ベクトルによるサンドイッチ積で回転変換する事を考えます。
双ベクトルは6種類あります。様子を見るために具体的に一方向だけを計算してみましょう。
$R=a+be_0e_1$とすると$\widetilde{R}=a-be_0e_1$になります。
$Cl(1,3)$では、
\[R\widetilde{R}=(a+be_0e_1)(a-be_0e_1)=a^2-b^2=1\]
を満たすように係数が決定される必要があります。
ご存知の様に、この数式に当てはまる適切な関数は双曲線関数です。
$e_0$が含まれるため、これは時間方向の回転であり、ローレンツブーストを意味しています。
$e_0e_2,e_0e_3$についても同様と考えられます。
双ベクトルの内、$e_0$が含まれる双ベクトルは時間的な双ベクトルと考えられます。

一方、$e_0$が含まれない双ベクトルは空間的な双ベクトルと考えられます。
$R=a+be_1e_2$で計算してみましょう。$\widetilde{R}=a-be_1e_2$になり、
\[R\widetilde{R}=(a+be_0e_1)(a-be_0e_1)=a^2+b^2=1\]
空間的双ベクトルによるサンドイッチ積では、満たすべき係数の条件は単位円上という事になります。
これは空間方向の回転ですので、ローレンツブーストにはなりません。
$e_1e_3,e_2e_3$についても同様と考えられます。

ここまでは双四元数で実現できる計算と変わりません。

\section{四元数の直積との対比}

双四元数は四元数の係数を複素数にしましたが、同じように係数を四元数にしてみます。
四元数としてお互いに干渉しない虚数単位のセットを2つ定義します。
\begin{gather}
  i^2=j^2=k^2=-1\\
  I^2=J^2=K^2=-1\\
  ij=k,~jk=i,~ki=j,~ji=-k,~kj=-i,~ik=-j\\
  IJ=K,~JK=I,~KI=J,~JI=-K,~KJ=-I,~IK=-J\\
  iI=Ii,~jI=Ij,~kI=Ik\\
  iJ=Ji,~jJ=Jj,~kJ=Jk\\
  iK=Ki,~jK=Kj,~kK=Kk
\end{gather}
これらを組み合わせて2つの四元数の直積を作りますが、利便性の為に以後「四双四元数」(クアクォータニオン)と呼称します。
\begin{equation}
  \begin{split}
    \texttt{Quaquaternion}:\mathbb{H}\otimes\mathbb{H}\coloneq ~&a+bi+cj+dk+m+ni+rj+sk\\
    &a,b,c,d\in\mathbb{R}\\
    &m=m_1I+m_2J+m_3K,\quad m_1,m_2,m_3\in\mathbb{R}\\
    &n=n_1I+n_2J+n_3K,\quad n_1,n_2,n_3\in\mathbb{R}\\
    &r=r_1I+r_2J+r_3K,\quad r_1,r_2,r_3\in\mathbb{R}\\
    &s=s_1I+s_2J+s_3K,\quad s_1,s_2,s_3\in\mathbb{R}\\
  \end{split}
\end{equation}
四双四元数はクリフォード代数$ Cl(3,1) $の変形と同型になっています。以下の様に各ブレードに割り当てます。
\begin{itemize}
  \item スカラー: $ a $
  \item 1-ベクトル: $ c j + s k = c j + s_1 kI + s_2 kJ + s_3 kK $
  \item 双ベクトル: $ n i + m = n_1 iI + n_2 iJ + n_3 iK + m_1 I + m_2 J + m_3 K $
  \item 3-ベクトル: $ dk + rj = dk + r_1 jI+r_2 jJ+r_3 jK $
  \item 擬スカラー: $ bi $
\end{itemize}
$ Cl(3,1) $の変形は以下の様に一部の順序を置き換えます。
\begin{itemize}
  \item 双ベクトルの一部: $ e_1 e_2 , e_2 e_3 \rightarrow e_2 e_1 , e_3 e_2 $
  \item 3-ベクトルの一部: $ e_0 e_1 e_2 , e_0 e_2 e_3 \rightarrow e_0 e_2 e_1 , e_0 e_3 e_2 $
\end{itemize}
全体的に順序を整えます。
\begin{itemize}
  \item 1-ベクトル: $ e_0 , e_1 , e_2 , e_3 $
  \item 双ベクトル: $ e_0e_1 , e_0e_2 , e_0e_3 , e_3e_2 , e_1e_3 , e_2e_1 $
  \item 3-ベクトル: $ e_1e_2e_3 , e_0e_3e_2 , e_0e_1e_3 , e_0e_2e_1 $
  \item 擬スカラー: $ e_0e_1e_2e_3 $
\end{itemize}
乗積表を作って確かめてみます。

\clearpage

\begin{table}[ht]
  \begin{center}
    \caption{四双四元数の乗積表}
    \small
    \begin{tabular}{r|rrrrrrrrrrrrrrr}
      &$j$&$kI$&$kJ$&$kK$&$iI$&$iJ$&$iK$&$I$&$J$&$K$&$k$&$jI$&$jJ$&$jK$&$i$ \\ \hline
      j&$-1$&$+iI$&$+iJ$&$+iK$&$-kI$&$-kJ$&$-kK$&$+jI$&$+jJ$&$+jK$&$+i$&$-I$&$-J$&$-K$&$-k$ \\ \hline
      kI&$-iI$&$+1$&$-K$&$+J$&$-j$&$+jK$&$-jJ$&$-k$&$+kK$&$-kJ$&$-I$&$+i$&$-iK$&$+iJ$&$+jI$ \\ \hline
      kJ&$-iJ$&$+K$&$+1$&$-I$&$-jK$&$-j$&$+jI$&$-kK$&$-k$&$+kI$&$-J$&$+iK$&$+i$&$-iI$&$+jJ$ \\ \hline
      kK&$-iK$&$-J$&$+I$&$+1$&$+jJ$&$-jI$&$-j$&$+kJ$&$-kI$&$-k$&$-K$&$-iJ$&$+iI$&$+i$&$+jK$ \\ \hline
      iI&$+kI$&$+j$&$-jK$&$+jJ$&$+1$&$-K$&$+J$&$-i$&$+iK$&$-iJ$&$-jI$&$-k$&$+kK$&$-kJ$&$-I$ \\ \hline
      iJ&$+kJ$&$+jK$&$+j$&$-jI$&$+K$&$+1$&$-I$&$-iK$&$-i$&$+iI$&$-jJ$&$-kK$&$-k$&$+kI$&$-J$ \\ \hline
      iK&$+kK$&$-jJ$&$+jI$&$+j$&$-J$&$+I$&$+1$&$+iJ$&$-iI$&$-i$&$-jK$&$+kJ$&$-kI$&$-k$&$-K$ \\ \hline
      I&$+jI$&$-k$&$+kK$&$-kJ$&$-i$&$+iK$&$-iJ$&$-1$&$+K$&$-J$&$+kI$&$-j$&$+jK$&$-jJ$&$+iI$ \\ \hline
      J&$+jJ$&$-kK$&$-k$&$+kI$&$-iK$&$-i$&$+iI$&$-K$&$-1$&$+I$&$+kJ$&$-jK$&$-j$&$+jI$&$+iJ$ \\ \hline
      K&$+jK$&$+kJ$&$-kI$&$-k$&$+iJ$&$-iI$&$-i$&$+J$&$-I$&$-1$&$+kK$&$+jJ$&$-jI$&$-j$&$+iK$ \\ \hline
      k&$-i$&$-I$&$-J$&$-K$&$+jI$&$+jJ$&$+jK$&$+kI$&$+kJ$&$+kK$&$-1$&$-iI$&$-iJ$&$-iK$&$+j$ \\ \hline
      jI&$-I$&$-i$&$+iK$&$-iJ$&$+k$&$-kK$&$+kJ$&$-j$&$+jK$&$-jJ$&$+iI$&$+1$&$-K$&$+J$&$-kI$ \\ \hline
      jJ&$-J$&$-iK$&$-i$&$+iI$&$+kK$&$+k$&$-kI$&$-jK$&$-j$&$+jI$&$+iJ$&$+K$&$+1$&$-I$&$-kJ$ \\ \hline
      jK&$-K$&$+iJ$&$-iI$&$-i$&$-kJ$&$+kI$&$+k$&$+jJ$&$-jI$&$-j$&$+iK$&$-J$&$+I$&$+1$&$-kK$ \\ \hline
      i&$+k$&$-jI$&$-jJ$&$-jK$&$-I$&$-J$&$-K$&$+iI$&$+iJ$&$+iK$&$-j$&$+kI$&$+kJ$&$+kK$&$-1$ \\ \hline
    \end{tabular}
  \end{center}
\end{table}
\begin{table}[ht]
  \begin{center}
    \caption{Cl(3,1)変化形の乗積表(e表記を省略)}
    \scriptsize
    \begin{tabular}{r|rrrrrrrrrrrrrrr}
      &$0$&$1$&$2$&$3$&$01$&$02$&$03$&$32$&$13$&$21$&$123$&$032$&$013$&$021$&$0123$\\\hline
      $0$&$-$&$+01$&$+02$&$+03$&$-1$&$-2$&$-3$&$+032$&$+013$&$+021$&$+0123$&$-32$&$-13$&$-21$&$-123$\\
      $1$&$-01$&$+$&$-21$&$+13$&$-0$&$+021$&$-013$&$-123$&$+3$&$-2$&$-32$&$+0123$&$-03$&$+02$&$+032$\\
      $2$&$-02$&$+21$&$+$&$-32$&$-021$&$-0$&$+032$&$-3$&$-123$&$+1$&$-13$&$+03$&$+0123$&$-01$&$+013$\\
      $3$&$-03$&$-13$&$+32$&$+$&$+013$&$-032$&$-0$&$+2$&$-1$&$-123$&$-21$&$-02$&$+01$&$+0123$&$+021$\\
      $01$&$+1$&$+0$&$-021$&$+013$&$+$&$-21$&$+13$&$-0123$&$+03$&$-02$&$-032$&$-123$&$+3$&$-2$&$-32$\\
      $02$&$+2$&$+021$&$+0$&$-032$&$+21$&$+$&$-32$&$-03$&$-0123$&$+01$&$-013$&$-3$&$-123$&$+1$&$-13$\\
      $03$&$+3$&$-013$&$+032$&$+0$&$-13$&$+32$&$+$&$+02$&$-01$&$-0123$&$-021$&$+2$&$-1$&$-123$&$-21$\\
      $32$&$+032$&$-123$&$+3$&$-2$&$-0123$&$+03$&$-02$&$-$&$+21$&$-13$&$+1$&$-0$&$+021$&$-013$&$+01$\\
      $13$&$+013$&$-3$&$-123$&$+1$&$-03$&$-0123$&$+01$&$-21$&$-$&$+32$&$+2$&$-021$&$-0$&$+032$&$+02$\\
      $21$&$+021$&$+2$&$-1$&$-123$&$+02$&$-01$&$-0123$&$+13$&$-32$&$-$&$+3$&$+013$&$-032$&$-0$&$+03$\\
      $123$&$-0123$&$-32$&$-13$&$-21$&$+032$&$+013$&$+021$&$+1$&$+2$&$+3$&$-$&$-01$&$-02$&$-03$&$+0$\\
      $032$&$-32$&$-0123$&$+03$&$-02$&$+123$&$-3$&$+2$&$-0$&$+021$&$-013$&$+01$&$+$&$-21$&$+13$&$-1$\\
      $013$&$-13$&$-03$&$-0123$&$+01$&$+3$&$+123$&$-1$&$-021$&$-0$&$+032$&$+02$&$+21$&$+$&$-32$&$-2$\\
      $021$&$-21$&$+02$&$-01$&$-0123$&$-2$&$+1$&$+123$&$+013$&$-032$&$-0$&$+03$&$-13$&$+32$&$+$&$-3$\\
      $0123$&$+123$&$-032$&$-013$&$-021$&$-32$&$-13$&$-21$&$+01$&$+02$&$+03$&$-0$&$+1$&$+2$&$+3$&$-$\\
    \end{tabular}
  \end{center}
\end{table}
これらは一致しています。

\section{四双四元数の内積と外積}

さて、四元数の積には以下のように内積と外積が組み込まれています。
\begin{equation}
  \begin{split}
    a=~&a_0+a_1i+a_2j+a_3k=a_0+u\\
    b=~&b_0+b_1i+b_2j+b_3k=b_0+v\\
  \end{split}
\end{equation}
と置いて、
\begin{equation}
  \begin{split}
    ab=~&a_0b_0-a_1b_1-a_2b_2-a_3b_3\\
    &+(a_0b_1+a_1b_0+a_2b_3-a_3b_2)i\\
    &+(a_0b_2+a_2b_0+a_3b_1-a_1b_3)j\\
    &+(a_0b_3+a_3b_0+a_1b_2-a_2b_1)k\\
    = ~ & a_0 b_0 - u \cdot v + a_0 v + b_0 u + u \times v
  \end{split}
\end{equation}
スカラー部分を取り除くと、
\begin{equation}
  uv=-u\cdot v+u\times v
\end{equation}

1-ベクトルの内積と外積を考えます。1-ベクトルを実数係数で具体化すると以下のようになっています。
\begin{equation}
  \begin{split}
    p = ~ & p_0 e_0 + p_1 e_1 + p_2 e_2 + p_3 e_3 \\
    q = ~ & q_0 e_0 + q_1 e_1 + q_2 e_2 + q_3 e_3 , \quad
    p_0 , ... , q_3 , q_0 , ... , q_3 \in \mathbb{R}
  \end{split}
\end{equation}
内積と外積の極めて一般的な定義としてはウェッジ積を用いて以下の定義になります。
\begin{equation}
  pq=\lambda p\cdot q+\mu p\wedge q,\quad \lambda,\mu\in\mathbb{R}
\end{equation}
ここで$\lambda=\mu=-1$を選択すると、
\begin{equation}
  \begin{split}
    pq=~&-p\cdot q-p\wedge q\\
    qp=~&-q\cdot p-q\wedge p=-p\cdot q+p\wedge q
  \end{split}
\end{equation}
\begin{equation}
  \begin{split}
    2p\cdot q=~&-(pq+qp)\\
    2p\wedge q=~&-(pq-qp)\\
    p \cdot q = ~ & - \frac{1}{2} (p q + q p) \\
    p \wedge q = ~ & - \frac{1}{2} (p q - q p)
  \end{split}
\end{equation}
一般的に、1-ベクトル$ p $とnグレードのマルチベクトル \footnote{複数回ウェッジ積を重ねた結果の事で、双ベクトルや3-ベクトルが構成される}
$ A_n = p_1 \wedge p_2 ... \wedge p_n $の
内積と外積は以下のようになります。
\begin{equation}
  \begin{split}
    2p \cdot A_n = ~ & (-1)^n (p A_n - (-1)^n A_n p) \\
    2p \wedge A_n = ~ & (-1)^n (p A_n + (-1)^n A_n p) \\
  \end{split}
\end{equation}
特に双ベクトル$ B $の場合は、$ n = 2 $なので、$ 2p \cdot B = p B - B p $となります。
具体的に、$ p = p_0 j + p_1 kI, B = b_0 iI + b_1 iJ $の内積について計算すると、
\begin{equation}
  \begin{split}
    pB = ~ & (p_0 j + p_1 kI)(b_0 iI + b_1 iJ) = - p_0 b_0 kI - p_0 b_1 kJ - p_1 b_0 j + p_1 b_1 jK \\
    Bp = ~ & (b_0 iI + b_1 iJ)(p_0 j + p_1 kI) = p_0 b_0 kI + p_0 b_1 kJ + p_1 b_0 j + p_1 b_1 jK \\
    2 p \cdot B = ~ & p B - B p = - 2 p_0 b_0 kI - 2 p_0 b_1 kJ - 2 p_1 b_0 j \\
    p \cdot B = ~ & - p_1 b_0 j - p_0 b_0 kI - p_0 b_1 kJ \\
  \end{split}
\end{equation}
または、
\begin{equation}
  \begin{split}
    pB = ~ & (p_0 e_0 + p_1 e_1)(b_0 e_0e_1 + b_1 e_0e_2) = - p_0 b_0 e_1 - p_0 b_1 e_2 - p_1 b_0 e_0 + p_1 b_1 e_0e_2e_1 \\
    Bp = ~ & (b_0 e_0e_1 + b_1 e_0e_2)(p_0 e_0 + p_1 e_1) = p_0 b_0 e_1 + p_0 b_1 e_2 + p_1 b_0 e_0 + p_1 b_1 e_0e_2e_1 \\
    p \cdot B = ~ & - p_1 b_0 e_0 - p_0 b_0 e_1 - p_0 b_1 e_2 \\
  \end{split}
\end{equation}

\section{四双四元数による擬ユークリッド空間}

1-ベクトルの実数係数を以下のように定めて内積を計算すると、
\begin{gather}
  s = t j + x kI + y kJ + z kK, \quad t, x, y, z \in \mathbb{R} \\
  s \cdot s = - ss = t^2 - x^2 - y^2 - z^2
\end{gather}
内積は相対論的不変量になります。ローレンツ変換は以下のように定義されます。
\begin{gather}
  X'= R X R^*
\end{gather}
ここで$ R $は$ R R^* = 1 $を満たす偶数グレード(通常、双ベクトル)です。
同様に任意の3-ベクトル$ A $に対して、
\begin{gather}
  A = a_0 k + a_1 jI + a_2 jJ + a_3 jK
\end{gather}
\begin{gather}
  A' = R A R^*
\end{gather}
が成り立ち、相対論的不変量$ A \cdot A $が得られます。
ラピディティ$ \theta $が、$ \tanh \theta = v/c $で与えられ、方向を表す単位ベクトル(ベルソル)が、
\begin{gather}
  d=d_1I+d_2J+d_3K,\quad d_1^2+d_2^2+d_3^2=1,\quad d_1,d_2,d_3\in\mathbb{R}
\end{gather}
で与えられる時、
\begin{equation}
  \begin{split}
    R = ~ & \exp \left( \frac{1}{2} \theta d i \right) \\
    = ~ & \cosh \frac{\theta}{2} + d i \sinh \frac{\theta}{2} \\
    = ~ & \cosh \frac{\theta}{2} + (d_1 iI + d_2 iJ + d_3 iK) \sinh \frac{\theta}{2}
  \end{split}
\end{equation}
として$ R $が定まります。

\documentclass[a4paper,12pt,notitlepage]{jsreport}
\usepackage[left=10truemm,right=10truemm,top=25truemm,bottom=20truemm]{geometry}
\usepackage{mathtools}
\usepackage{amsmath}
\usepackage{amsfonts}
\usepackage{bm}
\usepackage{setspace}
\usepackage{wrapfig}
\usepackage[dvipdfmx]{hyperref}
\usepackage{pxjahyper}
\usepackage{docmute}
\DeclareMathOperator\arctanh{arctanh}
\DeclareMathOperator\arccosh{arccosh}

\begin{document}

\chapter{クリフォード代数と一般相対性理論}

\section{斜交座標と基底ベクトル}

曲がった空間を表現するためにはこれまで使用してきた直交座標は適用できず斜交座標を使用します。
まず二次元で考えます。
斜交座標ではベクトルの成分は共変と反変の2種類あります。
二次元ベクトル$A$の共変成分$A_1,A_2$と、反変成分$A^1,A^2$があるとします。
\footnote{アインシュタインの縮約記法により、共変成分を下添え字、反変成分を上添え字で表記します。}
反変成分を定義する基底ベクトル$e_1,e_2$と共変成分を定義する双対基底ベクトル$e^1,e^2$については以下のような関係があります。
\footnote{基底ベクトルについては添え字の上下が入れ替わります。}
\begin{equation}
  \begin{split}
    &A=A^1e_1+A^2e_2=A_1e^1+A_2e^2\\
    &e_1\cdot e^2=e_2\cdot e^1=0,\quad e_1\cdot e^1=e_2\cdot e^2=1
  \end{split}
\end{equation}
$e_1,e_2,e^1,e^2$は座標の基準となるベクトルですが長さは$1$とは限らないです。
長さが$1$になる場合は、$e_1,e^1$の成す角度が$0$の場合で、そうなると$e_2,e^2$の成す角度も$0$になり、
$e_1=e^1,e_2=e^2$になり、長さが一斉に$1$にそろい直交座標となる場合です。

次に$A$ベクトルの微小変位$dA$の長さを考えると以下のようになります。
\begin{equation}
  N(dA)=dA\cdot dA=dA_1dA^1+dA_2dA^2=dA^1dA_1+dA^2dA_2
\end{equation}
そもそも長さをこのように定義したいがために基底ベクトルを選んでいます。
アインシュタインの縮約記法で表記すると、
\begin{equation}
  N(dA)=dA_idA^i=dA^idA_i
\end{equation}

ここで共変成分だけを使って長さを表すようにした時、
\begin{equation}
  N(dA)=
  \begin{pmatrix}
    dA_1&dA_2
  \end{pmatrix}
  \begin{pmatrix}
    g_{11}&g_{12}\\g_{21}&g_{22}
  \end{pmatrix}
  \begin{pmatrix}
    dA_1\\dA_2
  \end{pmatrix}
\end{equation}
となるように、
\begin{equation}
  g =
  \begin{pmatrix}
    g_{11}&g_{12}\\g_{21}&g_{22}
  \end{pmatrix}
\end{equation}
を定めると$ g $が空間の特徴を表すようになります。
$ g $を計量テンソルと呼びます。通常の平面であれば、
\begin{equation}
  g =
  \begin{pmatrix}
    1&0\\0&1
  \end{pmatrix}
\end{equation}
となります。$g$は一般的に対称テンソルです。
四次元ベクトルを持ってきて、以下のように$g$を定めれば、
\begin{equation}
  g =
  \begin{pmatrix}
    1&0&0&0\\
    0&-1&0&0\\
    0&0&-1&0\\
    0&0&0&-1\\
  \end{pmatrix}
\end{equation}
特殊相対論的時空の不変量が表現できます。
特殊相対論の時間と空間を論じる時に斜交座標で説明される事もありますが、リーマン幾何学の斜交座標とは異なるので注意しましょう。
特殊相対論では計量$ g $は変化しませんが、リーマン幾何学では$ g $が変化していきます。

斜交座標を持ってきても物差しが変わっただけで空間は曲がっていない事に注意してください。
$ g $を変化させると内積の計算規則が変更されていきます。
$ g $は場所によって変化していくので、位置の関数として表現されます。
空間が曲がるとは、場所によって計量$ g(x) $が変化していく事によって検出されます。
位置の関数となる計量$ g(x) $が、平行移動したときに計量の様態がどのように変化するかが空間の歪曲となります。
ある位置の$ g(x) $が標準と異なっていたとしても、平行移動して$ g(x) $に変化がなければ空間は曲がっていません。

逆に言うと、空間の歪曲を算出するにあたって$ g(x) $の関数の中身を知る必要はありません。
微小移動量$dx$で、$ g(x) $と$ g(x + dx) $の違いが分かればそれで歪曲の度合いが算出できます。

\subsection{クリフォード代数で計量が異なるという事の意味}

クリフォード代数での計量は幾何積のルールとして現れ、平坦な時空でCl(1,3)を選択すれば、計量は、
\begin{equation}
  \eta = [1,-1,-1,-1]
\end{equation}
となり、内積は、
\begin{equation}
  e_0 \cdot e_0 = 1, e_1 \cdot e_1 = e_2 \cdot e_2 = e_3 \cdot e_3 = -1
\end{equation}
\begin{equation}
  e_i \cdot e_i = \eta_i , \quad i=0,1,2,3
\end{equation}
となります。これらが新しい基底$ \nu_i $で表現し直されるとすると、係数を$ \omega_{ij} $として、
\begin{equation}
  \nu_i = \omega_{ij} e_j , \quad \omega_{ij} \in \mathbb{R} , \quad i,j=0,1,2,3
\end{equation}
\begin{equation}
  \nu_i \cdot \nu_i = (\omega_{ij})^2 e_j \cdot e_j = (\omega_{ij})^2 \eta_j
\end{equation}
$ \omega_{ij} $は16個の係数です。つまり、各$ \nu_i $について全ての基本基底のスカラー合成の範囲内にあるとします。
新しい基底$ \nu_i $の新しい計量は、
\begin{equation} g_i = (\omega_{ij})^2 \eta_j \end{equation}
になりました。これでとりあえず入れ物が用意されました。
次に制約を考えていきます。これらは不変量を保つという制約があります。
\begin{equation}
  S^2 = g_0 t^2 + g_1 x^2 + g_2 y^2 + g_3 z^2 = \eta_0 t^2 + \eta_1 x^2 + \eta_2 y^2 + \eta_3 z^2 = t^2 - x^2 - y^2 - z^2
\end{equation}
この制約をクリフォード代数で表現すると回転子による回転に帰着します。
四次元時空では、6種類の双ベクトルにより構成されます。
\begin{equation}
  R = \exp(\gamma_0 e_0e_1 + \gamma_1 e_0e_2 + \gamma_2 e_0e_3 + \gamma_3 e_1e_2 + \gamma_4 e_1e_3 + \gamma_5 e_2e_3)
\end{equation}
四双四元数では、
\begin{equation}
  R = \exp(\omega_0 iI + \omega_1 iJ + \omega_2 iK + \omega_3 I + \omega_4 J + \omega_5 K)
\end{equation}
変形Cl(3,1)では、
\begin{equation}
  R = \exp(\omega_0 \epsilon_0\epsilon_1 + \omega_1 \epsilon_0\epsilon_2 + \omega_2 \epsilon_0\epsilon_3 + 
  \omega_3 \epsilon_3\epsilon_2 + \omega_4 \epsilon_1\epsilon_3 + \omega_5 \epsilon_2\epsilon_1)
\end{equation}
これらの内、左側の時空的な3成分がもたらす回転はローレンツブーストになり、右側の空間的な3成分がもたらす回転は空間回転になります。
\begin{equation}
  S' = R S R^* , \quad R R^* = 1 , \quad S \cdot S = S' \cdot S'
\end{equation}
この回転はまた、双四元数や分解型八元数による回転と同型の、双曲線関数と三角関数の混合タイプになっています。

時空の歪みを算出するにあたっては、$ \omega_{ij} $が具体的にどんな値なのかを知る必要はない事に注意してください。
更に、計量$ g_i $についても知る必要はありません。
平行移動でどれだけ計量$ g_i $が変化するかを知れば歪みがわかりますが、
ここで位置の関数である$ g(x) $が$ R $で表現できる事が分かったわけですから、$ R $も位置による関数$ R(x) $です。
クリフォード代数で計量が異なるという事の意味は、回転子が異なるという意味に他なりません。
時空の歪みは回転子の変化量で表現できるという事になります。

\section{クリフォード代数の時空超立方体}

$Cl(1,3)$において、擬スカラー$e_0e_1e_2e_3$を1-ベクトルにかけると以下のようになります。
\begin{equation}
  \begin{split}
    (te_0+xe_1+ye_2+ze_3)e_0e_1e_2e_3=te_1e_2e_3+xe_0e_2e_3-ye_0e_1e_3+ze_0e_1e_2\\
    e_0e_1e_2e_3(te_1e_2e_3+xe_0e_2e_3-ye_0e_1e_3+ze_0e_1e_2)=te_0+xe_1+ye_2+ze_3
  \end{split}
\end{equation}
左からかけると元に戻ります。
y軸については符号が反転しています。
\begin{equation}
  \begin{split}
    (te_1e_2e_3+xe_0e_2e_3-ye_0e_1e_3+ze_0e_1e_2)^2
    =~&t^2+txe_0e_1+tye_0e_2+tze_0e_3\\
    &-txe_0e_1-x^2-xye_1e_2+xze_1e_3\\
    &-tye_0e_2-xye_1e_2-y^2+zye_2e_3\\
    &-tze_0e_3-xze_1e_3-yze_2e_3-z^2\\
    =~&t^2-x^2-y^2-z^2
  \end{split}
\end{equation}
二乗すると不変量になります。
この様に1-ベクトルに擬スカラーをかけて3-ベクトルにする操作を双対(Duality)と呼ぶ事とします。
時空とその双対時空の和のノルムを計算してみます。

\begin{equation}
  S=te_0+xe_1+ye_2+ze_3,\quad S^\dag=te_1e_2e_3+xe_0e_2e_3-ye_0e_1e_3+ze_0e_1e_2
\end{equation}
と置くと、
\begin{equation}
  Se_0e_1e_2e_3=S^\dag, \quad e_0e_1e_2e_3S^\dag=S
\end{equation}
\begin{equation}
  S^2=(S^\dag)^2=t^2-x^2-y^2-z^2
\end{equation}
\begin{equation}
  \begin{split}
    SS^\dag=~&(te_0+xe_1+ye_2+ze_3)(te_1e_2e_3+xe_0e_2e_3-ye_0e_1e_3+ze_0e_1e_2)\\
    &+t^2e_1e_2e_3e_4+txe_2e_3-tye_1e_3+tze_1e_2\\
    &-txe_2e_3-x^2e_0e_1e_2e_3-xye_0e_3+xze_0e_2\\
    &+tye_1e_3+xye_0e_3-y^2e_0e_1e_2e_3-yze_0e_1\\
    &-tze_1e_2-xze_0e_2+yze_0e_1-z^2e_0e_1e_2e_3\\
    =~&(t^2-x^2-y^2-z^2)e_1e_2e_3e_4\\
    =~&S^2e_1e_2e_3e_4\\
  \end{split}
\end{equation}
\begin{equation}
  \begin{split}
    S^\dag S=~&(te_1e_2e_3+xe_0e_2e_3-ye_0e_1e_3+ze_0e_1e_2)(te_0+xe_1+ye_2+ze_3)\\
    &-t^2e_1e_2e_3e_4-txe_2e_3-tye_1e_3-tze_1e_2\\
    &+txe_2e_3+x^2e_0e_1e_2e_3-xye_0e_3-xze_0e_2\\
    &+tye_1e_3+xye_0e_3+y^2e_0e_1e_2e_3-yze_0e_1\\
    &+tze_1e_2+xze_0e_2+yze_0e_1+z^2e_0e_1e_2e_3\\
    =~&-(t^2-x^2-y^2-z^2)e_1e_2e_3e_4\\
    =~&-S^2e_1e_2e_3e_4\\
  \end{split}
\end{equation}
\begin{equation}
  \begin{split}
    (S+S^\dag)^2=~&S^2+(S^\dag)^2+SS^\dag+S^\dag S\\
    =~&2S^2\\
  \end{split}
\end{equation}
\begin{equation}
  \frac{1}{2}(S+S^\dag)^2=t^2-x^2-y^2-z^2
\end{equation}
和ついても不変量となっている様子です。
双対時空にさらに右から擬スカラーをかけると、
\begin{equation}
  \begin{split}
    (t e_1e_2e_3 + x e_0e_2e_3 - y e_0e_1e_3 + z e_0e_1e_2) e_0e_1e_2e_3 = 
    - t e_0 - x e_1 - y e_2 - z e_3 \\
  \end{split}
\end{equation}
となり全符号が反転します。従って時空に擬スカラーを4回かけるごとに元に戻ります。

\subsection{$ Cl(1,3) $の超立方体}

擬スカラー$ e_0 e_1 e_2 e_3 $は幾何代数的には時空の4次元超立方体の向き付きの体積を表しています。
向きの情報は、$ e_0 e_1 e_2 e_3 $自体が持っています。
双対で算出した3-ベクトルは、4次元超立方体を構成する3次元超平面を形成しています。
係数はその超平面の面積(3次元体積)を表しています。
$ e_1 e_2 e_3 $は、空間超平面を構成しています。
他の$ e_0 $が関わる3つ$ e_0 e_1 e_2, e_0 e_1 e_3, e_0 e_2 e_3 $は、それぞれ時空超平面を構成しています。

単位超立方体を考えてみます。時間と空間を同じスケールで考えると光円錐外になってしまうので、空間は光速と比べ十分小さい単位で考えます。
\begin{itemize}
  \item 単位時空: $ (t, x, y, z) = (c, 1, 1, 1) $
  \item 単位時空超平面: $ (xyz, tyz, txz, txy) = (1, c, c, c) $
\end{itemize}
単位体積は光速$ c $になり、$ e_1 e_2 e_3 $の単位超面積は$ 1 $で、
$ e_0 e_2 e_3, e_0 e_1 e_3, e_0 e_1 e_2 $の単位超面積は$ c $です。
原点と座標$ (c t, x, y, z) $が張る超立方体は、超体積が$ c t x y z $で超面積は、
$ x y z, c t y z, c t x z, c t x y $になります。

\subsection{超平面の法線}

クリフォード代数では、擬ベクトル(ここでは3-ベクトル)の超平面に擬スカラーをかける操作で法線が求まります。
この法線の方向は単位超立方体の辺を構成していた時空座標の基底と一致していると考えられます。

さて、ここで空間の歪みを考えてみましょう。3次元の立方体を思い浮かべてください。
歪んでいない立方体では、x軸の方向と立方体のyz平面の法線は一致しています。
しかし、歪んだ立方体になるとx軸と立方体のyz平面の法線は一致しなくなります。

同様に歪んだ四次元時空についても法線不一致が起きると考えられます。
時空と双対時空超平面の法線の関係は、丁度リーマン幾何学で言う反変ベクトルと共変ベクトルの関係に似ています。

\subsection{双対時空で考える時空の歪曲}

クリフォード代数の構造の中で、1-ブレードと3-ブレードが独立した変数を持ち、その不一致が斜交座標とみなせるとしました。
どのように不一致が起きるのかについては、3-ブレードを不変量を保ったまま変形させると考えます。
そのような変形については、2-ブレードの時空的回転子である、$ e_0 e_1, e_0 e_2, e_0 e_3 $の指数関数による双曲線関数型の回転と、
空間的回転子である、$ e_1 e_2, e_1 e_3, e_2 e_3 $の指数関数による三角関数型の回転の合成になると考えられます。
これらの回転子が組み合わさって、3-ブレードへサンドイッチ積する事によって、超立方体が変形します。

重要なのはこの超立方体の歪みを内部ステータスとして持っていると考えた時、
リーマン幾何学で行われる平行移動はもはや必要なくなるという事です。
従って、時空が滑らかに連続している必要もなくなり、時空連続体仮説が棄却されます。

重力の影響はこの時空立方体へ作用し、それが時間進捗に従って自然に時空へと反映されます。
もし時空立方体の歪みが時空へ反映される事に抗えば、加速度として感じる力に変換されます。
逆に言うと物体が加速度を持つとき、時空側が先に歪んで時空立方体との差が出現する事になると考えます。

\subsection{双対空間回転}

一般的にどの様に3-ブレードへの回転を考えれば良いのか探っていきます。
ここでは、1-ブレードの時空に対する2-ブレードによる回転操作に相当する3-ブレードへの作用が、
2-ブレードの双対による回転操作になると仮定して計算してみます。

具体的に1方向についてだけ計算して様子を見てみましょう。
双ベクトル$e_1e_2$によって、x-y平面における角度$2\theta$の回転を行ってみます。
物体は一定の速さ$|\bm{v}|$で移動しているとします。
移動の方向転換になるのでローレンツブーストにはなりませんが、円運動になるので遠心力が働いていると考えられます。
物体の速度$ \bm{v} $によっては移動しながら回転したり、螺旋運動をしています。
\begin{itemize}
  \item 初期時空: $S=te_0+xe_1+ye_2+ze_3$
  \item 空間的回転子: $R=\exp(\theta e_1e_2)=\cos\theta+e_1e_2\sin\theta$
  \item 空間的回転子リバース: $\widetilde{R}=\cos\theta-e_1e_2\sin\theta$
  \item 初期双対時空: $ S^\dag = t ~ e_1e_2e_3 + x ~ e_0e_2e_3 - y ~ e_0e_1e_3 + z ~ e_0e_1e_2 $
  \item 空間的双対回転子: $R^\dag=\exp(\theta e_1e_2e_0e_1e_2e_3)=\exp(-\theta e_0e_3)=\cosh\theta-e_0e_3\sinh\theta$
  \item 空間的双対回転子リバース: $\widetilde{R}^\dag=\cosh\theta+e_0e_3\sinh\theta$
\end{itemize}

$RS\widetilde{R}$を計算すると、x-y平面での回転になります。こちらの計算の詳細は省略します。

$R^\dag S^\dag\widetilde{R}^\dag$を計算してみます。
\begin{equation}
  \begin{split}
    R^\dag S^\dag=~&(\cosh\theta-e_0e_3\sinh\theta)
    (t e_1e_2e_3 + x e_0e_2e_3 - y e_0e_1e_3 + z e_0e_1e_2) \\
    = ~ & S^\dag \cosh \theta + (t e_0e_1e_2 + x e_2 - y e_1 + z  e_1e_2e_3) \sinh \theta \\
  \end{split}
\end{equation}
\begin{equation}
  \begin{split}
    R^\dag S^\dag \widetilde{R}^\dag = ~ & (S^\dag \cosh \theta
    + (t e_0e_1e_2 + x e_2 - y e_1 + z e_1e_2e_3)\sinh \theta)(\cosh \theta + e_0e_3 \sinh \theta)\\
    =~&S^\dag\cosh^2\theta+(te_0e_1e_2+xe_2-ye_1+ze_1e_2e_3)e_0e_3\sinh^2\theta\\
    &+(S^\dag e_0e_3+te_0e_1e_2+xe_2-ye_1+ze_1e_2e_3)\cosh\theta\sinh\theta\\
    =~&(te_1e_2e_3+xe_0e_2e_3-ye_0e_1e_3+ze_0e_1e_2)\cosh^2\theta+(te_1e_2e_3-xe_0e_2e_3+ye_0e_1e_3+ze_0e_1e_2)\sinh^2\theta\\
    &+((te_1e_2e_3+xe_0e_2e_3-ye_0e_1e_3+ze_0e_1e_2)e_0e_3+te_0e_1e_2+xe_2-ye_1+ze_1e_2e_3)\cosh\theta\sinh\theta\\
    =~&(te_1e_2e_3+xe_0e_2e_3-ye_0e_1e_3+ze_0e_1e_2)\cosh^2\theta+(te_1e_2e_3-xe_0e_2e_3+ye_0e_1e_3+ze_0e_1e_2)\sinh^2\theta\\
    &+(te_0e_1e_2-xe_2+ye_1+ze_1e_2e_3+te_0e_1e_2+xe_2-ye_1+ze_1e_2e_3)\cosh\theta\sinh\theta\\
    =~&(te_1e_2e_3+xe_0e_2e_3-ye_0e_1e_3+ze_0e_1e_2)\cosh^2\theta+(te_1e_2e_3-xe_0e_2e_3+ye_0e_1e_3+ze_0e_1e_2)\sinh^2\theta\\
    &+2(te_0e_1e_2+ze_1e_2e_3)\cosh\theta\sinh\theta\\
    =~&(t\cosh^2\theta+t\sinh^2\theta+2z\cosh\theta\sinh\theta)e_1e_2e_3\\
    &+(x\cosh^2\theta-x\sinh^2\theta)e_0e_2e_3+(-y\cosh^2\theta+y\sinh^2\theta)e_0e_1e_3\\
    &+(z\cosh^2\theta+z\sinh^2\theta+2t\cosh\theta\sinh\theta)e_0e_1e_2\\
    =~&(t\cosh^2\theta+t\sinh^2\theta+2z\cosh\theta\sinh\theta)e_1e_2e_3+xe_0e_2e_3-ye_0e_1e_3\\
    &+(z\cosh^2\theta+z\sinh^2\theta+2t\cosh\theta\sinh\theta)e_0e_1e_2\\
    =~&(t\cosh(2\theta)+z\sinh(2\theta))e_1e_2e_3+xe_0e_2e_3-ye_0e_1e_3
    +(z\cosh(2\theta)+t\sinh(2\theta))e_0e_1e_2\\
  \end{split}
\end{equation}
$\theta^\dag=-\theta$とすると、
\begin{equation}
  \begin{bmatrix}
    t'\\z'
  \end{bmatrix}
  =
  \begin{bmatrix}
    \cosh(2\theta^\dag) & -\sinh(2\theta^\dag) \\
    -\sinh(2\theta^\dag) & \cosh(2\theta^\dag) \\
  \end{bmatrix}
  \begin{bmatrix}
    t\\z
  \end{bmatrix}
\end{equation}
式の形からローレンツブーストを表している事がわかります。

少しまとめます。1-ベクトルよりなる時空に擬スカラーをかけると双対時空が現れました。
時空の円運動を表現する空間回転の回転子に擬ベクトルをかけ、円運動の双対回転子を構成しました。
双対回転子により双対時空をサンドイッチ積すると、ローレンツブーストになりました。
時空体積を表現していると考えられる擬ベクトルによる双対時空が双曲線関数的に歪むと解釈できます。
円運動する物体が同時に双対時空の歪みを内部状態として持つと考えれば、双対回転子による計算が、
円運動による加速度、つまり遠心力を表現している可能性が考えられます。

逆にこの双対時空のz方向へのローレンツブーストが先に加速度として作用した場合を考えます。
$R^\dag=\exp(-\theta e_0e_3)$の双対計算をすると、
\begin{equation}
  \begin{split}
    (R^\dag)^\dag=~&\exp(-\theta e_0e_3e_0e_1e_2e_3)=\exp(-\theta e_1e_2)=\cos\theta-e_1e_2\sin\theta
  \end{split}
\end{equation}
この式は最初の円運動の回転の逆を意味しています。双対時空が通常時空に作用する結果は逆回転になるという事ですから、
円運動の中心方向と逆方向の遠心力を表現していると考えられます。
双対時空の時間方向の歪みは遠心力による時間の遅れになると解釈できます。
また、双曲線関数なので角度が大きくなると加速度が小さくなるという事はありません。

\subsection{双対時空回転}

それでは次に、物体が加速して速さが増大する計算をしてみます。
\begin{itemize}
  \item ラピディティ(速さの増大分): $\tanh\phi=\Delta v/c$
  \item 初期時空: $S=te_0+xe_1+ye_2+ze_3$
  \item 時空的回転子: $R=\exp(\phi e_0e_1)=\cosh\phi+e_0e_1\sinh\phi$
  \item 時空的回転子リバース: $\widetilde{R}=\cosh\phi-e_0e_1\sinh\phi$
  \item 初期双対時空: $S^\dag=te_1e_2e_3+xe_0e_2e_3-ye_0e_1e_3+ze_0e_1e_2$
  \item 時空的双対回転子: $R^\dag=\exp(\phi e_0e_1e_0e_1e_2e_3)=\exp(\phi e_2e_3)=\cos\phi+e_2e_3\sin\phi$
  \item 時空的双対回転子リバース: $\widetilde{R}^\dag=\cos\phi-e_2e_3\sin\phi$
\end{itemize}

$RS\widetilde{R}$を計算すると、t-x時空平面でのローレンツブーストになります。こちらの計算の詳細は省略します。

$R^\dag S^\dag\widetilde{R}^\dag$を計算してみます。
\begin{equation}
  \begin{split}
    R^\dag S^\dag=~&(\cos\phi+e_2e_3\sin\phi)(te_1e_2e_3+xe_0e_2e_3-ye_0e_1e_3+ze_0e_1e_2)\\
    =~&S^\dag\cos\phi+(-te_1-xe_0+ye_0e_1e_2+ze_0e_1e_3)\sin\phi\\
  \end{split}
\end{equation}
\begin{equation}
  \begin{split}
    R^\dag S^\dag\widetilde{R}^\dag=~&(S^\dag\cos\phi+(-te_1-xe_0+ye_0e_1e_2+ze_0e_1e_3)\sin\phi)(\cos\phi-e_2e_3\sin\phi)\\
    =~&S^\dag\cos^2\phi+(te_1+xe_0-ye_0e_1e_2-ze_0e_1e_3)e_2e_3\sin^2\phi\\
    &+(-S^\dag e_2e_3-te_1-xe_0+ye_0e_1e_2+ze_0e_1e_3)\cos\phi\sin\phi\\
    =~&(te_1e_2e_3+xe_0e_2e_3-ye_0e_1e_3+ze_0e_1e_2)\cos^2\phi+(te_1e_2e_3+xe_0e_2e_3+ye_0e_1e_3-ze_0e_1e_2)\sin^2\phi\\
    &+(-(te_1e_2e_3+xe_0e_2e_3-ye_0e_1e_3+ze_0e_1e_2)e_2e_3-te_1-xe_0+ye_0e_1e_2+ze_0e_1e_3)\cos\phi\sin\phi\\
    =~&(te_1e_2e_3+xe_0e_2e_3-ye_0e_1e_3+ze_0e_1e_2)\cos^2\phi+(te_1e_2e_3+xe_0e_2e_3+ye_0e_1e_3-ze_0e_1e_2)\sin^2\phi\\
    &+(te_0+xe_0+ye_0e_1e_2+ze_0e_1e_3-te_1-xe_0+ye_0e_1e_2+ze_0e_1e_3)\cos\phi\sin\phi\\
    =~&(te_1e_2e_3+xe_0e_2e_3-ye_0e_1e_3+ze_0e_1e_2)\cos^2\phi+(te_1e_2e_3+xe_0e_2e_3+ye_0e_1e_3-ze_0e_1e_2)\sin^2\phi\\
    &+2(ye_0e_1e_2+ze_0e_1e_3)\cos\phi\sin\phi\\
    =~&(t\cos^2\phi+t\sin^2\phi)e_1e_2e_3+(x\cos^2\phi+x\sin^2\phi)e_0e_2e_3\\
    &+(-y\cos^2\phi+y\sin^2\phi+2z\cos\phi\sin\phi)e_0e_1e_3\\
    &+(z\cos^2\phi-z\sin^2\phi+2y\cos\phi\sin\phi)e_0e_1e_2\\
    =~&te_1e_2e_3+xe_0e_2e_3+(-y\cos^2\phi+y\sin^2\phi+2z\cos\phi\sin\phi)e_0e_1e_3\\
    &+(z\cos^2\phi-z\sin^2\phi+2y\cos\phi\sin\phi)e_0e_1e_2\\
    =~&te_1e_2e_3+xe_0e_2e_3(-y\cos(2\phi)+z\sin(2\phi))e_0e_1e_3+(z\cos(2\phi)+y\sin(2\phi))e_0e_1e_2\\
  \end{split}
\end{equation}
\begin{equation}
  \begin{bmatrix}
    -y'\\z'
  \end{bmatrix}
  =
  \begin{bmatrix}
    \cos(2\phi) & -\sin(2\phi) \\
    \sin(2\phi) & \cos(2\phi) \\
  \end{bmatrix}
  \begin{bmatrix}
    y\\z
  \end{bmatrix}
\end{equation}
$ e_0e_1e_2 , e_0e_1e_3 $の超平面面積が交換されます。$e_0e_1e_3$の大きさは符号反転していますが、
これは$ Cl(1,3) $双対時空の特徴と合致しています。

こちらの式においても加速度を表していると考えられますが、循環関数なのである程度強い加速では、
加速度による負荷を感じなくなると解釈できます。
つまり非常に瞬間的な加速や減速を行う乗り物が成立する事になります。

$R^\dag=\exp(\phi e_2e_3)$の双対計算をすると、
\begin{equation}
  (R^\dag)^\dag=\exp(\phi e_2e_3e_0e_1e_2e_3)=\exp(-\phi e_0e_1)=\cosh\phi-e_0e_1\sinh\phi
\end{equation}

こちらの$ (R^\dag)^\dag $では、時空にサンドイッチ積されると逆方向のブーストになります。
加速では加速方向と逆に加速度を受ける事になりますが、
その加速度を重力によって最初に作用させれば、自然と自由落下として移動が起きると考えます。

\section{クリフォード代数での重力計算}

双対時空の歪みとその歪みによる速度の変化について、多体シミュレーションを意識しつつ、三次元で一般化された形を整えていきます。
双対空間回転と双対時空回転の様子見から、加速度と重力の数理的な本質が見えてきました。
1-ブレードで表現された時空の回転により、3-ブレードで表現された双対時空とのズレが生じ加速度の慣性力となります。
逆に双対時空から時空への双対の双対回転子による作用が重力による圧のない自然落下となると考えられます。
重力による時空超平面の変形は、空間回転子とブースト回転子の混合による事が見えてきました。
物体の進行方向と重力の作用が直角の場合は空間回転子になり、平行の場合はブースト回転子になり、
それ以外は、両者の角度により配合率が決定されると考えるのが自然です。
ベルソルを使ってまとめてみます。
\begin{table}[ht]
  \centering
  \caption{時空回転のパラメーター}
  \begin{tabular}{|l|c|l|l|l|} \hline
    回転種別 & 角度 & 方向 & ベルソル & 双対回転 \\ \hline
    ブースト & $ 2 \phi $ & 進行方向 & $ \bm{u}^2 = {u_1}^2 + {u_2}^2 + {u_3}^2 = 1 $ & 空間回転 \\ \hline
    空間回転 & $ 2 \theta $ & 進行方向と直角 & $ \bm{v}^2 = -({v_1}^2 + {v_2}^2 + {v_3}^2) = -1 $ & ブースト \\ \hline
  \end{tabular}
\end{table}
\begin{equation}
  \begin{split}
    \bm{u} = ~ & u_1 iI + u_2 iJ + u_3 iK \\
    \bm{u}^2 = ~ & (u_1 iI + u_2 iJ + u_3 iK)(u_1 iI + u_2 iJ + u_3 iK) \\
    = ~ & {u_1}^2 - u_1 u_2 K + u_1 u_3 J
    + u_1 u_2 K + {u_2}^2 - u_2 u_3 I
    - u_1 u_3 J + u_2 u_3 I + {u_3}^2 \\
    = ~ & {u_1}^2 + {u_2}^2 + {u_3}^2 = 1 \\
    \bm{v} = ~ & v_1 I + v_2 J + v_3 K \\
    \bm{v}^2 = ~ & (v_1 I + v_2 J + v_3 K)(v_1 I + v_2 J + v_3 K) \\
    = ~ & - {v_1}^2 + v_1 v_2 K - v_1 v_3 J
    - v_1 v_2 K - {v_2}^2 + v_2 v_3 I
    + v_1 v_3 J - v_2 v_3 I - {v_3}^2 \\
    = ~ & - {v_1}^2 - {v_2}^2 - {v_3}^2 = -1 \\
  \end{split}
\end{equation}
\begin{equation}
  \begin{split}
    R_{boost} = ~ & \exp(\gamma_0 e_0e_1 + \gamma_1 e_0e_2 + \gamma_2 e_0e_3)
    = \exp ( \phi \bm{u} )
    = \cosh \phi + \bm{u} \sinh \phi \\
    R_{space} = ~ & \exp(\gamma_3 e_1e_2 + \gamma_4 e_1e_3 + \gamma_5 e_2e_3)
    = \exp ( \theta \bm{v} )
    = \cos \theta + \bm{v} \sin \theta \\
  \end{split}
\end{equation}
\begin{equation}
  \begin{split}
    R_{hybrid} = ~ & \exp(\gamma_0 e_0e_1 + \gamma_1 e_0e_2 + \gamma_2 e_0e_3
    + \gamma_3 e_1e_2 + \gamma_4 e_1e_3 + \gamma_5 e_2e_3) \\
    = ~ & \exp (\phi \bm{u} + \theta \bm{v}) \\
    = ~ & \exp (u_1 \phi iI + u_2 \phi iJ + u_3 \phi iK
    + \theta v_1 I + \theta v_2 J + \theta v_3 K) \\
    = ~ & \exp \Big( (\theta v_1 + \phi u_1 i)I + (\theta v_2 + \phi i u_2)J 
    + (\theta v_3 + \phi u_3 i)K \Big) \\
  \end{split}
\end{equation}
$ \eta_n $で置き換えて見通しを良くします。
\begin{equation}
  \eta_n = \theta v_n + \phi u_n i, \quad n = 1,2,3
\end{equation}
\begin{equation}
  R_{hybrid} = \exp (\eta_1 I + \eta_2 J + \eta_3 K)
\end{equation}
ベルソルの方向としてはお互いが直角になっているので、
\begin{equation}
  v_1 u_1 + v_2 u_2 + v_3 u_3 = 0
\end{equation}
また二乗に関しては以下の通りです。
\begin{equation}
  {\eta_1}^2 = (v_1 \theta + u_1 \phi i)^2 = {v_1}^2 \theta^2 - {u_1}^2 \phi^2 +2 v_1 u_1 \theta \phi
\end{equation}
\begin{equation}
  {\eta_1}^2 + {\eta_2}^2 + {\eta_3}^2
  = (v_1 \theta + u_1 \phi i)^2 + (v_2 \theta + u_2 \phi i)^2 + (v_3 \theta + u_3 \phi i)^2
  = \theta^2 - \phi^2
\end{equation}
\begin{equation}
  (\eta_1 I + \eta_2 J + \eta_3 K)^2 = - {\eta_1}^2 - {\eta_2}^2 - {\eta_3}^2
  = - (\theta^2 - \phi^2)
\end{equation}
$ \alpha = \phi^2 - \theta^2 , \quad \alpha \in \mathbb{R} 
, \quad \eta = \eta_1 I + \eta_2 J + \eta_3 K $と置くと、
\begin{equation}
  \eta^2 = \alpha
  , \quad R_{hybrid} = \exp (\eta)
\end{equation}
$ \alpha $の平方根については、
\begin{equation}
  \eta = \sqrt{\alpha} = \phi \bm{u} + \theta \bm{v}
\end{equation}
ここで$ \bm{u}, \bm{v} $について更に調べると、
\begin{equation}
  \begin{split}
    \bm{u} \bm{v} = ~ & (u_1 iI + u_2 iJ + u_3 iK)(v_1 I + v_2 J + v_3 K) \\
    = ~ & - u_1 v_1 i + u_1 v_2 iK - u_1 v_3 iJ
    - u_2 v_1 iK - u_2 v_2 i + u_2 v_3 iI
    + u_3 v_1 iJ - u_3 v_2 iI - u_3 v_3 i \\
    = ~ & (u_2 v_3 - u_3 v_2) iI + (u_3 v_1 - u_1 v_3) iJ + (u_1 v_2 - u_2 v_1) iK \\
    \bm{v} \bm{u} = ~ & (v_1 I + v_2 J + v_3 K)(u_1 iI + u_2 iJ + u_3 iK) \\
    = ~ & - u_1 v_1 i + u_2 v_1 iK - u_3 v_1 iJ
    - u_1 v_2 iK - u_2 v_2 i + u_3 v_2 iI
    + u_1 v_3 iJ - u_2 v_3 iI - u_3 v_3 i \\
    = ~ & (u_3 v_2 - u_2 v_3) iI + (u_1 v_3 - u_3 v_1) iJ + (u_2 v_1 - u_1 v_2) iK \\
  \end{split}
\end{equation}
\begin{equation}
  \bm{u}^2 = 1 , \quad \bm{v}^2 = -1 , \quad \bm{u} \bm{v} = - \bm{v} \bm{u}
\end{equation}
になっているので、$ \bm{u}, \bm{v} $については、$ Cl(1,1) $の基底と同等になります。
\begin{equation}
  (\phi \bm{u} + \theta \bm{v})^2 = \phi^2 - \theta^2
\end{equation}
テイラー展開を行うと、
\begin{equation}
  \begin{split}
    R_{hybrid} = ~ & \sum_{n=0}^{∞} \frac{1}{n!} \eta^n \\
    = ~ & \sum_{n=0}^{∞} \frac{1}{(2n)!} \eta^{2n}
    + \sum_{n=0}^{∞} \frac{1}{(2n + 1)!} \eta^{2n + 1} \\
  \end{split}
\end{equation}

\subsection{混合回転の具体化}

$ \alpha $の符号によって場合分けします。

\begin{itemize}
  \item $ \alpha > 0 : $
\end{itemize}
\begin{equation}
  \alpha^n = |\alpha|^n
\end{equation}
\begin{equation}
  \begin{split}
    R_{hybrid} = ~ & \sum_{n=0}^{∞} \frac{1}{(2n)!} \alpha^n
    + \eta \sum_{n=0}^{∞} \frac{1}{(2n + 1)!} \alpha^n \\
    = ~ & \sum_{n=0}^{∞} \frac{1}{(2n)!} (\sqrt{\alpha})^{2n}
    + \sum_{n=0}^{∞} \frac{1}{(2n + 1)!} (\sqrt{\alpha})^{2n + 1} \\
    = ~ & \cosh(\sqrt{\alpha}) + \sinh(\sqrt{\alpha}) \\
  \end{split}
\end{equation}
双曲線関数の加法定理から、
\begin{equation}
  \begin{split}
    \cosh(\phi \bm{u} + \theta \bm{v})
    = ~ & \cosh(\phi \bm{u}) \cosh(\theta \bm{v}) + \sinh(\phi \bm{u}) \sinh(\theta \bm{v}) \\
    = ~ & \cosh \phi \cos \theta + \bm{u} \bm{v} \sinh \phi \sin \theta \\
    \sinh(\phi \bm{u} + \theta \bm{v})
    = ~ & \sinh(\phi \bm{u}) \cosh(\theta \bm{v}) + \cosh(\phi \bm{u}) \sinh(\theta \bm{v}) \\
    = ~ & \bm{u} \sinh \phi \cos \theta + \bm{v} \cosh \phi \sin \theta \\
  \end{split}
\end{equation}
\begin{equation}
  \begin{split}
    R_{hybrid} = ~ & \cosh \phi \cos \theta + \bm{u} \sinh \phi \cos \theta 
    + \bm{v} \cosh \phi \sin \theta + \bm{u} \bm{v} \sinh \phi \sin \theta \\
    = ~ & (\cosh \phi + \bm{u} \sinh \phi)(\cos \theta + \bm{v} \sin \theta) \\
  \end{split}
\end{equation}

\begin{itemize}
  \item $ \alpha < 0 : $
\end{itemize}
\begin{equation}
  \alpha^n = (-1)^n |\alpha|^n
\end{equation}
\begin{equation}
  \begin{split}
    R_{hybrid} = ~ & \sum_{n=0}^{∞} \frac{(-1)^n}{(2n)!} |\alpha|^n
    + \eta \sum_{n=0}^{∞} \frac{(-1)^n}{(2n + 1)!} |\alpha|^n \\
    = ~ & \sum_{n=0}^{∞} \frac{(-1)^n}{(2n)!} (\sqrt{|\alpha|})^{2n}
    + \frac{\eta}{\sqrt{|\alpha|}} \sum_{n=0}^{∞} \frac{(-1)^n}{(2n + 1)!} (\sqrt{|\alpha|})^{2n + 1} \\
    = ~ & \cos(\sqrt{|\alpha|}) + \sin(\sqrt{|\alpha|}) \\
  \end{split}
\end{equation}
ここで、$ |\alpha| = \theta^2 - \phi^2 > 0 $だから、
$ \sqrt{|\alpha|} = \theta \bm{u} + \phi \bm{v} $
\begin{equation}
  \begin{split}
    \frac{\eta}{\sqrt{|\alpha|}} = ~ & \frac{\phi \bm{u} + \theta \bm{v}}{\theta \bm{u} + \phi \bm{v}}
    = \frac{(\phi \bm{u} + \theta \bm{v})(\theta \bm{u} + \phi \bm{v})}{(\theta \bm{u} + \phi \bm{v})^2}
    = \frac{(\phi \bm{u} + \theta \bm{v})(\theta \bm{u} + \phi \bm{v})}{\theta^2 - \phi^2} \\
    = ~ & \frac{\phi \theta - \phi \theta + (\phi^2 - \theta^2) \bm{u} \bm{v}}{\theta^2 - \phi^2} \\
    = ~ & \bm{u} \bm{v} \\
  \end{split}
\end{equation}
三角関数の加法定理から、
\begin{equation}
  \begin{split}
    \cos(\theta \bm{u} + \phi \bm{v})
    = ~ & \cos(\theta \bm{u}) \cos(\phi \bm{v}) - \sin(\theta \bm{u}) \sin(\phi \bm{v}) \\
    = ~ & \cosh \phi \cos \theta - \bm{u} \bm{v} \sinh \phi \sin \theta \\
    \sinh(\theta \bm{u} + \phi \bm{v})
    = ~ & \sin(\theta \bm{u}) \cos(\phi \bm{v}) + \cos(\theta \bm{u}) \sin(\phi \bm{v}) \\
    = ~ & \bm{u} \cosh \phi \sin \theta + \bm{v} \sinh \phi \cos \theta \\
  \end{split}
\end{equation}
\begin{equation}
  \begin{split}
    R_{hybrid} = ~ & \cosh \phi \cos \theta - \bm{u} \bm{v} \sinh \phi \sin \theta
    + \bm{u} \bm{v} (\bm{u} \cosh \phi \sin \theta + \bm{v} \sinh \phi \cos \theta) \\
    = ~ & \cosh \phi \cos \theta - \bm{u} \bm{v} \sinh \phi \sin \theta
    - \bm{v} \cosh \phi \sin \theta - \bm{u} \sinh \phi \cos \theta \\
    = ~ & \cosh \phi \cos \theta - \bm{u} \sinh \phi \cos \theta 
    - \bm{v} \cosh \phi \sin \theta - \bm{u} \bm{v} \sinh \phi \sin \theta \\
    = ~ & (\cos \theta - \bm{v} \sin \theta)(\cosh \phi - \bm{u} \sinh \phi) \\
  \end{split}
\end{equation}

\begin{itemize}
  \item $ \alpha = 0 : $
\end{itemize}
\begin{equation}
  R_{hybrid} = \sum_{n=0}^{∞} \frac{1}{n!} \eta^n = 1 + \phi \bm{u} + \theta \bm{v}
\end{equation}

\subsection{ユニタリ性の確認}

それぞれについての共役を選択してユニタリ性を確かめていきます。
\begin{itemize}
  \item $ \alpha = 0 : $
\end{itemize}
\begin{equation}
  \begin{split}
    R R^* = ~ & (1 + \phi \bm{u} + \theta \bm{v})(1 - \phi \bm{u} - \theta \bm{v}) \\
    = ~ & 1 - \phi \bm{u} - \theta \bm{v}
    + \phi \bm{u} - \phi^2 - \phi \theta \bm{u} \bm{v}
    + \theta \bm{v} + \phi \theta \bm{u} \bm{v} + \theta^2 \\
    = ~ & 1 - \phi^2 + \theta^2 \\
    = ~ & 1 \\
  \end{split}
\end{equation}

\begin{itemize}
  \item $ \alpha > 0: $
\end{itemize}
\begin{equation}
  \begin{split}
    R = ~ & \cosh \phi \cos \theta + \bm{u} \sinh \phi \cos \theta 
    + \bm{v} \cosh \phi \sin \theta + \bm{u} \bm{v} \sinh \phi \sin \theta \\
    R^* = ~ & \cosh \phi \cos \theta - \bm{u} \sinh \phi \cos \theta 
    - \bm{v} \cosh \phi \sin \theta - \bm{u} \bm{v} \sinh \phi \sin \theta \\
  \end{split}
\end{equation}
ここの$ R^* $は、$ \alpha < 0 $の場合の$ R $と同じになっています。
式の短縮のために$\cosh \phi = a, \sinh \phi = b, \cos \theta = p, \sin \theta = q$と置いて、
\begin{equation}
  \begin{split}
    R R^* = ~ & (a p + b p \bm{u} + a q \bm{v} + b q \bm{u} \bm{v})
    (a p - b p \bm{u} - a q \bm{v} - b q \bm{u} \bm{v}) \\
    = ~ & a^2 p^2 - a b p^2 \bm{u} - a^2 p q \bm{v} - a b p q \bm{u} \bm{v}
    + a b p^2 \bm{u} - b^2 p^2 - a b p q \bm{u} \bm{v} - b^2 p q \bm{v} \\
    & + a^2 p q \bm{v} + a b p q \bm{u} \bm{v} + a^2 q^2 - a b q^2 \bm{u}
    + a b p q \bm{u} \bm{v} + b^2 p q \bm{v} + a b q^2 \bm{u} - b^2 q^2 \\
    = ~ & a^2 p^2 - b^2 p^2 + a^2 q^2 - b^2 q^2 \\
    = ~ & (a^2 - b^2) (p^2 + q^2) \\
    = ~ & 1 \\
  \end{split}
\end{equation}

\begin{itemize}
  \item $ \alpha < 0: $
\end{itemize}
逆も同様なので、
\begin{equation}
  R R^* = R^* R = 1
\end{equation}

\subsection{混合回転の双対}

双対混合回転を計算します。
$ Cl(1,3) $で考えると、
\begin{equation}
  \begin{split}
    R = ~ & \exp(\gamma_0 e_0e_1 + \gamma_1 e_0e_2 + \gamma_2 e_0e_3
    + \gamma_3 e_1e_2 + \gamma_4 e_1e_3 + \gamma_5 e_2e_3) \\
    R^\dag = ~ & \exp((\gamma_0 e_0e_1 + \gamma_1 e_0e_2 + \gamma_2 e_0e_3
    + \gamma_3 e_1e_2 + \gamma_4 e_1e_3 + \gamma_5 e_2e_3)e_0e_1e_2e_3) \\
    = ~ & \exp(\gamma_0 e_2e_3 - \gamma_1 e_1e_3 + \gamma_2 e_1e_2
    - \gamma_3 e_0e_3 + \gamma_4 e_0e_2 - \gamma_5 e_0e_1) \\
  \end{split}
\end{equation}
四双四元数で考えると、
\begin{equation}
  \begin{split}
    R = ~ & \exp(\omega_0 iI + \omega_1 iJ + \omega_2 iK + \omega_3 I + \omega_4 J + \omega_5 K) \\
    R^\dag = ~ & \exp((\omega_0 iI + \omega_1 iJ + \omega_2 iK + \omega_3 I + \omega_4 J + \omega_5 K)i) \\
    = ~ & \exp(- \omega_0 I - \omega_1 J - \omega_2 K + \omega_3 iI + \omega_4 iJ + \omega_5 iK) \\
  \end{split}
\end{equation}
ベルソルの双対は、
\begin{equation}
  \begin{split}
    \bm{u}^\dag = ~ & i \bm{u} = - u_1 I - u_2 J - u_3 K \\
    (\bm{u}^\dag)^2 = ~ & - {u_1}^2 - {u_2}^2 - {u_3}^2 = -1 \\
    \bm{v}^\dag = ~ & i \bm{v} = v_1 iI + v_2 iJ + v_3 iK \\
    (\bm{v}^\dag)^2 = ~ & {v_1}^2 + {v_2}^2 + {v_3}^2 = 1 \\
  \end{split}
\end{equation}
\begin{equation}
  \begin{split}
    R_{boost}^\dag = ~ & \exp(- \omega_0 I - \omega_1 J - \omega_2 K)
    = \exp ( \phi \bm{u}^\dag )
    = \cos \phi + \bm{u}^\dag \sin \phi \\
    R_{space}^\dag = ~ & \exp(\omega_3 iI + \omega_4 iJ + \omega_5 iK)
    = \exp ( \theta \bm{v}^\dag )
    = \cosh \theta + \bm{v}^\dag \sinh \theta \\
  \end{split}
\end{equation}

\begin{table}[ht]
  \centering
  \caption{時空歪曲の混合回転}
  \begin{tabular}{|c|l|l|} \hline
    $ \alpha = \phi^2 - \theta^2 $ & 回転子(Rotor) & 双対回転子(Duality Rotor) \\ \hline
    $ \alpha > 0 $ & $ (\cosh \phi + \bm{u} \sinh \phi)(\cos \theta + \bm{v} \sin \theta) $
    & $ (\cos \phi + \bm{u}^\dag \sin \phi)(\cosh \theta + \bm{v}^\dag \sinh \theta) $ \\ \hline
    $ \alpha < 0 $ & $ (\cos \theta - \bm{v} \sin \theta)(\cosh \phi - \bm{u} \sinh \phi) $
    & $ (\cosh \theta - \bm{v}^\dag \sinh \theta)(\cos \phi - \bm{u}^\dag \sin \phi) $ \\ \hline
    $ \alpha = 0 $ & $ 1 + \phi \bm{u} + \theta \bm{v} $
    & $ 1 + \phi \bm{u}^\dag + \theta \bm{v}^\dag $ \\ \hline
  \end{tabular}
\end{table}

物体の進行方向と重力の作用する方向のなす角度を$ \sigma $、重力の大きさを$ \rho $とすると、
\begin{equation}
  \phi = \frac{1}{2} \rho \cos \sigma , \quad \theta = \frac{1}{2} \rho \sin \sigma
  , \quad 0 \le \sigma \le \pi , \quad \rho \ge 0 , \quad \sigma,\rho \in \mathbb{R}
\end{equation}

\subsection{クリフォード代数の4元運動量}

相対性理論では、運動量$\bm{p}$、エネルギー$E$、質量$m$の間に以下の関係があります。
\begin{equation}
  E^2=(\bm{p}c)^2+(mc^2)^2
\end{equation}
質量を左辺にして整理すると、
\begin{equation}
  (mc^2)^2=E^2-(\bm{p}c)^2=E^2-(p_1c)^2-(p_2c)^2-(p_3c)^2
\end{equation}
クリフォード代数$Cl(1,3)$に当てはめて、
\begin{equation}
  \begin{split}
    P = ~ & E e_0 + c p_1 e_1 + c p_2 e_2 + c p_3 e_3 \\
    P^2 = ~ & E^2 - c^2 {p_1}^2 - c^2 {p_2}^2 - c^2 {p_3}^2 = E^2 - c^2 \bm{p}^2 = (m c^2)^2 \\
    |P| = ~ & m c^2, \quad (m \ge 0) \\
  \end{split}
\end{equation}
この様に1-ブレードで表現しても不変量は保たれますが、$ c $の位置が丁度、時空超平面の単位面積になっています。
\begin{equation}
  E (e_1e_2e_3), \quad p_1 (c ~ e_0e_2e_3), \quad p_2 (c ~ e_0e_1e_3), \quad p_3 (c ~ e_0e_1e_2)
\end{equation}
双対時空として表現すると、
\begin{equation}
  \begin{split}
    P^\dag = ~ & E e_1e_2e_3 + c p_1 e_0e_2e_3 - c p_2 e_0e_1e_3 + c p_3 e_0e_1e_2 \\
    (P^\dag)^2 = ~ & E^2 - c^2 {p_1}^2 - c^2 {p_2}^2 - c^2 {p_3}^2 = E^2 - c^2 \bm{p}^2 = (m c^2)^2 \\
    |P^\dag| = ~ & m c^2, \quad (m \ge 0) \\
  \end{split}
\end{equation}
クリフォード代数表現からすると、4元運動量はどうやら時空超立方体と関連が深い様子です。
1-ブレードの4元速度$ U $を考えてみます。
3次元速度を$ \bm{v} = (v_1, v_2, v_3) $、固有時間を$ \tau $とすると、
ローレンツ因子$ \gamma $は、
\begin{equation}
  \gamma = \frac{dt}{d\tau} = \frac{1}{\sqrt{1 - \bm{v}^2 / c^2}}
\end{equation}
\begin{equation}
  \begin{split}
    U = ~ & \frac{c dt}{d\tau} e_0 + \frac{dx}{d\tau} e_1 + \frac{dy}{d\tau} e_2 + \frac{dz}{d\tau} e_3 \\
    = ~ & \gamma c e_0 + \gamma v_1 e_1 + \gamma v_2 e_2 + \gamma v_3 e_3 \\
    = ~ & \gamma (c e_0 + v_1 e_1 + v_2 e_2 + v_3 e_3) \\
    U^2 = ~ & \gamma^2 (c^2 - v_1^2 - v_2^2 - v_3^2) \\
    = ~ & \gamma^2 (c^2 - \bm{v}^2) = c^2 \gamma^2 (1 - \bm{v}^2 / c^2) \\
    = ~ & c^2 \\
    |U| = ~ & c \\
  \end{split}
\end{equation}
4元速度のノルムと4元運動量のノルムは以下のような関係になっています。
\begin{equation}
  |U| m c = |P^\dag|
\end{equation}
$ c $の位置から4元速度は、単位時空基底で構成されています。
\begin{equation}
  \gamma (c e_0), \quad \gamma v_1 (e_1), \quad \gamma v_2 (e_2), \quad \gamma v_3 (e_3)
\end{equation}

\begin{itemize}
  \item 4元速度: 単位時空基底で構成
  \item 4元運動量: 時空超平面の単位面積基底で構成
\end{itemize}

4元速度を扱う場合、単純に速度を足していくと光速を超えてしまう事になるので出来ませんが、
4元運動量であれば、3次元運動量として加算していっても、エネルギーが自動的に調整されるので、
相対論的に破綻する事がありません。

\subsubsection{光の場合}

質量がゼロなら光的になります。$ h $をプランク定数、$ \bm{\nu} $を振動数として、
\begin{equation}
  E^2 - c^2 \bm{p}^2 = E^2 - h^2 \bm{\nu}^2 = 0
\end{equation}
\begin{equation}
  \begin{split}
    P^\dag_{light} = ~ & E e_1e_2e_3 + c p_1 e_0e_2e_3 - c p_2 e_0e_1e_3 + c p_3 e_0e_1e_2 \\
    = ~ & E e_1e_2e_3 + h \nu_1 e_0e_2e_3 - h \nu_2 e_0e_1e_3 + h \nu_3 e_0e_1e_2 \\
    (P^\dag_{light})^2 = ~ & |P^\dag_{light}| = 0 \\
  \end{split}
\end{equation}
光の場合、4元速度は固有時間$ \tau = 0 $のため定義できませんが、速度の方向と大きさは明確なので算出は容易です。
光は進行方向の空間の長さがローレンツ収縮によりゼロになっています。
このようなゼロの空間で光の活動がどの様になっているかを考える事は困難です。
しかし、時空超立方体で光の振動が起きていると考えると、光に関する円偏光などの現象が理解しやすくなります。
また、重力によって時空超立方体が変形すると光の進行方向が変化すると想定できます。

\subsection{4元運動量とラピディティの関係}

相対性理論では運動量$\bm{p}$と速度$\bm{v}$は以下の関係があります。
\begin{equation}
  \bm{p}=\gamma m\bm{v}, \quad \gamma=\frac{1}{\sqrt{1-\beta^2}}, \quad \beta=\frac{|\bm{v}|}{c}
\end{equation}
ここで$\bm{p}$および$\bm{v}$は三次元ベクトルの範囲とします。展開して変形していきます。
$p=|\bm{p}|, v=|\bm{v}|$とします。
\begin{equation}
  \begin{split}
    &\bm{p} = \frac{m \bm{v}}{\sqrt{1 - \frac{v^2}{c^2}}} \quad \Leftrightarrow \quad
    p^2(1 - \frac{v^2}{c^2}) = m^2 v^2 \quad \Leftrightarrow \quad
    m^2 v^2 + \frac{p^2 v^2}{c^2} = p^2 \quad \Leftrightarrow \quad
    (m^2 + \frac{p^2}{c^2}) v^2 = p^2\\
    &\quad \Leftrightarrow \quad (\frac{m^2 c^2 + p^2}{c^2}) v^2 = p^2 \quad \Leftrightarrow \quad
    v^2 = p^2(\frac{c^2}{m^2 c^2 + p^2})\\
    &v = \frac{pc}{\sqrt{m^2 c^2 + p^2}}, \quad v \to c \quad (p \to \infty, \quad m \to \infty),
    \quad v = c \quad (m = 0)\\
  \end{split}
\end{equation}
運動量や質量が無限大に増大しても破綻しません。速さは光速に近づいていきます。
質量がゼロの時、速度は光速になります。
ラピディティ$ \Theta $を求めていくと、
\begin{equation}
  \begin{split}
    &\beta = \frac{v}{c} = \tanh \Theta = \frac{p}{\sqrt{m^2 c^2 + p^2}} = \frac{pc}{E} < 1
    , \quad (p \to \infty, \quad m \to \infty)\\
    &\texttt{Rapidity}: \quad \Theta = \arctanh \frac{p}{\sqrt{m^2 c^2 + p^2}}
    , \quad \Theta \to \infty \quad (p \to \infty)\\
  \end{split}
\end{equation}

\subsection{双対時空での重力計算}

ここまで時空連続体仮説を棄却して代わりに時空超立方体の性質を調べ、
運動量は時空超立方体に関連した量と考えられる事を見てきました。
重力として物体に働く力は、力というより時空超立方体の歪みであり、同時に運動量を歪ませる量と考えて来ています。
物体もしくは粒子に働く力は、基本的に運動量の交換になります。
物体が時空超立方体の歪みに抵抗し、物体の運動量の変化に反映されなかった場合は、それは力が働いたとは言えない事になります。
日常的には時空超立方体の歪みに抵抗した力を重力と感じています。
双対時空で考えると運動量交換の原理は変わらず、$ P^\dag $の内、
$ (p_1, p_2, p_3) $の三次元ベクトルにおいて交換されると考えられます。
不変量$ (P^\dag)^2 $は$ E $の大きさを調整することで保たれます。
従って、自由落下のみを計算する多体シミュレーションで考えた場合では、運動量の交換量が線形的に加算されます。

では、具体的な計算に入りましょう。まず2体で考えます。物体のパラメーターを以下の様に定義します。
\\\\
\begin{tabular}{|l|c|c|c|} \hline
  & 質量 & 運動量(3次元ベクトル) & 位置(3次元座標) \\ \hline
  物体A & $ m_a $ & $ \bm{p}_a $ & $ \bm{x}_a $ \\ \hline
  物体B & $ m_b $ & $ \bm{p}_b $ & $ \bm{x}_b $ \\ \hline
\end{tabular}
\\\\
物体Aと物体Bは重力によって運動量を交換し、交換の結果、運動量が反対向きに同じ量だけ変化します。
\begin{itemize}
  \item 物体Aの運動量変化: $ \Delta \bm{p}_a = \Delta \bm{p} $
  \item 物体Bの運動量変化: $ \Delta \bm{p}_b = - \Delta \bm{p} $
  \item 物体A,B間の距離: $ r = |\bm{x}_b - \bm{x}_a| $
  \item 物体AからBへの方向: $ \hat{\bm{r}} = (\bm{x}_b - \bm{x}_a) / r $
  \item 万有引力定数: $G$
  \item 微小時間ステップ: $\Delta t$
\end{itemize}
重力による力積をニュートン近似を用いて次のように計算します。
\begin{equation}
  \Delta \bm{p} = \Delta t \frac{G m_a m_b}{r^2} \hat{\bm{r}} = \Delta t \frac{G m_a m_b}{r^3} (\bm{x}_b - \bm{x}_a)
\end{equation}
物体Aを含めてN体の物体があるとすると、$ N(N - 1) / 2 $の運動量の交換関係があります。物体Aに関して和を計算します。
\begin{equation}
  \bm{p}_{total} = \sum_{n = 1}^{N - 1} \Delta \bm{p}_n
\end{equation}
運動量に関しては線形的にどれだけでも加算することが出来るという前提です。
この$ \bm{p}_{total} $が物体の持つ双対時空に作用します。
つまり、その物体が状態として持つ局所的な時空立方体を歪めます。
ある瞬間に浮遊する物体が持っている双対時空は物体の持つ座標と一致しています。
\begin{equation}
  \begin{split}
    S = ~ & t e_0 + x e_1 + y e_2 + z e_3 \\
    S^\dag = ~ & (t e_0 + x e_1 + y e_2 + z e_3)e_0e_1e_2e_3 
    = t e_1e_2e_3 + x e_0e_2e_3 - y e_0e_1e_3 + z e_0e_1e_2 \\
  \end{split}
\end{equation}
四双四元数で表現すると、
\begin{equation}
  \begin{split}
    S = ~ & t j + x kI + y kJ + z kK \\
    S^\dag = ~ & i(t j + x kI + y kJ + z kK) = t k - x jI - y jJ - z jK \\
  \end{split}
\end{equation}
\begin{itemize}
  \item ある瞬間の物体の双対時空: $ S^\dag = t e_1e_2e_3 + x e_0e_2e_3 - y e_0e_1e_3 + z e_0e_1e_2 $
\end{itemize}
双対混合回転子で表現すると、
\begin{equation}
  \bm{u} = \frac{\bm{p}_a}{|\bm{p}_a|}
  , \quad \bm{p} = \frac{\bm{p}_{total}}{|\bm{p}_{total}|}
  , \quad \bm{u} \cdot \bm{v} = 0
\end{equation}
\begin{equation}
  \cos \sigma = \bm{u} \cdot \bm{p}
  , \quad \phi = a \cos \sigma
  , \quad \theta = a \sin \sigma
\end{equation}
\begin{equation}
  \bm{v} = \frac{\bm{p} - \bm{u} \cos \sigma}{\sin \sigma}
  , \quad 0 < \sigma < \pi
\end{equation}
\begin{equation}
  R^\dag = \left(\cos \frac{\phi}{2} + \bm{u}^\dag \sin \frac{\phi}{2} \right)
  \left(\cosh \frac{\theta}{2} + \bm{v}^\dag \sinh \frac{\theta}{2} \right)
\end{equation}
運動量の方向と重力の方向が一致するとして、左側だけを計算していきます。
\begin{equation}
  R = \exp \left(\frac{\phi}{2} \bm{u} \right) 
  = \exp \left(\frac{\phi}{2} (u_1 e_0e_1 + u_2 e_0e_2 + u_3 e_0e_3) \right)
\end{equation}
時空的双対回転子は、$ \bm{u} $に$ e_0e_1e_2e_3 $をかけて構成するので、
\begin{equation}
  \begin{split}
    \bm{u}^\dag = \bm{u} e_0e_1e_2e_3 = ~ & u_1 e_0e_1e_0e_1e_2e_3 + u_2 e_0e_2e_0e_1e_2e_3 + u_3 e_0e_3e_0e_1e_2e_3 \\
    = ~ & u_1 e_2e_3 - u_2 e_1e_3 + u_3 e_1e_2 \\
    = ~ & u_3 e_1e_2 - u_2 e_1e_3 + u_1 e_2e_3 \\
  \end{split}
\end{equation}
\begin{equation}
  \begin{split}
    (\bm{u}^\dag)^2 = ~ & (u_3 e_1e_2 - u_2 e_1e_3 + u_1 e_2e_3)^2 \\
    = ~ & u_3^2 e_1e_2e_1e_2 - u_2 u_3 e_1e_2e_1e_3 + u_1 u_3 e_1e_2e_2e_3 \\
    & - u_2 u_3 e_1e_3e_1e_2 + u_2^2 e_1e_3e_1e_3 - u_1 u_2 e_1e_3e_2e_3 \\
    & + u_1 u_3 e_2e_3e_1e_2 - u_1 u_2 e_2e_3e_1e_3 + u_1^2 e_2e_3e_2e_3 \\
    = ~ & - u_3^2 - u_2 u_3 e_2e_3 - u_1 u_3 e_1e_3 \\
    & + u_2 u_3 e_2e_3 - u_2^2 - u_1 u_2 e_1e_2 \\
    & + u_1 u_3 e_1e_3 + u_1 u_2 e_1e_2 - u_1^2 \\
    = ~ & - u_1^2 - u_2^2 - u_3^2 \\
  \end{split}
\end{equation}
従って、$ \bm{u}^\dag $は四元数を使ったベルソルと同型となっています。
\begin{equation}
  R^\dag = \exp \left(\frac{\phi}{2} \bm{u}^\dag \right)
  = \exp \left(\frac{\phi}{2}(u_3 e_1e_2 - u_2 e_1e_3 + u_1 e_2e_3) \right)
  = \cos \frac{\phi}{2} + \bm{u}^\dag \sin \frac{\phi}{2}
\end{equation}
サンドイッチ積により重力の影響で歪んだ双対時空$ (S^\dag)' = R^\dag S^\dag \widetilde{R^\dag} $を計算します。
$ p = \cos \frac{\phi}{2} , q = \sin \frac{\phi}{2} $とします。
\begin{equation}
  \begin{split}
    R^\dag S^\dag \widetilde{R^\dag} = ~ & (p + \bm{u}^\dag q)S^\dag(p - \bm{u}^\dag q) \\
    = ~ & (p S^\dag + q \bm{u}^\dag S^\dag)(p - \bm{u}^\dag q) \\
    = ~ & p^2 S^\dag + p q \bm{u}^\dag S^\dag - p q S^\dag \bm{u}^\dag - q^2 \bm{u}^\dag S^\dag \bm{u}^\dag \\
  \end{split}
\end{equation}
\begin{equation}
  \begin{split}
    \bm{u}^\dag S^\dag = ~ & (u_3 e_1e_2 - u_2 e_1e_3 + u_1 e_2e_3)(z e_0e_1e_2 - y e_0e_1e_3 + x e_0e_2e_3 + t e_1e_2e_3) \\
    = ~ & z u_3 e_1e_2 e_0e_1e_2 - y u_3 e_1e_2 e_0e_1e_3 + x u_3 e_1e_2 e_0e_2e_3 + t u_3 e_1e_2 e_1e_2e_3 \\
    & - z u_2 e_1e_3 e_0e_1e_2 + y u_2 e_1e_3 e_0e_1e_3 - x u_2 e_1e_3 e_0e_2e_3 - t u_2 e_1e_3 e_1e_2e_3 \\
    & + z u_1 e_2e_3 e_0e_1e_2 - y u_1 e_2e_3 e_0e_1e_3 + x u_1 e_2e_3 e_0e_2e_3 + t u_1 e_2e_3 e_1e_2e_3 \\
    = ~ & - z u_3 e_0 - y u_3 e_0e_2e_3 - x u_3 e_0e_1e_3 - t u_3 e_3 \\
    & + z u_2 e_0e_2e_3 - y u_2 e_0 - x u_2 e_0e_1e_2 - t u_2 e_2 \\
    & + z u_1 e_0e_1e_3 + y u_1 e_0e_1e_2 - x u_1 e_0 - t u_1 e_1 \\
    = ~ & (- x u_1 - y u_2 - z u_3) e_0 - t u_1 e_1 - t u_2 e_2 - t u_3 e_3 \\
    & + (y u_1 - x u_2) e_0e_1e_2 + (z u_1 - x u_3) e_0e_1e_3 + (z u_2 - y u_3) e_0e_2e_3\\
  \end{split}
\end{equation}
\begin{equation}
  \begin{split}
    S^\dag \bm{u}^\dag = ~ & (z e_0e_1e_2 - y e_0e_1e_3 + x e_0e_2e_3 + t e_1e_2e_3)(u_3 e_1e_2 - u_2 e_1e_3 + u_1 e_2e_3) \\
    = ~ & z u_3 e_0e_1e_2 e_1e_2 - y u_3 e_0e_1e_3 e_1e_2 + x u_3 e_0e_2e_3 e_1e_2 + t u_3 e_1e_2e_3 e_1e_2 \\
    & - z u_2 e_0e_1e_2 e_1e_3 + y u_2 e_0e_1e_3 e_1e_3 - x u_2 e_0e_2e_3 e_1e_3 - t u_2 e_1e_2e_3 e_1e_3 \\
    & + z u_1 e_0e_1e_2 e_2e_3 - y u_1 e_0e_1e_3 e_2e_3 + x u_1 e_0e_2e_3 e_2e_3 + t u_1 e_1e_2e_3 e_2e_3 \\
    = ~ & - z u_3 e_0 + y u_3 e_0e_2e_3 + x u_3 e_0e_1e_3 - t u_3 e_3 \\
    & - z u_2 e_0e_2e_3 - y u_2 e_0 + x u_2 e_0e_1e_2 - t u_2 e_2 \\
    & - z u_1 e_0e_1e_3 - y u_1 e_0e_1e_2 - x u_1 e_0 - t u_1 e_1 \\
    = ~ & (- x u_1 - y u_2 - z u_3) e_0 - t u_1 e_1 - t u_2 e_2 - t u_3 e_3 \\
    & - (y u_1 - x u_2) e_0e_1e_2 - (z u_1 - x u_3) e_0e_1e_3 - (z u_2 - y u_3) e_0e_2e_3\\
  \end{split}
\end{equation}
\begin{equation}
  A_0 = - x u_1 - y u_2 - z u_3 , \quad
  A_1 = y u_1 - x u_2 , \quad
  A_2 = z u_1 - x u_3 , \quad
  A_3 = z u_2 - y u_3
\end{equation}
と置いて整理すると、
\begin{equation}
  \begin{split}
    \bm{u}^\dag S^\dag = ~ & A_0 e_0 - t u_1 e_1 - t u_2 e_2 - t u_3 e_3 + A_1 e_0e_1e_2 + A_2 e_0e_1e_3 + A_3 e_0e_2e_3\\
    S^\dag \bm{u}^\dag = ~ & A_0 e_0 - t u_1 e_1 - t u_2 e_2 - t u_3 e_3 - A_1 e_0e_1e_2 - A_2 e_0e_1e_3 - A_3 e_0e_2e_3\\
  \end{split}
\end{equation}
\begin{equation}
  \begin{split}
    p q \bm{u}^\dag S^\dag - p q S^\dag \bm{u}^\dag = ~ & 2 p q (A_1 e_0e_1e_2 + A_2 e_0e_1e_3 + A_3 e_0e_2e_3) \\
    = ~ & 2 p q A_1 e_0e_1e_2 + 2 p q A_2 e_0e_1e_3 + 2 p q A_3 e_0e_2e_3 \\
    = ~ & 2 p q (y u_1 - x u_2) e_0e_1e_2 + 2 p q (z u_1 - x u_3) e_0e_1e_3 + 2 p q (z u_2 - y u_3) e_0e_2e_3 \\
  \end{split}
\end{equation}
\begin{equation}
  \begin{split}
    \bm{u}^\dag S^\dag \bm{u}^\dag
    = ~ & (A_0 e_0 - t u_1 e_1 - t u_2 e_2 - t u_3 e_3 + A_1 e_0e_1e_2 + A_2 e_0e_1e_3 + A_3 e_0e_2e_3)
    (u_3 e_1e_2 - u_2 e_1e_3 + u_1 e_2e_3) \\
    = ~ & A_0 u_3 e_0e_1e_2 + t u_1 u_3 e_2 - t u_2 u_3 e_1 - t u_3^2 e_1e_2e_3 - A_1 u_3 e_0 - A_2 u_3 e_0e_2e_3 + A_3 u_3 e_0e_1e_3 \\
    & - A_0 u_2 e_0e_1e_3 - t u_1 u_2 e_3 - t u_2^2 e_1e_2e_3 + t u_2 u_3 e_1 - A_1 u_2 e_0e_2e_3 + A_2 u_2 e_0 + A_3 u_2 e_0e_1e_2 \\
    & + A_0 u_1 e_0e_2e_3 - t u_1^2 e_1e_2e_3 + t u_1 u_2 e_3 - t u_1 u_3 e_2 - A_1 u_1 e_0e_1e_3 + A_2 u_1 e_0e_1e_2 - A_3 u_1 e_0 \\
    = ~ & (- A_3 u_1 + A_2 u_2 - A_1 u_3) e_0 + (A_2 u_1 + A_3 u_2 + A_0 u_3) e_0e_1e_2 + (- A_1 u_1 - A_0 u_2 + A_3 u_3) e_0e_1e_3 \\
    & + (A_0 u_1 - A_1 u_2 - A_2 u_3) e_0e_2e_3 - t e_1e_2e_3 \\
  \end{split}
\end{equation}
係数を展開すると、
\begin{equation}
  \begin{split}
    - A_3 u_1 + A_2 u_2 - A_1 u_3 = ~ & - (z u_2 - y u_3) u_1 + (z u_1 - x u_3) u_2 - (y u_1 - x u_2) u_3 \\
    = ~ & - z u_1 u_2 + y u_1 u_3 + z u_1 u_2 - x u_2 u_3 - y u_1 u_3 + x u_2 u_3 \\
    = ~ & 0 \\
    A_2 u_1 + A_3 u_2 + A_0 u_3 = ~ & (z u_1 - x u_3) u_1 + (z u_2 - y u_3) u_2 + (- x u_1 - y u_2 - z u_3) u_3 \\
    = ~ & z u_1^2 - x u_1 u_3 + z u_2^2 - y u_2 u_3 - x u_1 u_3 - y u_2 u_3 - z u_3^2 \\
    = ~ & - 2 x u_1 u_3 - 2 y u_2 u_3 + z (u_1^2 + u_2^2 - u_3^2) \\
    - A_1 u_1 - A_0 u_2 + A_3 u_3 = ~ & - (y u_1 - x u_2) u_1 - (- x u_1 - y u_2 - z u_3) u_2 + (z u_2 - y u_3) u_3 \\
    = ~ & - y u_1^2 + x u_1 u_2 + x u_1  u_2 + y u_2^2 + z u_2 u_3 + z u_2 u_3 - y u_3^2 \\
    = ~ & 2 x u_1 u_2 + y ( - u_1^2 + u_2^2 - u_3^2) + 2 z u_2 u_3 \\
    A_0 u_1 - A_1 u_2 - A_2 u_3 = ~ & (- x u_1 - y u_2 - z u_3) u_1 - (y u_1 - x u_2) u_2 - (z u_1 - x u_3) u_3 \\
    = ~ & - x u_1^2 - y u_1 u_2 - z u_1 u_3 - y u_1 u_2 + x u_2^2 - z u_1 u_3 + x u_3^2 \\
    = ~ & x ( - u_1^2 + u_2^2 + u_3^2) - 2 y u_1 u_2 - 2 z u_1 u_3 \\
  \end{split}
\end{equation}
\begin{equation}
  \begin{split}
    \bm{u}^\dag S^\dag \bm{u}^\dag
    = ~ & (- 2 x u_1 u_3 - 2 y u_2 u_3 + z (u_1^2 + u_2^2 - u_3^2)) e_0e_1e_2 \\
    & + (2 x u_1 u_2 + y ( - u_1^2 + u_2^2 - u_3^2) + 2 z u_2 u_3) e_0e_1e_3 \\
    & + (x ( - u_1^2 + u_2^2 + u_3^2) - 2 y u_1 u_2 - 2 z u_1 u_3) e_0e_2e_3 - t e_1e_2e_3 \\
  \end{split}
\end{equation}
\begin{equation}
  \begin{split}
    R^\dag S^\dag \widetilde{R^\dag} 
    = ~ & p^2 S^\dag + p q \bm{u}^\dag S^\dag - p q S^\dag \bm{u}^\dag - q^2 \bm{u}^\dag S^\dag \bm{u}^\dag \\
    = ~ & p^2 z e_0e_1e_2 - p^2 y e_0e_1e_3 + p^2 x e_0e_2e_3 + p^2 t e_1e_2e_3 \\
    & + 2 p q (y u_1 - x u_2) e_0e_1e_2 + 2 p q (z u_1 - x u_3) e_0e_1e_3 + 2 p q (z u_2 - y u_3) e_0e_2e_3 \\
    & - q^2 (- 2 x u_1 u_3 - 2 y u_2 u_3 + z (u_1^2 + u_2^2 - u_3^2)) e_0e_1e_2 \\
    & - q^2 (2 x u_1 u_2 + y ( - u_1^2 + u_2^2 - u_3^2) + 2 z u_2 u_3) e_0e_1e_3 \\
    & - q^2 (x ( - u_1^2 + u_2^2 + u_3^2) - 2 y u_1 u_2 - 2 z u_1 u_3) e_0e_2e_3 \\
    & + q^2 t e_1e_2e_3 \\
    = ~ &(p^2 z + 2 p q (y u_1 - x u_2) - q^2 (- 2 x u_1 u_3 - 2 y u_2 u_3 + z (u_1^2 + u_2^2 - u_3^2))) e_0e_1e_2 \\
    & + (- p^2 y + 2 p q (z u_1 - x u_3) - q^2 (2 x u_1 u_2 + y ( - u_1^2 + u_2^2 - u_3^2) + 2 z u_2 u_3)) e_0e_1e_3 \\
    & + (p^2 x + 2 p q (z u_2 - y u_3) - q^2 (x ( - u_1^2 + u_2^2 + u_3^2) - 2 y u_1 u_2 - 2 z u_1 u_3)) e_0e_2e_3 \\
    & + t e_1e_2e_3 \\
    = ~ &(p^2 z + 2 p q (y u_1 - x u_2) + q^2 (2 x u_1 u_3 + 2 y u_2 u_3 + z ( - u_1^2 - u_2^2 + u_3^2))) e_0e_1e_2 \\
    & - (p^2 y + 2 p q (x u_3 - z u_1) + q^2 (2 x u_1 u_2 + y ( - u_1^2 + u_2^2 - u_3^2) + 2 z u_2 u_3)) e_0e_1e_3 \\
    & + (p^2 x + 2 p q (z u_2 - y u_3) + q^2 (x (u_1^2 - u_2^2 - u_3^2) + 2 y u_1 u_2 + 2 z u_1 u_3)) e_0e_2e_3 \\
    & + t e_1e_2e_3 \\
  \end{split}
\end{equation}
これは四元数による三次元の回転と同型と考えられます。以下の様に置き換えます。
\begin{equation}
  q_1 = q u_1 , \quad q_2 = q u_2 , \quad q_3 = q u_3
\end{equation}
\begin{equation}
  R^\dag = \cos \frac{\phi}{2} + \bm{u}^\dag \sin \frac{\phi}{2}
  = p + q u_3 e_1e_2 - q u_2 e_1e_3 + q u_1 e_2e_3
  = p + q_3 e_1e_2 - q_2 e_1e_3 + q_1 e_2e_3
\end{equation}
\begin{equation}
  \left\{
    \begin{aligned}
      &x' = p^2 x + 2 p q (z u_2 - y u_3) + q^2 (x (u_1^2 - u_2^2 - u_3^2) + 2 y u_1 u_2 + 2 z u_1 u_3) \\
      &y' = p^2 y + 2 p q (x u_3 - z u_1) + q^2 (2 x u_1 u_2 + y ( - u_1^2 + u_2^2 - u_3^2) + 2 z u_2 u_3) \\
      &z' = p^2 z + 2 p q (y u_1 - x u_2) + q^2 (2 x u_1 u_3 + 2 y u_2 u_3 + z ( - u_1^2 - u_2^2 + u_3^2)) \\
    \end{aligned}
  \right.
\end{equation}
\begin{equation}
  \Longleftrightarrow
  \left\{
    \begin{aligned}
      &x' = p^2 x + 2 p (z q_2 - y q_3) + x (q_1^2 - q_2^2 - q_3^2) + 2 y q_1 q_2 + 2 z q_1 q_3 \\
      &y' = p^2 y + 2 p (x q_3 - z q_1) + 2 x q_1 q_2 + y ( - q_1^2 + q_2^2 - q_3^2) + 2 z q_2 q_3 \\
      &z' = p^2 z + 2 p (y q_1 - x q_2) + 2 x q_1 q_3 + 2 y q_2 q_3 + z ( - q_1^2 - q_2^2 + q_3^2) \\
    \end{aligned}
  \right.
\end{equation}
\begin{equation}
  \Longleftrightarrow
  \left\{
    \begin{aligned}
      &x' = (p^2 + q_1^2 - q_2^2 - q_3^2) x + 2 (z p q_2 - y p q_3 + y q_1 q_2 + z q_1 q_3) \\
      &y' = (p^2 - q_1^2 + q_2^2 - q_3^2) y + 2 (x p q_3 - z p q_1 + x q_1 q_2 + z q_2 q_3) \\
      &z' = (p^2 - q_1^2 - q_2^2 + q_3^2) z + 2 (y p q_1 - x p q_2 + x q_1 q_3 + y q_2 q_3) \\
    \end{aligned}
  \right.
\end{equation}
\begin{equation}
  \Longleftrightarrow
  \left\{
    \begin{aligned}
      &x' = (p^2 + q_1^2 - q_2^2 - q_3^2) x + 2 ((q_1 q_2 - p q_3) y + (q_1 q_3 + p q_2) z) \\
      &y' = (p^2 - q_1^2 + q_2^2 - q_3^2) y + 2 ((q_1 q_2 + p q_3) x + (q_2 q_3 - p q_1) z) \\
      &z' = (p^2 - q_1^2 - q_2^2 + q_3^2) z + 2 ((q_1 q_3 - p q_2) x + (q_2 q_3 + p q_1) y) \\
    \end{aligned}
  \right.
\end{equation}
四元数による三次元の回転計算とは符号が異なる部分がありますが、
サンドイッチ積の順序が逆になっていて回転角度が逆になったと考えられます。
本質的な部分は変わりません。

一方向だけで計算してみましょう。
$ u_1 = 1 , u_2 = 0 , u_3 = 0$とすると、
\begin{equation}
  \begin{split}
    R^\dag S^\dag \widetilde{R^\dag} 
    = ~ & (p^2 z + 2 p q y - q^2 z) e_0e_1e_2 - (p^2 y - 2 p q z - q^2 y) e_0e_1e_3
    + x e_0e_2e_3 + t e_1e_2e_3 \\
  \end{split}
\end{equation}
\begin{equation}
  \begin{split}
    2 p q = 2 \cos \frac{\phi}{2} \sin \frac{\phi}{2} = \sin \phi , \quad
    p^2 - q^2 = \cos \phi
  \end{split}
\end{equation}
\begin{equation}
  \begin{split}
    R^\dag S^\dag \widetilde{R^\dag} 
    = ~ & (\sin \phi y + \cos \phi z) e_0e_1e_2 - (\cos \phi y - \sin \phi z) e_0e_1e_3
    + x e_0e_2e_3 + t e_1e_2e_3 \\
  \end{split}
\end{equation}
角度を中心にまとめると、
\begin{equation}
  \left\{
    \begin{aligned}
      &x' = p^2 x + 2 p q (z u_2 - y u_3) + q^2 (x (u_1^2 - u_2^2 - u_3^2) + 2 y u_1 u_2 + 2 z u_1 u_3) \\
      &y' = p^2 y + 2 p q (x u_3 - z u_1) + q^2 (2 x u_1 u_2 + y ( - u_1^2 + u_2^2 - u_3^2) + 2 z u_2 u_3) \\
      &z' = p^2 z + 2 p q (y u_1 - x u_2) + q^2 (2 x u_1 u_3 + 2 y u_2 u_3 + z ( - u_1^2 - u_2^2 + u_3^2)) \\
    \end{aligned}
  \right.
\end{equation}
\begin{equation}
  \Longleftrightarrow
  \left\{
    \begin{aligned}
      &x' = (1 - q^2) x + 2 p q (z u_2 - y u_3) + q^2 (x (u_1^2 - u_2^2 - u_3^2) + 2 y u_1 u_2 + 2 z u_1 u_3) \\
      &y' = (1 - q^2) y + 2 p q (x u_3 - z u_1) + q^2 (y ( - u_1^2 + u_2^2 - u_3^2) + 2 x u_1 u_2 + 2 z u_2 u_3) \\
      &z' = (1 - q^2) z + 2 p q (y u_1 - x u_2) + q^2 (z ( - u_1^2 - u_2^2 + u_3^2) + 2 x u_1 u_3 + 2 y u_2 u_3) \\
    \end{aligned}
  \right.
\end{equation}
\begin{equation}
  \Longleftrightarrow
  \left\{
    \begin{aligned}
      &x' = x + 2 p q (z u_2 - y u_3) + q^2 (x (u_1^2 - u_2^2 - u_3^2 - 1) + 2 (y u_1 u_2 + z u_1 u_3)) \\
      &y' = y + 2 p q (x u_3 - z u_1) + q^2 (y (u_2^2 - u_3^2 - u_1^2 - 1) + 2 (x u_1 u_2 + z u_2 u_3)) \\
      &z' = z + 2 p q (y u_1 - x u_2) + q^2 (z (u_3^2 - u_1^2 - u_2^2 - 1) + 2 (x u_1 u_3 + y u_2 u_3)) \\
    \end{aligned}
  \right.
\end{equation}
\begin{equation}
  \Longleftrightarrow
  \left\{
    \begin{aligned}
      &x' = x + (z u_2 - y u_3) \sin \phi + (x (u_1^2 - u_2^2 - u_3^2 - 1) + 2 (y u_1 u_2 + z u_1 u_3)) \sin^2 \frac{\phi}{2} \\
      &y' = y + (x u_3 - z u_1) \sin \phi + (y (u_2^2 - u_3^2 - u_1^2 - 1) + 2 (x u_1 u_2 + z u_2 u_3)) \sin^2 \frac{\phi}{2} \\
      &z' = z + (y u_1 - x u_2) \sin \phi + (z (u_3^2 - u_1^2 - u_2^2 - 1) + 2 (x u_1 u_3 + y u_2 u_3)) \sin^2 \frac{\phi}{2} \\
    \end{aligned}
  \right.
\end{equation}
\begin{equation}
  \Longleftrightarrow
  \left\{
    \begin{aligned}
      &x' = x + (z u_2 - y u_3) \sin \phi + ( - 2 x (1 - u_1^2) + 2 (y u_1 u_2 + z u_1 u_3)) \sin^2 \frac{\phi}{2} \\
      &y' = y + (x u_3 - z u_1) \sin \phi + ( - 2 y (1 - u_2^2) + 2 (x u_1 u_2 + z u_2 u_3)) \sin^2 \frac{\phi}{2} \\
      &z' = z + (y u_1 - x u_2) \sin \phi + ( - 2 z (1 - u_3^2) + 2 (x u_1 u_3 + y u_2 u_3)) \sin^2 \frac{\phi}{2} \\
    \end{aligned}
  \right.
\end{equation}
\begin{equation}
  \Longleftrightarrow
  \left\{
    \begin{aligned}
      &x' = x + (z u_2 - y u_3) \sin \phi + 2 (x (u_1^2 - 1) + y u_1 u_2 + z u_1 u_3) \sin^2 \frac{\phi}{2} \\
      &y' = y + (x u_3 - z u_1) \sin \phi + 2 (y (u_2^2 - 1) + x u_1 u_2 + z u_2 u_3) \sin^2 \frac{\phi}{2} \\
      &z' = z + (y u_1 - x u_2) \sin \phi + 2 (z (u_3^2 - 1) + x u_1 u_3 + y u_2 u_3) \sin^2 \frac{\phi}{2} \\
    \end{aligned}
  \right.
\end{equation}
循環関数になりますので重力の増大の結果、運動量への影響が反転するという事を言っています。
重力反転が起きるという事は信じられない主張かと思いますが、天文学的観測結果との整合性について後ほど議論していきたいと思います。

どうして、直接的に時空に重力の影響を考えるのではなく、双対時空へかける計算を行うか、その必然性を説明します。
ラピディティの総和を直接時空にかけても重力多体シミュレーションとして一応成立します。
例えば、水星と太陽の要素をインプットすれば回転する楕円軌道が再現できます。
しかし、このシミュレーション方式では重力が大きくなりすぎると無限大でエラーになって粒子の動きが止まってしまいます。
重力の増大で計算不能になる原因はコンピューターの能力の限界ではなく、
物理法則としての実在性について問題がある事を示していると考えます。
無限大が出現する原因は双曲線関数の適用にあるので、三角関数による回転によればその矛盾は回避されると考えました。

\subsection{双対時空から加速度として速度に反映}

重力による影響の結果、双対時空の時空立方体が回転し歪む様子を、三角関数と双曲線関数の混合回転で見てきました。
物体の運動量は時空立方体に沿って進行すると考えられるので、時空立方体が変形すれば運動量が見かけ上変化するように見えます。
物体側からすると運動量を保っているので、慣性力は感じられず無重力浮遊の状態です。
双対時空で双対回転子によって回転した角度をもって、今度は時空側に通常回転子としてサンドイッチ積を行う事で、
最終的に速度を得る事が出来ます。

双対時空を今度は四双四元数で表現してみます。
\begin{equation}
  \begin{split}
    S = ~ & t j + x kI + y kJ + z kK \\
    S^\dag = ~ & (t j + x kI + y kJ + z kK)i = - t k + x jI + y jJ + z jK \\
  \end{split}
\end{equation}
\begin{itemize}
  \item ある瞬間の物体の双対時空: $ S^\dag = - t k + x jI + y jJ + z jK $
  \item $ Cl(1,3) $表現 : $ S^\dag = t e_1e_2e_3 + x e_0e_2e_3 - y e_0e_1e_3 + z e_0e_1e_2 $
\end{itemize}
四双四元数の方が符号が整っている印象です。クリフォード代数でも基底の順序を入れ替える事で符号を揃えることは可能です。

双対時空$ S^\dag $が重力の影響で$ (S^\dag)' $に変換されているとします。
\begin{equation}
  S^\dag = - t k + x jI + y jJ + z jK, \quad
  (S^\dag)' = - t' k + x' jI + y' jJ + z' jK
\end{equation}
幾何積から内積からこれらの角度を計算します。
\begin{equation}
  \begin{split}
    S^\dag (S^\dag)' = ~ & (- t k + x jI + y jJ + z jK)(- t' k + x' jI + y' jJ + z' jK) \\
    = ~ & - t t' + t x' iI + t y' iJ + t z' iK \\
    & - x t' iI + x x' - x y' K + x z' J \\
    & - y t' iJ + y x' K + y y' - y z' I \\
    & - z t' iK - z x' J + z y' I + z z' \\
    = ~ & - t t' + x x' + y y' + z z' \\
    & + t x' iI - x t' iI + t y' iJ - y t' iJ + t z' iK - z t' iK \\
    & + z y' I - y z' I + x z' J - z x' J + y x' K - x y' K \\
    = ~ & - t t' + x x' + y y' + z z' \\
    & + (t x' - x t') iI + (t y' - y t') iJ + (t z' - z t') iK
    + (z y' - y z') I + (x z' - z x') J + (y x' - x y') K \\
  \end{split}
\end{equation}
内積の符号は四双四元数が$ Cl(3,1) $と同型である事を反映しています。
\begin{equation}
  \begin{split}
    S^\dag (S^\dag)' = ~ & (z e_0e_1e_2 - y e_0e_1e_3 + x e_0e_2e_3 + t e_1e_2e_3)(z' e_0e_1e_2 - y' e_0e_1e_3 + x' e_0e_2e_3 + t' e_1e_2e_3) \\
    = ~ & z z' e_0e_1e_2 e_0e_1e_2 - z y' e_0e_1e_2 e_0e_1e_3 + z x' e_0e_1e_2 e_0e_2e_3 + z t' e_0e_1e_2 e_1e_2e_3 \\
    & - y z' e_0e_1e_3 e_0e_1e_2 + y y' e_0e_1e_3 e_0e_1e_3 - y x' e_0e_1e_3 e_0e_2e_3 - y t' e_0e_1e_3 e_1e_2e_3 \\
    & + x z' e_0e_2e_3 e_0e_1e_2 - x y' e_0e_2e_3 e_0e_1e_3 + x x' e_0e_2e_3 e_0e_2e_3 + x t' e_0e_2e_3 e_1e_2e_3 \\
    & + t z' e_0e_2e_3 e_0e_1e_2 - t y' e_0e_2e_3 e_0e_1e_3 + t x' e_0e_2e_3 e_0e_2e_3 + t t' e_0e_2e_3 e_1e_2e_3 \\
    = ~ & - z z' - z y' e_2e_3 - z x' e_1e_3 - z t' e_0e_3 \\
    & + y z' e_2e_3 - y y' - y x' e_1e_2 - y t' e_0e_2 \\
    & + x z' e_1e_3 + x y' e_1e_2 - x x' - x t' e_0e_1 \\
    & + t z' e_0e_3 + t y' e_0e_2 + t x' e_0e_1 + t t' \\
    = ~ & t t' - x x' - y y' - z z' \\
    & + (t x' - x t') e_0e_1 + (t y' - y t') e_0e_2 + (t z' - z t') e_0e_3
    + (x y' - y x') e_1e_2 + (x z' - z x') e_1e_3 + (y z' - z y') e_2e_3 \\
  \end{split}
\end{equation}
外積については、四双四元数と$ Cl(1,3) $は一致しています。
\begin{equation}
  S^\dag \cdot (S^\dag)' = t t' - x x' - y y' - z z'
\end{equation}
扱う時空が時間的であるとすると、$ S^\dag \cdot S^\dag > 0 $かつ$ (S^\dag)' \cdot (S^\dag)' > 0 $となり、
ミンコフスキー時空の内積の性質から、
\begin{equation}
  S^\dag \cdot (S^\dag)' = \sqrt{S^\dag \cdot S^\dag} \sqrt{(S^\dag)' \cdot (S^\dag)'} \cosh \theta
\end{equation}
$ (S^\dag)' $の不変量は$ S^\dag $の不変量と一致しているはずですから、
\begin{equation}
  S^\dag \cdot (S^\dag)' = \sqrt{(S^\dag)^2} \sqrt{(S^\dag)^2} \cosh \theta = (S^\dag)^2 \cosh \theta
\end{equation}
従って、
\begin{equation}
  \theta = \arccosh \frac{S^\dag \cdot (S^\dag)'}{(S^\dag)^2}
  = \arccosh \frac{t t' - x x' - y y' - z z'}{t^2 - x^2 - y^2 - z^2}
\end{equation}
しかし、この方法では1種類の角度しか求められません。欲しいのは2種類の角度になります。
双対時空へのブーストはそのまま時空への空間回転になるので再計算の必要はないと考えられます。

\subsection{空間部分の回転}

双対時空の空間回転の双対は時空へのブーストになります。
\begin{itemize}
  \item 双対時空単位基底: $ k, c jI, c jJ, c jK $
  \item 回転後: $ k + b_1 c jI + b_2 c jJ + b_3 c jK $
  \item 進行方向の回転子ベルソル: $ \bm{u} = u_1 iI + u_2 iJ + u_3 iK $
  \item 進行方向の回転角: $ \phi $
\end{itemize}
\begin{equation}
  \left\{
    \begin{aligned}
      &x' = x + (z u_2 - y u_3) \sin \phi + 2 (x (u_1^2 - 1) + y u_1 u_2 + z u_1 u_3) \sin^2 \frac{\phi}{2} \\
      &y' = y + (x u_3 - z u_1) \sin \phi + 2 (y (u_2^2 - 1) + x u_1 u_2 + z u_2 u_3) \sin^2 \frac{\phi}{2} \\
      &z' = z + (y u_1 - x u_2) \sin \phi + 2 (z (u_3^2 - 1) + x u_1 u_3 + y u_2 u_3) \sin^2 \frac{\phi}{2} \\
    \end{aligned}
  \right.
\end{equation}
\begin{equation}
  \left\{
    \begin{aligned}
      &b_1 = 1 + (u_2 - u_3) \sin \phi + 2 ((u_1^2 - 1) + u_1 u_2 + u_1 u_3) \sin^2 \frac{\phi}{2} \\
      &b_2 = 1 + (u_3 - u_1) \sin \phi + 2 ((u_2^2 - 1) + u_1 u_2 + u_2 u_3) \sin^2 \frac{\phi}{2} \\
      &b_3 = 1 + (u_1 - u_2) \sin \phi + 2 ((u_3^2 - 1) + u_1 u_3 + u_2 u_3) \sin^2 \frac{\phi}{2} \\
    \end{aligned}
  \right.
\end{equation}
空間部分の内積から、
\begin{equation}
  \begin{split}
    (c jI + c jJ + c jK) \cdot (b_1 c jI + b_2 c jJ + b_3 c jK)
    = ~ & \sqrt{c^2 + c^2 + c^2} \sqrt{b_1^2 c^2 + b_2^2 c^2 + b_3^2 c^2} \cos \phi' \\
    b_1 c^2 + b_2 c^2 + b_3 c^2
    = ~ & c \sqrt{3} c \sqrt{b_1^2 + b_2^2 + b_3^2} \cos \phi' \\
    b_1 + b_2 + b_3
    = ~ & \sqrt{3} \sqrt{b_1^2 + b_2^2 + b_3^2} \cos \phi' \\
  \end{split}
\end{equation}
回転後の長さは変わらないので、
\begin{equation}
  \begin{split}
    & c^2 + c^2 + c^2 = b_1^2 c^2 + b_2^2 c^2 + b_3^2 c^2 \\
    & b_1^2 + b_2^2 + b_3^2 = 3 \\
  \end{split}
\end{equation}
\begin{equation}
  \phi' = \arccos \frac{b_1 + b_2 + b_3}{3}
\end{equation}

\subsubsection{時空立方体の歪みによる時間の遅れ}

$ u_1 = 1, u_2 = 0, u_3 = 0 $の時の回転後の係数は、
\begin{equation}
  \left\{
    \begin{aligned}
      &b_1 = 1 \\
      &b_2 = 1 - \sin \phi - 2 \sin^2 \frac{\phi}{2} \\
      &b_3 = 1 + \sin \phi - 2 \sin^2 \frac{\phi}{2} \\
    \end{aligned}
  \right.
\end{equation}
\begin{equation}
  \cos \phi' = \frac{b_1 + b_2 + b_3}{3} = 1 - \frac{4}{3} \sin^2 \frac{\phi}{2}
\end{equation}
この式は時空超立方体における時間が関わる超平面の3つの合計値が小さくなる事を意味しています。
つまり、物体が速度を持つことによる時間の遅れ(特殊相対論的な時間の遅れ)とは別に、
時空超立方体が歪むことによる時間の遅れを定義する事が出来ます。
全方位的に計算してみると、
\begin{equation}
  \left\{
    \begin{aligned}
      &b_1 = 1 + (u_2 - u_3) \sin \phi + (2 u_1^2 - 2 + 2 u_1 u_2 + 2 u_1 u_3) \sin^2 \frac{\phi}{2} \\
      &b_2 = 1 + (u_3 - u_1) \sin \phi + (2 u_2^2 - 2 + 2 u_1 u_2 + 2 u_2 u_3) \sin^2 \frac{\phi}{2} \\
      &b_3 = 1 + (u_1 - u_2) \sin \phi + (2 u_3^2 - 2 + 2 u_1 u_3 + 2 u_2 u_3) \sin^2 \frac{\phi}{2} \\
    \end{aligned}
  \right.
\end{equation}
\begin{equation}
  \begin{split}
    \frac{b_1 + b_2 + b_3}{3} = ~ & 1 + \frac{4}{3} (-1 + u_1 u_2 + u_1 u_3 + u_2 u_3) \sin^2 \frac{\phi}{2} \\
    = ~ & 1 - \frac{4}{3} \sin^2 \frac{\phi}{2} + \frac{4}{3} (u_1 u_2 + u_1 u_3 + u_2 u_3) \sin^2 \frac{\phi}{2} \\
  \end{split}
\end{equation}
立方体の斜め方向に回転すると値が変わって来ますが、実は直交座標表現の立方体というのは虚実であって、
根底にある角度という本質への変換に必要な項目と考えられます。
座標の方向に合わせた場合については、通常の時間の流れの速さを1とすると時間の遅れは以下の式になります。
\begin{equation}
  \texttt{Time dilation}: \frac{4}{3} \sin^2 \frac{\phi}{2}
\end{equation}
この値は1よりも大きくなるので時間反転を起こすことを意味しています。
時間反転の境界の条件は、
\begin{equation}
  \begin{split}
    & \frac{4}{3} \sin^2 \frac{\phi}{2} = 1
    \quad \leftrightarrow \quad \sin^2 \frac{\phi}{2} = \frac{3}{4}
    \quad \leftrightarrow \quad \sin \frac{\phi}{2} = \frac{\sqrt{3}}{2} \\
    & \leftrightarrow \quad \phi = 2 \arcsin \frac{\sqrt{3}}{2} 
    = \frac{2}{3} \pi, \frac{4}{3} \pi, ... = 120^\circ, 240^\circ, ... \\
  \end{split}
\end{equation}
時間の流れが変われば加速度$ a $も影響を受けるので、時間成分を掛け合わせたラピディティは、
\begin{equation}
  \Theta = \phi' \cos \phi'
\end{equation}
速度ゼロからの加速度で考えると、
\begin{equation}
  \frac{a}{c} = \tanh( \phi' \cos \phi' )
\end{equation}
$ 0 \le \phi < \frac{2}{3} \pi $の範囲で重力が正の方向で、
$ \frac{2}{3} \pi < \phi < \frac{4}{3} \pi $の範囲では重力が負の方向になり、重力反転を起こします。
$ \frac{4}{3} \pi < \phi < \frac{8}{3} \pi $の範囲では重力は再び正の方向で、
$ \frac{8}{3} \pi < \phi < \frac{10}{3} \pi $の範囲で再度重力反転を起こします。

なお、ここで扱っている回転子の角度については一切の物理定数が関わってこない事が式から判明しています。
従って、角度と実際の物理定数との突合は観測によらなければならないです。
地球表面上での時間の遅れの観測値は、時間の流れを1とした時、$ 6.95 \times 10^{-10} $です。
\begin{equation}
  dt = \frac{4}{3} \sin^2 \frac{\phi}{2}
  \quad \leftrightarrow \quad \sin \frac{\phi}{2} = \sqrt{\frac{3}{4} dt}
  \quad \leftrightarrow \quad \phi = 2 \arcsin \sqrt{\frac{3}{4} dt}
\end{equation}
微小角近似$\arcsin x \approx x$を使って、
\begin{equation}
  \phi = 2 \sqrt{\frac{3}{4} dt} = \sqrt{3 dt} = 4.566 \times 10^{-5}
\end{equation}
地球表面の重力加速度$ 9.8 m/s^2 $に比例しているとすると比例定数$ k $は、
\begin{equation}
  k = 4.66 \times 10^{-6}
\end{equation}

\subsection{双対時空への双曲的歪曲}

重力による双対時空への双曲的歪曲$ \phi $の影響を計算していきます。
$ \phi $はラピディティとは異なるので注意しましょう。\\
相対速度の方向のベルソルを$ \bm{v}^\dag = v_1 iI + v_2 iJ + v_3 iK, \quad {v_1}^2 + {v_2}^2 + {v_3}^2 = 1 $として
指数関数の回転子を展開します。
\begin{equation}
  \exp(\bm{v}^\dag) = r_0 + r_1 iI + r_2 iJ + r_3 iK, \quad
  r_0 = \cosh \frac{\theta}{2}, \quad
  (r_1, r_2, r_3) = (v_1, v_2, v_3) \sinh \frac{\theta}{2}
\end{equation}
と置くと、双対時空のブースト$  $は、
\begin{equation}
  \begin{split}
    =~&(r_0^2+r_1^2+r_2^2+r_3^2)t-2r_0(r_1x+r_2y+r_3z)\\
    x'=~&(r_0^2+r_1^2-r_2^2-r_3^2)x-2r_1(r_0t-r_2y-r_3z)\\
    y'=~&(r_0^2-r_1^2+r_2^2-r_3^2)y-2r_2(r_0t-r_1x-r_3z)\\
    z'=~&(r_0^2-r_1^2-r_2^2+r_3^2)z-2r_3(r_0t-r_1x-r_2y)
  \end{split}
\end{equation}
\begin{equation}
  \begin{split}
    t'=~&(r_0^2+r_1^2+r_2^2+r_3^2)t-2r_0(r_1x+r_2y+r_3z)\\
    x'=~&(r_0^2+r_1^2-r_2^2-r_3^2)x-2r_1(r_0t-r_2y-r_3z)\\
    y'=~&(r_0^2-r_1^2+r_2^2-r_3^2)y-2r_2(r_0t-r_1x-r_3z)\\
    z'=~&(r_0^2-r_1^2-r_2^2+r_3^2)z-2r_3(r_0t-r_1x-r_2y)
  \end{split}
\end{equation}


\end{document}

\end{document}