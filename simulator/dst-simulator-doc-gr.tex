\documentclass[a4paper,12pt,notitlepage]{jsreport}
\usepackage[left=10truemm,right=10truemm,top=25truemm,bottom=20truemm]{geometry}
\usepackage{mathtools}
\usepackage{amsmath}
\usepackage{amsfonts}
\usepackage{bm}
\usepackage{setspace}
\usepackage{wrapfig}
\usepackage[dvipdfmx]{hyperref}
\usepackage{pxjahyper}
\usepackage{docmute}
\DeclareMathOperator\arctanh{arctanh}
\DeclareMathOperator\arccosh{arccosh}

\begin{document}

\chapter{クリフォード代数と一般相対性理論}

\section{斜交座標と基底ベクトル}

曲がった空間を表現するためにはこれまで使用してきた直交座標は適用できず斜交座標を使用します。
まず二次元で考えます。
斜交座標ではベクトルの成分は共変と反変の2種類あります。
二次元ベクトル$A$の共変成分$A_1,A_2$と、反変成分$A^1,A^2$があるとします。
\footnote{アインシュタインの縮約記法により、共変成分を下添え字、反変成分を上添え字で表記します。}
反変成分を定義する基底ベクトル$e_1,e_2$と共変成分を定義する双対基底ベクトル$e^1,e^2$については以下のような関係があります。
\footnote{基底ベクトルについては添え字の上下が入れ替わります。}
\begin{equation}
  \begin{split}
    &A=A^1e_1+A^2e_2=A_1e^1+A_2e^2\\
    &e_1\cdot e^2=e_2\cdot e^1=0,\quad e_1\cdot e^1=e_2\cdot e^2=1
  \end{split}
\end{equation}
$e_1,e_2,e^1,e^2$は座標の基準となるベクトルですが長さは$1$とは限らないです。
長さが$1$になる場合は、$e_1,e^1$の成す角度が$0$の場合で、そうなると$e_2,e^2$の成す角度も$0$になり、
$e_1=e^1,e_2=e^2$になり、長さが一斉に$1$にそろい直交座標となる場合です。

次に$A$ベクトルの微小変位$dA$の長さを考えると以下のようになります。
\begin{equation}
  N(dA)=dA\cdot dA=dA_1dA^1+dA_2dA^2=dA^1dA_1+dA^2dA_2
\end{equation}
そもそも長さをこのように定義したいがために基底ベクトルを選んでいます。
アインシュタインの縮約記法で表記すると、
\begin{equation}
  N(dA)=dA_idA^i=dA^idA_i
\end{equation}

ここで共変成分だけを使って長さを表すようにした時、
\begin{equation}
  N(dA)=
  \begin{pmatrix}
    dA_1&dA_2
  \end{pmatrix}
  \begin{pmatrix}
    g_{11}&g_{12}\\g_{21}&g_{22}
  \end{pmatrix}
  \begin{pmatrix}
    dA_1\\dA_2
  \end{pmatrix}
\end{equation}
となるように、
\begin{equation}
  g =
  \begin{pmatrix}
    g_{11}&g_{12}\\g_{21}&g_{22}
  \end{pmatrix}
\end{equation}
を定めると$ g $が空間の特徴を表すようになります。
$ g $を計量テンソルと呼びます。通常の平面であれば、
\begin{equation}
  g =
  \begin{pmatrix}
    1&0\\0&1
  \end{pmatrix}
\end{equation}
となります。$g$は一般的に対称テンソルです。
四次元ベクトルを持ってきて、以下のように$g$を定めれば、
\begin{equation}
  g =
  \begin{pmatrix}
    1&0&0&0\\
    0&-1&0&0\\
    0&0&-1&0\\
    0&0&0&-1\\
  \end{pmatrix}
\end{equation}
特殊相対論的時空の不変量が表現できます。
特殊相対論の時間と空間を論じる時に斜交座標で説明される事もありますが、リーマン幾何学の斜交座標とは異なるので注意しましょう。
特殊相対論では計量$ g $は変化しませんが、リーマン幾何学では$ g $が変化していきます。

斜交座標を持ってきても物差しが変わっただけで空間は曲がっていない事に注意してください。
$ g $を変化させると内積の計算規則が変更されていきます。
$ g $は場所によって変化していくので、位置の関数として表現されます。
空間が曲がるとは、場所によって計量$ g(x) $が変化していく事によって検出されます。
位置の関数となる計量$ g(x) $が、平行移動したときに計量の様態がどのように変化するかが空間の歪曲となります。
ある位置の$ g(x) $が標準と異なっていたとしても、平行移動して$ g(x) $に変化がなければ空間は曲がっていません。

逆に言うと、空間の歪曲を算出するにあたって$ g(x) $の関数の中身を知る必要はありません。
微小移動量$dx$で、$ g(x) $と$ g(x + dx) $の違いが分かればそれで歪曲の度合いが算出できます。

\subsection{クリフォード代数で計量が異なるという事の意味}

クリフォード代数での計量は幾何積のルールとして現れ、平坦な時空でCl(1,3)を選択すれば、計量は、
\begin{equation}
  \eta = [1,-1,-1,-1]
\end{equation}
となり、内積は、
\begin{equation}
  e_0 \cdot e_0 = 1, e_1 \cdot e_1 = e_2 \cdot e_2 = e_3 \cdot e_3 = -1
\end{equation}
\begin{equation}
  e_i \cdot e_i = \eta_i , \quad i=0,1,2,3
\end{equation}
となります。これらが新しい基底$ \nu_i $で表現し直されるとすると、係数を$ \omega_{ij} $として、
\begin{equation}
  \nu_i = \omega_{ij} e_j , \quad \omega_{ij} \in \mathbb{R} , \quad i,j=0,1,2,3
\end{equation}
\begin{equation}
  \nu_i \cdot \nu_i = (\omega_{ij})^2 e_j \cdot e_j = (\omega_{ij})^2 \eta_j
\end{equation}
$ \omega_{ij} $は16個の係数です。つまり、各$ \nu_i $について全ての基本基底のスカラー合成の範囲内にあるとします。
新しい基底$ \nu_i $の新しい計量は、
\begin{equation} g_i = (\omega_{ij})^2 \eta_j \end{equation}
になりました。これでとりあえず入れ物が用意されました。
次に制約を考えていきます。これらは不変量を保つという制約があります。
\begin{equation}
  S^2 = g_0 t^2 + g_1 x^2 + g_2 y^2 + g_3 z^2 = \eta_0 t^2 + \eta_1 x^2 + \eta_2 y^2 + \eta_3 z^2 = t^2 - x^2 - y^2 - z^2
\end{equation}
この制約をクリフォード代数で表現すると回転子による回転に帰着します。
四次元時空では、6種類の双ベクトルにより構成されます。
\begin{equation}
  R = \exp(\gamma_0 e_0e_1 + \gamma_1 e_0e_2 + \gamma_2 e_0e_3 + \gamma_3 e_1e_2 + \gamma_4 e_1e_3 + \gamma_5 e_2e_3)
\end{equation}
四双四元数では、
\begin{equation}
  R = \exp(\omega_0 iI + \omega_1 iJ + \omega_2 iK + \omega_3 I + \omega_4 J + \omega_5 K)
\end{equation}
変形Cl(3,1)では、
\begin{equation}
  R = \exp(\omega_0 \epsilon_0\epsilon_1 + \omega_1 \epsilon_0\epsilon_2 + \omega_2 \epsilon_0\epsilon_3 + 
  \omega_3 \epsilon_3\epsilon_2 + \omega_4 \epsilon_1\epsilon_3 + \omega_5 \epsilon_2\epsilon_1)
\end{equation}
これらの内、左側の時空的な3成分がもたらす回転はローレンツブーストになり、右側の空間的な3成分がもたらす回転は空間回転になります。
\begin{equation}
  S' = R S R^* , \quad R R^* = 1 , \quad S \cdot S = S' \cdot S'
\end{equation}
この回転はまた、双四元数や分解型八元数による回転と同型の、双曲線関数と三角関数の混合タイプになっています。

時空の歪みを算出するにあたっては、$ \omega_{ij} $が具体的にどんな値なのかを知る必要はない事に注意してください。
更に、計量$ g_i $についても知る必要はありません。
平行移動でどれだけ計量$ g_i $が変化するかを知れば歪みがわかりますが、
ここで位置の関数である$ g(x) $が$ R $で表現できる事が分かったわけですから、$ R $も位置による関数$ R(x) $です。
クリフォード代数で計量が異なるという事の意味は、回転子が異なるという意味に他なりません。
時空の歪みは回転子の変化量で表現できるという事になります。

\section{クリフォード代数の時空超立方体}

$Cl(1,3)$において、擬スカラー$e_0e_1e_2e_3$を1-ベクトルにかけると以下のようになります。
\begin{equation}
  \begin{split}
    (te_0+xe_1+ye_2+ze_3)e_0e_1e_2e_3=te_1e_2e_3+xe_0e_2e_3-ye_0e_1e_3+ze_0e_1e_2\\
    e_0e_1e_2e_3(te_1e_2e_3+xe_0e_2e_3-ye_0e_1e_3+ze_0e_1e_2)=te_0+xe_1+ye_2+ze_3
  \end{split}
\end{equation}
左からかけると元に戻ります。
y軸については符号が反転しています。
\begin{equation}
  \begin{split}
    (te_1e_2e_3+xe_0e_2e_3-ye_0e_1e_3+ze_0e_1e_2)^2
    =~&t^2+txe_0e_1+tye_0e_2+tze_0e_3\\
    &-txe_0e_1-x^2-xye_1e_2+xze_1e_3\\
    &-tye_0e_2-xye_1e_2-y^2+zye_2e_3\\
    &-tze_0e_3-xze_1e_3-yze_2e_3-z^2\\
    =~&t^2-x^2-y^2-z^2
  \end{split}
\end{equation}
二乗すると不変量になります。
この様に1-ベクトルに擬スカラーをかけて3-ベクトルにする操作を双対(Duality)と呼ぶ事とします。
時空とその双対時空の和のノルムを計算してみます。

\begin{equation}
  S=te_0+xe_1+ye_2+ze_3,\quad S^\dag=te_1e_2e_3+xe_0e_2e_3-ye_0e_1e_3+ze_0e_1e_2
\end{equation}
と置くと、
\begin{equation}
  Se_0e_1e_2e_3=S^\dag, \quad e_0e_1e_2e_3S^\dag=S
\end{equation}
\begin{equation}
  S^2=(S^\dag)^2=t^2-x^2-y^2-z^2
\end{equation}
\begin{equation}
  \begin{split}
    SS^\dag=~&(te_0+xe_1+ye_2+ze_3)(te_1e_2e_3+xe_0e_2e_3-ye_0e_1e_3+ze_0e_1e_2)\\
    &+t^2e_1e_2e_3e_4+txe_2e_3-tye_1e_3+tze_1e_2\\
    &-txe_2e_3-x^2e_0e_1e_2e_3-xye_0e_3+xze_0e_2\\
    &+tye_1e_3+xye_0e_3-y^2e_0e_1e_2e_3-yze_0e_1\\
    &-tze_1e_2-xze_0e_2+yze_0e_1-z^2e_0e_1e_2e_3\\
    =~&(t^2-x^2-y^2-z^2)e_1e_2e_3e_4\\
    =~&S^2e_1e_2e_3e_4\\
  \end{split}
\end{equation}
\begin{equation}
  \begin{split}
    S^\dag S=~&(te_1e_2e_3+xe_0e_2e_3-ye_0e_1e_3+ze_0e_1e_2)(te_0+xe_1+ye_2+ze_3)\\
    &-t^2e_1e_2e_3e_4-txe_2e_3-tye_1e_3-tze_1e_2\\
    &+txe_2e_3+x^2e_0e_1e_2e_3-xye_0e_3-xze_0e_2\\
    &+tye_1e_3+xye_0e_3+y^2e_0e_1e_2e_3-yze_0e_1\\
    &+tze_1e_2+xze_0e_2+yze_0e_1+z^2e_0e_1e_2e_3\\
    =~&-(t^2-x^2-y^2-z^2)e_1e_2e_3e_4\\
    =~&-S^2e_1e_2e_3e_4\\
  \end{split}
\end{equation}
\begin{equation}
  \begin{split}
    (S+S^\dag)^2=~&S^2+(S^\dag)^2+SS^\dag+S^\dag S\\
    =~&2S^2\\
  \end{split}
\end{equation}
\begin{equation}
  \frac{1}{2}(S+S^\dag)^2=t^2-x^2-y^2-z^2
\end{equation}
和ついても不変量となっている様子です。
双対時空にさらに右から擬スカラーをかけると、
\begin{equation}
  \begin{split}
    (t e_1e_2e_3 + x e_0e_2e_3 - y e_0e_1e_3 + z e_0e_1e_2) e_0e_1e_2e_3 = 
    - t e_0 - x e_1 - y e_2 - z e_3 \\
  \end{split}
\end{equation}
となり全符号が反転します。従って時空に擬スカラーを4回かけるごとに元に戻ります。

\subsection{$ Cl(1,3) $の超立方体}

擬スカラー$ e_0 e_1 e_2 e_3 $は幾何代数的には時空の4次元超立方体の向き付きの体積を表しています。
向きの情報は、$ e_0 e_1 e_2 e_3 $自体が持っています。
双対で算出した3-ベクトルは、4次元超立方体を構成する3次元超平面を形成しています。
係数はその超平面の面積(3次元体積)を表しています。
$ e_1 e_2 e_3 $は、空間超平面を構成しています。
他の$ e_0 $が関わる3つ$ e_0 e_1 e_2, e_0 e_1 e_3, e_0 e_2 e_3 $は、それぞれ時空超平面を構成しています。

単位超立方体を考えてみます。時間と空間を同じスケールで考えると光円錐外になってしまうので、空間は光速と比べ十分小さい単位で考えます。
\begin{itemize}
  \item 単位時空: $ (t, x, y, z) = (c, 1, 1, 1) $
  \item 単位時空超平面: $ (xyz, tyz, txz, txy) = (1, c, c, c) $
\end{itemize}
単位体積は光速$ c $になり、$ e_1 e_2 e_3 $の単位超面積は$ 1 $で、
$ e_0 e_2 e_3, e_0 e_1 e_3, e_0 e_1 e_2 $の単位超面積は$ c $です。
原点と座標$ (c t, x, y, z) $が張る超立方体は、超体積が$ c t x y z $で超面積は、
$ x y z, c t y z, c t x z, c t x y $になります。

\subsection{超平面の法線}

クリフォード代数では、擬ベクトル(ここでは3-ベクトル)の超平面に擬スカラーをかける操作で法線が求まります。
この法線の方向は単位超立方体の辺を構成していた時空座標の基底と一致していると考えられます。

さて、ここで空間の歪みを考えてみましょう。3次元の立方体を思い浮かべてください。
歪んでいない立方体では、x軸の方向と立方体のyz平面の法線は一致しています。
しかし、歪んだ立方体になるとx軸と立方体のyz平面の法線は一致しなくなります。

同様に歪んだ四次元時空についても法線不一致が起きると考えられます。
時空と双対時空超平面の法線の関係は、丁度リーマン幾何学で言う反変ベクトルと共変ベクトルの関係に似ています。

\subsection{双対時空で考える時空の歪曲}

クリフォード代数の構造の中で、1-ブレードと3-ブレードが独立した変数を持ち、その不一致が斜交座標とみなせるとしました。
どのように不一致が起きるのかについては、3-ブレードを不変量を保ったまま変形させると考えます。
そのような変形については、2-ブレードの時空的回転子である、$ e_0 e_1, e_0 e_2, e_0 e_3 $の指数関数による双曲線関数型の回転と、
空間的回転子である、$ e_1 e_2, e_1 e_3, e_2 e_3 $の指数関数による三角関数型の回転の合成になると考えられます。
これらの回転子が組み合わさって、3-ブレードへサンドイッチ積する事によって、超立方体が変形します。

重要なのはこの超立方体の歪みを内部ステータスとして持っていると考えた時、
リーマン幾何学で行われる平行移動はもはや必要なくなるという事です。
従って、時空が滑らかに連続している必要もなくなり、時空連続体仮説が棄却されます。

重力の影響はこの時空立方体へ作用し、それが時間進捗に従って自然に時空へと反映されます。
もし時空立方体の歪みが時空へ反映される事に抗えば、加速度として感じる力に変換されます。
逆に言うと物体が加速度を持つとき、時空側が先に歪んで時空立方体との差が出現する事になると考えます。

\subsection{双対空間回転}

一般的にどの様に3-ブレードへの回転を考えれば良いのか探っていきます。
ここでは、1-ブレードの時空に対する2-ブレードによる回転操作に相当する3-ブレードへの作用が、
2-ブレードの双対による回転操作になると仮定して計算してみます。

具体的に1方向についてだけ計算して様子を見てみましょう。
双ベクトル$e_1e_2$によって、x-y平面における角度$2\theta$の回転を行ってみます。
物体は一定の速さ$|\bm{v}|$で移動しているとします。
移動の方向転換になるのでローレンツブーストにはなりませんが、円運動になるので遠心力が働いていると考えられます。
物体の速度$ \bm{v} $によっては移動しながら回転したり、螺旋運動をしています。
\begin{itemize}
  \item 初期時空: $S=te_0+xe_1+ye_2+ze_3$
  \item 空間的回転子: $R=\exp(\theta e_1e_2)=\cos\theta+e_1e_2\sin\theta$
  \item 空間的回転子リバース: $\widetilde{R}=\cos\theta-e_1e_2\sin\theta$
  \item 初期双対時空: $ S^\dag = t ~ e_1e_2e_3 + x ~ e_0e_2e_3 - y ~ e_0e_1e_3 + z ~ e_0e_1e_2 $
  \item 空間的双対回転子: $R^\dag=\exp(\theta e_1e_2e_0e_1e_2e_3)=\exp(-\theta e_0e_3)=\cosh\theta-e_0e_3\sinh\theta$
  \item 空間的双対回転子リバース: $\widetilde{R}^\dag=\cosh\theta+e_0e_3\sinh\theta$
\end{itemize}

$RS\widetilde{R}$を計算すると、x-y平面での回転になります。こちらの計算の詳細は省略します。

$R^\dag S^\dag\widetilde{R}^\dag$を計算してみます。
\begin{equation}
  \begin{split}
    R^\dag S^\dag=~&(\cosh\theta-e_0e_3\sinh\theta)
    (t e_1e_2e_3 + x e_0e_2e_3 - y e_0e_1e_3 + z e_0e_1e_2) \\
    = ~ & S^\dag \cosh \theta + (t e_0e_1e_2 + x e_2 - y e_1 + z  e_1e_2e_3) \sinh \theta \\
  \end{split}
\end{equation}
\begin{equation}
  \begin{split}
    R^\dag S^\dag \widetilde{R}^\dag = ~ & (S^\dag \cosh \theta
    + (t e_0e_1e_2 + x e_2 - y e_1 + z e_1e_2e_3)\sinh \theta)(\cosh \theta + e_0e_3 \sinh \theta)\\
    =~&S^\dag\cosh^2\theta+(te_0e_1e_2+xe_2-ye_1+ze_1e_2e_3)e_0e_3\sinh^2\theta\\
    &+(S^\dag e_0e_3+te_0e_1e_2+xe_2-ye_1+ze_1e_2e_3)\cosh\theta\sinh\theta\\
    =~&(te_1e_2e_3+xe_0e_2e_3-ye_0e_1e_3+ze_0e_1e_2)\cosh^2\theta+(te_1e_2e_3-xe_0e_2e_3+ye_0e_1e_3+ze_0e_1e_2)\sinh^2\theta\\
    &+((te_1e_2e_3+xe_0e_2e_3-ye_0e_1e_3+ze_0e_1e_2)e_0e_3+te_0e_1e_2+xe_2-ye_1+ze_1e_2e_3)\cosh\theta\sinh\theta\\
    =~&(te_1e_2e_3+xe_0e_2e_3-ye_0e_1e_3+ze_0e_1e_2)\cosh^2\theta+(te_1e_2e_3-xe_0e_2e_3+ye_0e_1e_3+ze_0e_1e_2)\sinh^2\theta\\
    &+(te_0e_1e_2-xe_2+ye_1+ze_1e_2e_3+te_0e_1e_2+xe_2-ye_1+ze_1e_2e_3)\cosh\theta\sinh\theta\\
    =~&(te_1e_2e_3+xe_0e_2e_3-ye_0e_1e_3+ze_0e_1e_2)\cosh^2\theta+(te_1e_2e_3-xe_0e_2e_3+ye_0e_1e_3+ze_0e_1e_2)\sinh^2\theta\\
    &+2(te_0e_1e_2+ze_1e_2e_3)\cosh\theta\sinh\theta\\
    =~&(t\cosh^2\theta+t\sinh^2\theta+2z\cosh\theta\sinh\theta)e_1e_2e_3\\
    &+(x\cosh^2\theta-x\sinh^2\theta)e_0e_2e_3+(-y\cosh^2\theta+y\sinh^2\theta)e_0e_1e_3\\
    &+(z\cosh^2\theta+z\sinh^2\theta+2t\cosh\theta\sinh\theta)e_0e_1e_2\\
    =~&(t\cosh^2\theta+t\sinh^2\theta+2z\cosh\theta\sinh\theta)e_1e_2e_3+xe_0e_2e_3-ye_0e_1e_3\\
    &+(z\cosh^2\theta+z\sinh^2\theta+2t\cosh\theta\sinh\theta)e_0e_1e_2\\
    =~&(t\cosh(2\theta)+z\sinh(2\theta))e_1e_2e_3+xe_0e_2e_3-ye_0e_1e_3
    +(z\cosh(2\theta)+t\sinh(2\theta))e_0e_1e_2\\
  \end{split}
\end{equation}
$\theta^\dag=-\theta$とすると、
\begin{equation}
  \begin{bmatrix}
    t'\\z'
  \end{bmatrix}
  =
  \begin{bmatrix}
    \cosh(2\theta^\dag) & -\sinh(2\theta^\dag) \\
    -\sinh(2\theta^\dag) & \cosh(2\theta^\dag) \\
  \end{bmatrix}
  \begin{bmatrix}
    t\\z
  \end{bmatrix}
\end{equation}
式の形からローレンツブーストを表している事がわかります。

少しまとめます。1-ベクトルよりなる時空に擬スカラーをかけると双対時空が現れました。
時空の円運動を表現する空間回転の回転子に擬ベクトルをかけ、円運動の双対回転子を構成しました。
双対回転子により双対時空をサンドイッチ積すると、ローレンツブーストになりました。
時空体積を表現していると考えられる擬ベクトルによる双対時空が双曲線関数的に歪むと解釈できます。
円運動する物体が同時に双対時空の歪みを内部状態として持つと考えれば、双対回転子による計算が、
円運動による加速度、つまり遠心力を表現している可能性が考えられます。

逆にこの双対時空のz方向へのローレンツブーストが先に加速度として作用した場合を考えます。
$R^\dag=\exp(-\theta e_0e_3)$の双対計算をすると、
\begin{equation}
  \begin{split}
    (R^\dag)^\dag=~&\exp(-\theta e_0e_3e_0e_1e_2e_3)=\exp(-\theta e_1e_2)=\cos\theta-e_1e_2\sin\theta
  \end{split}
\end{equation}
この式は最初の円運動の回転の逆を意味しています。双対時空が通常時空に作用する結果は逆回転になるという事ですから、
円運動の中心方向と逆方向の遠心力を表現していると考えられます。
双対時空の時間方向の歪みは遠心力による時間の遅れになると解釈できます。
また、双曲線関数なので角度が大きくなると加速度が小さくなるという事はありません。

\subsection{双対時空回転}

それでは次に、物体が加速して速さが増大する計算をしてみます。
\begin{itemize}
  \item ラピディティ(速さの増大分): $\tanh\phi=\Delta v/c$
  \item 初期時空: $S=te_0+xe_1+ye_2+ze_3$
  \item 時空的回転子: $R=\exp(\phi e_0e_1)=\cosh\phi+e_0e_1\sinh\phi$
  \item 時空的回転子リバース: $\widetilde{R}=\cosh\phi-e_0e_1\sinh\phi$
  \item 初期双対時空: $S^\dag=te_1e_2e_3+xe_0e_2e_3-ye_0e_1e_3+ze_0e_1e_2$
  \item 時空的双対回転子: $R^\dag=\exp(\phi e_0e_1e_0e_1e_2e_3)=\exp(\phi e_2e_3)=\cos\phi+e_2e_3\sin\phi$
  \item 時空的双対回転子リバース: $\widetilde{R}^\dag=\cos\phi-e_2e_3\sin\phi$
\end{itemize}

$RS\widetilde{R}$を計算すると、t-x時空平面でのローレンツブーストになります。こちらの計算の詳細は省略します。

$R^\dag S^\dag\widetilde{R}^\dag$を計算してみます。
\begin{equation}
  \begin{split}
    R^\dag S^\dag=~&(\cos\phi+e_2e_3\sin\phi)(te_1e_2e_3+xe_0e_2e_3-ye_0e_1e_3+ze_0e_1e_2)\\
    =~&S^\dag\cos\phi+(-te_1-xe_0+ye_0e_1e_2+ze_0e_1e_3)\sin\phi\\
  \end{split}
\end{equation}
\begin{equation}
  \begin{split}
    R^\dag S^\dag\widetilde{R}^\dag=~&(S^\dag\cos\phi+(-te_1-xe_0+ye_0e_1e_2+ze_0e_1e_3)\sin\phi)(\cos\phi-e_2e_3\sin\phi)\\
    =~&S^\dag\cos^2\phi+(te_1+xe_0-ye_0e_1e_2-ze_0e_1e_3)e_2e_3\sin^2\phi\\
    &+(-S^\dag e_2e_3-te_1-xe_0+ye_0e_1e_2+ze_0e_1e_3)\cos\phi\sin\phi\\
    =~&(te_1e_2e_3+xe_0e_2e_3-ye_0e_1e_3+ze_0e_1e_2)\cos^2\phi+(te_1e_2e_3+xe_0e_2e_3+ye_0e_1e_3-ze_0e_1e_2)\sin^2\phi\\
    &+(-(te_1e_2e_3+xe_0e_2e_3-ye_0e_1e_3+ze_0e_1e_2)e_2e_3-te_1-xe_0+ye_0e_1e_2+ze_0e_1e_3)\cos\phi\sin\phi\\
    =~&(te_1e_2e_3+xe_0e_2e_3-ye_0e_1e_3+ze_0e_1e_2)\cos^2\phi+(te_1e_2e_3+xe_0e_2e_3+ye_0e_1e_3-ze_0e_1e_2)\sin^2\phi\\
    &+(te_0+xe_0+ye_0e_1e_2+ze_0e_1e_3-te_1-xe_0+ye_0e_1e_2+ze_0e_1e_3)\cos\phi\sin\phi\\
    =~&(te_1e_2e_3+xe_0e_2e_3-ye_0e_1e_3+ze_0e_1e_2)\cos^2\phi+(te_1e_2e_3+xe_0e_2e_3+ye_0e_1e_3-ze_0e_1e_2)\sin^2\phi\\
    &+2(ye_0e_1e_2+ze_0e_1e_3)\cos\phi\sin\phi\\
    =~&(t\cos^2\phi+t\sin^2\phi)e_1e_2e_3+(x\cos^2\phi+x\sin^2\phi)e_0e_2e_3\\
    &+(-y\cos^2\phi+y\sin^2\phi+2z\cos\phi\sin\phi)e_0e_1e_3\\
    &+(z\cos^2\phi-z\sin^2\phi+2y\cos\phi\sin\phi)e_0e_1e_2\\
    =~&te_1e_2e_3+xe_0e_2e_3+(-y\cos^2\phi+y\sin^2\phi+2z\cos\phi\sin\phi)e_0e_1e_3\\
    &+(z\cos^2\phi-z\sin^2\phi+2y\cos\phi\sin\phi)e_0e_1e_2\\
    =~&te_1e_2e_3+xe_0e_2e_3(-y\cos(2\phi)+z\sin(2\phi))e_0e_1e_3+(z\cos(2\phi)+y\sin(2\phi))e_0e_1e_2\\
  \end{split}
\end{equation}
\begin{equation}
  \begin{bmatrix}
    -y'\\z'
  \end{bmatrix}
  =
  \begin{bmatrix}
    \cos(2\phi) & -\sin(2\phi) \\
    \sin(2\phi) & \cos(2\phi) \\
  \end{bmatrix}
  \begin{bmatrix}
    y\\z
  \end{bmatrix}
\end{equation}
$ e_0e_1e_2 , e_0e_1e_3 $の超平面面積が交換されます。$e_0e_1e_3$の大きさは符号反転していますが、
これは$ Cl(1,3) $双対時空の特徴と合致しています。

こちらの式においても加速度を表していると考えられますが、循環関数なのである程度強い加速では、
加速度による負荷を感じなくなると解釈できます。
つまり非常に瞬間的な加速や減速を行う乗り物が成立する事になります。

$R^\dag=\exp(\phi e_2e_3)$の双対計算をすると、
\begin{equation}
  (R^\dag)^\dag=\exp(\phi e_2e_3e_0e_1e_2e_3)=\exp(-\phi e_0e_1)=\cosh\phi-e_0e_1\sinh\phi
\end{equation}

こちらの$ (R^\dag)^\dag $では、時空にサンドイッチ積されると逆方向のブーストになります。
加速では加速方向と逆に加速度を受ける事になりますが、
その加速度を重力によって最初に作用させれば、自然と自由落下として移動が起きると考えます。

\section{クリフォード代数での重力計算}

双対時空の歪みとその歪みによる速度の変化について、多体シミュレーションを意識しつつ、三次元で一般化された形を整えていきます。
双対空間回転と双対時空回転の様子見から、加速度と重力の数理的な本質が見えてきました。
1-ブレードで表現された時空の回転により、3-ブレードで表現された双対時空とのズレが生じ加速度の慣性力となります。
逆に双対時空から時空への双対の双対回転子による作用が重力による圧のない自然落下となると考えられます。
重力による時空超平面の変形は、空間回転子とブースト回転子の混合による事が見えてきました。
物体の進行方向と重力の作用が直角の場合は空間回転子になり、平行の場合はブースト回転子になり、
それ以外は、両者の角度により配合率が決定されると考えるのが自然です。
ベルソルを使ってまとめてみます。
\begin{table}[ht]
  \centering
  \caption{時空回転のパラメーター}
  \begin{tabular}{|l|c|l|l|l|} \hline
    回転種別 & 角度 & 方向 & ベルソル & 双対回転 \\ \hline
    ブースト & $ 2 \phi $ & 進行方向 & $ \bm{u}^2 = {u_1}^2 + {u_2}^2 + {u_3}^2 = 1 $ & 空間回転 \\ \hline
    空間回転 & $ 2 \theta $ & 進行方向と直角 & $ \bm{v}^2 = -({v_1}^2 + {v_2}^2 + {v_3}^2) = -1 $ & ブースト \\ \hline
  \end{tabular}
\end{table}
\begin{equation}
  \begin{split}
    \bm{u} = ~ & u_1 iI + u_2 iJ + u_3 iK \\
    \bm{u}^2 = ~ & (u_1 iI + u_2 iJ + u_3 iK)(u_1 iI + u_2 iJ + u_3 iK) \\
    = ~ & {u_1}^2 - u_1 u_2 K + u_1 u_3 J
    + u_1 u_2 K + {u_2}^2 - u_2 u_3 I
    - u_1 u_3 J + u_2 u_3 I + {u_3}^2 \\
    = ~ & {u_1}^2 + {u_2}^2 + {u_3}^2 = 1 \\
    \bm{v} = ~ & v_1 I + v_2 J + v_3 K \\
    \bm{v}^2 = ~ & (v_1 I + v_2 J + v_3 K)(v_1 I + v_2 J + v_3 K) \\
    = ~ & - {v_1}^2 + v_1 v_2 K - v_1 v_3 J
    - v_1 v_2 K - {v_2}^2 + v_2 v_3 I
    + v_1 v_3 J - v_2 v_3 I - {v_3}^2 \\
    = ~ & - {v_1}^2 - {v_2}^2 - {v_3}^2 = -1 \\
  \end{split}
\end{equation}
\begin{equation}
  \begin{split}
    R_{boost} = ~ & \exp(\gamma_0 e_0e_1 + \gamma_1 e_0e_2 + \gamma_2 e_0e_3)
    = \exp ( \phi \bm{u} )
    = \cosh \phi + \bm{u} \sinh \phi \\
    R_{space} = ~ & \exp(\gamma_3 e_1e_2 + \gamma_4 e_1e_3 + \gamma_5 e_2e_3)
    = \exp ( \theta \bm{v} )
    = \cos \theta + \bm{v} \sin \theta \\
  \end{split}
\end{equation}
\begin{equation}
  \begin{split}
    R_{hybrid} = ~ & \exp(\gamma_0 e_0e_1 + \gamma_1 e_0e_2 + \gamma_2 e_0e_3
    + \gamma_3 e_1e_2 + \gamma_4 e_1e_3 + \gamma_5 e_2e_3) \\
    = ~ & \exp (\phi \bm{u} + \theta \bm{v}) \\
    = ~ & \exp (u_1 \phi iI + u_2 \phi iJ + u_3 \phi iK
    + \theta v_1 I + \theta v_2 J + \theta v_3 K) \\
    = ~ & \exp \Big( (\theta v_1 + \phi u_1 i)I + (\theta v_2 + \phi i u_2)J 
    + (\theta v_3 + \phi u_3 i)K \Big) \\
  \end{split}
\end{equation}
$ \eta_n $で置き換えて見通しを良くします。
\begin{equation}
  \eta_n = \theta v_n + \phi u_n i, \quad n = 1,2,3
\end{equation}
\begin{equation}
  R_{hybrid} = \exp (\eta_1 I + \eta_2 J + \eta_3 K)
\end{equation}
ベルソルの方向としてはお互いが直角になっているので、
\begin{equation}
  v_1 u_1 + v_2 u_2 + v_3 u_3 = 0
\end{equation}
また二乗に関しては以下の通りです。
\begin{equation}
  {\eta_1}^2 = (v_1 \theta + u_1 \phi i)^2 = {v_1}^2 \theta^2 - {u_1}^2 \phi^2 +2 v_1 u_1 \theta \phi
\end{equation}
\begin{equation}
  {\eta_1}^2 + {\eta_2}^2 + {\eta_3}^2
  = (v_1 \theta + u_1 \phi i)^2 + (v_2 \theta + u_2 \phi i)^2 + (v_3 \theta + u_3 \phi i)^2
  = \theta^2 - \phi^2
\end{equation}
\begin{equation}
  (\eta_1 I + \eta_2 J + \eta_3 K)^2 = - {\eta_1}^2 - {\eta_2}^2 - {\eta_3}^2
  = - (\theta^2 - \phi^2)
\end{equation}
$ \alpha = \phi^2 - \theta^2 , \quad \alpha \in \mathbb{R} 
, \quad \eta = \eta_1 I + \eta_2 J + \eta_3 K $と置くと、
\begin{equation}
  \eta^2 = \alpha
  , \quad R_{hybrid} = \exp (\eta)
\end{equation}
$ \alpha $の平方根については、
\begin{equation}
  \eta = \sqrt{\alpha} = \phi \bm{u} + \theta \bm{v}
\end{equation}
ここで$ \bm{u}, \bm{v} $について更に調べると、
\begin{equation}
  \begin{split}
    \bm{u} \bm{v} = ~ & (u_1 iI + u_2 iJ + u_3 iK)(v_1 I + v_2 J + v_3 K) \\
    = ~ & - u_1 v_1 i + u_1 v_2 iK - u_1 v_3 iJ
    - u_2 v_1 iK - u_2 v_2 i + u_2 v_3 iI
    + u_3 v_1 iJ - u_3 v_2 iI - u_3 v_3 i \\
    = ~ & (u_2 v_3 - u_3 v_2) iI + (u_3 v_1 - u_1 v_3) iJ + (u_1 v_2 - u_2 v_1) iK \\
    \bm{v} \bm{u} = ~ & (v_1 I + v_2 J + v_3 K)(u_1 iI + u_2 iJ + u_3 iK) \\
    = ~ & - u_1 v_1 i + u_2 v_1 iK - u_3 v_1 iJ
    - u_1 v_2 iK - u_2 v_2 i + u_3 v_2 iI
    + u_1 v_3 iJ - u_2 v_3 iI - u_3 v_3 i \\
    = ~ & (u_3 v_2 - u_2 v_3) iI + (u_1 v_3 - u_3 v_1) iJ + (u_2 v_1 - u_1 v_2) iK \\
  \end{split}
\end{equation}
\begin{equation}
  \bm{u}^2 = 1 , \quad \bm{v}^2 = -1 , \quad \bm{u} \bm{v} = - \bm{v} \bm{u}
\end{equation}
になっているので、$ \bm{u}, \bm{v} $については、$ Cl(1,1) $の基底と同等になります。
\begin{equation}
  (\phi \bm{u} + \theta \bm{v})^2 = \phi^2 - \theta^2
\end{equation}
テイラー展開を行うと、
\begin{equation}
  \begin{split}
    R_{hybrid} = ~ & \sum_{n=0}^{∞} \frac{1}{n!} \eta^n \\
    = ~ & \sum_{n=0}^{∞} \frac{1}{(2n)!} \eta^{2n}
    + \sum_{n=0}^{∞} \frac{1}{(2n + 1)!} \eta^{2n + 1} \\
  \end{split}
\end{equation}

\subsection{混合回転の具体化}

$ \alpha $の符号によって場合分けします。

\begin{itemize}
  \item $ \alpha > 0 : $
\end{itemize}
\begin{equation}
  \alpha^n = |\alpha|^n
\end{equation}
\begin{equation}
  \begin{split}
    R_{hybrid} = ~ & \sum_{n=0}^{∞} \frac{1}{(2n)!} \alpha^n
    + \eta \sum_{n=0}^{∞} \frac{1}{(2n + 1)!} \alpha^n \\
    = ~ & \sum_{n=0}^{∞} \frac{1}{(2n)!} (\sqrt{\alpha})^{2n}
    + \sum_{n=0}^{∞} \frac{1}{(2n + 1)!} (\sqrt{\alpha})^{2n + 1} \\
    = ~ & \cosh(\sqrt{\alpha}) + \sinh(\sqrt{\alpha}) \\
  \end{split}
\end{equation}
双曲線関数の加法定理から、
\begin{equation}
  \begin{split}
    \cosh(\phi \bm{u} + \theta \bm{v})
    = ~ & \cosh(\phi \bm{u}) \cosh(\theta \bm{v}) + \sinh(\phi \bm{u}) \sinh(\theta \bm{v}) \\
    = ~ & \cosh \phi \cos \theta + \bm{u} \bm{v} \sinh \phi \sin \theta \\
    \sinh(\phi \bm{u} + \theta \bm{v})
    = ~ & \sinh(\phi \bm{u}) \cosh(\theta \bm{v}) + \cosh(\phi \bm{u}) \sinh(\theta \bm{v}) \\
    = ~ & \bm{u} \sinh \phi \cos \theta + \bm{v} \cosh \phi \sin \theta \\
  \end{split}
\end{equation}
\begin{equation}
  \begin{split}
    R_{hybrid} = ~ & \cosh \phi \cos \theta + \bm{u} \sinh \phi \cos \theta 
    + \bm{v} \cosh \phi \sin \theta + \bm{u} \bm{v} \sinh \phi \sin \theta \\
    = ~ & (\cosh \phi + \bm{u} \sinh \phi)(\cos \theta + \bm{v} \sin \theta) \\
  \end{split}
\end{equation}

\begin{itemize}
  \item $ \alpha < 0 : $
\end{itemize}
\begin{equation}
  \alpha^n = (-1)^n |\alpha|^n
\end{equation}
\begin{equation}
  \begin{split}
    R_{hybrid} = ~ & \sum_{n=0}^{∞} \frac{(-1)^n}{(2n)!} |\alpha|^n
    + \eta \sum_{n=0}^{∞} \frac{(-1)^n}{(2n + 1)!} |\alpha|^n \\
    = ~ & \sum_{n=0}^{∞} \frac{(-1)^n}{(2n)!} (\sqrt{|\alpha|})^{2n}
    + \frac{\eta}{\sqrt{|\alpha|}} \sum_{n=0}^{∞} \frac{(-1)^n}{(2n + 1)!} (\sqrt{|\alpha|})^{2n + 1} \\
    = ~ & \cos(\sqrt{|\alpha|}) + \sin(\sqrt{|\alpha|}) \\
  \end{split}
\end{equation}
ここで、$ |\alpha| = \theta^2 - \phi^2 > 0 $だから、
$ \sqrt{|\alpha|} = \theta \bm{u} + \phi \bm{v} $
\begin{equation}
  \begin{split}
    \frac{\eta}{\sqrt{|\alpha|}} = ~ & \frac{\phi \bm{u} + \theta \bm{v}}{\theta \bm{u} + \phi \bm{v}}
    = \frac{(\phi \bm{u} + \theta \bm{v})(\theta \bm{u} + \phi \bm{v})}{(\theta \bm{u} + \phi \bm{v})^2}
    = \frac{(\phi \bm{u} + \theta \bm{v})(\theta \bm{u} + \phi \bm{v})}{\theta^2 - \phi^2} \\
    = ~ & \frac{\phi \theta - \phi \theta + (\phi^2 - \theta^2) \bm{u} \bm{v}}{\theta^2 - \phi^2} \\
    = ~ & \bm{u} \bm{v} \\
  \end{split}
\end{equation}
三角関数の加法定理から、
\begin{equation}
  \begin{split}
    \cos(\theta \bm{u} + \phi \bm{v})
    = ~ & \cos(\theta \bm{u}) \cos(\phi \bm{v}) - \sin(\theta \bm{u}) \sin(\phi \bm{v}) \\
    = ~ & \cosh \phi \cos \theta - \bm{u} \bm{v} \sinh \phi \sin \theta \\
    \sinh(\theta \bm{u} + \phi \bm{v})
    = ~ & \sin(\theta \bm{u}) \cos(\phi \bm{v}) + \cos(\theta \bm{u}) \sin(\phi \bm{v}) \\
    = ~ & \bm{u} \cosh \phi \sin \theta + \bm{v} \sinh \phi \cos \theta \\
  \end{split}
\end{equation}
\begin{equation}
  \begin{split}
    R_{hybrid} = ~ & \cosh \phi \cos \theta - \bm{u} \bm{v} \sinh \phi \sin \theta
    + \bm{u} \bm{v} (\bm{u} \cosh \phi \sin \theta + \bm{v} \sinh \phi \cos \theta) \\
    = ~ & \cosh \phi \cos \theta - \bm{u} \bm{v} \sinh \phi \sin \theta
    - \bm{v} \cosh \phi \sin \theta - \bm{u} \sinh \phi \cos \theta \\
    = ~ & \cosh \phi \cos \theta - \bm{u} \sinh \phi \cos \theta 
    - \bm{v} \cosh \phi \sin \theta - \bm{u} \bm{v} \sinh \phi \sin \theta \\
    = ~ & (\cos \theta - \bm{v} \sin \theta)(\cosh \phi - \bm{u} \sinh \phi) \\
  \end{split}
\end{equation}

\begin{itemize}
  \item $ \alpha = 0 : $
\end{itemize}
\begin{equation}
  R_{hybrid} = \sum_{n=0}^{∞} \frac{1}{n!} \eta^n = 1 + \phi \bm{u} + \theta \bm{v}
\end{equation}

\subsection{ユニタリ性の確認}

それぞれについての共役を選択してユニタリ性を確かめていきます。
\begin{itemize}
  \item $ \alpha = 0 : $
\end{itemize}
\begin{equation}
  \begin{split}
    R R^* = ~ & (1 + \phi \bm{u} + \theta \bm{v})(1 - \phi \bm{u} - \theta \bm{v}) \\
    = ~ & 1 - \phi \bm{u} - \theta \bm{v}
    + \phi \bm{u} - \phi^2 - \phi \theta \bm{u} \bm{v}
    + \theta \bm{v} + \phi \theta \bm{u} \bm{v} + \theta^2 \\
    = ~ & 1 - \phi^2 + \theta^2 \\
    = ~ & 1 \\
  \end{split}
\end{equation}

\begin{itemize}
  \item $ \alpha > 0: $
\end{itemize}
\begin{equation}
  \begin{split}
    R = ~ & \cosh \phi \cos \theta + \bm{u} \sinh \phi \cos \theta 
    + \bm{v} \cosh \phi \sin \theta + \bm{u} \bm{v} \sinh \phi \sin \theta \\
    R^* = ~ & \cosh \phi \cos \theta - \bm{u} \sinh \phi \cos \theta 
    - \bm{v} \cosh \phi \sin \theta - \bm{u} \bm{v} \sinh \phi \sin \theta \\
  \end{split}
\end{equation}
ここの$ R^* $は、$ \alpha < 0 $の場合の$ R $と同じになっています。
式の短縮のために$\cosh \phi = a, \sinh \phi = b, \cos \theta = p, \sin \theta = q$と置いて、
\begin{equation}
  \begin{split}
    R R^* = ~ & (a p + b p \bm{u} + a q \bm{v} + b q \bm{u} \bm{v})
    (a p - b p \bm{u} - a q \bm{v} - b q \bm{u} \bm{v}) \\
    = ~ & a^2 p^2 - a b p^2 \bm{u} - a^2 p q \bm{v} - a b p q \bm{u} \bm{v}
    + a b p^2 \bm{u} - b^2 p^2 - a b p q \bm{u} \bm{v} - b^2 p q \bm{v} \\
    & + a^2 p q \bm{v} + a b p q \bm{u} \bm{v} + a^2 q^2 - a b q^2 \bm{u}
    + a b p q \bm{u} \bm{v} + b^2 p q \bm{v} + a b q^2 \bm{u} - b^2 q^2 \\
    = ~ & a^2 p^2 - b^2 p^2 + a^2 q^2 - b^2 q^2 \\
    = ~ & (a^2 - b^2) (p^2 + q^2) \\
    = ~ & 1 \\
  \end{split}
\end{equation}

\begin{itemize}
  \item $ \alpha < 0: $
\end{itemize}
逆も同様なので、
\begin{equation}
  R R^* = R^* R = 1
\end{equation}

\subsection{混合回転の双対}

双対混合回転を計算します。
$ Cl(1,3) $で考えると、
\begin{equation}
  \begin{split}
    R = ~ & \exp(\gamma_0 e_0e_1 + \gamma_1 e_0e_2 + \gamma_2 e_0e_3
    + \gamma_3 e_1e_2 + \gamma_4 e_1e_3 + \gamma_5 e_2e_3) \\
    R^\dag = ~ & \exp((\gamma_0 e_0e_1 + \gamma_1 e_0e_2 + \gamma_2 e_0e_3
    + \gamma_3 e_1e_2 + \gamma_4 e_1e_3 + \gamma_5 e_2e_3)e_0e_1e_2e_3) \\
    = ~ & \exp(\gamma_0 e_2e_3 - \gamma_1 e_1e_3 + \gamma_2 e_1e_2
    - \gamma_3 e_0e_3 + \gamma_4 e_0e_2 - \gamma_5 e_0e_1) \\
  \end{split}
\end{equation}
四双四元数で考えると、
\begin{equation}
  \begin{split}
    R = ~ & \exp(\omega_0 iI + \omega_1 iJ + \omega_2 iK + \omega_3 I + \omega_4 J + \omega_5 K) \\
    R^\dag = ~ & \exp((\omega_0 iI + \omega_1 iJ + \omega_2 iK + \omega_3 I + \omega_4 J + \omega_5 K)i) \\
    = ~ & \exp(- \omega_0 I - \omega_1 J - \omega_2 K + \omega_3 iI + \omega_4 iJ + \omega_5 iK) \\
  \end{split}
\end{equation}
ベルソルの双対は、
\begin{equation}
  \begin{split}
    \bm{u}^\dag = ~ & i \bm{u} = - u_1 I - u_2 J - u_3 K \\
    (\bm{u}^\dag)^2 = ~ & - {u_1}^2 - {u_2}^2 - {u_3}^2 = -1 \\
    \bm{v}^\dag = ~ & i \bm{v} = v_1 iI + v_2 iJ + v_3 iK \\
    (\bm{v}^\dag)^2 = ~ & {v_1}^2 + {v_2}^2 + {v_3}^2 = 1 \\
  \end{split}
\end{equation}
\begin{equation}
  \begin{split}
    R_{boost}^\dag = ~ & \exp(- \omega_0 I - \omega_1 J - \omega_2 K)
    = \exp ( \phi \bm{u}^\dag )
    = \cos \phi + \bm{u}^\dag \sin \phi \\
    R_{space}^\dag = ~ & \exp(\omega_3 iI + \omega_4 iJ + \omega_5 iK)
    = \exp ( \theta \bm{v}^\dag )
    = \cosh \theta + \bm{v}^\dag \sinh \theta \\
  \end{split}
\end{equation}

\begin{table}[ht]
  \centering
  \caption{時空歪曲の混合回転}
  \begin{tabular}{|c|l|l|} \hline
    $ \alpha = \phi^2 - \theta^2 $ & 回転子(Rotor) & 双対回転子(Duality Rotor) \\ \hline
    $ \alpha > 0 $ & $ (\cosh \phi + \bm{u} \sinh \phi)(\cos \theta + \bm{v} \sin \theta) $
    & $ (\cos \phi + \bm{u}^\dag \sin \phi)(\cosh \theta + \bm{v}^\dag \sinh \theta) $ \\ \hline
    $ \alpha < 0 $ & $ (\cos \theta - \bm{v} \sin \theta)(\cosh \phi - \bm{u} \sinh \phi) $
    & $ (\cosh \theta - \bm{v}^\dag \sinh \theta)(\cos \phi - \bm{u}^\dag \sin \phi) $ \\ \hline
    $ \alpha = 0 $ & $ 1 + \phi \bm{u} + \theta \bm{v} $
    & $ 1 + \phi \bm{u}^\dag + \theta \bm{v}^\dag $ \\ \hline
  \end{tabular}
\end{table}

物体の進行方向と重力の作用する方向のなす角度を$ \sigma $、重力の大きさを$ \rho $とすると、
\begin{equation}
  \phi = \frac{1}{2} \rho \cos \sigma , \quad \theta = \frac{1}{2} \rho \sin \sigma
  , \quad 0 \le \sigma \le \pi , \quad \rho \ge 0 , \quad \sigma,\rho \in \mathbb{R}
\end{equation}

\subsection{クリフォード代数の4元運動量}

相対性理論では、運動量$\bm{p}$、エネルギー$E$、質量$m$の間に以下の関係があります。
\begin{equation}
  E^2=(\bm{p}c)^2+(mc^2)^2
\end{equation}
質量を左辺にして整理すると、
\begin{equation}
  (mc^2)^2=E^2-(\bm{p}c)^2=E^2-(p_1c)^2-(p_2c)^2-(p_3c)^2
\end{equation}
クリフォード代数$Cl(1,3)$に当てはめて、
\begin{equation}
  \begin{split}
    P = ~ & E e_0 + c p_1 e_1 + c p_2 e_2 + c p_3 e_3 \\
    P^2 = ~ & E^2 - c^2 {p_1}^2 - c^2 {p_2}^2 - c^2 {p_3}^2 = E^2 - c^2 \bm{p}^2 = (m c^2)^2 \\
    |P| = ~ & m c^2, \quad (m \ge 0) \\
  \end{split}
\end{equation}
この様に1-ブレードで表現しても不変量は保たれますが、$ c $の位置が丁度、時空超平面の単位面積になっています。
\begin{equation}
  E (e_1e_2e_3), \quad p_1 (c ~ e_0e_2e_3), \quad p_2 (c ~ e_0e_1e_3), \quad p_3 (c ~ e_0e_1e_2)
\end{equation}
双対時空として表現すると、
\begin{equation}
  \begin{split}
    P^\dag = ~ & E e_1e_2e_3 + c p_1 e_0e_2e_3 - c p_2 e_0e_1e_3 + c p_3 e_0e_1e_2 \\
    (P^\dag)^2 = ~ & E^2 - c^2 {p_1}^2 - c^2 {p_2}^2 - c^2 {p_3}^2 = E^2 - c^2 \bm{p}^2 = (m c^2)^2 \\
    |P^\dag| = ~ & m c^2, \quad (m \ge 0) \\
  \end{split}
\end{equation}
クリフォード代数表現からすると、4元運動量はどうやら時空超立方体と関連が深い様子です。
1-ブレードの4元速度$ U $を考えてみます。
3次元速度を$ \bm{v} = (v_1, v_2, v_3) $、固有時間を$ \tau $とすると、
ローレンツ因子$ \gamma $は、
\begin{equation}
  \gamma = \frac{dt}{d\tau} = \frac{1}{\sqrt{1 - \bm{v}^2 / c^2}}
\end{equation}
\begin{equation}
  \begin{split}
    U = ~ & \frac{c dt}{d\tau} e_0 + \frac{dx}{d\tau} e_1 + \frac{dy}{d\tau} e_2 + \frac{dz}{d\tau} e_3 \\
    = ~ & \gamma c e_0 + \gamma v_1 e_1 + \gamma v_2 e_2 + \gamma v_3 e_3 \\
    = ~ & \gamma (c e_0 + v_1 e_1 + v_2 e_2 + v_3 e_3) \\
    U^2 = ~ & \gamma^2 (c^2 - v_1^2 - v_2^2 - v_3^2) \\
    = ~ & \gamma^2 (c^2 - \bm{v}^2) = c^2 \gamma^2 (1 - \bm{v}^2 / c^2) \\
    = ~ & c^2 \\
    |U| = ~ & c \\
  \end{split}
\end{equation}
4元速度のノルムと4元運動量のノルムは以下のような関係になっています。
\begin{equation}
  |U| m c = |P^\dag|
\end{equation}
$ c $の位置から4元速度は、単位時空基底で構成されています。
\begin{equation}
  \gamma (c e_0), \quad \gamma v_1 (e_1), \quad \gamma v_2 (e_2), \quad \gamma v_3 (e_3)
\end{equation}

\begin{itemize}
  \item 4元速度: 単位時空基底で構成
  \item 4元運動量: 時空超平面の単位面積基底で構成
\end{itemize}

4元速度を扱う場合、単純に速度を足していくと光速を超えてしまう事になるので出来ませんが、
4元運動量であれば、3次元運動量として加算していっても、エネルギーが自動的に調整されるので、
相対論的に破綻する事がありません。

\subsubsection{光の場合}

質量がゼロなら光的になります。$ h $をプランク定数、$ \bm{\nu} $を振動数として、
\begin{equation}
  E^2 - c^2 \bm{p}^2 = E^2 - h^2 \bm{\nu}^2 = 0
\end{equation}
\begin{equation}
  \begin{split}
    P^\dag_{light} = ~ & E e_1e_2e_3 + c p_1 e_0e_2e_3 - c p_2 e_0e_1e_3 + c p_3 e_0e_1e_2 \\
    = ~ & E e_1e_2e_3 + h \nu_1 e_0e_2e_3 - h \nu_2 e_0e_1e_3 + h \nu_3 e_0e_1e_2 \\
    (P^\dag_{light})^2 = ~ & |P^\dag_{light}| = 0 \\
  \end{split}
\end{equation}
光の場合、4元速度は固有時間$ \tau = 0 $のため定義できませんが、速度の方向と大きさは明確なので算出は容易です。
光は進行方向の空間の長さがローレンツ収縮によりゼロになっています。
このようなゼロの空間で光の活動がどの様になっているかを考える事は困難です。
しかし、時空超立方体で光の振動が起きていると考えると、光に関する円偏光などの現象が理解しやすくなります。
また、重力によって時空超立方体が変形すると光の進行方向が変化すると想定できます。

\subsection{4元運動量とラピディティの関係}

相対性理論では運動量$\bm{p}$と速度$\bm{v}$は以下の関係があります。
\begin{equation}
  \bm{p}=\gamma m\bm{v}, \quad \gamma=\frac{1}{\sqrt{1-\beta^2}}, \quad \beta=\frac{|\bm{v}|}{c}
\end{equation}
ここで$\bm{p}$および$\bm{v}$は三次元ベクトルの範囲とします。展開して変形していきます。
$p=|\bm{p}|, v=|\bm{v}|$とします。
\begin{equation}
  \begin{split}
    &\bm{p} = \frac{m \bm{v}}{\sqrt{1 - \frac{v^2}{c^2}}} \quad \Leftrightarrow \quad
    p^2(1 - \frac{v^2}{c^2}) = m^2 v^2 \quad \Leftrightarrow \quad
    m^2 v^2 + \frac{p^2 v^2}{c^2} = p^2 \quad \Leftrightarrow \quad
    (m^2 + \frac{p^2}{c^2}) v^2 = p^2\\
    &\quad \Leftrightarrow \quad (\frac{m^2 c^2 + p^2}{c^2}) v^2 = p^2 \quad \Leftrightarrow \quad
    v^2 = p^2(\frac{c^2}{m^2 c^2 + p^2})\\
    &v = \frac{pc}{\sqrt{m^2 c^2 + p^2}}, \quad v \to c \quad (p \to \infty, \quad m \to \infty),
    \quad v = c \quad (m = 0)\\
  \end{split}
\end{equation}
運動量や質量が無限大に増大しても破綻しません。速さは光速に近づいていきます。
質量がゼロの時、速度は光速になります。
ラピディティ$ \Theta $を求めていくと、
\begin{equation}
  \begin{split}
    &\beta = \frac{v}{c} = \tanh \Theta = \frac{p}{\sqrt{m^2 c^2 + p^2}} = \frac{pc}{E} < 1
    , \quad (p \to \infty, \quad m \to \infty)\\
    &\texttt{Rapidity}: \quad \Theta = \arctanh \frac{p}{\sqrt{m^2 c^2 + p^2}}
    , \quad \Theta \to \infty \quad (p \to \infty)\\
  \end{split}
\end{equation}

\subsection{双対時空での重力計算}

ここまで時空連続体仮説を棄却して代わりに時空超立方体の性質を調べ、
運動量は時空超立方体に関連した量と考えられる事を見てきました。
重力として物体に働く力は、力というより時空超立方体の歪みであり、同時に運動量を歪ませる量と考えて来ています。
物体もしくは粒子に働く力は、基本的に運動量の交換になります。
物体が時空超立方体の歪みに抵抗し、物体の運動量の変化に反映されなかった場合は、それは力が働いたとは言えない事になります。
日常的には時空超立方体の歪みに抵抗した力を重力と感じています。
双対時空で考えると運動量交換の原理は変わらず、$ P^\dag $の内、
$ (p_1, p_2, p_3) $の三次元ベクトルにおいて交換されると考えられます。
不変量$ (P^\dag)^2 $は$ E $の大きさを調整することで保たれます。
従って、自由落下のみを計算する多体シミュレーションで考えた場合では、運動量の交換量が線形的に加算されます。

では、具体的な計算に入りましょう。まず2体で考えます。物体のパラメーターを以下の様に定義します。
\\\\
\begin{tabular}{|l|c|c|c|} \hline
  & 質量 & 運動量(3次元ベクトル) & 位置(3次元座標) \\ \hline
  物体A & $ m_a $ & $ \bm{p}_a $ & $ \bm{x}_a $ \\ \hline
  物体B & $ m_b $ & $ \bm{p}_b $ & $ \bm{x}_b $ \\ \hline
\end{tabular}
\\\\
物体Aと物体Bは重力によって運動量を交換し、交換の結果、運動量が反対向きに同じ量だけ変化します。
\begin{itemize}
  \item 物体Aの運動量変化: $ \Delta \bm{p}_a = \Delta \bm{p} $
  \item 物体Bの運動量変化: $ \Delta \bm{p}_b = - \Delta \bm{p} $
  \item 物体A,B間の距離: $ r = |\bm{x}_b - \bm{x}_a| $
  \item 物体AからBへの方向: $ \hat{\bm{r}} = (\bm{x}_b - \bm{x}_a) / r $
  \item 万有引力定数: $G$
  \item 微小時間ステップ: $\Delta t$
\end{itemize}
重力による力積をニュートン近似を用いて次のように計算します。
\begin{equation}
  \Delta \bm{p} = \Delta t \frac{G m_a m_b}{r^2} \hat{\bm{r}} = \Delta t \frac{G m_a m_b}{r^3} (\bm{x}_b - \bm{x}_a)
\end{equation}
物体Aを含めてN体の物体があるとすると、$ N(N - 1) / 2 $の運動量の交換関係があります。物体Aに関して和を計算します。
\begin{equation}
  \bm{p}_{total} = \sum_{n = 1}^{N - 1} \Delta \bm{p}_n
\end{equation}
運動量に関しては線形的にどれだけでも加算することが出来るという前提です。
この$ \bm{p}_{total} $が物体の持つ双対時空に作用します。
つまり、その物体が状態として持つ局所的な時空立方体を歪めます。
ある瞬間に浮遊する物体が持っている双対時空は物体の持つ座標と一致しています。
\begin{equation}
  \begin{split}
    S = ~ & t e_0 + x e_1 + y e_2 + z e_3 \\
    S^\dag = ~ & (t e_0 + x e_1 + y e_2 + z e_3)e_0e_1e_2e_3 
    = t e_1e_2e_3 + x e_0e_2e_3 - y e_0e_1e_3 + z e_0e_1e_2 \\
  \end{split}
\end{equation}
四双四元数で表現すると、
\begin{equation}
  \begin{split}
    S = ~ & t j + x kI + y kJ + z kK \\
    S^\dag = ~ & i(t j + x kI + y kJ + z kK) = t k - x jI - y jJ - z jK \\
  \end{split}
\end{equation}
\begin{itemize}
  \item ある瞬間の物体の双対時空: $ S^\dag = t e_1e_2e_3 + x e_0e_2e_3 - y e_0e_1e_3 + z e_0e_1e_2 $
\end{itemize}
双対混合回転子で表現すると、
\begin{equation}
  \bm{u} = \frac{\bm{p}_a}{|\bm{p}_a|}
  , \quad \bm{p} = \frac{\bm{p}_{total}}{|\bm{p}_{total}|}
  , \quad \bm{u} \cdot \bm{v} = 0
\end{equation}
\begin{equation}
  \cos \sigma = \bm{u} \cdot \bm{p}
  , \quad \phi = a \cos \sigma
  , \quad \theta = a \sin \sigma
\end{equation}
\begin{equation}
  \bm{v} = \frac{\bm{p} - \bm{u} \cos \sigma}{\sin \sigma}
  , \quad 0 < \sigma < \pi
\end{equation}
\begin{equation}
  R^\dag = \left(\cos \frac{\phi}{2} + \bm{u}^\dag \sin \frac{\phi}{2} \right)
  \left(\cosh \frac{\theta}{2} + \bm{v}^\dag \sinh \frac{\theta}{2} \right)
\end{equation}
運動量の方向と重力の方向が一致するとして、左側だけを計算していきます。
\begin{equation}
  R = \exp \left(\frac{\phi}{2} \bm{u} \right) 
  = \exp \left(\frac{\phi}{2} (u_1 e_0e_1 + u_2 e_0e_2 + u_3 e_0e_3) \right)
\end{equation}
時空的双対回転子は、$ \bm{u} $に$ e_0e_1e_2e_3 $をかけて構成するので、
\begin{equation}
  \begin{split}
    \bm{u}^\dag = \bm{u} e_0e_1e_2e_3 = ~ & u_1 e_0e_1e_0e_1e_2e_3 + u_2 e_0e_2e_0e_1e_2e_3 + u_3 e_0e_3e_0e_1e_2e_3 \\
    = ~ & u_1 e_2e_3 - u_2 e_1e_3 + u_3 e_1e_2 \\
    = ~ & u_3 e_1e_2 - u_2 e_1e_3 + u_1 e_2e_3 \\
  \end{split}
\end{equation}
\begin{equation}
  \begin{split}
    (\bm{u}^\dag)^2 = ~ & (u_3 e_1e_2 - u_2 e_1e_3 + u_1 e_2e_3)^2 \\
    = ~ & u_3^2 e_1e_2e_1e_2 - u_2 u_3 e_1e_2e_1e_3 + u_1 u_3 e_1e_2e_2e_3 \\
    & - u_2 u_3 e_1e_3e_1e_2 + u_2^2 e_1e_3e_1e_3 - u_1 u_2 e_1e_3e_2e_3 \\
    & + u_1 u_3 e_2e_3e_1e_2 - u_1 u_2 e_2e_3e_1e_3 + u_1^2 e_2e_3e_2e_3 \\
    = ~ & - u_3^2 - u_2 u_3 e_2e_3 - u_1 u_3 e_1e_3 \\
    & + u_2 u_3 e_2e_3 - u_2^2 - u_1 u_2 e_1e_2 \\
    & + u_1 u_3 e_1e_3 + u_1 u_2 e_1e_2 - u_1^2 \\
    = ~ & - u_1^2 - u_2^2 - u_3^2 \\
  \end{split}
\end{equation}
従って、$ \bm{u}^\dag $は四元数を使ったベルソルと同型となっています。
\begin{equation}
  R^\dag = \exp \left(\frac{\phi}{2} \bm{u}^\dag \right)
  = \exp \left(\frac{\phi}{2}(u_3 e_1e_2 - u_2 e_1e_3 + u_1 e_2e_3) \right)
  = \cos \frac{\phi}{2} + \bm{u}^\dag \sin \frac{\phi}{2}
\end{equation}
サンドイッチ積により重力の影響で歪んだ双対時空$ (S^\dag)' = R^\dag S^\dag \widetilde{R^\dag} $を計算します。
$ p = \cos \frac{\phi}{2} , q = \sin \frac{\phi}{2} $とします。
\begin{equation}
  \begin{split}
    R^\dag S^\dag \widetilde{R^\dag} = ~ & (p + \bm{u}^\dag q)S^\dag(p - \bm{u}^\dag q) \\
    = ~ & (p S^\dag + q \bm{u}^\dag S^\dag)(p - \bm{u}^\dag q) \\
    = ~ & p^2 S^\dag + p q \bm{u}^\dag S^\dag - p q S^\dag \bm{u}^\dag - q^2 \bm{u}^\dag S^\dag \bm{u}^\dag \\
  \end{split}
\end{equation}
\begin{equation}
  \begin{split}
    \bm{u}^\dag S^\dag = ~ & (u_3 e_1e_2 - u_2 e_1e_3 + u_1 e_2e_3)(z e_0e_1e_2 - y e_0e_1e_3 + x e_0e_2e_3 + t e_1e_2e_3) \\
    = ~ & z u_3 e_1e_2 e_0e_1e_2 - y u_3 e_1e_2 e_0e_1e_3 + x u_3 e_1e_2 e_0e_2e_3 + t u_3 e_1e_2 e_1e_2e_3 \\
    & - z u_2 e_1e_3 e_0e_1e_2 + y u_2 e_1e_3 e_0e_1e_3 - x u_2 e_1e_3 e_0e_2e_3 - t u_2 e_1e_3 e_1e_2e_3 \\
    & + z u_1 e_2e_3 e_0e_1e_2 - y u_1 e_2e_3 e_0e_1e_3 + x u_1 e_2e_3 e_0e_2e_3 + t u_1 e_2e_3 e_1e_2e_3 \\
    = ~ & - z u_3 e_0 - y u_3 e_0e_2e_3 - x u_3 e_0e_1e_3 - t u_3 e_3 \\
    & + z u_2 e_0e_2e_3 - y u_2 e_0 - x u_2 e_0e_1e_2 - t u_2 e_2 \\
    & + z u_1 e_0e_1e_3 + y u_1 e_0e_1e_2 - x u_1 e_0 - t u_1 e_1 \\
    = ~ & (- x u_1 - y u_2 - z u_3) e_0 - t u_1 e_1 - t u_2 e_2 - t u_3 e_3 \\
    & + (y u_1 - x u_2) e_0e_1e_2 + (z u_1 - x u_3) e_0e_1e_3 + (z u_2 - y u_3) e_0e_2e_3\\
  \end{split}
\end{equation}
\begin{equation}
  \begin{split}
    S^\dag \bm{u}^\dag = ~ & (z e_0e_1e_2 - y e_0e_1e_3 + x e_0e_2e_3 + t e_1e_2e_3)(u_3 e_1e_2 - u_2 e_1e_3 + u_1 e_2e_3) \\
    = ~ & z u_3 e_0e_1e_2 e_1e_2 - y u_3 e_0e_1e_3 e_1e_2 + x u_3 e_0e_2e_3 e_1e_2 + t u_3 e_1e_2e_3 e_1e_2 \\
    & - z u_2 e_0e_1e_2 e_1e_3 + y u_2 e_0e_1e_3 e_1e_3 - x u_2 e_0e_2e_3 e_1e_3 - t u_2 e_1e_2e_3 e_1e_3 \\
    & + z u_1 e_0e_1e_2 e_2e_3 - y u_1 e_0e_1e_3 e_2e_3 + x u_1 e_0e_2e_3 e_2e_3 + t u_1 e_1e_2e_3 e_2e_3 \\
    = ~ & - z u_3 e_0 + y u_3 e_0e_2e_3 + x u_3 e_0e_1e_3 - t u_3 e_3 \\
    & - z u_2 e_0e_2e_3 - y u_2 e_0 + x u_2 e_0e_1e_2 - t u_2 e_2 \\
    & - z u_1 e_0e_1e_3 - y u_1 e_0e_1e_2 - x u_1 e_0 - t u_1 e_1 \\
    = ~ & (- x u_1 - y u_2 - z u_3) e_0 - t u_1 e_1 - t u_2 e_2 - t u_3 e_3 \\
    & - (y u_1 - x u_2) e_0e_1e_2 - (z u_1 - x u_3) e_0e_1e_3 - (z u_2 - y u_3) e_0e_2e_3\\
  \end{split}
\end{equation}
\begin{equation}
  A_0 = - x u_1 - y u_2 - z u_3 , \quad
  A_1 = y u_1 - x u_2 , \quad
  A_2 = z u_1 - x u_3 , \quad
  A_3 = z u_2 - y u_3
\end{equation}
と置いて整理すると、
\begin{equation}
  \begin{split}
    \bm{u}^\dag S^\dag = ~ & A_0 e_0 - t u_1 e_1 - t u_2 e_2 - t u_3 e_3 + A_1 e_0e_1e_2 + A_2 e_0e_1e_3 + A_3 e_0e_2e_3\\
    S^\dag \bm{u}^\dag = ~ & A_0 e_0 - t u_1 e_1 - t u_2 e_2 - t u_3 e_3 - A_1 e_0e_1e_2 - A_2 e_0e_1e_3 - A_3 e_0e_2e_3\\
  \end{split}
\end{equation}
\begin{equation}
  \begin{split}
    p q \bm{u}^\dag S^\dag - p q S^\dag \bm{u}^\dag = ~ & 2 p q (A_1 e_0e_1e_2 + A_2 e_0e_1e_3 + A_3 e_0e_2e_3) \\
    = ~ & 2 p q A_1 e_0e_1e_2 + 2 p q A_2 e_0e_1e_3 + 2 p q A_3 e_0e_2e_3 \\
    = ~ & 2 p q (y u_1 - x u_2) e_0e_1e_2 + 2 p q (z u_1 - x u_3) e_0e_1e_3 + 2 p q (z u_2 - y u_3) e_0e_2e_3 \\
  \end{split}
\end{equation}
\begin{equation}
  \begin{split}
    \bm{u}^\dag S^\dag \bm{u}^\dag
    = ~ & (A_0 e_0 - t u_1 e_1 - t u_2 e_2 - t u_3 e_3 + A_1 e_0e_1e_2 + A_2 e_0e_1e_3 + A_3 e_0e_2e_3)
    (u_3 e_1e_2 - u_2 e_1e_3 + u_1 e_2e_3) \\
    = ~ & A_0 u_3 e_0e_1e_2 + t u_1 u_3 e_2 - t u_2 u_3 e_1 - t u_3^2 e_1e_2e_3 - A_1 u_3 e_0 - A_2 u_3 e_0e_2e_3 + A_3 u_3 e_0e_1e_3 \\
    & - A_0 u_2 e_0e_1e_3 - t u_1 u_2 e_3 - t u_2^2 e_1e_2e_3 + t u_2 u_3 e_1 - A_1 u_2 e_0e_2e_3 + A_2 u_2 e_0 + A_3 u_2 e_0e_1e_2 \\
    & + A_0 u_1 e_0e_2e_3 - t u_1^2 e_1e_2e_3 + t u_1 u_2 e_3 - t u_1 u_3 e_2 - A_1 u_1 e_0e_1e_3 + A_2 u_1 e_0e_1e_2 - A_3 u_1 e_0 \\
    = ~ & (- A_3 u_1 + A_2 u_2 - A_1 u_3) e_0 + (A_2 u_1 + A_3 u_2 + A_0 u_3) e_0e_1e_2 + (- A_1 u_1 - A_0 u_2 + A_3 u_3) e_0e_1e_3 \\
    & + (A_0 u_1 - A_1 u_2 - A_2 u_3) e_0e_2e_3 - t e_1e_2e_3 \\
  \end{split}
\end{equation}
係数を展開すると、
\begin{equation}
  \begin{split}
    - A_3 u_1 + A_2 u_2 - A_1 u_3 = ~ & - (z u_2 - y u_3) u_1 + (z u_1 - x u_3) u_2 - (y u_1 - x u_2) u_3 \\
    = ~ & - z u_1 u_2 + y u_1 u_3 + z u_1 u_2 - x u_2 u_3 - y u_1 u_3 + x u_2 u_3 \\
    = ~ & 0 \\
    A_2 u_1 + A_3 u_2 + A_0 u_3 = ~ & (z u_1 - x u_3) u_1 + (z u_2 - y u_3) u_2 + (- x u_1 - y u_2 - z u_3) u_3 \\
    = ~ & z u_1^2 - x u_1 u_3 + z u_2^2 - y u_2 u_3 - x u_1 u_3 - y u_2 u_3 - z u_3^2 \\
    = ~ & - 2 x u_1 u_3 - 2 y u_2 u_3 + z (u_1^2 + u_2^2 - u_3^2) \\
    - A_1 u_1 - A_0 u_2 + A_3 u_3 = ~ & - (y u_1 - x u_2) u_1 - (- x u_1 - y u_2 - z u_3) u_2 + (z u_2 - y u_3) u_3 \\
    = ~ & - y u_1^2 + x u_1 u_2 + x u_1  u_2 + y u_2^2 + z u_2 u_3 + z u_2 u_3 - y u_3^2 \\
    = ~ & 2 x u_1 u_2 + y ( - u_1^2 + u_2^2 - u_3^2) + 2 z u_2 u_3 \\
    A_0 u_1 - A_1 u_2 - A_2 u_3 = ~ & (- x u_1 - y u_2 - z u_3) u_1 - (y u_1 - x u_2) u_2 - (z u_1 - x u_3) u_3 \\
    = ~ & - x u_1^2 - y u_1 u_2 - z u_1 u_3 - y u_1 u_2 + x u_2^2 - z u_1 u_3 + x u_3^2 \\
    = ~ & x ( - u_1^2 + u_2^2 + u_3^2) - 2 y u_1 u_2 - 2 z u_1 u_3 \\
  \end{split}
\end{equation}
\begin{equation}
  \begin{split}
    \bm{u}^\dag S^\dag \bm{u}^\dag
    = ~ & (- 2 x u_1 u_3 - 2 y u_2 u_3 + z (u_1^2 + u_2^2 - u_3^2)) e_0e_1e_2 \\
    & + (2 x u_1 u_2 + y ( - u_1^2 + u_2^2 - u_3^2) + 2 z u_2 u_3) e_0e_1e_3 \\
    & + (x ( - u_1^2 + u_2^2 + u_3^2) - 2 y u_1 u_2 - 2 z u_1 u_3) e_0e_2e_3 - t e_1e_2e_3 \\
  \end{split}
\end{equation}
\begin{equation}
  \begin{split}
    R^\dag S^\dag \widetilde{R^\dag} 
    = ~ & p^2 S^\dag + p q \bm{u}^\dag S^\dag - p q S^\dag \bm{u}^\dag - q^2 \bm{u}^\dag S^\dag \bm{u}^\dag \\
    = ~ & p^2 z e_0e_1e_2 - p^2 y e_0e_1e_3 + p^2 x e_0e_2e_3 + p^2 t e_1e_2e_3 \\
    & + 2 p q (y u_1 - x u_2) e_0e_1e_2 + 2 p q (z u_1 - x u_3) e_0e_1e_3 + 2 p q (z u_2 - y u_3) e_0e_2e_3 \\
    & - q^2 (- 2 x u_1 u_3 - 2 y u_2 u_3 + z (u_1^2 + u_2^2 - u_3^2)) e_0e_1e_2 \\
    & - q^2 (2 x u_1 u_2 + y ( - u_1^2 + u_2^2 - u_3^2) + 2 z u_2 u_3) e_0e_1e_3 \\
    & - q^2 (x ( - u_1^2 + u_2^2 + u_3^2) - 2 y u_1 u_2 - 2 z u_1 u_3) e_0e_2e_3 \\
    & + q^2 t e_1e_2e_3 \\
    = ~ &(p^2 z + 2 p q (y u_1 - x u_2) - q^2 (- 2 x u_1 u_3 - 2 y u_2 u_3 + z (u_1^2 + u_2^2 - u_3^2))) e_0e_1e_2 \\
    & + (- p^2 y + 2 p q (z u_1 - x u_3) - q^2 (2 x u_1 u_2 + y ( - u_1^2 + u_2^2 - u_3^2) + 2 z u_2 u_3)) e_0e_1e_3 \\
    & + (p^2 x + 2 p q (z u_2 - y u_3) - q^2 (x ( - u_1^2 + u_2^2 + u_3^2) - 2 y u_1 u_2 - 2 z u_1 u_3)) e_0e_2e_3 \\
    & + t e_1e_2e_3 \\
    = ~ &(p^2 z + 2 p q (y u_1 - x u_2) + q^2 (2 x u_1 u_3 + 2 y u_2 u_3 + z ( - u_1^2 - u_2^2 + u_3^2))) e_0e_1e_2 \\
    & - (p^2 y + 2 p q (x u_3 - z u_1) + q^2 (2 x u_1 u_2 + y ( - u_1^2 + u_2^2 - u_3^2) + 2 z u_2 u_3)) e_0e_1e_3 \\
    & + (p^2 x + 2 p q (z u_2 - y u_3) + q^2 (x (u_1^2 - u_2^2 - u_3^2) + 2 y u_1 u_2 + 2 z u_1 u_3)) e_0e_2e_3 \\
    & + t e_1e_2e_3 \\
  \end{split}
\end{equation}
これは四元数による三次元の回転と同型と考えられます。以下の様に置き換えます。
\begin{equation}
  q_1 = q u_1 , \quad q_2 = q u_2 , \quad q_3 = q u_3
\end{equation}
\begin{equation}
  R^\dag = \cos \frac{\phi}{2} + \bm{u}^\dag \sin \frac{\phi}{2}
  = p + q u_3 e_1e_2 - q u_2 e_1e_3 + q u_1 e_2e_3
  = p + q_3 e_1e_2 - q_2 e_1e_3 + q_1 e_2e_3
\end{equation}
\begin{equation}
  \left\{
    \begin{aligned}
      &x' = p^2 x + 2 p q (z u_2 - y u_3) + q^2 (x (u_1^2 - u_2^2 - u_3^2) + 2 y u_1 u_2 + 2 z u_1 u_3) \\
      &y' = p^2 y + 2 p q (x u_3 - z u_1) + q^2 (2 x u_1 u_2 + y ( - u_1^2 + u_2^2 - u_3^2) + 2 z u_2 u_3) \\
      &z' = p^2 z + 2 p q (y u_1 - x u_2) + q^2 (2 x u_1 u_3 + 2 y u_2 u_3 + z ( - u_1^2 - u_2^2 + u_3^2)) \\
    \end{aligned}
  \right.
\end{equation}
\begin{equation}
  \Longleftrightarrow
  \left\{
    \begin{aligned}
      &x' = p^2 x + 2 p (z q_2 - y q_3) + x (q_1^2 - q_2^2 - q_3^2) + 2 y q_1 q_2 + 2 z q_1 q_3 \\
      &y' = p^2 y + 2 p (x q_3 - z q_1) + 2 x q_1 q_2 + y ( - q_1^2 + q_2^2 - q_3^2) + 2 z q_2 q_3 \\
      &z' = p^2 z + 2 p (y q_1 - x q_2) + 2 x q_1 q_3 + 2 y q_2 q_3 + z ( - q_1^2 - q_2^2 + q_3^2) \\
    \end{aligned}
  \right.
\end{equation}
\begin{equation}
  \Longleftrightarrow
  \left\{
    \begin{aligned}
      &x' = (p^2 + q_1^2 - q_2^2 - q_3^2) x + 2 (z p q_2 - y p q_3 + y q_1 q_2 + z q_1 q_3) \\
      &y' = (p^2 - q_1^2 + q_2^2 - q_3^2) y + 2 (x p q_3 - z p q_1 + x q_1 q_2 + z q_2 q_3) \\
      &z' = (p^2 - q_1^2 - q_2^2 + q_3^2) z + 2 (y p q_1 - x p q_2 + x q_1 q_3 + y q_2 q_3) \\
    \end{aligned}
  \right.
\end{equation}
\begin{equation}
  \Longleftrightarrow
  \left\{
    \begin{aligned}
      &x' = (p^2 + q_1^2 - q_2^2 - q_3^2) x + 2 ((q_1 q_2 - p q_3) y + (q_1 q_3 + p q_2) z) \\
      &y' = (p^2 - q_1^2 + q_2^2 - q_3^2) y + 2 ((q_1 q_2 + p q_3) x + (q_2 q_3 - p q_1) z) \\
      &z' = (p^2 - q_1^2 - q_2^2 + q_3^2) z + 2 ((q_1 q_3 - p q_2) x + (q_2 q_3 + p q_1) y) \\
    \end{aligned}
  \right.
\end{equation}
四元数による三次元の回転計算とは符号が異なる部分がありますが、
サンドイッチ積の順序が逆になっていて回転角度が逆になったと考えられます。
本質的な部分は変わりません。

一方向だけで計算してみましょう。
$ u_1 = 1 , u_2 = 0 , u_3 = 0$とすると、
\begin{equation}
  \begin{split}
    R^\dag S^\dag \widetilde{R^\dag} 
    = ~ & (p^2 z + 2 p q y - q^2 z) e_0e_1e_2 - (p^2 y - 2 p q z - q^2 y) e_0e_1e_3
    + x e_0e_2e_3 + t e_1e_2e_3 \\
  \end{split}
\end{equation}
\begin{equation}
  \begin{split}
    2 p q = 2 \cos \frac{\phi}{2} \sin \frac{\phi}{2} = \sin \phi , \quad
    p^2 - q^2 = \cos \phi
  \end{split}
\end{equation}
\begin{equation}
  \begin{split}
    R^\dag S^\dag \widetilde{R^\dag} 
    = ~ & (\sin \phi y + \cos \phi z) e_0e_1e_2 - (\cos \phi y - \sin \phi z) e_0e_1e_3
    + x e_0e_2e_3 + t e_1e_2e_3 \\
  \end{split}
\end{equation}
角度を中心にまとめると、
\begin{equation}
  \left\{
    \begin{aligned}
      &x' = p^2 x + 2 p q (z u_2 - y u_3) + q^2 (x (u_1^2 - u_2^2 - u_3^2) + 2 y u_1 u_2 + 2 z u_1 u_3) \\
      &y' = p^2 y + 2 p q (x u_3 - z u_1) + q^2 (2 x u_1 u_2 + y ( - u_1^2 + u_2^2 - u_3^2) + 2 z u_2 u_3) \\
      &z' = p^2 z + 2 p q (y u_1 - x u_2) + q^2 (2 x u_1 u_3 + 2 y u_2 u_3 + z ( - u_1^2 - u_2^2 + u_3^2)) \\
    \end{aligned}
  \right.
\end{equation}
\begin{equation}
  \Longleftrightarrow
  \left\{
    \begin{aligned}
      &x' = (1 - q^2) x + 2 p q (z u_2 - y u_3) + q^2 (x (u_1^2 - u_2^2 - u_3^2) + 2 y u_1 u_2 + 2 z u_1 u_3) \\
      &y' = (1 - q^2) y + 2 p q (x u_3 - z u_1) + q^2 (y ( - u_1^2 + u_2^2 - u_3^2) + 2 x u_1 u_2 + 2 z u_2 u_3) \\
      &z' = (1 - q^2) z + 2 p q (y u_1 - x u_2) + q^2 (z ( - u_1^2 - u_2^2 + u_3^2) + 2 x u_1 u_3 + 2 y u_2 u_3) \\
    \end{aligned}
  \right.
\end{equation}
\begin{equation}
  \Longleftrightarrow
  \left\{
    \begin{aligned}
      &x' = x + 2 p q (z u_2 - y u_3) + q^2 (x (u_1^2 - u_2^2 - u_3^2 - 1) + 2 (y u_1 u_2 + z u_1 u_3)) \\
      &y' = y + 2 p q (x u_3 - z u_1) + q^2 (y (u_2^2 - u_3^2 - u_1^2 - 1) + 2 (x u_1 u_2 + z u_2 u_3)) \\
      &z' = z + 2 p q (y u_1 - x u_2) + q^2 (z (u_3^2 - u_1^2 - u_2^2 - 1) + 2 (x u_1 u_3 + y u_2 u_3)) \\
    \end{aligned}
  \right.
\end{equation}
\begin{equation}
  \Longleftrightarrow
  \left\{
    \begin{aligned}
      &x' = x + (z u_2 - y u_3) \sin \phi + (x (u_1^2 - u_2^2 - u_3^2 - 1) + 2 (y u_1 u_2 + z u_1 u_3)) \sin^2 \frac{\phi}{2} \\
      &y' = y + (x u_3 - z u_1) \sin \phi + (y (u_2^2 - u_3^2 - u_1^2 - 1) + 2 (x u_1 u_2 + z u_2 u_3)) \sin^2 \frac{\phi}{2} \\
      &z' = z + (y u_1 - x u_2) \sin \phi + (z (u_3^2 - u_1^2 - u_2^2 - 1) + 2 (x u_1 u_3 + y u_2 u_3)) \sin^2 \frac{\phi}{2} \\
    \end{aligned}
  \right.
\end{equation}
\begin{equation}
  \Longleftrightarrow
  \left\{
    \begin{aligned}
      &x' = x + (z u_2 - y u_3) \sin \phi + ( - 2 x (1 - u_1^2) + 2 (y u_1 u_2 + z u_1 u_3)) \sin^2 \frac{\phi}{2} \\
      &y' = y + (x u_3 - z u_1) \sin \phi + ( - 2 y (1 - u_2^2) + 2 (x u_1 u_2 + z u_2 u_3)) \sin^2 \frac{\phi}{2} \\
      &z' = z + (y u_1 - x u_2) \sin \phi + ( - 2 z (1 - u_3^2) + 2 (x u_1 u_3 + y u_2 u_3)) \sin^2 \frac{\phi}{2} \\
    \end{aligned}
  \right.
\end{equation}
\begin{equation}
  \Longleftrightarrow
  \left\{
    \begin{aligned}
      &x' = x + (z u_2 - y u_3) \sin \phi + 2 (x (u_1^2 - 1) + y u_1 u_2 + z u_1 u_3) \sin^2 \frac{\phi}{2} \\
      &y' = y + (x u_3 - z u_1) \sin \phi + 2 (y (u_2^2 - 1) + x u_1 u_2 + z u_2 u_3) \sin^2 \frac{\phi}{2} \\
      &z' = z + (y u_1 - x u_2) \sin \phi + 2 (z (u_3^2 - 1) + x u_1 u_3 + y u_2 u_3) \sin^2 \frac{\phi}{2} \\
    \end{aligned}
  \right.
\end{equation}
循環関数になりますので重力の増大の結果、運動量への影響が反転するという事を言っています。
重力反転が起きるという事は信じられない主張かと思いますが、天文学的観測結果との整合性について後ほど議論していきたいと思います。

どうして、直接的に時空に重力の影響を考えるのではなく、双対時空へかける計算を行うか、その必然性を説明します。
ラピディティの総和を直接時空にかけても重力多体シミュレーションとして一応成立します。
例えば、水星と太陽の要素をインプットすれば回転する楕円軌道が再現できます。
しかし、このシミュレーション方式では重力が大きくなりすぎると無限大でエラーになって粒子の動きが止まってしまいます。
重力の増大で計算不能になる原因はコンピューターの能力の限界ではなく、
物理法則としての実在性について問題がある事を示していると考えます。
無限大が出現する原因は双曲線関数の適用にあるので、三角関数による回転によればその矛盾は回避されると考えました。

\subsection{双対時空から加速度として速度に反映}

重力による影響の結果、双対時空の時空立方体が回転し歪む様子を、三角関数と双曲線関数の混合回転で見てきました。
物体の運動量は時空立方体に沿って進行すると考えられるので、時空立方体が変形すれば運動量が見かけ上変化するように見えます。
物体側からすると運動量を保っているので、慣性力は感じられず無重力浮遊の状態です。
双対時空で双対回転子によって回転した角度をもって、今度は時空側に通常回転子としてサンドイッチ積を行う事で、
最終的に速度を得る事が出来ます。

双対時空を今度は四双四元数で表現してみます。
\begin{equation}
  \begin{split}
    S = ~ & t j + x kI + y kJ + z kK \\
    S^\dag = ~ & (t j + x kI + y kJ + z kK)i = - t k + x jI + y jJ + z jK \\
  \end{split}
\end{equation}
\begin{itemize}
  \item ある瞬間の物体の双対時空: $ S^\dag = - t k + x jI + y jJ + z jK $
  \item $ Cl(1,3) $表現 : $ S^\dag = t e_1e_2e_3 + x e_0e_2e_3 - y e_0e_1e_3 + z e_0e_1e_2 $
\end{itemize}
四双四元数の方が符号が整っている印象です。クリフォード代数でも基底の順序を入れ替える事で符号を揃えることは可能です。

双対時空$ S^\dag $が重力の影響で$ (S^\dag)' $に変換されているとします。
\begin{equation}
  S^\dag = - t k + x jI + y jJ + z jK, \quad
  (S^\dag)' = - t' k + x' jI + y' jJ + z' jK
\end{equation}
幾何積から内積からこれらの角度を計算します。
\begin{equation}
  \begin{split}
    S^\dag (S^\dag)' = ~ & (- t k + x jI + y jJ + z jK)(- t' k + x' jI + y' jJ + z' jK) \\
    = ~ & - t t' + t x' iI + t y' iJ + t z' iK \\
    & - x t' iI + x x' - x y' K + x z' J \\
    & - y t' iJ + y x' K + y y' - y z' I \\
    & - z t' iK - z x' J + z y' I + z z' \\
    = ~ & - t t' + x x' + y y' + z z' \\
    & + t x' iI - x t' iI + t y' iJ - y t' iJ + t z' iK - z t' iK \\
    & + z y' I - y z' I + x z' J - z x' J + y x' K - x y' K \\
    = ~ & - t t' + x x' + y y' + z z' \\
    & + (t x' - x t') iI + (t y' - y t') iJ + (t z' - z t') iK
    + (z y' - y z') I + (x z' - z x') J + (y x' - x y') K \\
  \end{split}
\end{equation}
内積の符号は四双四元数が$ Cl(3,1) $と同型である事を反映しています。
\begin{equation}
  \begin{split}
    S^\dag (S^\dag)' = ~ & (z e_0e_1e_2 - y e_0e_1e_3 + x e_0e_2e_3 + t e_1e_2e_3)(z' e_0e_1e_2 - y' e_0e_1e_3 + x' e_0e_2e_3 + t' e_1e_2e_3) \\
    = ~ & z z' e_0e_1e_2 e_0e_1e_2 - z y' e_0e_1e_2 e_0e_1e_3 + z x' e_0e_1e_2 e_0e_2e_3 + z t' e_0e_1e_2 e_1e_2e_3 \\
    & - y z' e_0e_1e_3 e_0e_1e_2 + y y' e_0e_1e_3 e_0e_1e_3 - y x' e_0e_1e_3 e_0e_2e_3 - y t' e_0e_1e_3 e_1e_2e_3 \\
    & + x z' e_0e_2e_3 e_0e_1e_2 - x y' e_0e_2e_3 e_0e_1e_3 + x x' e_0e_2e_3 e_0e_2e_3 + x t' e_0e_2e_3 e_1e_2e_3 \\
    & + t z' e_0e_2e_3 e_0e_1e_2 - t y' e_0e_2e_3 e_0e_1e_3 + t x' e_0e_2e_3 e_0e_2e_3 + t t' e_0e_2e_3 e_1e_2e_3 \\
    = ~ & - z z' - z y' e_2e_3 - z x' e_1e_3 - z t' e_0e_3 \\
    & + y z' e_2e_3 - y y' - y x' e_1e_2 - y t' e_0e_2 \\
    & + x z' e_1e_3 + x y' e_1e_2 - x x' - x t' e_0e_1 \\
    & + t z' e_0e_3 + t y' e_0e_2 + t x' e_0e_1 + t t' \\
    = ~ & t t' - x x' - y y' - z z' \\
    & + (t x' - x t') e_0e_1 + (t y' - y t') e_0e_2 + (t z' - z t') e_0e_3
    + (x y' - y x') e_1e_2 + (x z' - z x') e_1e_3 + (y z' - z y') e_2e_3 \\
  \end{split}
\end{equation}
外積については、四双四元数と$ Cl(1,3) $は一致しています。
\begin{equation}
  S^\dag \cdot (S^\dag)' = t t' - x x' - y y' - z z'
\end{equation}
扱う時空が時間的であるとすると、$ S^\dag \cdot S^\dag > 0 $かつ$ (S^\dag)' \cdot (S^\dag)' > 0 $となり、
ミンコフスキー時空の内積の性質から、
\begin{equation}
  S^\dag \cdot (S^\dag)' = \sqrt{S^\dag \cdot S^\dag} \sqrt{(S^\dag)' \cdot (S^\dag)'} \cosh \theta
\end{equation}
$ (S^\dag)' $の不変量は$ S^\dag $の不変量と一致しているはずですから、
\begin{equation}
  S^\dag \cdot (S^\dag)' = \sqrt{(S^\dag)^2} \sqrt{(S^\dag)^2} \cosh \theta = (S^\dag)^2 \cosh \theta
\end{equation}
従って、
\begin{equation}
  \theta = \arccosh \frac{S^\dag \cdot (S^\dag)'}{(S^\dag)^2}
  = \arccosh \frac{t t' - x x' - y y' - z z'}{t^2 - x^2 - y^2 - z^2}
\end{equation}
しかし、この方法では1種類の角度しか求められません。欲しいのは2種類の角度になります。
双対時空へのブーストはそのまま時空への空間回転になるので再計算の必要はないと考えられます。

\subsection{空間部分の回転}

双対時空の空間回転の双対は時空へのブーストになります。
\begin{itemize}
  \item 双対時空単位基底: $ k, c jI, c jJ, c jK $
  \item 回転後: $ k + b_1 c jI + b_2 c jJ + b_3 c jK $
  \item 進行方向の回転子ベルソル: $ \bm{u} = u_1 iI + u_2 iJ + u_3 iK $
  \item 進行方向の回転角: $ \phi $
\end{itemize}
\begin{equation}
  \left\{
    \begin{aligned}
      &x' = x + (z u_2 - y u_3) \sin \phi + 2 (x (u_1^2 - 1) + y u_1 u_2 + z u_1 u_3) \sin^2 \frac{\phi}{2} \\
      &y' = y + (x u_3 - z u_1) \sin \phi + 2 (y (u_2^2 - 1) + x u_1 u_2 + z u_2 u_3) \sin^2 \frac{\phi}{2} \\
      &z' = z + (y u_1 - x u_2) \sin \phi + 2 (z (u_3^2 - 1) + x u_1 u_3 + y u_2 u_3) \sin^2 \frac{\phi}{2} \\
    \end{aligned}
  \right.
\end{equation}
\begin{equation}
  \left\{
    \begin{aligned}
      &b_1 = 1 + (u_2 - u_3) \sin \phi + 2 ((u_1^2 - 1) + u_1 u_2 + u_1 u_3) \sin^2 \frac{\phi}{2} \\
      &b_2 = 1 + (u_3 - u_1) \sin \phi + 2 ((u_2^2 - 1) + u_1 u_2 + u_2 u_3) \sin^2 \frac{\phi}{2} \\
      &b_3 = 1 + (u_1 - u_2) \sin \phi + 2 ((u_3^2 - 1) + u_1 u_3 + u_2 u_3) \sin^2 \frac{\phi}{2} \\
    \end{aligned}
  \right.
\end{equation}
空間部分の内積から、
\begin{equation}
  \begin{split}
    (c jI + c jJ + c jK) \cdot (b_1 c jI + b_2 c jJ + b_3 c jK)
    = ~ & \sqrt{c^2 + c^2 + c^2} \sqrt{b_1^2 c^2 + b_2^2 c^2 + b_3^2 c^2} \cos \phi' \\
    b_1 c^2 + b_2 c^2 + b_3 c^2
    = ~ & c \sqrt{3} c \sqrt{b_1^2 + b_2^2 + b_3^2} \cos \phi' \\
    b_1 + b_2 + b_3
    = ~ & \sqrt{3} \sqrt{b_1^2 + b_2^2 + b_3^2} \cos \phi' \\
  \end{split}
\end{equation}
回転後の長さは変わらないので、
\begin{equation}
  \begin{split}
    & c^2 + c^2 + c^2 = b_1^2 c^2 + b_2^2 c^2 + b_3^2 c^2 \\
    & b_1^2 + b_2^2 + b_3^2 = 3 \\
  \end{split}
\end{equation}
\begin{equation}
  \phi' = \arccos \frac{b_1 + b_2 + b_3}{3}
\end{equation}

\subsubsection{時空立方体の歪みによる時間の遅れ}

$ u_1 = 1, u_2 = 0, u_3 = 0 $の時の回転後の係数は、
\begin{equation}
  \left\{
    \begin{aligned}
      &b_1 = 1 \\
      &b_2 = 1 - \sin \phi - 2 \sin^2 \frac{\phi}{2} \\
      &b_3 = 1 + \sin \phi - 2 \sin^2 \frac{\phi}{2} \\
    \end{aligned}
  \right.
\end{equation}
\begin{equation}
  \cos \phi' = \frac{b_1 + b_2 + b_3}{3} = 1 - \frac{4}{3} \sin^2 \frac{\phi}{2}
\end{equation}
この式は時空超立方体における時間が関わる超平面の3つの合計値が小さくなる事を意味しています。
つまり、物体が速度を持つことによる時間の遅れ(特殊相対論的な時間の遅れ)とは別に、
時空超立方体が歪むことによる時間の遅れを定義する事が出来ます。
全方位的に計算してみると、
\begin{equation}
  \left\{
    \begin{aligned}
      &b_1 = 1 + (u_2 - u_3) \sin \phi + (2 u_1^2 - 2 + 2 u_1 u_2 + 2 u_1 u_3) \sin^2 \frac{\phi}{2} \\
      &b_2 = 1 + (u_3 - u_1) \sin \phi + (2 u_2^2 - 2 + 2 u_1 u_2 + 2 u_2 u_3) \sin^2 \frac{\phi}{2} \\
      &b_3 = 1 + (u_1 - u_2) \sin \phi + (2 u_3^2 - 2 + 2 u_1 u_3 + 2 u_2 u_3) \sin^2 \frac{\phi}{2} \\
    \end{aligned}
  \right.
\end{equation}
\begin{equation}
  \begin{split}
    \frac{b_1 + b_2 + b_3}{3} = ~ & 1 + \frac{4}{3} (-1 + u_1 u_2 + u_1 u_3 + u_2 u_3) \sin^2 \frac{\phi}{2} \\
    = ~ & 1 - \frac{4}{3} \sin^2 \frac{\phi}{2} + \frac{4}{3} (u_1 u_2 + u_1 u_3 + u_2 u_3) \sin^2 \frac{\phi}{2} \\
  \end{split}
\end{equation}
立方体の斜め方向に回転すると値が変わって来ますが、実は直交座標表現の立方体というのは虚実であって、
根底にある角度という本質への変換に必要な項目と考えられます。
座標の方向に合わせた場合については、通常の時間の流れの速さを1とすると時間の遅れは以下の式になります。
\begin{equation}
  \texttt{Time dilation}: \frac{4}{3} \sin^2 \frac{\phi}{2}
\end{equation}
この値は1よりも大きくなるので時間反転を起こすことを意味しています。
時間反転の境界の条件は、
\begin{equation}
  \begin{split}
    & \frac{4}{3} \sin^2 \frac{\phi}{2} = 1
    \quad \leftrightarrow \quad \sin^2 \frac{\phi}{2} = \frac{3}{4}
    \quad \leftrightarrow \quad \sin \frac{\phi}{2} = \frac{\sqrt{3}}{2} \\
    & \leftrightarrow \quad \phi = 2 \arcsin \frac{\sqrt{3}}{2} 
    = \frac{2}{3} \pi, \frac{4}{3} \pi, ... = 120^\circ, 240^\circ, ... \\
  \end{split}
\end{equation}
時間の流れが変われば加速度$ a $も影響を受けるので、時間成分を掛け合わせたラピディティは、
\begin{equation}
  \Theta = \phi' \cos \phi'
\end{equation}
速度ゼロからの加速度で考えると、
\begin{equation}
  \frac{a}{c} = \tanh( \phi' \cos \phi' )
\end{equation}
$ 0 \le \phi < \frac{2}{3} \pi $の範囲で重力が正の方向で、
$ \frac{2}{3} \pi < \phi < \frac{4}{3} \pi $の範囲では重力が負の方向になり、重力反転を起こします。
$ \frac{4}{3} \pi < \phi < \frac{8}{3} \pi $の範囲では重力は再び正の方向で、
$ \frac{8}{3} \pi < \phi < \frac{10}{3} \pi $の範囲で再度重力反転を起こします。

なお、ここで扱っている回転子の角度については一切の物理定数が関わってこない事が式から判明しています。
従って、角度と実際の物理定数との突合は観測によらなければならないです。
地球表面上での時間の遅れの観測値は、時間の流れを1とした時、$ 6.95 \times 10^{-10} $です。
\begin{equation}
  dt = \frac{4}{3} \sin^2 \frac{\phi}{2}
  \quad \leftrightarrow \quad \sin \frac{\phi}{2} = \sqrt{\frac{3}{4} dt}
  \quad \leftrightarrow \quad \phi = 2 \arcsin \sqrt{\frac{3}{4} dt}
\end{equation}
微小角近似$\arcsin x \approx x$を使って、
\begin{equation}
  \phi = 2 \sqrt{\frac{3}{4} dt} = \sqrt{3 dt} = 4.566 \times 10^{-5}
\end{equation}
地球表面の重力加速度$ 9.8 m/s^2 $に比例しているとすると比例定数$ k $は、
\begin{equation}
  k = 4.66 \times 10^{-6}
\end{equation}

\subsection{双対時空への双曲的歪曲}

重力による双対時空への双曲的歪曲$ \phi $の影響を計算していきます。
$ \phi $はラピディティとは異なるので注意しましょう。\\
相対速度の方向のベルソルを$ \bm{v}^\dag = v_1 iI + v_2 iJ + v_3 iK, \quad {v_1}^2 + {v_2}^2 + {v_3}^2 = 1 $として
指数関数の回転子を展開します。
\begin{equation}
  \exp(\bm{v}^\dag) = r_0 + r_1 iI + r_2 iJ + r_3 iK, \quad
  r_0 = \cosh \frac{\theta}{2}, \quad
  (r_1, r_2, r_3) = (v_1, v_2, v_3) \sinh \frac{\theta}{2}
\end{equation}
と置くと、双対時空のブースト$  $は、
\begin{equation}
  \begin{split}
    =~&(r_0^2+r_1^2+r_2^2+r_3^2)t-2r_0(r_1x+r_2y+r_3z)\\
    x'=~&(r_0^2+r_1^2-r_2^2-r_3^2)x-2r_1(r_0t-r_2y-r_3z)\\
    y'=~&(r_0^2-r_1^2+r_2^2-r_3^2)y-2r_2(r_0t-r_1x-r_3z)\\
    z'=~&(r_0^2-r_1^2-r_2^2+r_3^2)z-2r_3(r_0t-r_1x-r_2y)
  \end{split}
\end{equation}
\begin{equation}
  \begin{split}
    t'=~&(r_0^2+r_1^2+r_2^2+r_3^2)t-2r_0(r_1x+r_2y+r_3z)\\
    x'=~&(r_0^2+r_1^2-r_2^2-r_3^2)x-2r_1(r_0t-r_2y-r_3z)\\
    y'=~&(r_0^2-r_1^2+r_2^2-r_3^2)y-2r_2(r_0t-r_1x-r_3z)\\
    z'=~&(r_0^2-r_1^2-r_2^2+r_3^2)z-2r_3(r_0t-r_1x-r_2y)
  \end{split}
\end{equation}


\end{document}